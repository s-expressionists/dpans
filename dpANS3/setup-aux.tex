% -*- Mode: TeX -*-
\overfullrule 0pt
\let\sub_		%subscripts
% fonts
\def\Font#1{\def\next{\fixfont#1}\afterassignment\next\font#1}
\def\fixfont#1{\fontdimen3#1=0pt\fontdimen4#1=0pt}
\def\sc {scaled}

\let\mh	 \magstephalf		\def\mi	  {\magstep1}
\def\mii {\magstep2}		\def\miii {\magstep3}

\def\beginImplNote
{\begingroup\advance\rightskip 3pc\advance\leftskip 3pc
{\bf Implementation Note: }\vrule width0pt depth 5pt\hfil\break}
\def\endImplNote{\par\endgroup}

\def\beginTermNote
{\begingroup\advance\rightskip 2pc\advance\leftskip 2pc
{\bf Terminology Note: }\hfil\break}                       
\def\endTermNote{\par\endgroup}

\input setup-cmfont
%\input setup-amfont

\newif \iftt
\newif \ifbf
\newif \ifsphy \sphyfalse

\def\tenpoint
{\let\bit\prbiten
\let\bbf\prbeleven
\def\bbfl{\prbtwelve}%
\def\brfl{\prmtwelve}%
%%
    \def\arg{\ssiten}%%
    \def\keyword{\tt}%%
    \def\function{\bbfnine}%%
   %\def\datatype{\bslten}%%
%!!! No longer used. -kmp 9-May-91
%    \def\datatype{\it}%%
    \def\word{\it}%%
% Experimentally removed.  This is defined by \let above, and I see no reason to override it.
% -kmp 6-Sep-91
%   \def\bit{\it}%%
    \def\constant{\ssqeight}%%
    \def\cltl{\bscten}%%
    \def\andarg{\bslten}%%
    \def\argument{\ssiten}%%
%    \def\cal{\calten}%%
%\def\rm{\fam0\prmten\textfont7\miten \textfont8\syten \ttfalse\bffalse}%
\def\bf{\fam4\prbten\textfont7\bmiten\textfont8\bsyten\ttfalse\bftrue}%
\def\it{\fam5\priten\textfont7\miten \textfont8\syten \ttfalse\bffalse}%
\def\tt{\catothers\fam6\lgnine\textfont7\miten\textfont8\syten\tttrue\bffalse}%
\textfont0=\prmten \scriptfont0=\prmseven \scriptscriptfont0=\prmfive
\textfont1=\miten  \scriptfont1=\miseven  \scriptscriptfont1=\mifive
\textfont2=\syten  \scriptfont2=\syseven  \scriptscriptfont2=\syfive
\textfont4=\prbten \scriptfont4=\prbseven	
\textfont5=\priten \scriptfont5=\priseven
\textfont6=\lgnine
\baselineskip 11pt\rm
}

\def\elevenpoint
{\let\bit\prbieleven
\let\bbf\prbtwelve
%\def\rm{%
%\fam0\prmeleven\textfont7\mieleven \textfont8\syeleven \ttfalse\bffalse}%
\def\bf{%
\fam4\prbeleven\textfont7\bmieleven\textfont8\bsyeleven\ttfalse\bftrue}%
\def\it{%
\fam5\prieleven\textfont7\mieleven \textfont8\syeleven \ttfalse\bffalse}%
\def\tt{\catothers
\fam6\lgnine   \textfont7\mieleven \textfont8\syeleven \tttrue \bffalse}%
\textfont0=\prmeleven \scriptfont0=\prmeight \scriptscriptfont0=\prmfive
\textfont1=\mieleven  \scriptfont1=\mieight  \scriptscriptfont1=\mifive
\textfont2=\syeleven  \scriptfont2=\syeight  \scriptscriptfont2=\syfive
\textfont4=\prbeleven \scriptfont4=\prbeight
\textfont5=\prieleven \scriptfont5=\prieight
\textfont6=\lgnine
\baselineskip 12pt\rm
}

\def\twelvepoint
{\let\rm\prmtwelve
\let\bf\prbtwelve
\baselineskip 13pt
\rm
}

\def\bften
{\fam4\prbten  \textfont7\bmiten  \textfont8\bsyten  \bftrue\ttfalse}
\def\bfeleven
{\fam4\prbeleven  \textfont7\bmieleven  \textfont8\bsyeleven  \bftrue\ttfalse}
\def\bftwelve
{\fam4\prbtwelve  \textfont7\bmitwelve  \textfont8\bsytwelve  \bftrue\ttfalse}
\def\bffourteen
{\fam4\prbfourteen\textfont7\bmifourteen\textfont8\bsyfourteen\bftrue\ttfalse}
\def\bfsixteen
{\fam4\prbsixteen \textfont7\bmisixteen \textfont8\bsysixteen \bftrue\ttfalse}
\def\bfeighteen
{\fam4\prbeighteen\textfont7\bmieighteen\textfont8\bsyeighteen\bftrue\ttfalse}

\let\normaltype=\elevenpoint
\normaltype
\def\marginstyle{\ttfalse\bffalse\vrule height6pt depth2pt width0pt\prmseven}
\let\df\tt

% page layout

\newskip  \normalleftskip	\normalleftskip   = 10pc
\newskip  \comleftskip		\comleftskip      = 6pc
\newskip  \hcomleftskip		\hcomleftskip     = 3pc
\newskip  \normalparskip	\normalparskip    = 1pc
\newdimen \combarht		\combarht         = 1pt

\hsize    40pc			\vsize    43pc
\topskip   2pc
\leftskip \normalleftskip	\rightskip 0pc plus 3pc
\parindent 0pc
\parskip  \normalparskip

\let\NIS=\nointerlineskip
\def\NIPS{\NIS\parskip 0pc\relax}
\def\removedepth{\ifdim \prevdepth>-1000pt \vskip -\prevdepth\fi}

\def\Vskip #1!{\endgraf
\removedepth
\ifdim \lastskip<#1 \ifdim \lastskip>0pc \removelastskip\fi \vskip#1\NIPS\fi}
\def\VPskip #1 plus #2!{\endgraf
\removedepth
\ifdim \lastskip<#1 \ifdim \lastskip>0pc \removelastskip\fi \vskip#1 plus #2\NIPS\fi}

\def\par{\ifvmode\else\endgraf
\removedepth
\NIS\parskip \normalparskip\relax\fi}

\let\normpar=\par

\def\shortpar{\begingroup\def\par{\endgraf\endgroup\normpar}
\advance\rightskip\leftskip}

\newdimen \fullhsize	\fullhsize=40pc
\def\fullline{\hbox to \fullhsize}

\newtoks \headline		\newtoks \footline
\countdef\pageno=0	\pageno=-1
\let\chapno = \empty

\newskip\iskip  \newskip\iiskip \newskip\iiiskip
\newbox \firstcolbox    	\newbox \othercolbox

\newwrite \tocfile		%\openout \tocfile \jobname.toc
\newwrite \figfile	      	%\openout \figfile \jobname.fig
\newwrite \idxfile		%\openout \idxfile \jobname.idx
\newwrite \issfile		%\openout \issfile \jobname.iss
\newwrite \reffile		%\openout \reffile \jobname.ref
\newwrite \secfile		%\openout \secfile \jobname.sec

% \write\issfile{}
% \write\idxfile{}

\newcount\capno			\capno=1

\newcount\secn			\secn=1
\newcount\ssecn			\ssecn=1
\newcount\sssecn		\sssecn=1
\newcount\ssssecn		\ssssecn=1
\newcount\sssssecn		\sssssecn=1
\newcount\ssssssecn		\ssssssecn=1

\newbox\comline

%% Used to say "Working Draft American National Standard for Information Systems---xxx"
%% but now we avoid mentioning the status since it might change overnight.  Say this
%% longer info only on the cover page.
\def\beginchapter#1#2#3#4{\xbeginchapter{#1}{\bookline}{#2}{#2}{#3}{#4}\par
\endTitlePage}

\def\xbeginchapter #1#2#3#4#5#6{%Open data files 
\immediate\openout \tocfile \jobname.toc
\immediate\openout \figfile \jobname.fig
\immediate\openout \idxfile \jobname.idx
\immediate\openout \issfile \jobname.iss
\immediate\openout \reffile \jobname.ref
\immediate\openout \secfile \jobname.sec
\write\issfile{}
\write\idxfile{}
\toctrue
%\toc4{}
\DefineChapter{#5}{#6}{#1}{#3}
\capno=1\ssecn=0\sssecn=0\ssssecn=0\sssssecn=0\ssssssecn=0\relax
\def\chapno{#1}\def\chapline{#4}\setbox\comline\null
\Head {#2}
\HeadI {#1. #3}
\begingroup 
\leftskip \normalleftskip \rightskip 6pc plus 2pc
\vfill}

\def\endTitlePage
{\par\endgroup\vskip 2pc\break\eject
\ifshowtoc\else{$ $}\vfill\eject\fi
\normaltype
\pageno=1}

\newinsert \idxins
\dimen\idxins=\maxdimen
\count\idxins=0
\skip\idxins = 0pt

\newif  \ifsilent
\newif  \ifbold

% output routines


\newif \iffooter        % if there a footer
\newif \ifticks         % positioning lines for output
\newif \ifcomfirst      % header in command chapter

%\hoffset 1.0in
%\voffset 1.0in  

\def\llbrac{\lbrack\!\lbrack\,}
\def\rrbrac{\,\rbrack\!\rbrack}

\newif \ifincom \incomfalse

\def\makepagerule{\hrule height1.5pt width \fullhsize}

% The reason this doesn't show you the name at the top of the first page
% is that TeX has already processed the next entry (the one that didn't fit)
% when it does the pagination.  So if Foo,Bar,Baz is the sequence, but only Foo
% and Bar would fit, the headline will be Baz even though only Foo and Bar fit.
%  --kmp 8-Apr-92
\def\makeheadline{\setbox0=\fullline{\the\headline}\ht0=1pc\dp0=4.5pt\box0}
\headline={\prbfourteen\ifcomfirst\else
\ifodd\pageno\hss\copy\comline\else\copy\comline\hss\fi\fi}

\def\makefootline{\iffooter\setbox0=\fullline{\the\footline}\dp0=.5pc\box0\fi}
\footline={\normaltype
\ifodd\pageno\hss\chapline\quad\folio\else\folio\quad\bookline\hss\fi}
%Was:
%\ifodd\pageno\hss\chapline\ \ \folio\else\folio\ \ \bookline\hss\fi
%but the "\ \ " seems to work poorly when in a \code...\endcode context,
%where "\ " has a locally bizarre meaning.

\def\folio{{\bf\pagenumber}}
\def\pagenumber
{\ifnum\pageno>0 \chapno--\the\pageno\else\romannumeral-\pageno\fi}

\def\advancepageno{\ifnum\pageno<0 \global\advance\pageno -1
  \else\global\advance\pageno 1\fi} 
                                          
\def\normalpage{\unvbox255\relax}

\def\onecolumnReally{\output{\closout\normalpage}}
\let\onecolumn=\onecolumnReally

\def\thedraftcomment{}
\def\draftcomment#1{\def\thedraftcomment{, #1}}

\def\closout#1{\shipout\vbox
{\ifdraft
\vbox to 0pt {\vss\baselineskip 12pt
\hbox{\prmeleven Version \rev\thedraftcomment}\hbox{\prmeleven \timestamp}
\vskip 2pc}
\fi
\offinterlineskip
\ifticks\topticks\fi
\makeheadline
\makepagerule
\vbox to 47pc {#1\vss\makefootline} %was 45pc -kmp 8-Apr-92
\ifticks\botticks\fi
}
\advancepageno
\global\comfirstfalse
\ifincom\else\global\setbox\comline\null\fi
}

%% This stuff is taken with modification from the TeX manual (8th printing, Aug86), p257 -kmp
\let\lr=L

\newbox\leftcolumn

\def\columnbox{\leftline{\pagebody}}
\def\twocolumn{\hsize 18pc\output{%
\if L\lr
  \global\setbox\leftcolumn=\columnbox \global\let\lr=R
\else
  \doubleformat
  \global\let\lr=L\fi
\ifnum\outputpenalty>187-20000 \else\dosupereject\fi}%
\def\onecolumn{\hsize 40pc%
\if R\lr\doubleformat\global\let\lr=L\fi
\global\let\onecolumn=\onecolumnReally
\onecolumnReally}}

\def\doubleformat{\shipout\vbox
{\ifdraft
\vbox to 0pt {\vss\baselineskip 12pt
\hbox{\prmeleven Version \rev\thedraftcomment}\hbox{\prmeleven \timestamp}
\vskip 2pc}
\fi
\offinterlineskip
\ifticks\topticks\fi
\makeheadline
\makepagerule
\vbox to 47pc {\hbox to 40pc{\vbox to 43pc{\box\leftcolumn\vfil}\hfil\vbox{\columnbox\vfil}}\vss\makefootline} %was 45pc -kmp 8-Apr-92
\ifticks\botticks\fi
}
\advancepageno
\global\comfirstfalse
\ifincom\else\global\setbox\comline\null\fi
}

\onecolumn
%% This gets defined in setup-version, so it will be consistent across all chapters.
%
% {
% \count0=\time
% \count2=\time\divide\count2 by 60\multiply\count2by60\advance\count0by-\count2
% \count2=\time\divide\count2 by 60
% 
% \xdef\timestamp{\ifcase\month\or
% Jan\or Feb\or Mar\or Apr\or May\or Jun\or Jul\or Aug\or
% Sep\or Oct\or Nov\or Dec\fi\space\number\day, \number\year\space\space
% \the\count2:\ifnum\count0<10\relax0\fi\the\count0}
% }

\def\topticks
{\setbox0=\fullline{\hskip-1pc\vrule height .2pt width 1pc\relax
\hskip -.2pt\vrule height 1pc width .2pt\hfil
\vrule height 1pc width .2pt\hskip -.2pt\relax
\vrule height .2pt width 1pc\hskip -1pc}\ht0=0pc\box0}

\def\botticks
{\setbox0=\fullline{\hskip-1pc\vrule height 0pt depth .2pt width 1pc\relax
\hskip -.2pt\vrule height 0pt depth 1pc width .2pt\hfil
\vrule height 0pt depth 1pc width .2pt\hskip -.2pt\relax
\vrule height 0pt depth .2pt width 1pc\hskip -1pc}\dp0=0pc\box0}

\def\pageticks{\tickstrue}
\def\nopageticks{\ticksfalse}
\def\footers{\footertrue}
\def\nofooters{\footerfalse}


\setbox\comline \null   \let\chapline\empty
\footers		\nopageticks

% Tables of Contents

\def\dotleader{\leaders\hbox to 6pt {\hfil\prmfive.\hfil}\hfill}

% start \numitem
\def\numhangsize{25pt}
\def\yskip{\penalty-50\vskip 3pt plus 3pt minus 2pt}
\def\numtextindent#1{\noindent\hbox to \numhangsize{\hskip 0pt plus 1000pt minus 1000pt#1\ }}
\def\numhang{\hangindent \numhangsize}
\def\numitem#1{\yskip\numhang\numtextindent{#1}}
% end \numitem

% {\obeylines
% \gdef\Czero#1
% {\Vskip1pc!\bbf #1\par}
% \gdef\Cone#1\!#2
% {\Vskip1pc!\bbf #1\dotleader#2\hskip-5pc\null\par}
% \gdef\Ctwo#1\!#2
% {\hangindent1pc\rm #1\dotleader#2\hskip-5pc\null\par}
% \gdef\Cthree#1
% {}
% \global\let\Cfour\Cthree
% \global\let\Pzero\Cthree
% \global\let\Pone \Cthree
% \gdef\Ptwo#1\!#2
% {\leftskip 0pt\hangindent 1pc\rm#1\dotleader#2\hskip-5pc\null\par}
% \gdef\Pthree#1\!#2
% {\leftskip 1pc\rm#1\dotleader#2\hskip-5pc\null\par}
% \global\let\Pfour\Cthree
% \gdef\Pfive#1\!#2
% {\leftskip 1pc\rm#1\dotleader#2\hskip-5pc\null\par}
% \gdef\Psix#1\!#2
% {\leftskip 1pc\rm#1\dotleader#2\hskip-5pc\null\par}
% }

{\obeylines
\gdef\Czero#1
{\Vskip1pc!\bbf #1\par}
\gdef\Cone#1\!#2
{\Vskip1pc!\bbf #1\dotleader#2\hskip-5pc\null\par}
\gdef\Ctwo#1\!#2
{\hangindent1pc\rm #1\dotleader#2\hskip-5pc\null\par}
\global\let\Cthree\Ctwo
\global\let\Cfour\Ctwo
\global\let\Cfive\Ctwo
\global\let\Csix\Ctwo
\gdef\Pempty#1
{}
\global\let\Pzero\Pempty
\global\let\Pone \Pempty
% \gdef\Ptwo#1\!#2
% {\leftskip 0pt\hangindent 1pc\rm#1\dotleader#2\hskip-5pc\null\par}
% \gdef\Pthree#1\!#2
% {\leftskip 1pc\rm#1\dotleader#2\hskip-5pc\null\par}
\gdef\Ptwo#1\!#2
{\leftskip 0pt\rm #1\dotleader#2\hskip-5pc\null\par}
\gdef\Pthree#1\!#2
{\leftskip 0pt\hangindent1pc\rm #1\dotleader#2\hskip-5pc\null\par}
\global\let\Pfour\Pthree
\global\let\Pfive\Pthree
\global\let\Psix\Pthree
\gdef\Pnine#1:#2\!#3
{\leftskip 0pt\rightskip 0pt\hangindent 1pc%
\ifx#1R{\clref{#2}}\else
\ifx#1C{\f{#2}}\else
\ifx#1K{\kwd{#2}}\else
\ifx#1T{{\rm #2}}\else
\ifx#1G{\term{#2}}\else
\ifx#1E{\f{#2}{\rm example}}\else
\ifx#1P{\packref{#2} \term{package}}\fi\fi\fi\fi\fi\fi\fi\quad#3\hfil\null\par}
}
%\tracingcommands=1

% These are new in an attempt to allow us to make index entries
% not just for function names, but for other things as well. -kmp 25-Apr-93
\def\idxref#1{\logidx{R}{#1}}
\def\idxkeyref#1{\logidx{R}{\&#1}}
\def\idxcode#1{\logidx{C}{#1}}
\def\idxkwd#1{\logidx{K}{#1}}
\def\idxtext#1{\logidx{T}{#1}}
\def\idxterm#1{\logidx{G}{#1}}          % G = Glossary
\def\idxexample#1{\logidx{E}{#1}}
\def\idxpackref#1{\logidx{P}{#1}}

\newif \iftoc \tocfalse
\newif \ifshowtoc \showtoctrue

\def\ThisSection{$mm$.$nn$}

\def\toc#1#2{\gdef\ThisSection{#2}\iftoc{\let\break=\empty
\xdef\writeit{\write\tocfile{!#1#2 !!\noexpand\pagenumber}}\writeit}\fi}

%Was \PTOC
\def\ShowContents{\ifshowtoc\iftoc\immediate\closeout\tocfile\global\tocfalse\fi\def\chapline{Contents}%
\Head{Table of Contents}%\HeadII{ CONTENTS}
{\let\0\Pzero\let\1\Pone\let\2\Ptwo\let\3\Pthree\let\4\Pfour\let\5\Pfive\let\6\Psix
\let\par=\endgraf\parskip 0pt\parfillskip 0pt
\rightskip 5pc plus 15pc\hangindent1pc
\obeylines\catcode`\!=0\relax\input\jobname.toc\relax}
\vfil\break\ifodd\pageno\else\null\vfil\break\fi\fi
}

\def\ShowIndex#1{\iftoc\immediate\closeout\idxfile\fi\def\chapline{#1}\Head{#1}%\HeadII{ INDEX}
{\let\0\Pzero\let\1\Pone\let\2\Ptwo\let\3\Pthree\let\4\Pfour\let\5\Pfive\let\6\Psix
\let\par=\endgraf\parskip 0pt\parfillskip 0pt
\rightskip 5pc plus 15pc\hangindent1pc
\obeylines\catcode`\!=0\relax\input\jobname.idx\relax}
\vfil\break\ifodd\pageno\else\null\vfil\break\fi
}


% Headers

% note: Head Levels 0 and  1 should appear only at the top of a page.
% note: same for \altHeadII
% \endSection does the page breaking, not \beginSection

\def\afterheaderbreak{\penalty100000 }
\def\beforeheaderbreak#1{\par\vskip 0pt plus #1pt minus 8pt\penalty-1000 }

\def\Head #1{\toc0{#1}{\bfeighteen\beforeheaderbreak{18}
\baselineskip 20pt\leftskip  0pt plus 1fill \rightskip 0pt
\vglue -10pt\null #1\par}%
\afterheaderbreak\Vskip 5pc!\afterheaderbreak}

\def\HeadI #1{\toc1{#1}{\bfeighteen\beforeheaderbreak{18}
\baselineskip 20pt\leftskip  0pt plus 1fill \rightskip 0pt
\vglue -10pt\null #1\par}%
\afterheaderbreak\Vskip 5pc!\afterheaderbreak}

\def\HeadIL #1{\toc1{#1}{\bfeighteen\beforeheaderbreak{18}
\baselineskip 20pt\leftskip  0pt \rightskip 0pt
\vglue -10pt\null#1\hfil\par}%
\afterheaderbreak\Vskip 5pc!\afterheaderbreak}

% Used to skip 3pc before, 2pc afterward!
\def\HeadII #1{\Vskip 2.5pc!%
\toc2{\chapno.\the\ssecn #1}{\bfsixteen\beforeheaderbreak{16}
\baselineskip 18pt\leftskip 0pt \rightskip 0pt plus 1fil
\setbox0=\hbox{\chapno.\the\ssecn\ }\hangindent\wd0{$ $}\box0\ignorespaces #1\par}
%\chapno.\the\ssecn \relax #1\par}%
\afterheaderbreak\Vskip 1pc!\afterheaderbreak}

% Used to skip 2pc before!
\def\HeadIII #1{\Vskip 1.5pc!%
\toc3{\chapno.\the\ssecn.\the\sssecn #1}{\bffourteen\beforeheaderbreak{14}
\baselineskip 16pt\leftskip 0pt \rightskip 0pt plus 1fil
\setbox0=\hbox{\chapno.\the\ssecn.\the\sssecn\ }\hangindent\wd0{$ $}\box0\ignorespaces #1\par}
%\chapno.\the\ssecn.\the\sssecn \relax #1\par}%
\afterheaderbreak\Vskip 1pc!\afterheaderbreak}

% Used to skip 2pc before!
\def\HeadIV #1{\Vskip 1.5pc!%
\toc4{\chapno.\the\ssecn.\the\sssecn.\the\ssssecn #1}{\bftwelve\beforeheaderbreak{12}
\baselineskip 13pt \leftskip 0pt \rightskip 0pt plus 1fil
\setbox0=\hbox{\chapno.\the\ssecn.\the\sssecn.\the\ssssecn\ }\hangindent\wd0{$ $}\box0\ignorespaces #1\par}
%\chapno.\the\ssecn.\the\sssecn.\the\ssssecn \relax #1\par}%
\afterheaderbreak\Vskip 1pc!\afterheaderbreak}

% Used to skip 1.5pc before!
\def\HeadV #1{\Vskip 1.25pc!%
\toc5{\chapno.\the\ssecn.\the\sssecn.\the\ssssecn.\the\sssssecn #1}%
{\bfeleven\beforeheaderbreak{11}
\baselineskip 12pt \leftskip 0pt \rightskip 0pt plus 1fil
\setbox0=\hbox{\chapno.\the\ssecn.\the\sssecn.\the\ssssecn.\the\sssssecn\ }\hangindent\wd0{$ $}\box0\ignorespaces #1\par}
%\chapno.\the\ssecn.\the\sssecn.\the\ssssecn.\the\sssssecn \relax #1\par}%
\afterheaderbreak\Vskip 1pc!\afterheaderbreak}

% Used to skip 1.2pc before!
\def\HeadVI #1{\Vskip 1pc!%
\toc6{\chapno.\the\ssecn.\the\sssecn.\the\ssssecn.\the\sssssecn.\the\ssssssecn #1}%
{\bften\beforeheaderbreak{10}
\baselineskip 11pt \leftskip 0pt \rightskip 0pt plus 1fil
\setbox0=\hbox{\chapno.\the\ssecn.\the\sssecn.\the\ssssecn.\the\sssssecn.\the\ssssssecn\ }\hangindent\wd0{$ $}\box0\ignorespaces #1\par}
%\chapno.\the\ssecn.\the\sssecn.\the\ssssecn.\the\sssssecn.\the\ssssssecn \relax #1\par}%
\afterheaderbreak\Vskip 1pc!\afterheaderbreak}

%\def\HeadVI  #1{\Vskip1.5pc!{\elevenpoint\bf\beforeheaderbreak{11}
%\leftskip 4pc \rightskip 0pt plus 1fil
%\relax #1\par}%
%\afterheaderbreak\Vskip 1pc!\afterheaderbreak}

%% This is apparently not used. -kmp 9-Oct-91
% \def\altHeadII #1{\refalt\toc2{\chapno.\the\ssecn #1}{\bfsixteen\beforeheaderbreak{16}
% \baselineskip 18pt \leftskip \normalleftskip \rightskip 0pt plus 1fil
% \chapno.\the\ssecn \relax #1\par}%
% \afterheaderbreak\Vskip 4pc!\afterheaderbreak}
%
% Ditto for this. -kmp 24-Oct-91
% \let\refalt\empty
% 
% \def\refHeadIV #1{\Vskip 1pc!
% \toc4{\chapno.\the\ssecn.\the\sssecn.\the\ssssecn #1}{\bftwelve\beforeheaderbreak{12}
% \baselineskip 13pt \leftskip 0pt \rightskip 0pt plus 1fil
% \chapno.\the\ssecn.\the\sssecn.\the\ssssecn \relax #1\par}%
% \afterheaderbreak\Vskip \normalparskip!\afterheaderbreak}

% Lists

\def\bull
{\ifmmode\bullet\else{$\bullet$}\fi}

\def\listlabel
#1{\noindent\hbox to 0pc{\hskip -1.5pc #1\hss}{\penalty20000}\ignorespaces}

% \def\item		#1{\par\leftskip\iskip  \listlabel{#1}}
% \def\itemitem		#1{\par\leftskip\iiskip \listlabel{#1}}
% \def\itemitemitem	#1{\par\leftskip\iiiskip\listlabel{#1}}

% These used to do \bigbreak,\medbreak,\smallbreak instead of \itemskip.
\def\itemskip#1#2#3{\endgraf\penalty #1\vskip #2 plus #3 minus #3}

\def\item		#1{\itemskip{-200}{5pt}{3pt}\leftskip\iskip  \listlabel{#1}}
\def\itemitem		#1{\itemskip{-100}{4pt}{2pt}\leftskip\iiskip \listlabel{#1}}
\def\itemitemitem	#1{\itemskip{ -50}{3pt}{1pt}\leftskip\iiiskip\listlabel{#1}}

\def\beginlist
{\begingroup\iiiskip=\leftskip	\advance\iiiskip 1.5pc\iskip  =\iiiskip
\advance\iiiskip 1.5pc\iiskip =\iiiskip		\advance\iiiskip 1.5pc
\Vskip 1pc!}

\def\endlist
{\par\endgroup\Vskip 1pc!}

% screen text examples

{\obeylines\gdef\eatcr#1
{}}

\chardef\bslash=`\\
\def\prompt{>}

\chardef\other=12
{\obeyspaces\global\let =\ }

\def\catothers
{\catcode`\&=\other	\catcode`\#=\other
%\catcode`\^=\other	\catcode`\^^A=\other
%\catcode`\^^X=\other
\catcode`\%=\other}                
            
\def\screen!{\ifvmode\Vskip\normalparskip!\fi\begingroup
\baselineskip 11pt\tt
\parfillskip 0pt plus1fil
\parskip 0pt
\def\par{\leavevmode\endgraf}%
\def\ {{}}
\catcode`\$=\other
\let\>=\prompt
%!!! Experimentally making { and } have their normal TeX meaning here. -kmp 7-May-91
%\catcode `\{=\other	\catcode`\}=\other
%!!! Experimentally making [ and ] normal alphabetic. -kmp 7-May-91
%\catcode `\[=\other	\catcode`\]=\other
\obeyspaces\obeylines\eatcr}

\def\endscreen!{\endgraf\endgroup\Vskip\normalparskip!}

\def\widescreen{\Vskip \normalparskip!\begingroup
\leftskip 0pc
\baselineskip 11pt\tt
\parfillskip 0pt plus1fil
\parskip 0pt
\def\par{\leavevmode\endgraf}
\catcode`\$=\other
\let\>=\prompt
\catcode `\{=\other	\catcode`\}=\other
\obeyspaces\obeylines\eatcr}

% notes

\def\note{\Vskip 1pc!{\bf Note:} }

\def\longnote{\Vskip 1pc!\begingroup \advance\leftskip 1.5pc {\bf Note:} }

\def\notes{\Vskip 1pc!{\bf Notes:}\par\beginlist}

\let\endlongnote=\endlist \let\endnotes =\endlist

% Figures

%Syntax:
%       \boxfig (or \cboxfig) (or \rulefig)   (or \fig)
%       { }
%       \caption{ }
%       \endfig

\gdef\figtype{0}

\def\fig
{\gdef\figtype{0}
\begingroup\leftskip0pt
\global\setbox1=\vbox}

\def\finishfig
{\endgroup
\Vskip1pc!
\moveright\leftskip\box1
{\penalty20000}
\vskip 1pc
{\penalty20000}
\docaption
\smallbreak
\Vskip1pc!
}

\def\rulefig
{\gdef\figtype{1}
\begingroup\leftskip0pt
\global\setbox1=\vbox}

\def\finishrulefig
{\endgroup
\Vskip1pc!
\fullline{\hskip\leftskip\leaders\hrule height1pt depth0pt\hfil}
\Vskip1.5pc!
\moveright\leftskip\box1
{\penalty20000}
\Vskip1pc!
{\penalty20000}
\docaption
\smallbreak
\Vskip1pc!
\fullline{\hskip\leftskip\leaders\hrule height1pt depth0pt\hfil}
\Vskip1pc!}

\def\boxfig
{\gdef\figtype{2}
\begingroup
\advance\hsize by -\leftskip\advance\hsize by -\rightskip
\advance\hsize -2pt
\leftskip 1pc
\rightskip 1pc plus 2pc
\global\setbox1=\vbox}

\def\kcnocboxfig
{\gdef\figtype{2}
\begingroup
\advance\hsize by -2\leftskip
\advance\hsize -2pt
\leftskip 2pc
\rightskip 2pc plus 2pc
\global\setbox1=\vbox}

\def\kcfinishboxfig
{\wd1=\hsize
\endgroup
\Vskip1pc!
\vglue 0pt
\moveright\leftskip\vbox
\Vskip1pc!
\Vskip1pc!
}
\def\kcendfig
{\ifcase\figtype\relax\finishfig\or\finishrulefig\or\finishboxfig\fi}

\def\caption#1{\gdef\captext{\chapno--\the\capno. #1}}

\def\cboxfig
{\gdef\figtype{2}
\begingroup
\advance\hsize by -2\leftskip
\advance\hsize -2pt
\leftskip 2pc
\rightskip 2pc plus 2pc
\global\setbox1=\vbox}

\def\finishboxfig
{\wd1=\hsize
\endgroup
\Vskip1pc!
\vglue 0pt
\moveright\leftskip\vbox{\hrule height1pt
\hbox{\vrule width 1pt
  \vbox{\hrule height0pt width \wd1\vskip1pc\unvbox1\Vskip1pc!}\vrule width1pt}
\NIS\hrule height1pt}
{\penalty20000}
\Vskip1pc!
{\penalty20000}
\docaption
\smallbreak
\Vskip1pc!
}

\def\endfig
{\ifcase\figtype\relax\finishfig\or\finishrulefig\or\finishboxfig\fi}

\def\caption#1{\gdef\captext{\chapno--\the\capno. #1}}

\let\captext=\empty

\def\makecapline
{\vbox{\leftskip 0pt\noindent\prbnine Figure \captext}\figlist2\global\advance\capno 1\relax}

\def\docaption
{\ifx\captext\empty
\else\moveright\leftskip\makecapline\smallbreak\fi\global\let\captext\empty}

\def\figlist
#1{\xdef\writeit{\write\figfile{!#1\captext!!\noexpand\pagenumber}}\writeit}

%\figlist4

% \simplecaption lets you put captions on things that are not 
% set up formally as figures.

\def\simplecaption#1{\caption{#1}
\vskip 1pc
\docaption
\Vskip1pc!}

\def\startSection{\advance\ssecn 1\sssecn=0\ssssecn=0\sssssecn=0\ssssssecn=0}
\def\beginSection #1{\startSection\leftskip\normalleftskip\HeadII{\ #1}}
%Dictionary sections don't start with a banner line announcing their section name.
\def\includeDictionary #1{\startSection\toc2{\chapno.\the\ssecn\ {\chapline} Dictionary}
\input #1
\endSection}
    
\def\startSubsection{\advance\sssecn 1\ssssecn=0\sssssecn=0\ssssssecn=0}
\def\beginSubsection #1{\startSubsection\HeadIII{\ #1}}

\def\startsubsubsection{\advance\ssssecn 1\sssssecn=0\ssssssecn=0}
\def\beginsubsubsection #1{\startsubsubsection\HeadIV{\ #1}}

\def\startsubsubsubsection{\advance\sssssecn 1\ssssssecn=0}
\def\beginsubsubsubsection #1{\startsubsubsubsection\HeadV{\ #1}}

\def\startsubsubsubsubsection{\advance\ssssssecn 1}
\def\beginsubsubsubsubsection #1{\advance\ssssssecn 1\HeadVI{\ #1}}

\let\beginsection\beginSection
\let\endsection\endSection
\let\endSubsection\empty
\let\endsubsubsection\empty
\let\endsubsubsubsection\empty
\let\endsubsubsubsubsection\empty
\let\beginSubSection=\beginSubsection
\let\beginsubSection=\beginSubsection
\let\beginsubsection=\beginSubsection
\let\endSubSection=\endSubsection
\let\endsubSection=\endSubsection
\let\endsubsection=\endSubsection
\def\endchapter{\endSimpleChapter
\normaltype
\pageno=-2
\ShowContents}%was \PTOC
\def\beginSimpleChapter#1{\HeadI{#1}
\pageno=-1
\def\chapline{#1}}
\def\beginSimpleChapterLeft#1{\HeadIL{#1}
\pageno=-1
\def\chapline{#1}}
\def\endSimpleChapter
{\endSection\ifodd\pageno\else\global\setbox\comline\null\null\vfil\break\fi}
\let\endAppendix\endchapter

%Switch these to get Sections on new pages
\def\endSection{\ifdim \pagetotal>0pc \vfil\break\fi}
%\let\endSection\empty
\let\endsection=\endSection

%\def\endcom{\global\incomfalse\Vskip 1pc!\hrule\hrule\hrule\hrule}
\def\endcom{\global\incomfalse\penalty20000\VPskip 1pc plus 3pc!\penalty20000\hrule\hrule\hrule\hrule\goodbreak}

% \def\label
% #1:{\bigbreak
% \noindent\hbox to 0pc{\bf\hskip-\comleftskip #1:\hss}
% \penalty20000
% \vskip 2pt
% \penalty20000
% \ignorespaces}
% \def\methodlabel
% #1:{\bigbreak
% \noindent\hbox to 0pc{\bf\hskip-\hcomleftskip #1:\hss}
% \penalty20000
% \vskip 2pt
% \penalty20000
% \ignorespaces}
% \def\methodrule{\bigbreak
% \Vskip 1pc!
% \penalty20000
% \line{\hbox to \hcomleftskip{}\leaders\hrule\hfill\hbox to \hcomleftskip{}}
% \penalty20000
% \Vskip 1pc!
% \penalty20000{}}
    
%!!! Sandra complained that this sometimes lets names get hyphenated even
%    in a code-font context. e.g., see the entry for SUBSTITUTE, which gets typeset as:
%
%  substitute, substitute-if, substitute-if-not, nsubsti-
%  tute, nsubstitute-if, nsubstitute-if-not	 Function
    
\def\begincom
{\begingroup\catcode`\,=\active\catcode`\-=\active\dobegincom}

\def\obegincom
{\begingroup\catcode`\,=\active\catcode`\-=\active\doobegincom}

\newdimen \dotw
\setbox0=\hbox{\prbfourteen, $\ldots$}
\global\dotw=\wd0

\def\hyphen{-}
\def\comma{,}
\def\specialcomma
{\discretionary{\kern\dotw\vrule width0pt}{}{,\kern\fontdimen2
\prbfourteen}\ignorespaces}

{\catcode`\,\active\catcode`\-=\active
\gdef\dospecialcomma{\let,\specialcomma\sphytrue}
\gdef\donormalcomma{\let,\comma}
\gdef\begincomindex#1{\global\let\NEXT\bcindex\bcindex #1, \endit, \relax}
\gdef\bcindex#1, {\ifx\endit#1\global\let\NEXT\empty
\else\silenttrue\boldtrue\xref{#1}\fi\NEXT}}

\donormalcomma

\def\dobegincom
#1\ftype #2{\leftskip0pc\rightskip0pc plus10pc\bfsixteen
\baselineskip 16pt
\dospecialcomma\let\break=\ignorespaces
\global\setbox1=\vbox{\hsize 40pc\lowercase{#1}}
\vbadness 10000
\setbox0=\vsplit1 to 16pt
\setbox0=\vbox{\unvbox0\global\setbox3\lastbox}
\global\setbox1\hbox{\unhbox3\relax\ifdim\ht1>0pt\llap{, $\ldots$}\fi}
\global\incomtrue
\global\setbox\comline\box1
\endgroup
\comfirsttrue
\leftskip \comleftskip
\Vskip 1pc!
\goodbreak
\idxref{#1}%\toc2{#1} 
%!!! ACW is bugged that sometimes a carriage return is generated in the
%    middle of arg2 (the \ftype).  Maybe change "\pritwelve #2" to be in "\hbox{...}"
{\raggedright\catcode`\-=\active\bfsixteen #1 
\hfill\pritwelve #2}
\penalty20000
\Vskip 1pc!			
\penalty20000
\hrule height \combarht
\penalty20000
\parskip \normalparskip
\penalty20000
\vbox to 0pc{}
}
\def\doobegincom
#1{\leftskip0pc\rightskip0pc plus10pc\bffourteen
\baselineskip 16pt
\dospecialcomma\let\break=\ignorespaces
\global\setbox1=\vbox{\hsize 40pc\lowercase{#1}}
\vbadness 10000
\setbox0=\vsplit1 to 16pt
\setbox0=\vbox{\unvbox0\global\setbox3\lastbox}
\global\setbox1\hbox{\unhbox3\relax\ifdim\ht1>0pt\llap{, $\ldots$}\fi}
\global\incomtrue
\global\setbox\comline\box1
\endgroup
\comfirsttrue
\goodbreak
\leftskip \comleftskip
\penalty20000
\Vskip 3pc!
\penalty20000
\hbox to \fullhsize{\idxref{#1}%\toc2{#1}
\bfsixteen #1\hfil}
\penalty20000
\Vskip 2pc!			
\penalty20000
\hrule height \combarht
\penalty20000
\parskip \normalparskip
\penalty20000
\vbox to 0pc{}
\penalty20000
}


\normaltype

\newdimen \changedepth
\changedepth=0.15\baselineskip

% character hacks

\let\barunderaccent=\b  %So that \b is free for other things (e.g., bold)
\let\dotlessi=\i        %So that \i is free for other things (e.g., italic)

\mathchardef \spLT "373C % < 
\mathchardef \spGT "373E % >
\mathchardef \spST "2803 % *
\mathchardef \spBS "086E % \
\mathchardef \spMI "2800 % -
\mathchardef \spVB "386A % |
\mathchardef \spTI "3818 % ~
\def\LT{\ifmmode\spLT\else\iftt<\else{$\spLT$}\fi\fi}
\def\GT{\ifmmode\spGT\else\iftt>\else{$\spGT$}\fi\fi}
\def\ST{\ifmmode\spST\else\ifbf{$\spST$}\else*\fi\fi}
\def\BSlash{\ifmmode\spBS\else\iftt\bslash\else{$\spBS$}\fi\fi}
%\def\VB{\ifmmode\spVB\else\iftt|\else{$\spVB$}\fi\fi}
\def\VB{\ifmmode\spVB\else\iftt|\else{$\vert$}\fi\fi}
\def\US{\iftt\_\fi}         
\def\MI{\ifbf\ifmmode\spMI\else\ifsphy\hbox{-}\else-\fi\fi\else-\fi}
% !!! Disabled again. -kmp 7-May-91
% %!!! Experimentally making @ normal alphabetic. -kmp 7-May-91 (was commented out)
% \def\AT{\iftt\hbox to.5em{\hss\prmseven@\hskip.5pt\hss}\else@\fi}
\def\TI{\iftt\char'176\relax\else\penalty10000\ \fi}
\let\LB=\{
\let\RB=\}
\def\{{\iftt\char'173\relax\else\LB\fi}
\def\}{\iftt\char'175\relax\else\RB\fi}
\def\ngt{>}
\def\nlt{<}
\def\nst{*}

\let\\=\BSlash

\catcode `\_=9\relax
\def\tilde{\ifbf$\spTI$\else{\tt\char126\relax}\fi}


\catcode `\<=\active	\global\let<=\LT
\catcode `\>=\active	\global\let>=\GT
\catcode `\|=\active	\global\let|=\VB
\catcode `\*=\active	\global\let*=\ST
% !!! Disabled again. -kmp 7-May-91
% %!!! Experimentally making @ normal alphabetic. -kmp 7-May-91 (was commented out)
% \catcode `\@=\active	\global\let@=\AT
\catcode `\_=\active 	\global\let_=\US
\catcode `\-=\active	\global\let-=\MI
			\global\let~=\TI

\gdef\setspecialdefs
{\let-=\hyphen	\let>=\ngt	\let<=\nlt	\let*=\nst
\def\bf{\string\bf}\def\\{\string\\}\def\it{\string\it}\def\tt{\string\tt}}

%\catcode `\^ = \active
           
\def\uspace{{\tentt\char'40}}

\normalleftskip=  4pc
\comleftskip   =  4pc
\hcomleftskip  =  2pc
\let\normaltype=\tenpoint
\normaltype

%!!! The idea here is to let the user write either:
%
%    \label Foo Bar::
%
%    This is text for the "Foo Bar" section.
%
% or else
%
%    \label Foo Bar:\None.
% or \label Foo Bar:\None!
% or \label Foo Bar:\None?
%
% Currently the punctuation is not managed formally, but the intent is to provide
% a level of confidence that the absence of information is purposeful.
%   "?" means unsure or questionable.
%   "." is neutral--no information content.
%   "!" means pretty sure this is right.
% 
% In the latter case, no text will appear.

\newif \ifnullabel  \nullabelfalse % might be overridden in setup-options

\def\None{None.}

\def\EatPunc #1{} % Takes care of eating a trailing "." or "!"

\def\label #1:#2{\ifx#2:\truelabel{#1}\else\labelNone{#1}\fi}

\def\labelNone#1{\ifnullabel\nullabel{#1}\fi\expandafter\EatPunc}

\newif \ifsmallab \smallabfalse

\long\def\ignorepar #1{#1}

\def\truelabel#1{\rm\Vskip \normalparskip!
\bigbreak
\ifsmallab
\noindent\hskip -4pc \hbox to 4pc{{\prbseven #1 }\hss}\expandafter\ignorepar
\else
\hbox{\prbtwelve #1:}
\penalty20000
\vskip 2pt plus 2pt
\penalty20000
\fi}

\def\nullabel#1{\rm\Vskip \normalparskip!
\bigbreak
\hbox{{\tenpoint\bf (#1: \rm None.)}}
\bigbreak
}

\newif \ifexamples  \examplestrue  % might be overridden in setup-options
\newif \ifnotes     \notestrue     % might be overridden in setup-options

% This implementation had some problems, but I'd like to do something like this. -kmp  
%
% \def\Examples{\ifexamples\label Examples::\else\expandafter\eatExamples\fi}
% \def\Notes{\ifnotes\label Notes::\else\expandafter\eatNotes\fi}
% 
% \long\def\eatExamples #1\label{\label}
% \long\def\eatNotes #1\endcom{\endcom}

%% I'm having troubles with \HeadIV.  Could this have been the problem?
%% Removed experimentally. -kmp 9-Oct-91
% \let\HeadIV=\refHeadIV
% \def\refalt{\vglue -8pt\null}

%\hoffset 1in

\advance\voffset by .5in

\newif \ifdraft  % put \drafttrue at beginning of a draft

%!!! Moon notes his concern about the introduction of extra whitespace
%    in the vicinity of issue markers.
%    I experimentally added "\ignorespaces" here. Did that fix it? -kmp 18-Dec-91
\newif \ifissdisp
\newif \ifisslog
\def\issue #1{\ifisslog\logissue1{#1}\fi
\ifissdisp\par\leftskip\iskip {\bf The following is from issue: } #1\par\fi
\ignorespaces}
\def\endissue #1{\ifisslog\logissue0{#1}\fi
\ifissdisp\par\leftskip\iskip {\bf End of issue: } #1\par\fi
\ignorespaces}
% \def\issue #1{\logissue1{#1}}
% \def\endissue #1{\logissue0{#1}}
% % \def\issue #1{\par\leftskip\iskip {\bf The following is from issue: } #1\par}
% % \def\endissue #1{\par\leftskip\iskip {\bf End of issue: } #1\par}%\indent

% Use \logissue1 to start an issue, or \logissue0 to end one.
% Puts in the file:
%  !1<issue> !!<page> for start of issue
%  !0<issue> !!<page> for end   of issue
\def\logissue
#1#2{{\let\break=\empty
\xdef\writeit{\write\issfile{!#1#2 !!\noexpand\pagenumber}}\writeit}}

\def\logidx#1#2{{\let\break=\empty
\xdef\writeit{\write\idxfile{!2#1:#2!!\noexpand\pagenumber}}\writeit}}

\def\DefineFigure#1{{\let\break=\empty
\xdef\writeit{\write\reffile{!#1 \chapno--\the\capno}}\writeit}}

\def\deffigrefs#1{\def\figref##1{{\def##1{\message{Figure ``\string##1'' is not defined.}%
Figure $nn$--$mm$ (\string##1)}#1##1}}}
% Stylistically, ``\Figref'' goes at start of sentence or ``\figref'' in middle,
% but since "Figure xxx" always comes out capitalized, they're implementationally
% just synonyms.
\def\Figref#1{\figref#1}

\def\DefineSection#1{{\let\break=\empty
\xdef\writeit{\write\secfile{!#1 Section \ThisSection}}\writeit}}

%Maybe replace with more complicated self-reference one day.
\def\Thissection{This section}
\def\thissection{this section}

\def\DefineChapter#1#2#3#4{{\let\break=\empty
\xdef\writeit{\write\secfile{!#1 Chapter #3 #4}}\writeit
\xdef\writeit{\write\secfile{!#2 Chapter #3 #4}}\writeit}}

\def\defsecrefs#1{\def\secref##1{{\def##1{\message{Section ``\string##1'' is not defined.}%
Section $mm$.$nn$ (\string##1)}#1##1}}\let\chapref=\secref}

\tolerance=2500
