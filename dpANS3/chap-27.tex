\input setup		      % -*- Mode: TeX -*-

\beginchapter{27}{Concurrency}{ChapTwentySeven}{Concurrency}

Concurrency constructs leave the order of some \term{evaluations}
unspecified, allowing the implementation to order them as they see
fit. In some situations it may be possible to perform multiple
\term{evaluations} simultaneously. This can greatly improve the
performance and capabilities of programs, but complicates their
semantics.

Concurrency in Lisp is expressed via \term{threads}. Each
\term{thread} proceeds as though it was performing an ordered sequence
of \term{evaluations} according to the usual rules specified in
\secref\EvaluationModel. The order \term{evaluations} in different
\term{threads} take place in, however, is unspecified, unless
controlled by what are called synchronization operations, also
described in this chapter.

It is important to note that, without the presence of synchronization
operations, the implementation is only constrained as to the
\term{evaluation} order within a \term{thread}. This means that
without synchronization, \term{conforming code} cannot rely on
\term{side effects} in another \term{thread} happening in any order,
including the order that seems clear from the program text.

Concurrency is explained more formally in the remainder of this
section.

\beginSection{Concurrency}
\input concept-concurrency
\endSection%{Concurrency}

\beginSection{Atomics}
\input concept-atomics
\endSection%{Atomics}

\beginSection{Interruption}
\input concept-interruption
\endSection%{Interruption}

\includeDictionary{dict-concurrency}

\endchapter

\bye
