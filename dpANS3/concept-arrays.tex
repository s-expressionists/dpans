% -*- Mode: TeX -*-

\beginsubsection{Array Elements}
\DefineSection{ArrayElements}

An \term{array} contains a set of \term{objects} called \term{elements}
that can be referenced individually according to a rectilinear coordinate system.

\beginsubsubsection{Array Indices}

%% 2.5.0 5
An \term{array} \term{element} is referred to by a (possibly empty) series of indices.
The length of the series must equal the \term{rank} of the \term{array}.
\issue{ARRAY-DIMENSION-LIMIT-IMPLICATIONS:ALL-FIXNUM}
Each index must be a non-negative \term{fixnum} 
\endissue{ARRAY-DIMENSION-LIMIT-IMPLICATIONS:ALL-FIXNUM}
%strictly
less than the corresponding \term{array} \term{dimension}.
\term{Array} indexing is zero-origin.

\endsubsubsection%{Array Indices}

\beginsubsubsection{Array Dimensions}

An axis of an \term{array} is called a \newterm{dimension}.

Each \term{dimension} is a non-negative 
\issue{ARRAY-DIMENSION-LIMIT-IMPLICATIONS:ALL-FIXNUM}
\term{fixnum};
\endissue{ARRAY-DIMENSION-LIMIT-IMPLICATIONS:ALL-FIXNUM}
 if any dimension of an \term{array} is zero, the \term{array} has no elements.
% Maybe this part isn't in glossary...I just moved it from somewhere else per 
% suggestion of Barmar.  -kmp 14-Jan-92
It is permissible for a \term{dimension} to be zero, 
in which case the \term{array} has no elements, 
and any attempt to \term{access} an \term{element}
is an error.  However, other properties of the \term{array},  
such as the \term{dimensions} themselves, may be used.

\beginsubsubsubsection{Implementation Limits on Individual Array Dimensions}

An \term{implementation} may impose a limit on \term{dimensions} of an \term{array},
but there is a minimum requirement on that limit.  \Seevar{array-dimension-limit}.

\endsubsubsubsection%{Implementation Limits on Individual Array Dimensions}

\endsubsubsection%{Array Dimensions}

\beginsubsubsection{Array Rank}

%% 2.5.0 3
%% 2.5.0 4

An \term{array} can have any number of \term{dimensions} (including zero).
The number of \term{dimensions} is called the \newterm{rank}.

If the rank of an \term{array} is zero then the \term{array} is said to have
no \term{dimensions}, and the product of the dimensions (see \funref{array-total-size})
is then 1; a zero-rank \term{array} therefore has a single element.

\beginsubsubsubsection{Vectors}

An \term{array} of \term{rank} one (\ie a one-dimensional \term{array})
is called a \newterm{vector}.

\beginsubsubsubsubsection{Fill Pointers}

A \newterm{fill pointer} is a non-negative \term{integer} no
larger than the total number of \term{elements} in a \term{vector}.
Not all \term{vectors} have \term{fill pointers}.
\Seefuns{make-array} and \funref{adjust-array}.

An \term{element} of a \term{vector} is said to be \newterm{active} if it has
an index that is greater than or equal to zero, 
but less than the \term{fill pointer} (if any).
For an \term{array} that has no \term{fill pointer},
all \term{elements} are considered \term{active}.

%% 17.5.0 4
Only \term{vectors} may have \term{fill pointers}; 
multidimensional \term{arrays} may not.
A multidimensional \term{array} that is displaced to a \term{vector} 
that has a \term{fill pointer} can be created.

\endsubsubsubsubsection%{Fill Pointers}

\endsubsubsubsection%{Vectors}

\beginsubsubsubsection{Multidimensional Arrays}

\beginsubsubsubsubsection{Storage Layout for Multidimensional Arrays}

%% 2.5.0 8
Multidimensional \term{arrays} store their components in row-major order;
that is, internally a multidimensional \term{array} is stored as a
one-dimensional \term{array}, with the multidimensional index sets
ordered lexicographically, last index varying fastest.  
 
\endsubsubsubsubsection%{Storage Layout for Multidimensional Arrays}

\beginsubsubsubsubsection{Implementation Limits on Array Rank}

An \term{implementation} may impose a limit on the \term{rank} of an \term{array},
but there is a minimum requirement on that limit.  See the
\term{constant variable} \conref{array-rank-limit}.

\endsubsubsubsubsection%{Implementation Limits on Array Rank}

\endsubsubsubsection%{Multidimensional Arrays}

\endsubsubsection%{Array Rank}

\endsubsection%{Array Elements}

\beginsubsection{Specialized Arrays}

%% 17.0.0 4
An \term{array} can be a \term{general} \term{array}, 
    meaning each \term{element} may be any \term{object},
or it may be a \term{specialized} \term{array},
    meaning that each \term{element} must be of a restricted \term{type}.

The phrasing ``an \term{array} \term{specialized} to \term{type} \metavar{type}''
is sometimes used to emphasize the \term{element type} of an \term{array}.
This phrasing is tolerated even when the \metavar{type} is \typeref{t},
even though an \term{array} \term{specialized} to \term{type} \term{t}
is a \term{general} \term{array}, not a \term{specialized} \term{array}.

\Thenextfigure\ lists some \term{defined names} that are applicable to \term{array} 
creation, \term{access}, and information operations.

%% Added ARRAY-DISPLACEMENT per Tom Shepard.  (X3J13 approved: May 4-5, 1994)
%% -kmp 9-May-94
\displaythree{General Purpose Array-Related Defined Names}{
adjust-array&array-has-fill-pointer-p&make-array\cr
adjustable-array-p&array-in-bounds-p&svref\cr
aref&array-rank&upgraded-array-element-type\cr
array-dimension&array-rank-limit&upgraded-complex-part-type\cr
array-dimension-limit&array-row-major-index&vector\cr
array-dimensions&array-total-size&vector-pop\cr
array-displacement&array-total-size-limit&vector-push\cr
array-element-type&fill-pointer&vector-push-extend\cr
}

\beginsubsubsection{Array Upgrading}
\DefineSection{ArrayUpgrading}

\issue{ARRAY-TYPE-ELEMENT-TYPE-SEMANTICS:UNIFY-UPGRADING}

% Some of the following was transplanted from the description 
% of UPGRADED-ARRAY-ELEMENT-TYPE.  Consider also stealing from
% SUBTYPEP and TYPEP.

The \newterm{upgraded array element type} of a \term{type} $T\sub 1$
is a \term{type} $T\sub 2$ that is a \term{supertype} of $T\sub 1$
and that is used instead of $T\sub 1$ whenever $T\sub 1$
is used as an \term{array element type} 
for object creation or type discrimination.

During creation of an \term{array},
the \term{element type} that was requested 
is called the \newterm{expressed array element type}.
The \term{upgraded array element type} of the \term{expressed array element type}
becomes the \newterm{actual array element type} of the \term{array} that is created.

%!!! Barmar thinks this should be removed.
\term{Type} \term{upgrading} implies a movement upwards in the type hierarchy lattice.
A \term{type} is always a \term{subtype} of its \term{upgraded array element type}.
Also, if a \term{type} $T\sub x$ is a \term{subtype} of another \term{type} $T\sub y$,
then
the \term{upgraded array element type} of $T\sub x$ 
must be a \term{subtype} of
the \term{upgraded array element type} of $T\sub y$.
Two \term{disjoint} \term{types} can be \term{upgraded} to the same \term{type}.

The \term{upgraded array element type} $T\sub 2$ of a \term{type} $T\sub 1$
is a function only of $T\sub 1$ itself;
that is, it is independent of any other property of the \term{array} 
for which $T\sub 2$ will be used,
such as \term{rank}, \term{adjustability}, \term{fill pointers}, or displacement.
%% This next sentence is interesting, but is just Rationale, so is omitted.
% The reason \term{rank} is included is because it would not
% be consistently possible to displace \term{arrays} to those of differing
% \term{rank} otherwise.
\Thefunction{upgraded-array-element-type} 
can be used by \term{conforming programs} to predict how the \term{implementation}
will \term{upgrade} a given \term{type}.

\endissue{ARRAY-TYPE-ELEMENT-TYPE-SEMANTICS:UNIFY-UPGRADING}

\endsubsubsection%{Array Upgrading}

\beginsubsubsection{Required Kinds of Specialized Arrays}
\DefineSection{RequiredSpecializedArrays}

%% 17.0.0 5
\term{Vectors} whose \term{elements} are restricted to \term{type}
\issue{CHARACTER-PROPOSAL:2-3-2}
\typeref{character} or a \term{subtype} of \typeref{character}
\endissue{CHARACTER-PROPOSAL:2-3-2}
are called \newtermidx{strings}{string}. 
\term{Strings} are \oftype{string}.
%% 18.0.0 7
%% 18.0.0 4
\Thenextfigure\ lists some \term{defined names} related to \term{strings}.

\term{Strings} are \term{specialized} \term{arrays} 
and might logically have been included in this chapter.
However, for purposes of readability
most information about \term{strings} does not appear in this chapter;
see instead \chapref\Strings.

%% 18.0.0 5
%% paragraph duplicated in descriptions of string-equal and string=
%% 18.0.0 6
%% paragraph duplicated in description of stringp

\displaythree{Operators that Manipulate Strings}{
char&string-equal&string-upcase\cr
make-string&string-greaterp&string{\tt /=}\cr
nstring-capitalize&string-left-trim&string{\tt <}\cr
nstring-downcase&string-lessp&string{\tt <=}\cr
nstring-upcase&string-not-equal&string{\tt =}\cr
schar&string-not-greaterp&string{\tt >}\cr
string&string-not-lessp&string{\tt >=}\cr
string-capitalize&string-right-trim&\cr
string-downcase&string-trim&\cr
}

\term{Vectors} whose \term{elements} are restricted to \term{type}
\typeref{bit} are called \newtermidx{bit vectors}{bit vector}.
\term{Bit vectors} are \oftype{bit-vector}.
\Thenextfigure\ lists some \term{defined names} for operations on \term{bit arrays}.

\displaythree{Operators that Manipulate Bit Arrays}{
bit&bit-ior&bit-orc2\cr
bit-and&bit-nand&bit-xor\cr
bit-andc1&bit-nor&sbit\cr
bit-andc2&bit-not&\cr
bit-eqv&bit-orc1&\cr
}

\endsubsubsection%{Required Kinds of Specialized Arrays}

\endsubsection%{Specialized Arrays}
