% -*- Mode: TeX -*-

\def\GFauxOptionsAndMethDesc{
\auxbnf{option}{\paren{\kwd{argument-precedence-order} \plusparam{parameter-name}} | \CR
		\paren{\misc{declare} \plusparam{declaration}}                     | \CR
		\paren{\kwd{documentation} \i{string}}                             | \CR
	        \paren{\kwd{method-combination} \i{symbol} \starparam{arg}}	   | \CR
		\paren{\kwd{generic-function-class} \i{class-name}}		   | \CR
		\paren{\kwd{method-class} \i{class-name}}}
\auxbnf{method-description}{\lparen\kwd{method} \CR
                            \ \starparam{method-qualifier} specialized-lambda-list \CR
                            \ {\DeclsAndDoc} \CR
                            \ \starparam{form}\rparen}
}

%%% ========== FUNCTION-KEYWORDS
\begincom{function-keywords}\ftype{Standard Generic Function}
 
%Barmar: This function should have been called ``method-keywords.''

\label Syntax::
 
\DefgenWithValues {function-keywords} {method} {keys, allow-other-keys-p}

\label Method Signatures::
 
\Defmeth {function-keywords} {\specparam{method}{standard-method}}
 
\label Arguments and Values::
 
\param{method}---a \term{method}.
 
\param{keys}---a \term{list}.

\param{allow-other-keys-p}---a \term{generalized boolean}.

\label Description::
 
Returns the keyword parameter specifiers for a \param{method}.

Two values are returned: 
 a \term{list} of the explicitly named keywords 
 and a \term{generalized boolean} that states whether \keyref{allow-other-keys}
  had been specified in the \param{method} definition.

\label Examples::

\code
 (defmethod gf1 ((a integer) &optional (b 2)
                 &key (c 3) ((:dee d) 4) e ((eff f)))
   (list a b c d e f))
\EV #<STANDARD-METHOD GF1 (INTEGER) 36324653>
 (find-method #'gf1 '() (list (find-class 'integer))) 
\EV #<STANDARD-METHOD GF1 (INTEGER) 36324653>
 (function-keywords *)
\EV (:C :DEE :E EFF), \term{false}
 (defmethod gf2 ((a integer))
   (list a b c d e f))
\EV #<STANDARD-METHOD GF2 (INTEGER) 42701775>
 (function-keywords (find-method #'gf1 '() (list (find-class 'integer))))
\EV (), \term{false}
 (defmethod gf3 ((a integer) &key b c d &allow-other-keys)
   (list a b c d e f))
 (function-keywords *)
\EV (:B :C :D), \term{true}
\endcode
 
\label Affected By::

\macref{defmethod}
 
\label Exceptional Situations:\None.
 
\label See Also::
 
\macref{defmethod}

\label Notes:\None.
 
\endcom

%%% ========== ENSURE-GENERIC-FUNCTION
\begincom{ensure-generic-function}\ftype{Function}
 
\label Syntax::
 
\DefunWithValuesNewline ensure-generic-function
			{function-name {\key}
			 \vtop{\hbox{argument-precedence-order declare}
                               \hbox{documentation environment}
                               \hbox{generic-function-class lambda-list}
                               \hbox{method-class method-combination}}}
			{generic-function}

\label Arguments and Values::
 
\param{function-name}---a \term{function name}.
 
The keyword arguments correspond to the \param{option} arguments of
\macref{defgeneric}, except that the \kwd{method-class} and
\kwd{generic-function-class} arguments can be \term{class} \term{object}s
as well as names.
 
%!!! What's a method combination object??
{\keyword Method-combination} -- method combination object.
 
{\keyword Environment} -- the same as the \keyref{environment} argument
to macro expansion functions and is used to distinguish between compile-time
and run-time environments.
%Barmar said (and I agree) that this just doesn't belong here.
% \issue{MACRO-ENVIRONMENT-EXTENT:DYNAMIC}
% The \keyref{environment} argument has 
% \term{dynamic extent}; the consequences are undefined if 
% the \keyref{environment} argument is 
% referred to outside the \term{dynamic extent} 
% of the macro expansion function.
% \endissue{MACRO-ENVIRONMENT-EXTENT:DYNAMIC}
 
\editornote{KMP: What about documentation. Missing from this arguments enumeration,
  and confusing in description below.}%!!!

\param{generic-function}---a \term{generic function} \term{object}.
 
\label Description::
 
\Thefunction{ensure-generic-function} is used to define 
a globally named \term{generic function} with no \term{methods} 
or to specify or modify options and declarations that pertain to 
a globally named \term{generic function} as a whole.
 
%!!! This used to refer to FBOUNDP but I changed it to use "fbound".
%    The question is, why is it looking in the global env if an environment
%    argument was passed.
If \param{function-name} is not \term{fbound} in the \term{global environment},
a new
\term{generic function} is created.  
%!!! Rewrite in terms of function cell contents?
If 
\issue{FUNCTION-NAME:LARGE}
\f{(fdefinition \param{function-name})} 
\endissue{FUNCTION-NAME:LARGE}
is an \term{ordinary function}, 
a \term{macro}, 
or a \term{special operator},
an error is signaled.
 
If \param{function-name} 
is a \term{list}, it must be of the
form \f{(setf \param{symbol})}.
If \param{function-name} specifies a \term{generic function} that has a
different value for any of the following arguments,
the \term{generic function} is modified to have the new value: 
\kwd{argument-precedence-order}, \kwd{declare}, \kwd{documentation},
\kwd{method-combination}.
 
If \param{function-name} specifies a \term{generic function} that has a
different value for the \kwd{lambda-list} argument, and the new value
is congruent with the \term{lambda lists} of all existing 
\term{methods} or there
are no \term{methods}, the value is changed; otherwise an error is signaled.
 
%!!! Barmar: What does this part about
%     "new generic function class is compatible with the old" mean?
If \param{function-name} specifies a \term{generic function} that has a
different value for the \kwd{generic-function-class} argument and if
the new generic function class is compatible with the old,
\funref{change-class} is called to change the \term{class} of the 
\term{generic function};
otherwise an error is signaled.
 
If \param{function-name} specifies a \term{generic function} that has a
different value for the \kwd{method-class} argument, the value is
changed, but any existing \term{methods} are not changed.
 
\label Examples:\None.

\label Affected By::

Existing function binding of \param{function-name}.
 
\label Exceptional Situations::
 
If 
\issue{FUNCTION-NAME:LARGE}
\f{(fdefinition \param{function-name})}
\endissue{FUNCTION-NAME:LARGE}
is an \term{ordinary function}, a \term{macro}, or a \term{special operator}, 
an error \oftype{error} is signaled.
 
If \param{function-name} specifies a 
\term{generic function} that has a
different value for the \kwd{lambda-list} argument, and the new value
is not congruent with the \term{lambda list} of any existing 
\term{method},
an error \oftype{error} is signaled.
 
If \param{function-name} specifies a 
\term{generic function} that has a
different value for the \kwd{generic-function-class} argument and if
the new generic function class not is compatible with the old,
an error \oftype{error} is signaled.
 
 
\label See Also::
 
\macref{defgeneric}
 
\label Notes:\None.
 
\endcom

%%% ========== ALLOCATE-INSTANCE
\begincom{allocate-instance}\ftype{Standard Generic Function}
 
\issue{ALLOCATE-INSTANCE:ADD}

\label Syntax::

\issue{INITIALIZATION-FUNCTION-KEYWORD-CHECKING}
\DefgenWithValues allocate-instance 
		  {class {\rest} initargs {\key} {\allowotherkeys}}
	          {new-instance}
\endissue{INITIALIZATION-FUNCTION-KEYWORD-CHECKING}

\label Method Signatures::
 
\Defmeth {allocate-instance} {\specparam{class}{standard-class} {\rest} initargs}

\Defmeth {allocate-instance} {\specparam{class}{structure-class} {\rest} initargs}

\label Arguments and Values::
 
\param{class}---a \term{class}.
 
\param{initargs}---a \term{list} of \term{keyword/value pairs} 
		   (initialization argument \term{names} and \term{values}).
 
\param{new-instance}---an \term{object} whose \term{class} is \param{class}.

\label Description::
 
The generic function \funref{allocate-instance} creates and returns
a new instance of the \param{class}, without initializing it.
When the \param{class} is a \term{standard class}, this means that
the \term{slots} are \term{unbound}; when the \term{class} is a
\term{structure class}, this means the \term{slots}' \term{values}
are unspecified.
 
The caller of \funref{allocate-instance} is expected to have
already checked the initialization arguments.
 
The \term{generic function} \funref{allocate-instance} is called by
\funref{make-instance}, as described in
\secref\ObjectCreationAndInit.

\label Affected By:\None. % ????

\label Exceptional Situations:\None. % ????

\label See Also::
 
\macref{defclass}, \funref{make-instance}, \funref{class-of},
{\secref\ObjectCreationAndInit}
 
\label Notes::

The consequences of adding \term{methods} to \funref{allocate-instance} is unspecified.
This capability might be added by the \term{Metaobject Protocol}.

% kmp: This section was copied with minor changes from
%  \funref{make-instance}, and then reviewed by Moon.

\endissue{ALLOCATE-INSTANCE:ADD}

\endcom

%%% ========== REINITIALIZE-INSTANCE
\begincom{reinitialize-instance}\ftype{Standard Generic Function}
 
\label Syntax::
 
\issue{INITIALIZATION-FUNCTION-KEYWORD-CHECKING}
\DefgenWithValues reinitialize-instance
		  {instance {\rest} initargs {\key} {\allowotherkeys}}
		  {instance}
\endissue{INITIALIZATION-FUNCTION-KEYWORD-CHECKING}

\label Method Signatures::
 
\Defmeth reinitialize-instance {\specparam{instance}{standard-object} {\rest} initargs} 

\label Arguments and Values::

\param{instance}---an \term{object}.
 
\param{initargs}---an \term{initialization argument list}.
 
\label Description::
 
The \term{generic function} \funref{reinitialize-instance} can be used to change
the values of \term{local slots} of an \param{instance} according to 
\param{initargs}.
%This generic function is called by the Meta-Object
%Protocol.  
This \term{generic function} can be called by users.
 
The system-supplied primary \term{method} for \funref{reinitialize-instance}
checks the validity of \param{initargs} and signals an error if
an \param{initarg} is supplied that is not declared as valid.
The \term{method} then calls the generic function \funref{shared-initialize}
with the following arguments:  the \param{instance}, 
\nil\ (which means no \term{slots}
should be initialized according to their initforms), and the
\param{initargs} it received.
 
\label Examples:\None.

\label Side Effects::

\TheGF{reinitialize-instance} changes the values of \term{local slots}.
 
\label Affected By:\None.
 
\label Exceptional Situations::

The system-supplied primary \term{method} for \funref{reinitialize-instance}
signals an error if an \param{initarg} is supplied that is not declared as valid.
 
\label See Also::
 
\funref{initialize-instance},
\funref{shared-initialize},
\funref{update-instance-for-redefined-class},
\funref{update-instance-for-different-class},
\funref{slot-boundp},
\funref{slot-makunbound},
{\secref\InstanceReInit},
{\secref\InitargRules},
{\secref\DeclaringInitargValidity}
 
\label Notes::
                                                         
\param{Initargs} are declared as valid by using the
\kwd{initarg} option to \macref{defclass}, or by defining 
\term{methods} for \funref{reinitialize-instance}
or \funref{shared-initialize}.  The keyword name
of each keyword parameter specifier in the \term{lambda list} of any 
\term{method}
defined on \funref{reinitialize-instance} or \funref{shared-initialize} is
declared as a valid initialization argument name for all 
\term{classes} for
which that \term{method} is applicable.
 
\endcom

%%% ========== SHARED-INITIALIZE
\begincom{shared-initialize}\ftype{Standard Generic Function}
 
\label Syntax::
 
\issue{INITIALIZATION-FUNCTION-KEYWORD-CHECKING}
\DefgenWithValues shared-initialize 
		  {instance slot-names {\rest} initargs {\key} {\allowotherkeys}}
		  {instance}
\endissue{INITIALIZATION-FUNCTION-KEYWORD-CHECKING}
 
\label Method Signatures::
 
\Defmeth shared-initialize {\specparam{instance}{standard-object} slot-names {\rest} initargs}
 
\label Arguments and Values::
 
\param{instance}---an \term{object}.
 
\param{slot-names}---a \term{list} or \t.
 
\param{initargs}---a \term{list} of \term{keyword/value pairs}
		   (of initialization argument \term{names} and \term{values}).
 
\label Description::
 
The generic function \funref{shared-initialize} is used to fill the 
\term{slots}                        
of an \param{instance} 
using \param{initargs} and \kwd{initform}
forms.  It is called when an instance is created, when an instance is
re-initialized, when an instance is updated to conform to a redefined
\term{class}, and when an instance is updated to conform to a different
\term{class}. The generic function \funref{shared-initialize} is called by the
system-supplied primary \term{method} for \funref{initialize-instance},
\funref{reinitialize-instance}, \funref{update-instance-for-redefined-class}, and
\funref{update-instance-for-different-class}.
 
The generic function \funref{shared-initialize} takes the following
arguments: the \param{instance} to be initialized, a specification of a set of
\param{slot-names} \term{accessible} in that \param{instance}, 
and any number of \param{initargs}.
The arguments after the first two must form an 
\term{initialization argument list}.  The system-supplied primary \term{method} on 
\funref{shared-initialize} initializes the \term{slots} with values according to the
\param{initargs} and supplied \kwd{initform} forms.  \param{Slot-names}
indicates which \term{slots} should be initialized according
to their \kwd{initform} forms if no \param{initargs} are
provided for those \term{slots}. 
 
The system-supplied primary \term{method} behaves as follows, 
regardless of whether the \term{slots} are local or shared: 
 
\beginlist
 
\itemitem{\bull}
 If an \param{initarg} in the \term{initialization argument list} 
 specifies a value for that \term{slot}, that
 value is stored into the \term{slot}, even if a value has
 already been stored in the \term{slot} before the \term{method} is run.
 
\itemitem{\bull}
 Any \term{slots} indicated by \param{slot-names} that are still unbound
 at this point are initialized according to their \kwd{initform} forms.
 For any such \term{slot} that has an \kwd{initform} form,
 that \term{form} is evaluated in the lexical environment of its defining 
 \macref{defclass} \term{form} and the result is stored into the \term{slot}.
 For example, if a \term{before method} stores a value in the \term{slot}, 
 the \kwd{initform} form will not be used to supply a value for the \term{slot}.
 
\itemitem{\bull} 
 The rules mentioned in {\secref\InitargRules} are obeyed.
 
\endlist
 
The \param{slots-names} argument specifies the \term{slots} that are to be
initialized according to their \kwd{initform} forms if no
initialization arguments apply.  It can be a \term{list} of slot \term{names}, 
which specifies the set of those slot \term{names}; or it can be the \term{symbol} \t, 
which specifies the set of all of the \term{slots}.

%The generic function \funref{shared-initialize} fills the \term{slots}
%of \param{instance}.
 
\label Examples:\None.

\label Affected By:\None.
 
\label Exceptional Situations:\None.

%% Removed per X3J13. -kmp 5-Oct-93
% An error \oftype{error} is signaled if an \param{initarg} 
% is supplied that is not declared as valid.

 
\label See Also::
 
\funref{initialize-instance},
\funref{reinitialize-instance},
\funref{update-instance-for-redefined-class},
\funref{update-instance-for-different-class},
\funref{slot-boundp},
\funref{slot-makunbound},
{\secref\ObjectCreationAndInit},
{\secref\InitargRules},
{\secref\DeclaringInitargValidity}
 
\label Notes::
 
\param{Initargs} are declared as valid by using the \kwd{initarg}
option to \macref{defclass}, or by defining 
\term{methods} for \funref{shared-initialize}. 
The keyword name of each keyword parameter
specifier in the \term{lambda list} of any \term{method} defined on 
\funref{shared-initialize} is declared as a valid \param{initarg}
name for all \term{classes} for which that \term{method} is applicable.
 
Implementations are permitted to optimize \kwd{initform} forms that 
neither produce nor depend on side effects, by evaluating these \term{forms}
and storing them into slots before running any 
\funref{initialize-instance} methods, rather than by handling them in the
primary \funref{initialize-instance} method.  (This optimization might
be implemented by having the \funref{allocate-instance} method copy a
prototype instance.)
 
Implementations are permitted to optimize default initial value forms
for \param{initargs} associated with slots by not actually
creating the complete initialization argument 
\term{list} when the only \term{method}
that would receive the complete \term{list} is the 
\term{method} on \typeref{standard-object}.
In this case default initial value forms can be 
treated like \kwd{initform} forms.  This optimization has no visible
effects other than a performance improvement.
 
\endcom

%%% ========== UPDATE-INSTANCE-FOR-DIFFERENT-CLASS
\begincom{update-instance-for-different-class}\ftype{Standard Generic Function}
 
\label Syntax::
 
\issue{INITIALIZATION-FUNCTION-KEYWORD-CHECKING}
\DefgenWithValues update-instance-for-different-class
		  {previous current 
		   {\rest} initargs
		   {\key} {\allowotherkeys}}
		  {\term{implementation-dependent}}
\endissue{INITIALIZATION-FUNCTION-KEYWORD-CHECKING}
 
\label Method Signatures::
 
\Defmeth {update-instance-for-different-class}
	 {\vtop{\hbox{\specparam{previous}{standard-object}}
	        \hbox{\specparam{current}{standard-object}}
	        \hbox{{\rest} initargs}}}
 
\label Arguments and Values::
 
\param{previous}---a copy of the original \term{instance}.
 
\param{current}---the original \term{instance} (altered).
 
\param{initargs}---an \term{initialization argument list}.
 
\label Description::
 
The generic function \funref{update-instance-for-different-class} is not
intended to be called by programmers.  Programmers may write
\term{methods} for it.  \Thefunction{update-instance-for-different-class}
is called only by \thefunction{change-class}.
 
The system-supplied primary \term{method} on 
\funref{update-instance-for-different-class} checks the validity of
\param{initargs} and signals an error if an \param{initarg}
is supplied that is not declared as valid.  This \term{method} then
initializes \term{slots} with values according to the \param{initargs},
and initializes the newly added \term{slots} with values according
to their \kwd{initform} forms.  It does this by calling the generic
function \funref{shared-initialize} with the following arguments: the 
instance (\param{current}),
a list of \term{names} of the newly added \term{slots}, and the \param{initargs}
it received.  Newly added \term{slots} are those \term{local slots} for which
no \term{slot} of the same name exists in the \param{previous} class.
 
\term{Methods} for \funref{update-instance-for-different-class} can be defined to
specify actions to be taken when an \term{instance} is updated.  If only 
\term{after methods} for \funref{update-instance-for-different-class} are
defined, they will be run after the system-supplied primary \term{method} for
initialization and therefore will not interfere with the default
behavior of \funref{update-instance-for-different-class}.
 
\term{Methods} on \funref{update-instance-for-different-class} can be defined to
initialize \term{slots} differently from \funref{change-class}.  The default
behavior of \funref{change-class} is described in 
\secref\ChangingInstanceClass.
 
The arguments to \funref{update-instance-for-different-class} are
computed by \funref{change-class}.  When \funref{change-class} is invoked on
an \term{instance}, a copy of that \term{instance} is made; \funref{change-class} then
destructively alters the original \term{instance}. The first argument to
\funref{update-instance-for-different-class}, \param{previous}, is that
copy; it holds the old \term{slot} values temporarily.  This argument has
dynamic extent within \funref{change-class}; if it is referenced in any
way once \funref{update-instance-for-different-class} returns, the
results are undefined.  The second argument to
\funref{update-instance-for-different-class}, \param{current}, is the altered
original \term{instance}.
The intended use of \param{previous} is to extract old \term{slot} values by using
\funref{slot-value} or \macref{with-slots} or by invoking 
a reader generic function, or to run other \term{methods} that were applicable to 
\term{instances} of
the original \term{class}.
 
%The system-supplied primary \term{method} on
%\funref{update-instance-for-different-class}
%initializes \term{slots} with values according to the initialization
%arguments, and initializes the newly added \term{slots} with values according
%to their \kwd{initform} forms.
 
\label Examples::
 
See the example for \thefunction{change-class}.
 
\label Affected By:\None.
 
\label Exceptional Situations::
The system-supplied primary \term{method} on
\funref{update-instance-for-different-class} signals an error if an
initialization argument is supplied that is not declared as valid.
 
\label See Also::
                            
\funref{change-class},
\funref{shared-initialize},
{\secref\ChangingInstanceClass},
{\secref\InitargRules},
{\secref\DeclaringInitargValidity}
 
\label Notes::
 
\param{Initargs} are declared as valid by using the \kwd{initarg}
option to \macref{defclass}, or by defining \term{methods}
for \funref{update-instance-for-different-class} or \funref{shared-initialize}.
The keyword name of each keyword parameter specifier in the \term{lambda list} of
any \term{method} defined on \funref{update-instance-for-different-class}
or \funref{shared-initialize} is declared as a valid \param{initarg} name
for all \term{classes} for which that \term{method} is applicable.

The value returned by \funref{update-instance-for-different-class} is
ignored by \funref{change-class}.
 
\endcom

%%% ========== UPDATE-INSTANCE-FOR-REDEFINED-CLASS
\begincom{update-instance-for-redefined-class}\ftype{Standard Generic Function}
 
\label Syntax::
 
\issue{INITIALIZATION-FUNCTION-KEYWORD-CHECKING}
\DefgenWithValuesNewline update-instance-for-redefined-class
                         {\vtop{\hbox{instance}
                                \hbox{added-slots discarded-slots}
                                \hbox{property-list}
                                \hbox{{\rest} initargs {\key} {\allowotherkeys}}}}
			 {\starparam{result}}
\endissue{INITIALIZATION-FUNCTION-KEYWORD-CHECKING}

\label Method Signatures::
 
\Defmeth {update-instance-for-redefined-class} 
	 {\vtop{\hbox{\specparam{instance}{standard-object}}
		\hbox{added-slots discarded-slots}
		\hbox{property-list}
		\hbox{{\rest} initargs}}}
 
\label Arguments and Values::

\param{instance}---an \term{object}.
 
\param{added-slots}---a \term{list}.
 
\param{discarded-slots}---a \term{list}.
 
\param{property-list}---a \term{list}.
 
\param{initargs}---an \term{initialization argument list}.
 
\param{result}---an \term{object}.
 
\label Description::
 
The \term{generic function} \funref{update-instance-for-redefined-class} 
is not intended to be called by programmers. Programmers may write
\term{methods} for it.  The \term{generic function} 
\funref{update-instance-for-redefined-class} is called by the mechanism
activated by \funref{make-instances-obsolete}.
 
The system-supplied primary \term{method} on 
\funref{update-instance-for-redefined-class} checks the validity of
\param{initargs} and signals an error if an \param{initarg}
is supplied that is not declared as valid.  This \term{method} then
initializes \term{slots} with values according to the \param{initargs},
and initializes the newly \param{added-slots} with values according
to their \kwd{initform} forms.  It does this by calling the generic
function \funref{shared-initialize} with the following arguments: 
the \param{instance},
a list of names of the newly \param{added-slots} to \param{instance},
and the \param{initargs}
it received.  Newly \param{added-slots} are those \term{local slots} for which
no \term{slot} of the same name exists in the old version of the \term{class}.
                                                                 
When \funref{make-instances-obsolete} is invoked or when a \term{class} has been
redefined and an \term{instance} is being updated, a \param{property-list} is created
that captures the slot names and values of all the \param{discarded-slots} with
values in the original \param{instance}.  The structure of the 
\param{instance} is
transformed so that it conforms to the current class definition.  The
arguments to \funref{update-instance-for-redefined-class} are this
transformed \param{instance}, a list of \param{added-slots} to the
\param{instance}, a list \param{discarded-slots} from the
\param{instance}, and the \param{property-list} 
containing the slot names and values for
\term{slots} that were discarded and had values.  Included in this list of
discarded \term{slots} are \term{slots} that were local in the old \term{class} and are
shared in the new \term{class}.
 
%The system-supplied primary \term{method} on 
%\funref{update-instance-for-redefined-class} 
%initializes \term{slots} with values according to the \param{initargs},
%and initializes the newly \param{added-slots} with values according
%to their \kwd{initform} forms.  It does this by calling the generic
%function \funref{shared-initialize}. 

%!!! So?
The value returned by \funref{update-instance-for-redefined-class} is ignored.

\label Examples::
 
\code
  
 (defclass position () ())
 
 (defclass x-y-position (position)
     ((x :initform 0 :accessor position-x)
      (y :initform 0 :accessor position-y)))
 
;;; It turns out polar coordinates are used more than Cartesian 
;;; coordinates, so the representation is altered and some new
;;; accessor methods are added.
 
 (defmethod update-instance-for-redefined-class :before
    ((pos x-y-position) added deleted plist &key)
   ;; Transform the x-y coordinates to polar coordinates
   ;; and store into the new slots.
   (let ((x (getf plist 'x))
         (y (getf plist 'y)))
     (setf (position-rho pos) (sqrt (+ (* x x) (* y y)))
           (position-theta pos) (atan y x))))
  
 (defclass x-y-position (position)
     ((rho :initform 0 :accessor position-rho)
      (theta :initform 0 :accessor position-theta)))
  
;;; All instances of the old x-y-position class will be updated
;;; automatically.
 
;;; The new representation is given the look and feel of the old one.
 
 (defmethod position-x ((pos x-y-position))  
    (with-slots (rho theta) pos (* rho (cos theta))))
 
 (defmethod (setf position-x) (new-x (pos x-y-position))
    (with-slots (rho theta) pos
      (let ((y (position-y pos)))
        (setq rho (sqrt (+ (* new-x new-x) (* y y)))
              theta (atan y new-x))
        new-x)))
 
 (defmethod position-y ((pos x-y-position))
    (with-slots (rho theta) pos (* rho (sin theta))))
 
 (defmethod (setf position-y) (new-y (pos x-y-position))
    (with-slots (rho theta) pos
      (let ((x (position-x pos)))
        (setq rho (sqrt (+ (* x x) (* new-y new-y)))
              theta (atan new-y x))
        new-y)))
 
\endcode
 
\label Affected By:\None.
 
\label Exceptional Situations::
The system-supplied primary \term{method} on 
\funref{update-instance-for-redefined-class} signals an error if an
\param{initarg} is supplied that is not declared as valid.
 
\label See Also::
 
\funref{make-instances-obsolete},
\funref{shared-initialize},
{\secref\ClassReDef},
{\secref\InitargRules},
{\secref\DeclaringInitargValidity}
 
\label Notes::
 
\param{Initargs} are declared as valid by using the \kwd{initarg}
option to \macref{defclass}, or by defining \term{methods} for
\funref{update-instance-for-redefined-class} or \funref{shared-initialize}.
The keyword name of each keyword parameter specifier in the \term{lambda list} of
any \term{method} defined on 
\funref{update-instance-for-redefined-class} or 
\funref{shared-initialize} is declared as a valid \param{initarg} name
for all \term{classes} for which that \term{method} is applicable.
 
\endcom

%%% ========== CHANGE-CLASS
\begincom{change-class}\ftype{Standard Generic Function}
 
\label Syntax::
 
\issue{CHANGE-CLASS-INITARGS:PERMIT}
\DefgenWithValues change-class
		  {instance new-class {\key} {\allowotherkeys}}
		  {instance}
\endissue{CHANGE-CLASS-INITARGS:PERMIT}

\label Method Signatures::
 
\issue{CHANGE-CLASS-INITARGS:PERMIT}
\Defmeth {change-class} {\specparam{instance}{standard-object}
			 \specparam{new-class}{standard-class}
			 {\rest} initargs}
 
\Defmeth {change-class} {\specparam{instance}{t}
		         \specparam{new-class}{symbol}
			 {\rest} initargs}
\endissue{CHANGE-CLASS-INITARGS:PERMIT}

\label Arguments and Values::
 
\param{instance}---an \term{object}.
                                            
\param{new-class}---a \term{class designator}.
 
\issue{CHANGE-CLASS-INITARGS:PERMIT}
\param{initargs}---an \term{initialization argument list}.
\endissue{CHANGE-CLASS-INITARGS:PERMIT}

\label Description::
                                      
The \term{generic function} \funref{change-class} changes the 
\term{class} of an \param{instance} to \param{new-class}.  
It destructively modifies and returns the \param{instance}.
 
If in the old \term{class} there is any \term{slot} of the 
same name as a local \term{slot} in the \param{new-class}, 
the value of that \term{slot} is retained.  This means that if 
the \term{slot} has a value, the value returned by \funref{slot-value}
after \funref{change-class} is invoked is \funref{eql} to the
value returned by \funref{slot-value} before \funref{change-class} is
invoked.  Similarly, if the \term{slot} was unbound, it remains
unbound.  The other \term{slots} are initialized as described in 
\secref\ChangingInstanceClass.
 
After completing all other actions, \funref{change-class} invokes
\funref{update-instance-for-different-class}.  The
generic function \funref{update-instance-for-different-class} can be used
to assign values to slots in the transformed instance.
\issue{CHANGE-CLASS-INITARGS:PERMIT}
\Seesection\InitNewLocalSlots.
\endissue{CHANGE-CLASS-INITARGS:PERMIT}
 
\issue{CHANGE-CLASS-INITARGS:PERMIT}
If the second of the above \term{methods} is selected, 
that \term{method} invokes \funref{change-class} 
on \param{instance}, \f{(find-class \param{new-class})},
and the \param{initargs}.
\endissue{CHANGE-CLASS-INITARGS:PERMIT}
 
\label Examples::
 
\code
 
 (defclass position () ())
  
 (defclass x-y-position (position)
     ((x :initform 0 :initarg :x)
      (y :initform 0 :initarg :y)))
  
 (defclass rho-theta-position (position)
     ((rho :initform 0)
      (theta :initform 0)))
  
 (defmethod update-instance-for-different-class :before ((old x-y-position) 
                                                         (new rho-theta-position)
                                                         &key)
   ;; Copy the position information from old to new to make new
   ;; be a rho-theta-position at the same position as old.
   (let ((x (slot-value old 'x))
         (y (slot-value old 'y)))
     (setf (slot-value new 'rho) (sqrt (+ (* x x) (* y y)))
           (slot-value new 'theta) (atan y x))))
  
;;; At this point an instance of the class x-y-position can be
;;; changed to be an instance of the class rho-theta-position using
;;; change-class:
 
 (setq p1 (make-instance 'x-y-position :x 2 :y 0))
  
 (change-class p1 'rho-theta-position)
  
;;; The result is that the instance bound to p1 is now an instance of
;;; the class rho-theta-position.   The update-instance-for-different-class
;;; method performed the initialization of the rho and theta slots based
;;; on the value of the x and y slots, which were maintained by
;;; the old instance.
 
\endcode
 
\label Examples:\None.

\label Affected By:\None.
 
\label Exceptional Situations:\None.
 
\label See Also::
 
\funref{update-instance-for-different-class},
{\secref\ChangingInstanceClass}
 
\label Notes::
 
The generic function \funref{change-class} has several semantic
difficulties.  First, it performs a destructive operation that can be
invoked within a \term{method} on an \term{instance} that was used to select that
\term{method}. 
When multiple \term{methods} are involved because \term{methods} are being
combined, the \term{methods} currently executing or about to be executed may
no longer be applicable.  Second, some implementations might use
compiler optimizations of slot \term{access}, and when the \term{class} of an
\term{instance} is changed the assumptions the compiler made might be
violated.  This implies that a programmer must not use
\funref{change-class} inside a \term{method} if any 
\term{methods} for that \term{generic function}
\term{access} any \term{slots}, or the results are undefined.
 
\endcom

%%% ========== SLOT-BOUNDP
\begincom{slot-boundp}\ftype{Function}
 
%!!! Barmar: 88-002R says this is generic.  I should check on this. -kmp 27-Aug-91

\label Syntax::
 
\DefunWithValues slot-boundp {instance slot-name} {generalized-boolean}
 
\label Arguments and Values::
               
\param{instance}---an \term{object}.
                                 
\param{slot-name}---a \term{symbol} naming a \term{slot} of \param{instance}.

\param{generalized-boolean}---a \term{generalized boolean}.

\label Description::
 
Returns \term{true} if the \term{slot} named \param{slot-name} in \param{instance} is bound;
otherwise, returns \term{false}.

\label Examples:\None.

\label Affected By:\None.
 
\label Exceptional Situations::

If no \term{slot} of the \term{name} \param{slot-name} exists in the 
\param{instance}, \funref{slot-missing} is called as follows:

\code
 (slot-missing (class-of \i{instance})
               \i{instance}
               \i{slot-name}
               'slot-boundp)
\endcode

\issue{SLOT-MISSING-VALUES:SPECIFY}
(If \funref{slot-missing} is invoked and returns a value,
a \term{boolean equivalent} to its \term{primary value} 
is returned by \funref{slot-boundp}.)
\endissue{SLOT-MISSING-VALUES:SPECIFY}

\issue{SLOT-VALUE-METACLASSES:LESS-MINIMAL}
The specific behavior depends on \param{instance}'s \term{metaclass}.
An error is never signaled if \param{instance} has \term{metaclass} \typeref{standard-class}.
An error is always signaled if \param{instance} has \term{metaclass} \typeref{built-in-class}.
The consequences are undefined if \param{instance} has any other \term{metaclass}--an error
might or might not be signaled in this situation.  Note in particular that the behavior
for \term{conditions} and \term{structures} is not specified.
\endissue{SLOT-VALUE-METACLASSES:LESS-MINIMAL}
 
\label See Also::
 
\funref{slot-makunbound},
\funref{slot-missing}
 
\label Notes::
            
\Thefunction{slot-boundp} allows for writing 
\term{after methods} on \funref{initialize-instance} in order to initialize only
those \term{slots} that have not already been bound.
 
\MentionMetaObjects{slot-boundp}{slot-boundp-using-class}

\endcom

%%% ========== SLOT-EXISTS-P
\begincom{slot-exists-p}\ftype{Function}

%!!! Barmar: 88-002R says this is generic.  I should check on this. -kmp 27-Aug-91

\label Syntax::
 
\DefunWithValues {slot-exists-p} {object slot-name} {generalized-boolean}

\label Arguments and Values::
 
\issue{SLOT-VALUE-METACLASSES:LESS-MINIMAL}
\param{object}---an \term{object}.
\endissue{SLOT-VALUE-METACLASSES:LESS-MINIMAL}
 
\param{slot-name}---a \term{symbol}.

\param{generalized-boolean}---a \term{generalized boolean}.
 
\label Description::
 
% \Thefunction{slot-exists-p} tests whether the \param{object} has
% a \term{slot} of the given \term{name}.
 
Returns \term{true} if the \param{object} has
a \term{slot} named \param{slot-name}.

\label Examples:\None.

\label Affected By::

\macref{defclass},
\macref{defstruct}
 
\label Exceptional Situations:\None.
 
\label See Also::
 
\macref{defclass},
\funref{slot-missing}

\label Notes::
 
\MentionMetaObjects{slot-exists-p}{slot-exists-p-using-class}
 
\endcom

%%% ========== SLOT-MAKUNBOUND
\begincom{slot-makunbound}\ftype{Function}

%!!! Barmar: 88-002R says this is generic.  I should check on this. -kmp 27-Aug-91

\label Syntax::
 
\DefunWithValues {slot-makunbound} {instance slot-name} {instance}
 
\label Arguments and Values::
 
\param{instance} -- instance.
 
\param{Slot-name}---a \term{symbol}.
 
\label Description::
 
\Thefunction{slot-makunbound} restores a \term{slot} 
of the name \param{slot-name} in an \param{instance} to
the unbound state.

\label Examples:\None.

\label Affected By:\None.
 
\label Exceptional Situations::

If no \term{slot} of the name \param{slot-name} exists in the 
\param{instance}, \funref{slot-missing} is called as follows:

\code
(slot-missing (class-of \i{instance})
              \i{instance}
              \i{slot-name}
              'slot-makunbound)
\endcode

\issue{SLOT-MISSING-VALUES:SPECIFY}
(Any values returned by \funref{slot-missing} in this case are
ignored by \funref{slot-makunbound}.)
\endissue{SLOT-MISSING-VALUES:SPECIFY}
 
\issue{SLOT-VALUE-METACLASSES:LESS-MINIMAL}
The specific behavior depends on \param{instance}'s \term{metaclass}.
An error is never signaled if \param{instance} has \term{metaclass} \typeref{standard-class}.
An error is always signaled if \param{instance} has \term{metaclass} \typeref{built-in-class}.
The consequences are undefined if \param{instance} has any other \term{metaclass}--an error
might or might not be signaled in this situation.  Note in particular that the behavior
for \term{conditions} and \term{structures} is not specified.
\endissue{SLOT-VALUE-METACLASSES:LESS-MINIMAL}
 
\label See Also::
 
\funref{slot-boundp},
\funref{slot-missing}
 
\label Notes::
 
\MentionMetaObjects{slot-makunbound}{slot-makunbound-using-class}
 
\endcom

%%% ========== SLOT-MISSING
\begincom{slot-missing}\ftype{Standard Generic Function}
 
\label Syntax::
 
\DefgenWithValues slot-missing
		  {class object slot-name operation {\opt} new-value}
		  {\starparam{result}}
 
\label Method Signatures::
 
\Defmeth slot-missing {\vtop{\hbox{\specparam{class}{t}
				   object slot-name}
			     \hbox{operation {\opt} new-value}}}
 
\label Arguments and Values::
 
\param{class}---the \term{class} of \param{object}.
 
\param{object}---an \term{object}.
 
\param{slot-name}---a \term{symbol} (the \term{name} of a would-be \term{slot}).
 
\param{operation}---one of the \term{symbols}
		    \funref{setf},
		    \funref{slot-boundp},
		    \funref{slot-makunbound},
		 or \funref{slot-value}.
 
\param{new-value}---an \term{object}.
 
\param{result}---an \term{object}.

\label Description::
 
The generic function \funref{slot-missing} is invoked when an attempt is
made to \term{access} a \term{slot} in an \param{object} whose 
\term{metaclass} is \typeref{standard-class}
and the \term{slot} of the name \param{slot-name}
is not a \term{name} of a
\term{slot} in that \term{class}. 
The default \term{method} signals an error.
 
The generic function \funref{slot-missing} is not intended to be called by
programmers.  Programmers may write \term{methods} for it.
 
The generic function \funref{slot-missing} may be called during
evaluation of \funref{slot-value}, \funref{(setf slot-value)}, 
\funref{slot-boundp}, and \funref{slot-makunbound}.  For each
of these operations the corresponding \term{symbol} 
for the \param{operation}
argument is \misc{slot-value}, \misc{setf}, \misc{slot-boundp},
and \misc{slot-makunbound} respectively.
 
The optional \param{new-value} argument to \funref{slot-missing} is used
when the operation is attempting to set the value of the \term{slot}.
 
\issue{SLOT-MISSING-VALUES:SPECIFY}
If \funref{slot-missing} returns, its values will be treated as follows:

\beginlist
\item{\bull}
If the \param{operation} is \misc{setf} or \misc{slot-makunbound},
any \term{values} will be ignored by the caller.

\item{\bull}
If the \param{operation} is \misc{slot-value},
only the \term{primary value} will be used by the caller,
and all other values will be ignored.

\item{\bull}
If the \param{operation} is \misc{slot-boundp},
any \term{boolean equivalent} of the \term{primary value}
of the \term{method} might be is used,
and all other values will be ignored.
\endlist
\endissue{SLOT-MISSING-VALUES:SPECIFY}

\label Examples:\None.

\label Affected By:\None.
 
\label Exceptional Situations::

The default \term{method} on \funref{slot-missing} 
signals an error \oftype{error}.
 
\label See Also::
 
\macref{defclass},
\funref{slot-exists-p},
\funref{slot-value}

\label Notes::
 
The set of arguments (including the \term{class} of the instance) facilitates
defining methods on the metaclass for \funref{slot-missing}.

\endcom

%%% ========== SLOT-UNBOUND
\begincom{slot-unbound}\ftype{Standard Generic Function}
 
\label Syntax::
                                
\DefgenWithValues {slot-unbound} {class instance slot-name} {\starparam{result}}
 
\label Method Signatures::
 
\Defmeth slot-unbound {\specparam{class}{t}
		       instance slot-name}
 
\label Arguments and Values::
 
\param{class}---the \term{class} of the \param{instance}.
 
\param{instance}---the \param{instance} in which an attempt
		   was made to \term{read} the \term{unbound} \term{slot}.
 
\param{slot-name}---the \term{name} of the \term{unbound} \term{slot}.
 
\param{result}---an \term{object}.

\label Description::
 
The generic function \funref{slot-unbound} is called when an
unbound \term{slot} is read in
an \param{instance} whose metaclass is \typeref{standard-class}.
The default \term{method} signals an error 
\issue{UNDEFINED-VARIABLES-AND-FUNCTIONS:COMPROMISE} 
\oftype{unbound-slot}.
The name slot of the 
\typeref{unbound-slot} \term{condition} is initialized
  to the name of the offending variable, and the instance slot
  of the \typeref{unbound-slot} \term{condition} is initialized to the offending instance.
\endissue{UNDEFINED-VARIABLES-AND-FUNCTIONS:COMPROMISE} 

 
The generic function \funref{slot-unbound} is not intended to be called
by programmers.  Programmers may write \term{methods} for it.
\Thefunction{slot-unbound} is called only 
%%Looks like metaobjects to me. -kmp 15-Jan-91
%by \thefunction{slot-value-using-class} and thus 
indirectly by \funref{slot-value}.
 
\issue{SLOT-MISSING-VALUES:SPECIFY}
%% Per X3J13. -kmp 05-Oct-93
% If \funref{slot-unbound} returns, its values will be treated as follows:
% 
% \beginlist
% \item{\bull}
% If the \param{operation} is \misc{setf} or \misc{slot-makunbound},
% any \term{values} will be ignored by the caller.
% 
% \item{\bull}
% If the \param{operation} is \misc{slot-value},
% only the \term{primary value} will be used by the caller,
% and all other values will be ignored.
% 
% \item{\bull}
% If the \param{operation} is \misc{slot-boundp},
% any \term{boolean equivalent} of the \term{primary value}
% of the \term{method} might be is used,
% and all other values will be ignored.
% \endlist
If \funref{slot-unbound} returns, 
only the \term{primary value} will be used by the caller,
and all other values will be ignored.
\endissue{SLOT-MISSING-VALUES:SPECIFY}

\label Examples:\None.

\label Affected By:\None.
 
\label Exceptional Situations::
\issue{UNDEFINED-VARIABLES-AND-FUNCTIONS:COMPROMISE} 
The default \term{method} on \funref{slot-unbound}
signals an error \oftype{unbound-slot}.
\endissue{UNDEFINED-VARIABLES-AND-FUNCTIONS:COMPROMISE} 
 
\label See Also::
 
\funref{slot-makunbound}
 
\label Notes::
                                                                               
An unbound \term{slot} may occur if no \kwd{initform} form was
specified for the \term{slot} and the \term{slot} value has not been set,
or if \funref{slot-makunbound} has been called on the \term{slot}.
 
\endcom

%%% ========== SLOT-VALUE
\begincom{slot-value}\ftype{Function}
 
\label Syntax::
 
\DefunWithValues {slot-value} {object slot-name} {value}
 
\label Arguments and Values::
 
\param{object}---an \term{object}.
 
\param{slot-name}---a \term{symbol}.
 
\param{value}---an \term{object}.
 
\label Description::
 
\Thefunction{slot-value} returns the \term{value} of the \term{slot}
named \param{slot-name} in the \param{object}.
If there is no \term{slot} named \param{slot-name}, \funref{slot-missing} is called.
If the \term{slot} is unbound, \funref{slot-unbound} is called.
 
%!!! Reflect this in the Syntax above? Or is (SETF SLOT-VALUE) described somewhere?
The macro \macref{setf} can be used with \funref{slot-value} 
to change the value of a \term{slot}. 

\label Examples::
 
\code
 (defclass foo () 
   ((a :accessor foo-a :initarg :a :initform 1)
    (b :accessor foo-b :initarg :b)
    (c :accessor foo-c :initform 3)))
\EV #<STANDARD-CLASS FOO 244020371>
 (setq foo1 (make-instance 'foo :a 'one :b 'two))
\EV #<FOO 36325624>
 (slot-value foo1 'a) \EV ONE
 (slot-value foo1 'b) \EV TWO
 (slot-value foo1 'c) \EV 3
 (setf (slot-value foo1 'a) 'uno) \EV UNO
 (slot-value foo1 'a) \EV UNO
 (defmethod foo-method ((x foo))
   (slot-value x 'a))
\EV #<STANDARD-METHOD FOO-METHOD (FOO) 42720573>
 (foo-method foo1) \EV UNO
\endcode

\label Affected By:\None.
 
\label Exceptional Situations::
 
If an attempt is made to read a \term{slot} and no \term{slot} of
the name \param{slot-name} exists in the \param{object}, 
\funref{slot-missing} is called as follows:

\code
 (slot-missing (class-of \i{instance})
               \i{instance}
               \i{slot-name}
               'slot-value)
\endcode
 
\issue{SLOT-MISSING-VALUES:SPECIFY}
(If \funref{slot-missing} is invoked, its \term{primary value} 
 is returned by \funref{slot-value}.)
\endissue{SLOT-MISSING-VALUES:SPECIFY}

If an attempt is made to write a \term{slot} and no \term{slot} of
the name \param{slot-name} exists in the \param{object},
\funref{slot-missing} is called as follows:

\code
 (slot-missing (class-of \i{instance})
               \i{instance}
               \i{slot-name}
               'setf
               \i{new-value})
\endcode

\issue{SLOT-MISSING-VALUES:SPECIFY}
(If \funref{slot-missing} returns in this case, any \term{values} are ignored.)
\endissue{SLOT-MISSING-VALUES:SPECIFY}

\issue{SLOT-VALUE-METACLASSES:LESS-MINIMAL}
The specific behavior depends on \param{object}'s \term{metaclass}.
An error is never signaled if \param{object} has \term{metaclass} \typeref{standard-class}.
An error is always signaled if \param{object} has \term{metaclass} \typeref{built-in-class}.
The consequences are 
%% Per X3J13. -kmp 05-Oct-93
%undefined
unspecified
if \param{object} has any other \term{metaclass}--an error
might or might not be signaled in this situation.  Note in particular that the behavior
for \term{conditions} and \term{structures} is not specified.
\endissue{SLOT-VALUE-METACLASSES:LESS-MINIMAL}

\label See Also::
 
\funref{slot-missing},
\funref{slot-unbound},
\macref{with-slots}
 
\label Notes::
 
\MentionMetaObjects{slot-value}{slot-value-using-class}
 
Implementations may optimize \funref{slot-value} by compiling it inline.
 
\endcom

%%% ========== METHOD-QUALIFIERS
\begincom{method-qualifiers}\ftype{Standard Generic Function}
 
\label Syntax::
 
\DefgenWithValues method-qualifiers {method} {qualifiers}
 
\label Method Signatures::
 
\Defmeth method-qualifiers {\specparam{method}{standard-method}}
 
\label Arguments and Values::
 
\param{method}---a \term{method}.
 
\param{qualifiers}---a \term{proper list}.
 
\label Description::
 
Returns a \term{list} of the \term{qualifiers} of the \param{method}.
 
\label Examples::
 
\code
 (defmethod some-gf :before ((a integer)) a)
\EV #<STANDARD-METHOD SOME-GF (:BEFORE) (INTEGER) 42736540>
 (method-qualifiers *) \EV (:BEFORE)
\endcode
 
\label Affected By:\None.
 
\label Exceptional Situations:\None.
 
\label See Also:: 
 
\macref{define-method-combination}
 
\label Notes:\None.
 
\endcom

%%% ========== NO-APPLICABLE-METHOD
\begincom{no-applicable-method}\ftype{Standard Generic Function}
 
\label Syntax::
 
\DefgenWithValues no-applicable-method 
		  {generic-function {\rest} function-arguments}
		  {\starparam{result}}
 
\label Method Signatures::
 
\Defmeth no-applicable-method {\vtop{\hbox{\specparam{generic-function}{t}}
				     \hbox{{\rest} function-arguments}}}
 
\label Arguments and Values::
 
%!!! But the signature above says T, not STANDARD-GENERIC-FUNCTION ...? -kmp 9-May-91
\param{generic-function}---a \term{generic function} 
%% Per X3J13. -kmp 05-Oct-93
%			   of the class \typeref{standard-generic-function}
			   on which no \term{applicable method} was found.  
 
\param{function-arguments}---\term{arguments} to the \param{generic-function}.
 
\param{result}---an \term{object}.

\label Description::
 
The generic function \funref{no-applicable-method} is called when a
\term{generic function} 
%% Per X3J13. -kmp 05-Oct-93
%of \theclass{standard-generic-function} 
is invoked
and no \term{method} on that \term{generic function} is applicable.
% "default \term{method}" => "\term{default method}" per X3J13. -kmp 05-Oct-93
The \term{default method} signals an error.
 
The generic function \funref{no-applicable-method} is not intended
to be called by programmers.  Programmers may write \term{methods} for it.
 
\label Examples:\None.
 
\label Affected By:\None.
 
\label Exceptional Situations::

The default \term{method} signals an error \oftype{error}.
 
\label See Also::
 
\label Notes:\None.
 
\endcom

%%% ========== NO-NEXT-METHOD
\begincom{no-next-method}\ftype{Standard Generic Function}
 
\label Syntax::
 
\DefgenWithValues no-next-method
		  {generic-function method {\rest} args}
		  {\starparam{result}}
 
\label Method Signatures::
 
\Defmeth no-next-method {\vtop{\hbox{\specparam{generic-function}{standard-generic-function}}
			       \hbox{\specparam{method}{standard-method}}
			       \hbox{{\rest} args}}}
 
\label Arguments and Values::
 
\param{generic-function} -- \term{generic function} to which \param{method} belongs.
 
\param{method} -- \term{method} that contained the call to
		  \funref{call-next-method} for which there is no next \term{method}.
 
\param{args} -- arguments to \funref{call-next-method}.
 
\param{result}---an \term{object}.

\label Description::
 
\TheGF{no-next-method} is called by \funref{call-next-method} 
when there is no \term{next method}.
 
\TheGF{no-next-method} is not intended to be called by programmers.
Programmers may write \term{methods} for it.
 
\label Examples:\None.

\label Affected By:\None.
 
\label Exceptional Situations::

The system-supplied \term{method} on \funref{no-next-method} 
signals an error \oftype{error}. \editornote{KMP: perhaps control-error??}
 
\label See Also::
 
\funref{call-next-method}
 
\label Notes:\None.
 
\endcom

%%% ========== REMOVE-METHOD
\begincom{remove-method}\ftype{Standard Generic Function}
 
\label Syntax::
 
\DefgenWithValues remove-method {generic-function method} {generic-function}
 
\label Method Signatures::
 
\Defmeth remove-method {\vtop{\hbox{\specparam{generic-function}{standard-generic-function}}
			      \hbox{method}}}
 
\label Arguments and Values::
 
\param{generic-function}---a \term{generic function}.
 
\param{method}---a \term{method}.

\label Description::
 
\TheGF{remove-method} removes a \term{method} from \param{generic-function}
by modifying the \param{generic-function} (if necessary).
 
\funref{remove-method} must not signal an error if the \term{method} 
is not one of the \term{methods} on the \param{generic-function}.
 
\label Examples:\None.

\label Affected By:\None.
 
\label Exceptional Situations:\None.
 
\label See Also::
 
\funref{find-method}
 
\label Notes:\None.
 
\endcom

\issue{GENERIC-FLET-POORLY-DESIGNED:DELETE}
% %%% ========== GENERIC-FLET
% %%% ========== GENERIC-LABELS
% 
% \begincom{generic-flet, generic-labels}\ftype{Special Operator}
%  
% \label Syntax::
%  
% \DefspecWithValuesNewline generic-flet
%        {\vtop{\hbox{\paren{\starparen{function-name 
% 			   	      lambda-list
%                                       \interleave{\down{option} |
%                                                   \stardown{method-description}}}}}
%               \hbox{\starparam{form}}}}
%        {\starparam{result}}
% 
% \DefspecWithValuesNewline generic-labels
%        {\vtop{\hbox{\paren{\starparen{function-name 
% 			   	      lambda-list
%                                       \interleave{\down{option} |
%                                                   \stardown{method-description}}}}}
%               \hbox{\starparam{form}}}}
%        {\starparam{result}}
% 
% {\GFauxOptionsAndMethDesc}
%  
% \label Arguments and Values::
%  
% \editornote{KMP: Treatment of documentation?}%!!!
% 
% \param{function-name}, \param{lambda-list}, \param{option}, \param{method-qualifier},
% \param{specialized-lambda-list}---the same as for \macref{defgeneric}.
% 
% \param{method-definition}---the same as for \macref{defmethod}.
%  
% \param{forms}---an \term{implicit progn}.
% 
% \param{results}---the \term{values} returned by the \param{forms}.
%  
% \label Description::
%  
% \Thespecop{generic-flet} is analogous to \thespecop{flet}, and
% \thespecop{generic-labels} is analogous to \thespecop{labels}.
% 
% The \term{forms} are \term{evaluated} in a \term{lexical environment}
% in which \term{function} \term{bindings} for the \param{function-names} have
% been \term{established}.
% The \term{value} of each \term{binding} is a \term{fresh} \term{generic function},
% with \term{methods} as specified by the corresponding \param{method-description}.
% The \term{bindings} of the \term{function-names} have \term{lexical scope}, 
% and the \term{generic functions} (and their \term{methods}) have \term{indefinite extent}.
% 
% For \specref{generic-flet},
% the \term{scope} of the \term{bindings} for the \param{function-names}
% includes only the \param{forms},
% not the bodies of the local definitions of the \term{methods}.
% Within the method bodies, references to any of the \param{function-names}
% refer to global \term{functions} with coincidentally similar names,
% not to the \term{generic functions} established for use by the \param{forms}.
% It is thus not possible to define recursive \term{functions} 
% with \specref{generic-flet}.
%  
% For \specref{generic-labels},
% the \term{scope} of the \term{bindings} for the \param{function-names}
% includes the entire \specref{generic-labels} \term{form},
% including not only the \param{forms}, 
% but also the bodies of the local definitions of the \term{methods}.
% It is thus possible to define recursive \term{functions} 
% with \specref{generic-labels}.
% 
% The body of each \term{method} is enclosed in an \term{implicit block}.  
% If \param{function-name} is a \term{symbol}, 
% the \term{implicit block} has that \term{name}.
% If \param{function-name} is a \term{list} of the form \f{(setf \param{sym})},
% the \term{name} of the \term{implicit block} is \param{sym}.
%  
% \label Examples:\None.
% 
% \label Affected By:\None.
%  
% \label Exceptional Situations:\None.
%  
% \label See Also::
% 
% \specref{flet}, 
% \specref{labels}, 
% \macref{defmethod},
% \macref{defgeneric},
% \macref{generic-function}
%  
% \label Notes:\None.
%  
% \endcom
% 
% 
% %%% ========== GENERIC-FUNCTION
% \begincom{generic-function}\ftype{Macro}
%  
% \label Syntax::
%  
% \DefmacWithValues generic-function
% 		  {lambda-list \interleave{\down{option} | \starparam{method-description}}}
% 		  {generic-function}
% 
% {\GFauxOptionsAndMethDesc}
% 
% \label Arguments and Values::
%  
% \editornote{KMP: Treatment of documentation?}%!!!
% 
% \param{Option}, \param{method-qualifier}, 
% and \param{specialized-lambda-list} are the same as for
% \macref{defgeneric}.
%  
% \param{generic-function}---a \term{generic function} \term{object}.
%  
% \label Description::
%  
% Creates an \term{anonymous} \term{generic function} with the
% set of \term{methods} specified by its \param{method-descriptions}.
%  
% %%I don't think this is really needed. -kmp 15-Jan-91
% % If no \param{method-descriptions} are supplied,
% % an \term{anonymous} \term{generic function} with no \term{methods} is created.
%  
% \label Examples:\None.
% 
% \label Affected By:\None.
%  
% \label Exceptional Situations:\None.
%  
% \label See Also::
%  
% \macref{defgeneric},
% \specref{generic-flet},
% \specref{generic-labels},
% \macref{defmethod}
%  
% \label Notes:\None.
%  
% \endcom
\endissue{GENERIC-FLET-POORLY-DESIGNED:DELETE}

%%% ========== MAKE-INSTANCE
\begincom{make-instance}\ftype{Standard Generic Function}
 
\label Syntax::
 
\issue{INITIALIZATION-FUNCTION-KEYWORD-CHECKING}
\DefgenWithValues make-instance
		  {class {\rest} initargs {\key} {\allowotherkeys}}
		  {instance}
\endissue{INITIALIZATION-FUNCTION-KEYWORD-CHECKING}
 
\label Method Signatures::
 
\Defmeth make-instance {\specparam{class}{standard-class} {\rest} initargs}
 
\Defmeth make-instance {\specparam{class}{symbol} {\rest} initargs}

\label Arguments and Values::
 
%!!! I'd rather this were called class-name. -kmp 15-Jan-91
%!!! Or maybe class designator? -kmp 18-Feb-91
\param{class}---a \term{class},
	     or a \term{symbol} that names a \term{class}.
 
\param{initargs}---an \term{initialization argument list}.
 
\param{instance}---a \term{fresh} \term{instance} of \term{class} \param{class}.
 
\label Description::
 
The \term{generic function} \funref{make-instance} 
creates and returns a new \term{instance} of the given \param{class}.
 
If the second of the above \term{methods} is selected, 
that \term{method} invokes \funref{make-instance} on the arguments
\f{(find-class \param{class})} and \param{initargs}.
 
The initialization arguments are checked within \funref{make-instance}.
 
The \term{generic function} \funref{make-instance} 
may be used as described in \secref\ObjectCreationAndInit.
 
\label Affected By:\None.
 
\label Exceptional Situations::
 
If any of the initialization arguments has not
been declared as valid, an error \oftype{error} is signaled.
 
\label See Also::
 
\macref{defclass},
\funref{class-of},
\funref{allocate-instance},
\funref{initialize-instance},
{\secref\ObjectCreationAndInit}
 
%% Per X3J13. -kmp 05-Oct-93
\label Notes:\None.
 
%The meta-object protocol can be used to define new methods on
%\funref{make-instance} to replace the object-creation protocol.
 
\endcom

%%% ========== MAKE-INSTANCES-OBSOLETE
\begincom{make-instances-obsolete}\ftype{Standard Generic Function}
 
\label Syntax::
 
\DefgenWithValues make-instances-obsolete {class} {class}
 
\label Method Signatures::
 
\Defmeth make-instances-obsolete {\specparam{class}{standard-class}}
 
\Defmeth make-instances-obsolete {\specparam{class}{symbol}}
 
\label Arguments and Values::
 
\param{class}---a \term{class designator}.
 
\label Description::
 
\Thefunction{make-instances-obsolete} has the effect of
initiating the process of updating the instances of the
\term{class}. During updating, the generic function
\funref{update-instance-for-redefined-class} will be invoked.
 
The generic function \funref{make-instances-obsolete} is invoked
automatically by the system when \macref{defclass} has been used to
redefine an existing standard class and the set of local 
\term{slots} \term{accessible} in an
instance is changed or the order of \term{slots} in storage is changed.  It
can also be explicitly invoked by the user.
 
If the second of the above \term{methods} is selected, that 
\term{method} invokes
\funref{make-instances-obsolete} on \f{(find-class \param{class})}.
 
\label Examples:\None.
 
\label Affected By:\None.
 
\label Exceptional Situations:\None.
 
\label See Also::
 
\funref{update-instance-for-redefined-class},
{\secref\ClassReDef}
 
\label Notes:\None.
 
\endcom

%%% ========== MAKE-LOAD-FORM
\begincom{make-load-form}\ftype{Standard Generic Function}
\issue{MAKE-LOAD-FORM-CONFUSION:REWRITE}
\issue{LOAD-OBJECTS:MAKE-LOAD-FORM}
 
\label Syntax::
 
\DefgenWithValues {make-load-form}
        	  {object {\opt} environment}
        	  {creation-form\brac{, initialization-form}}
 
\label Method Signatures::
 
\Defmeth make-load-form {\specparam{object}{standard-object}  {\opt} environment}
\Defmeth make-load-form {\specparam{object}{structure-object} {\opt} environment}
\Defmeth make-load-form {\specparam{object}{condition}        {\opt} environment}
\Defmeth make-load-form {\specparam{object}{class}            {\opt} environment}

\label Arguments and Values::
 
\param{object}---an \term{object}.
 
% Barrett: Moved up from end, so arguments first, then values.
\param{environment}---an \term{environment object}.

\param{creation-form}---a \term{form}.

\param{initialization-form}---a \term{form}.
 
\label Description::
          
\TheGF{make-load-form} creates and returns 
one or two \term{forms},
     a \param{creation-form}
 and an \param{initialization-form},
that enable \funref{load} to construct an \term{object}
equivalent to \param{object}.
%% Added per item 1 of MAKE-LOAD-FORM-CONFUSION:
\param{Environment} is an \term{environment object} 
corresponding to the \term{lexical environment} 
in which the \term{forms} will be processed.

% %% Added with rewrites per item 1.1 of MAKE-LOAD-FORM-CONFUSION.
% %% I had to do some minor fooling with the wording to make it read
% %% better using `modern' terminology. -kmp 11-Feb-92
% \funref{make-load-form} is called by the \term{file compiler} when
% \param{object} is referenced as a \term{literal object}
% in a \term{file} being compiled
% %by \funref{compile-file}
% if the \term{object} is a \term{generalized instance}
% of \typeref{standard-object}, \typeref{structure-object},
% \typeref{condition}, or any of a (possibly empty) \term{implementation-dependent}
% %list
% set of other \term{classes}.
% The \term{file compiler} will only call \funref{make-load-form} once for the \term{same}
% \term{object} within a single \term{file}.
% Barrett: Commented out the above, instead referencing the appropriate
%          section describing how the compiler performs constant processing.
The \term{file compiler} calls \funref{make-load-form} to process certain
\term{classes} of \term{literal objects}; \seesection\CallingMakeLoadForm.

%% Added per item 1.2 of MAKE-LOAD-FORM-CONFUSION:
\term{Conforming programs} may call \funref{make-load-form} directly,
providing \param{object} is a \term{generalized instance} of
% Barrett: Fix dangling reference, now that list has been moved.
% one of the
% explicitly named \term{classes} mentioned previously.
\typeref{standard-object}, \typeref{structure-object}, 
or \typeref{condition}.

The creation form is a \term{form} that, when evaluated at
\funref{load} time, should return an \term{object} that 
is equivalent to \param{object}.  The exact meaning of
equivalent depends on the \term{type} of \term{object} 
and is up to the programmer who defines a \term{method} for
\funref{make-load-form};
\seesection\LiteralsInCompiledFiles.
%This is the same type of equivalence discussed
%  in issue CONSTANT-COMPILABLE-TYPES.
 
The initialization form is a \term{form} that, when evaluated at \funref{load} time, 
should perform further initialization of the \term{object}.  
The value returned by the initialization form is ignored.
% Barrett: Flush "method" -- its the value returned by the function that's
%          being talked about.
%If the \funref{make-load-form} method
If \funref{make-load-form}
returns only one value, 
the initialization form is \nil, which has no effect.
If \param{object} appears as a constant in the initialization form,
at \funref{load} time it will be replaced by the equivalent \term{object} 
constructed by the creation form;
this is how the further initialization gains access to the \term{object}.
 
% Both the creation and initialization forms can contain references to \term{objects} 
% of user-defined \term{types} (defined precisely below).
%% Rewritten per item 1.4 of MAKE-LOAD-FORM-CONFUSION:
Both the \param{creation-form} and the \param{initialization-form} may contain references
to any \term{externalizable object}.
However, there must not be any circular dependencies in creation forms.
An example of a circular dependency is when the creation form for the
object \f{X} contains a reference to the object \f{Y},
and the creation form for the object \f{Y} contains a reference to the object \f{X}.  
%A simpler
%  example would be when the creation form for the object X contains
%  a reference to X itself.  
Initialization forms are not subject to any restriction against circular dependencies, 
which is the reason that initialization forms exist; 
see the example of circular data structures below.
 
%   The creation form for an \term{object} is always evaluated before the
%   initialization form for that \term{object}.  
% When either the creation form or
%   the initialization form references other \term{objects} 
% of user-defined \term{types}
%   that have not been referenced earlier in the 
% \funref{compile-file}, the
%   compiler collects all of the creation and initialization forms.  Each
%   initialization form is evaluated as soon as possible after its
%   creation form, as determined by data flow.  If the initialization form
%   for an \term{object} does not reference any other 
% \term{objects} of user-defined
%   \term{types} 
% that have not been referenced earlier in the \funref{compile-file}, the
%   initialization form is evaluated immediately after the creation form.
%   If a creation or initialization form \f{F} references other 
% \term{objects} of
%   user-defined \term{types} that have not been referenced earlier in the
%   \funref{compile-file}, 
% the creation forms for those other \term{objects} are evaluated
%   before \f{F}, and the initialization forms for those other 
% \term{objects} are
%   also evaluated before \f{F} whenever they do not depend on the 
% \term{object}
%   created or initialized by 
% \f{F}.  Where the above rules do not uniquely
%   determine an order of evaluation, which of the possible orders of
%   evaluation is chosen is \term{implementation-dependent}.
%   \idxtext{order of evaluation}\idxtext{evaluation order}
%% Rewritten per item 1.5 of MAKE-LOAD-FORM-CONFUSION:

The creation form for an \term{object} is always \term{evaluated} before the
initialization form for that \term{object}.  When either the creation form or
the initialization form references other \term{objects} that have not been
referenced earlier in the \term{file} being \term{compiled}, the \term{compiler} ensures
that all of the referenced \term{objects} have been created before \term{evaluating}
the referencing \term{form}.  When the referenced \term{object} is of a \term{type} which
%\funref{compile-file}
the \term{file compiler} processes using \funref{make-load-form},
this involves \term{evaluating}
the creation form returned for it.  (This is the reason for the
prohibition against circular references among creation forms).

Each initialization form is \term{evaluated} as soon as possible after its
associated creation form, as determined by data flow.  If the
initialization form for an \term{object} does not reference any other \term{objects}
not referenced earlier in the \term{file} and processed by 
%\funref{compile-file} 
the \term{file compiler}
using
\funref{make-load-form}, the initialization form is evaluated immediately after
the creation form.  If a creation or initialization form $F$ does contain
references to such \term{objects}, the creation forms for those other objects
are evaluated before $F$, and the initialization forms for those other
\term{objects} are also evaluated before $F$ whenever they do not depend on the
\term{object} created or initialized by $F$.  Where these rules do not uniquely
determine an order of \term{evaluation} between two creation/initialization
forms, the order of \term{evaluation} is unspecified.
 
  While these creation and initialization forms are being evaluated, the
  \term{objects} are possibly in an uninitialized state, 
analogous to the state
  of an \term{object} 
between the time it has been created by \funref{allocate-instance}
  and it has been processed fully by 
\funref{initialize-instance}.  Programmers
  writing \term{methods} for 
\funref{make-load-form} must take care in manipulating
  \term{objects} not to depend on 
\term{slots} that have not yet been initialized.
 
  It is \term{implementation-dependent}
whether \funref{load} calls \funref{eval} on the 
\term{forms} or does some
  other operation that has an equivalent effect.  For example, the
  \term{forms} might be translated into different but equivalent 
\term{forms} and
  then evaluated, they might be compiled and the resulting functions
  called by \funref{load}, 
or they might be interpreted by a special-purpose
function different from \funref{eval}.  
All that is required is that the
  effect be equivalent to evaluating the \term{forms}.

% Barrett: Add method descriptions, per make-load-form-confusion.
The \term{method} \term{specialized} on \typeref{class} returns a creation
\term{form} using the \term{name} of the \term{class} if the \term{class} has
a \term{proper name} in \param{environment}, signaling an error \oftype{error}
if it does not have a \term{proper name}.  \term{Evaluation} of the creation
\term{form} uses the \term{name} to find the \term{class} with that
\term{name}, as if by \term{calling} \funref{find-class}.  If a \term{class}
with that \term{name} has not been defined, then a \term{class} may be
computed in an \term{implementation-defined} manner.  If a \term{class}
cannot be returned as the result of \term{evaluating} the creation
\term{form}, then an error \oftype{error} is signaled.

%!!! Barrett: MAKE-LOAD-FORM-CONFUSION sayed "... may define additional
%             \term{conforming methods} ...", but we never actually came up
%             with a definition for that term.  That whole issue is not
%             currently addressed by the standard, so I'm not going to try
%             to fix it just here.
Both \term{conforming implementations} and \term{conforming programs} may
further \term{specialize} \funref{make-load-form}.
 
\label Examples::

\code
 (defclass obj ()
    ((x :initarg :x :reader obj-x)
     (y :initarg :y :reader obj-y)
     (dist :accessor obj-dist)))
\EV #<STANDARD-CLASS OBJ 250020030>
 (defmethod shared-initialize :after ((self obj) slot-names &rest keys)
   (declare (ignore slot-names keys))
   (unless (slot-boundp self 'dist)
     (setf (obj-dist self)
           (sqrt (+ (expt (obj-x self) 2) (expt (obj-y self) 2))))))
\EV #<STANDARD-METHOD SHARED-INITIALIZE (:AFTER) (OBJ T) 26266714>
 (defmethod make-load-form ((self obj) &optional environment)
   (declare (ignore environment))
   ;; Note that this definition only works because X and Y do not
   ;; contain information which refers back to the object itself.
   ;; For a more general solution to this problem, see revised example below.
   `(make-instance ',(class-of self)
                   :x ',(obj-x self) :y ',(obj-y self)))
\EV #<STANDARD-METHOD MAKE-LOAD-FORM (OBJ) 26267532>
 (setq obj1 (make-instance 'obj :x 3.0 :y 4.0)) \EV #<OBJ 26274136>
 (obj-dist obj1) \EV 5.0
 (make-load-form obj1) \EV (MAKE-INSTANCE 'OBJ :X '3.0 :Y '4.0)
\endcode

In the above example, an equivalent \term{instance} of \f{obj} is
reconstructed by using the values of two of its \term{slots}.  
The value of the third \term{slot} is derived from those two values.

\medbreak
Another way to write the \funref{make-load-form} \term{method}
in that example is to use \funref{make-load-form-saving-slots}.
The code it generates might yield a slightly different result 
from the \funref{make-load-form} \term{method} shown above,
but the operational effect will be the same.  For example:
 
\smallbreak
\code
 ;; Redefine method defined above.
 (defmethod make-load-form ((self obj) &optional environment)
    (make-load-form-saving-slots self
                                 :slot-names '(x y)
                                 :environment environment))
\EV #<STANDARD-METHOD MAKE-LOAD-FORM (OBJ) 42755655>
 ;; Try MAKE-LOAD-FORM on object created above.
 (make-load-form obj1)
\EV (ALLOCATE-INSTANCE '#<STANDARD-CLASS OBJ 250020030>),
    (PROGN
      (SETF (SLOT-VALUE '#<OBJ 26274136> 'X) '3.0)
      (SETF (SLOT-VALUE '#<OBJ 26274136> 'Y) '4.0)
      (INITIALIZE-INSTANCE '#<OBJ 26274136>))
\endcode

\medbreak
In the following example, \term{instances} of \f{my-frob} are ``interned'' 
in some way.  An equivalent \term{instance} is reconstructed by using the 
value of the name slot as a key for searching existing \term{objects}.
In this case the programmer has chosen to create a new \term{object} 
if no existing \term{object} is found; alternatively an error could 
have been signaled in that case.

\smallbreak
\code
 (defclass my-frob ()
    ((name :initarg :name :reader my-name)))
 (defmethod make-load-form ((self my-frob) &optional environment)
   (declare (ignore environment))
   `(find-my-frob ',(my-name self) :if-does-not-exist :create))
\endcode

\medbreak
In the following example, the data structure to be dumped is circular, 
because each parent has a list of its children and each child has a reference
back to its parent.  If \funref{make-load-form} is called on one   
\term{object} in such a structure,  the creation form creates an equivalent 
\term{object} and fills in the children slot, which forces creation of equivalent
\term{objects} for all of its children, grandchildren, etc.  At this point
none of the parent \term{slots} have been filled in.  
The initialization form fills in the parent \term{slot}, which forces creation 
of an equivalent \term{object} for the parent if it was not already created.
Thus the entire tree is recreated at \funref{load} time.  
At compile time, \funref{make-load-form} is called once for each \term{object} 
in the tree.  
All of the creation forms are evaluated,
in \term{implementation-dependent} order,
and then all of the initialization forms are evaluated, 
also in \term{implementation-dependent} order.
 
\smallbreak
\code
 (defclass tree-with-parent () ((parent :accessor tree-parent)
                                (children :initarg :children)))
 (defmethod make-load-form ((x tree-with-parent) &optional environment)
   (declare (ignore environment))
   (values
     ;; creation form
     `(make-instance ',(class-of x) :children ',(slot-value x 'children))
     ;; initialization form
     `(setf (tree-parent ',x) ',(slot-value x 'parent))))
\endcode

\medbreak
In the following example, the data structure to be dumped has no special
properties and an equivalent structure can be reconstructed
simply by reconstructing the \term{slots}' contents.
 
\smallbreak
\code
 (defstruct my-struct a b c)
 (defmethod make-load-form ((s my-struct) &optional environment)
    (make-load-form-saving-slots s :environment environment))
\endcode

\label Affected By:\None.
 
\label Exceptional Situations::

% \funref{make-load-form} of an \term{object}
% of \term{metaclass} \typeref{standard-class} or \typeref{structure-class}
% for which no user-defined \term{method} is applicable
% signals an error \oftype{error}.  
% It is valid to implement this either by defining default
% methods on \typeref{standard-object} and \typeref{structure-object}
% that signal an error \oftype{error}
% or by having no \term{applicable method} for those \term{classes}.
%% Rewritten per item 1.3 of MAKE-LOAD-FORM-CONFUSION:
The \term{methods} \term{specialized} on 
     \typeref{standard-object},
     \typeref{structure-object},
 and \typeref{condition}
all signal an error \oftype{error}.
 
% Barrett: Add per make-load-form-confusion.
It is \term{implementation-dependent} whether \term{calling}
\funref{make-load-form} on a \term{generalized instance} of a
\term{system class} signals an error or returns creation and
initialization \term{forms}.

\label See Also::

\funref{compile-file}, 
\funref{make-load-form-saving-slots},
{\secref\CallingMakeLoadForm}
{\secref\Evaluation},
{\secref\Compilation}
 
\label Notes::
 
The \term{file compiler}
%\funref{compile-file}
calls \funref{make-load-form} in specific circumstances
detailed in \secref\CallingMakeLoadForm.
%% Removed per item 1.2 of MAKE-LOAD-FORM-CONFUSION:
% It is valid for user programs to 
% call \funref{make-load-form} in other circumstances, providing \param{object}'s
% \term{metaclass} is neither \typeref{built-in-class} 
% nor a \term{subclass} of \typeref{built-in-class}.

% Barrett: Add per make-load-form-confusion.
Some \term{implementations} may provide facilities for defining new
\term{subclasses} of \term{classes} which are specified as
\term{system classes}.  (Some likely candidates include
\typeref{generic-function}, \typeref{method}, and \typeref{stream}).  Such
\term{implementations} should document how the \term{file compiler} processes
\term{instances} of such \term{classes} when encountered as
\term{literal objects}, and should document any relevant \term{methods}
for \funref{make-load-form}.

\endissue{LOAD-OBJECTS:MAKE-LOAD-FORM}
\endissue{MAKE-LOAD-FORM-CONFUSION:REWRITE}
\endcom

%%% ========== MAKE-LOAD-FORM-SAVING-SLOTS
\begincom{make-load-form-saving-slots}\ftype{Function}

\issue{MAKE-LOAD-FORM-CONFUSION:REWRITE}
% Barrett: Issue reference for LOAD-OBJECTS was missing.
\issue{LOAD-OBJECTS:MAKE-LOAD-FORM}

\label Syntax::

\DefunWithValuesNewline make-load-form-saving-slots 
        	        {object {\key} slot-names environment}
        	        {creation-form, initialization-form}

\label Arguments and Values::

\param{object}---an \term{object}.

\param{slot-names}---a \term{list}.

\param{environment}---an \term{environment object}.

\param{creation-form}---a \term{form}.

\param{initialization-form}---a \term{form}.

\label Description::

\issue{MAKE-LOAD-FORM-SAVING-SLOTS:NO-INITFORMS}
% Returns \term{forms} that, when \term{evaluated}, will construct an
% \term{object} equivalent to \param{object} using \funref{make-instance}
% and \macref{setf} of \funref{slot-value} for \term{slots} with values, 
% or \funref{slot-makunbound} for \term{slots} without values, 
% or using other \term{functions} of equivalent effect.
% \funref{make-load-form-saving-slots} works for any \term{object}
% of \term{metaclass} \typeref{standard-class} or \typeref{structure-class}.  
Returns \term{forms} that, when \term{evaluated}, will construct an
\term{object} equivalent to \param{object}, without \term{executing}
\term{initialization forms}.  The \term{slots} in the new \term{object}
that correspond to initialized \term{slots} in \param{object} are
initialized using the values from \param{object}.  Uninitialized \term{slots}
in \param{object} are not initialized in the new \term{object}.
\funref{make-load-form-saving-slots} works for any \term{instance} of
\typeref{standard-object} or \typeref{structure-object}.
\endissue{MAKE-LOAD-FORM-SAVING-SLOTS:NO-INITFORMS}

%Barmar: What does it return if object is a structure?
%  MAKE-INSTANCE can't be used on structure classes, can it?
%KMP: I think that's what "functions of equivalent effect" is supposed to answer.

\param{Slot-names} is a \term{list} of the names of the 
\term{slots} to preserve. If \param{slot-names} is not
supplied, its value is all of the \term{local slots}.  

\funref{make-load-form-saving-slots} returns two values,
thus it can deal with circular structures.
Whether the result is useful in an application depends on
whether the \param{object}'s \term{type} and slot contents
fully capture the application's idea of the \param{object}'s state.

\param{Environment} is the environment in which the forms will be processed.

\label Examples:\None.

\label Side Effects:\None.

\label Affected By:\None.

\label Exceptional Situations:\None.

\label See Also::

\funref{make-load-form},
\funref{make-instance},
\macref{setf},
\funref{slot-value},
\funref{slot-makunbound}

\label Notes::

\funref{make-load-form-saving-slots} can be useful in user-written
\funref{make-load-form} methods.

% Barrett: Remember to put in end marker.
\endissue{LOAD-OBJECTS:MAKE-LOAD-FORM}
\endissue{MAKE-LOAD-FORM-CONFUSION:REWRITE}

\issue{MAKE-LOAD-FORM-SAVING-SLOTS:NO-INITFORMS}
When the \term{object} is an \term{instance} of \typeref{standard-object},
\funref{make-load-form-saving-slots} could return a creation form that
\term{calls} \funref{allocate-instance} and an initialization form that
contains \term{calls} to \macref{setf} of \funref{slot-value} and
\funref{slot-makunbound}, though other \term{functions} of similar effect
might actually be used.
\endissue{MAKE-LOAD-FORM-SAVING-SLOTS:NO-INITFORMS}

\endcom



%%% ========== WITH-ACCESSORS
\begincom{with-accessors}\ftype{Macro}
 
\issue{DECLS-AND-DOC}

\label Syntax::  
 
\DefmacWithValuesNewline with-accessors
		  {{\paren{\starparam{slot-entry}}} 
		   instance-form
 		   \starparam{declaration} \starparam{form}}
		  {\starparam{result}}

\auxbnf{slot-entry}{\paren{variable-name accessor-name}}
 
\label Arguments and Values::
 
\param{variable-name}---a \term{variable name}; \noeval.

\param{accessor-name}---a \term{function name}; \noeval.
 
\param{instance-form}---a \term{form}; \eval.
                                                                  
\param{declaration}---a \misc{declare} \term{expression}; \noeval.

\param{forms}---an \term{implicit progn}.
 
\param{results}---the \term{values} returned by the \param{forms}.
 
\label Description::
 
Creates a lexical environment in which
the slots specified by
\param{slot-entry} are lexically available through their accessors as if
they were variables.  The macro \macref{with-accessors} invokes the
appropriate accessors to \param{access} the \term{slots} specified
by \param{slot-entry}.  Both \macref{setf}
and \specref{setq} can be used to set the value of the \term{slot}.
 
\label Examples::
 
\code
 (defclass thing ()
           ((x :initarg :x :accessor thing-x)
            (y :initarg :y :accessor thing-y)))
\EV #<STANDARD-CLASS THING 250020173>
 (defmethod (setf thing-x) :before (new-x (thing thing))
   (format t "~&Changing X from ~D to ~D in ~S.~%"
           (thing-x thing) new-x thing))
 (setq thing1 (make-instance 'thing :x 1 :y 2)) \EV #<THING 43135676>
 (setq thing2 (make-instance 'thing :x 7 :y 8)) \EV #<THING 43147374>
 (with-accessors ((x1 thing-x) (y1 thing-y))
                 thing1
   (with-accessors ((x2 thing-x) (y2 thing-y))
                   thing2
     (list (list x1 (thing-x thing1) y1 (thing-y thing1)
                 x2 (thing-x thing2) y2 (thing-y thing2))
           (setq x1 (+ y1 x2))
           (list x1 (thing-x thing1) y1 (thing-y thing1)
                 x2 (thing-x thing2) y2 (thing-y thing2))
           (setf (thing-x thing2) (list x1))
           (list x1 (thing-x thing1) y1 (thing-y thing1)
                 x2 (thing-x thing2) y2 (thing-y thing2)))))
\OUT Changing X from 1 to 9 in #<THING 43135676>.
\OUT Changing X from 7 to (9) in #<THING 43147374>.
\EV ((1 1 2 2 7 7 8 8)
     9
     (9 9 2 2 7 7 8 8) 
     (9)
     (9 9 2 2 (9) (9) 8 8))
\endcode
 
\label Affected By::

\macref{defclass}
 
\label Exceptional Situations::

The consequences are undefined if any \param{accessor-name} is not the name
of an accessor for the \param{instance}.

\label See Also::
 
\macref{with-slots},
\specref{symbol-macrolet}
 
\label Notes::
 
A \macref{with-accessors} expression of the form:
 
$$\openup1\jot\vbox{\settabs\+\cr
\+{\tt (with-accessors} (${\hbox{\i{slot-entry}}}\sub 1%
\ldots{\hbox{\i{slot-entry}}}\sub n$) \i{instance-form}
${\hbox{\i{form}}}\sub 1%
\ldots{\hbox{\i{form}}}\sub k$)\cr}$$
 
\noindent expands into the equivalent of
 
$$\openup1\jot\vbox{\settabs\+\cr
\+{\tt (}&{\tt let ((}$in$ \i{instance-form}{\tt ))}\cr
\+&{\tt (symbol-macrolet (}${\hbox{\i{Q}}}\sub 1\ldots%
{\hbox{\i{Q}}}\sub n${\tt )} ${\hbox{\i{form}}}\sub 1%
\ldots{\hbox{\i{form}}}\sub k${\tt ))}\cr}$$
 
\noindent where ${\hbox{\i{Q}}}\sub i$ is 
 
$${\vbox{\hbox{{\tt (}${\hbox{\i{variable-name}}}\sub i$ () %
{\tt (${\hbox{\i{accessor-name}}}\sub{i}\ in$))}}}}$$

\endissue{DECLS-AND-DOC}

\endcom

%%% ========== WITH-SLOTS
\begincom{with-slots}\ftype{Macro}
 
\issue{DECLS-AND-DOC}

\label Syntax::  
 
\DefmacWithValuesNewline with-slots
		  	 {\paren{\starparam{slot-entry}}
		          instance-form 
                          \starparam{declaration} \starparam{form}}
			 {\starparam{result}}
 
\auxbnf{slot-entry}{slot-name | \paren{variable-name slot-name}}
                     
\label Arguments and Values::

\param{slot-name}---a \term{slot} \term{name}; \noeval.
 
\param{variable-name}---a \term{variable name}; \noeval.

\param{instance-form}---a \term{form}; evaluted to produce \param{instance}.
 
\param{instance}---an \term{object}.

\param{declaration}---a \misc{declare} \term{expression}; \noeval.

\param{forms}---an \term{implicit progn}.
 
\param{results}---the \term{values} returned by the \param{forms}.
 
\label Description::
 
The macro \macref{with-slots} \term{establishes} a
%lexical context
\term{lexical environment}
for referring to the \term{slots} in the \param{instance} 
named by the given \param{slot-names} 
as though they were \term{variables}.  Within such a context
the value of the \term{slot} can be specified by using its slot name, as if
it were a lexically bound variable.  Both \macref{setf} and \specref{setq}
can be used to set the value of the \term{slot}.
 
The macro \macref{with-slots} translates an appearance of the slot 
name as a \term{variable} into a call to \funref{slot-value}.
 
\label Examples::
                             
\code
 (defclass thing ()
           ((x :initarg :x :accessor thing-x)
            (y :initarg :y :accessor thing-y)))
\EV #<STANDARD-CLASS THING 250020173>
 (defmethod (setf thing-x) :before (new-x (thing thing))
   (format t "~&Changing X from ~D to ~D in ~S.~%"
           (thing-x thing) new-x thing))
 (setq thing (make-instance 'thing :x 0 :y 1)) \EV #<THING 62310540>
 (with-slots (x y) thing (incf x) (incf y)) \EV 2
 (values (thing-x thing) (thing-y thing)) \EV 1, 2
 (setq thing1 (make-instance 'thing :x 1 :y 2)) \EV #<THING 43135676>
 (setq thing2 (make-instance 'thing :x 7 :y 8)) \EV #<THING 43147374>
 (with-slots ((x1 x) (y1 y))
             thing1
   (with-slots ((x2 x) (y2 y))
               thing2
     (list (list x1 (thing-x thing1) y1 (thing-y thing1)
                 x2 (thing-x thing2) y2 (thing-y thing2))
           (setq x1 (+ y1 x2))
           (list x1 (thing-x thing1) y1 (thing-y thing1)
                 x2 (thing-x thing2) y2 (thing-y thing2))
           (setf (thing-x thing2) (list x1))
           (list x1 (thing-x thing1) y1 (thing-y thing1)
                 x2 (thing-x thing2) y2 (thing-y thing2)))))
\OUT Changing X from 7 to (9) in #<THING 43147374>.
\EV ((1 1 2 2 7 7 8 8)
     9
     (9 9 2 2 7 7 8 8) 
     (9)
     (9 9 2 2 (9) (9) 8 8))
\endcode
 
\label Affected By::

\macref{defclass}
 
\label Exceptional Situations::

The consequences are undefined if any \param{slot-name} is not the name
of a \term{slot} in the \param{instance}.
 
\label See Also::
 
\macref{with-accessors},
\funref{slot-value},
\specref{symbol-macrolet}
 
\label Notes::
 
A \macref{with-slots} expression of the form:
 
$$\openup1\jot\vbox{\settabs\+\cr
\+{\tt (with-slots} (${\hbox{\i{slot-entry}}}\sub 1%
\ldots{\hbox{\i{slot-entry}}}\sub n$) \i{instance-form}
${\hbox{\i{form}}}\sub 1%
\ldots{\hbox{\i{form}}}\sub k$)\cr}$$
 
\noindent expands into the equivalent of
 
$$\openup1\jot\vbox{\settabs\+\cr
\+{\tt (}&{\tt let ((}$in$ \i{instance-form}{\tt ))}\cr
\+&{\tt (symbol-macrolet (}${\hbox{\i{Q}}}\sub 1\ldots%
{\hbox{\i{Q}}}\sub n${\tt )} ${\hbox{\i{form}}}\sub 1%
\ldots{\hbox{\i{form}}}\sub k${\tt ))}\cr}$$
 
\noindent where ${\hbox{\i{Q}}}\sub i$ is 
 
$$\vbox{\hbox{{\tt (}${\hbox{\i{slot-entry}}}\sub i$ () %
{\tt (slot-value }$in$ '${\hbox{\i{slot-entry}}}\sub i${\tt ))}}}$$
 
\noindent if ${\hbox{\i{slot-entry}}}\sub i$ is a \term{symbol}
and is
 
$${\vbox{\hbox{{\tt (}${\hbox{\i{variable-name}}}\sub i$ () %
{\tt (slot-value }$in$ '${\hbox{\i{slot-name}}}\sub i${\tt ))}}}}$$
 
\noindent if ${\hbox{\i{slot-entry}}}\sub i$
is of the form 
 
$$\vbox{\hbox{{\tt (}${\hbox{\i{variable-name}}}\sub i$ %
${\hbox{\i{slot-name}}}\sub i${\tt )}}}$$


\endissue{DECLS-AND-DOC}

\endcom

%%% ========== DEFCLASS
\begincom{defclass}\ftype{Macro}
 
\label Syntax::
 
%%Syntax fixed per X3J13. -kmp 05-Oct-93
\DefmacWithValuesNewline defclass
        {\param{class-name} \paren{\star{\curly{\param{superclass-name}}}}
\paren{\star{\curly{\i{slot-specifier}}}}
 $\lbrack\!\lbrack\downarrow\!\hbox{\i{class-option}}\,\rbrack\!\rbrack$}
	{new-class}
 
\settabs\+\hskip\leftskip&\cr
\+&\cleartabs\i{slot-specifier}::$=$ &\i{slot-name} $\vert$ 
(\i{slot-name}  
$\lbrack\!\lbrack\downarrow\!\hbox{\i{slot-option}}\,\rbrack\!\rbrack$)\cr
\Vskip 1pc!
\+&\i{slot-name}::$=$ \term{symbol}\cr
\Vskip 1pc!
\+&\cleartabs\i{slot-option}::$=$ &\star{\curly{\kwd{reader} 
\param{reader-function-name}}} $\vert$ \cr
\+&&\star{\curly{\kwd{writer} \param{writer-function-name}}} $\vert$ \cr
\+&&\star{\curly{\kwd{accessor} \param{reader-function-name}}} $\vert$ \cr
\+&&\curly{\kwd{allocation} \param{allocation-type}} $\vert$ \cr
\+&&\star{\curly{\kwd{initarg} \param{initarg-name}}} $\vert$ \cr
\+&&\curly{\kwd{initform} \param{form}} $\vert$ \cr
\+&&\curly{\kwd{type} \param{type-specifier}} $\vert$ \cr
\+&&\curly{\kwd{documentation} \term{string}} \cr
\Vskip 1pc!                                        
\+&\i{function-name}::$=$ \curly{\term{symbol} 
$\vert$ {\tt (setf \term{symbol})}}\cr
\Vskip 1pc!
\+&\cleartabs\param{class-option}::$=$ &(\kwd{default-initargs} \f{.}
\param{initarg-list}) $\vert$ \cr
\+&&(\kwd{documentation} \term{string}) $\vert$ \cr
\+&&(\kwd{metaclass} \param{class-name}) \cr
\Vskip 1pc!
 
\label Arguments and Values::
 
%!!! Treatment of documentation?

\param{Class-name}---a \term{non-nil} \term{symbol}.
                                                    
\param{Superclass-name}--a \term{non-nil} \term{symbol}.
 
\param{Slot-name}--a \term{symbol}.
  The \param{slot-name} argument is 
%!!! Isn't there a more concise way to say this?
  a \term{symbol} that is syntactically valid for use as a variable name.
 
\param{Reader-function-name}---a \term{non-nil} \term{symbol}.
 \kwd{reader} can be supplied more than once for a given \term{slot}.
          
\param{Writer-function-name}---a \term{generic function} name.
 \kwd{writer} can be supplied more than once for a given \term{slot}.
 
\param{Reader-function-name}---a \term{non-nil} \term{symbol}.
 \kwd{accessor} can be supplied more than once for a given \term{slot}.
 
\param{Allocation-type}---(member \kwd{instance} \kwd{class}).
 \kwd{allocation} can be supplied once at most for a given \term{slot}.
 
\param{Initarg-name}---a \term{symbol}.
 \kwd{initarg} can be supplied more than once for a given \term{slot}.  
 
\param{Form}---a \term{form}. 
 \kwd{init-form} can be supplied once at most for a given \term{slot}.  
 
\param{Type-specifier}---a \term{type specifier}.
 \kwd{type} can be supplied once at most for a given \term{slot}. 
 
\param{Class-option}--- refers to the \term{class} as a whole or to all class \term{slots}.

\param{Initarg-list}---a \term{list} of alternating initialization argument
		        \term{names} and default initial value \term{forms}.
 \kwd{default-initargs} can be supplied at most once.

\param{Class-name}---a \term{non-nil} \term{symbol}.
 \kwd{metaclass} can be supplied once at most.
 
%The \i{class-name} argument is a \term{non-nil} symbol.
 
%Each \i{superclass-name} argument is a \term{non-nil} symbol.  
 
%Each \i{slot-specifier} argument is the name of the slot or a list
%consisting of the slot name followed by zero or more slot options.
 
 
%The \i{reader-function-name} argument is a \term{non-nil} symbol.
 
%The \i{writer-function-name} argument is a function specifier.
 
%The \i{initarg-name} argument is a symbol.
 
%!!! Ugh. Surely this should not be so. -kmp 24-Apr-91
\param{new-class}---the new \term{class} \term{object}.
 
\label Description::
                                        
The macro \macref{defclass} defines a new named \term{class}.  It returns
the new \term{class} \term{object} as its result.
 
The syntax of \macref{defclass} provides options for specifying
initialization arguments for \term{slots}, for specifying default
initialization values for \term{slots}, and for requesting that 
\term{methods} on specified \term{generic functions} be automatically 
generated for reading and writing the values of \term{slots}.  
No reader or writer functions are defined by default; 
their generation must be explicitly requested.  However,
\term{slots} can always be \term{accessed} using \funref{slot-value}.
 
Defining a new \term{class} also causes a \term{type} of the same name to be
defined.  The predicate \f{(typep \param{object} \param{class-name})} returns
true if the \term{class} of the given \param{object} is 
the \term{class} named by \param{class-name} itself or
a subclass of the class \param{class-name}.  A \term{class} \term{object} 
can be used as a \term{type specifier}.  
Thus \f{(typep \param{object} \param{class})} returns \term{true}
if the \term{class} of the \param{object} is 
\param{class} itself or a subclass of \param{class}.   
                            
The \param{class-name} argument specifies the \term{proper name} 
of the new \term{class}.  
%If a \term{class} 
%with the same proper name already exists and that
%\term{class} is an \term{instance} of \typeref{standard-class}, and if the
%\macref{defclass} form for the definition of the new \term{class}
%specifies a \term{class} of class \typeref{standard-class}, the
%definition of the existing \term{class} is replaced.
If a \term{class} with the same \term{proper name} already exists 
 and that \term{class} is an \term{instance} of \typeref{standard-class}, 
 and if the \macref{defclass} form for the definition of the new \term{class}
      specifies a \term{class} of \term{class} \typeref{standard-class},
the existing \term{class} is redefined,
and instances of it (and its \term{subclasses}) are updated 
 to the new definition at the time that they are next \term{accessed}.
For details, \seesection\ClassReDef.

% addressed in the packages chapter.  --sjl 5 Mar 92 
%\issue{LISP-SYMBOL-REDEFINITION:MAR89-X3J13}
%The consequences are undefined if a \term{symbol} in \thepackage{common-lisp}
%is used as the \param{class-name} argument.
%\endissue{LISP-SYMBOL-REDEFINITION:MAR89-X3J13}
 
Each \param{superclass-name} argument 
specifies a direct \term{superclass} of the new \term{class}.  
%% gray addition
If the \term{superclass} list is empty, then the \term{superclass}
defaults depending on the \term{metaclass}, 
with \typeref{standard-object} being the
default for \typeref{standard-class}.

The new \term{class} will
inherit \term{slots} and \term{methods} 
from each of its direct \term{superclasses}, from
their direct \term{superclasses}, and so on.  
For a discussion of how \term{slots} and \term{methods} are inherited,
\seesection\Inheritance.
 
%Each \i{slot-specifier} argument is the \term{name} of the slot or a list
%consisting of the slot name followed by zero or more slot options.
%The \i{slot-name} argument is a symbol that is syntactically valid
%for use as a Common Lisp variable name.  If there are any duplicate
%slot names, an error is signaled.
 
The following slot options are available:
 
\beginlist
 
\itemitem{\bull}
The \kwd{reader} slot option specifies that an \term{unqualified method} is
to be defined on the \term{generic function} named \param{reader-function-name}
to read the value of the given \term{slot}.
 
\itemitem{\bull} 
The \kwd{writer} slot option specifies that an \term{unqualified method} is
to be defined on the \term{generic function} named \param{writer-function-name}
to write the value of the \term{slot}.
 
\itemitem{\bull} 
The \kwd{accessor} slot option specifies that an \term{unqualified method}
is to be defined on the generic function named \param{reader-function-name}
to read the value of the given \term{slot}
and that an \term{unqualified method} is to be defined on the 
\term{generic function} named \f{(setf \param{reader-function-name})} to be
used with \macref{setf} to modify the value of the \term{slot}.
 
\itemitem{\bull} 
The \kwd{allocation} slot option is used to specify where storage is
to be allocated for the given \term{slot}.  Storage for a 
\term{slot} can be located
in each instance or in the \term{class} \term{object} itself.
The value of the \param{allocation-type} argument can be 
either the keyword \kwd{instance}
or the keyword \kwd{class}.    If the \kwd{allocation}
slot option is not specified, the effect is the same as specifying
\f{:allocation :instance}.
\beginlist 
\itemitem{--}                   
If \param{allocation-type} is \kwd{instance}, a \term{local slot} of
the name \param{slot-name} is allocated in each instance of the 
\term{class}.
 
\itemitem{--}                   
If \param{allocation-type} is \kwd{class}, a shared 
\term{slot} of the given                                      
name is allocated in the \term{class} \term{object} created by this \macref{defclass}
form.  The value of the \term{slot} is shared by all 
\term{instances} of the \term{class}.
If a class $C\sub1$ defines such a \term{shared slot}, any 
subclass $C\sub2$ of
$C\sub1$ will share this single \term{slot} unless the \macref{defclass} form
for $C\sub2$ specifies a \term{slot} of the same \term{name} or there is a
superclass of $C\sub2$ that precedes $C\sub1$ in the class precedence
list of $C\sub2$ and that defines a \term{slot} of the same \term{name}.
\endlist              
\itemitem{\bull} The \kwd{initform} slot option is used to provide a default
initial value form to be used in the initialization of the \term{slot}.  This
\term{form} is evaluated every time it is used to initialize the 
\term{slot}.  The
lexical environment in which this \term{form} is evaluated is the lexical
environment in which the \macref{defclass} form was evaluated.
Note that the lexical environment refers both to variables and to
functions.  For \term{local slots}, the dynamic environment is the dynamic
environment in which \funref{make-instance} is called; for shared
\term{slots}, the dynamic environment is the dynamic environment in which the
\macref{defclass} form was evaluated.  
\Seesection\ObjectCreationAndInit.
 
No implementation is permitted to extend the syntax of \macref{defclass}
to allow \f{(\param{slot-name} \param{form})} as an abbreviation for 
\f{(\param{slot-name} :initform \param{form})}.
\reviewer{Barmar: Can you extend this to mean something else?}
 
\itemitem{\bull}
The \kwd{initarg} slot option declares an initialization
argument named \param{initarg-name} and specifies that this
initialization argument initializes the given \term{slot}.  If the
initialization argument has a value in the call to
\funref{initialize-instance}, the value will be stored into the given \term{slot},
and the slot's \kwd{initform} slot option, if any, is not
evaluated.  If none of the initialization arguments specified for a
given \term{slot} has a value, the \term{slot} is initialized according to the
\kwd{initform} slot option, if specified.  
 
\itemitem{\bull}
The \kwd{type} slot option specifies that the contents of the
\term{slot} will always be of the specified data type.  It effectively
declares the result type of the reader generic function when applied
to an \term{object} of this \term{class}.  The consequences of attempting to store in a
\term{slot} a value that does not satisfy the type of the \term{slot} are undefined.
The \kwd{type} slot option is further discussed in 
\secref\SlotInheritance.
 
\itemitem{\bull}
The \kwd{documentation} slot option provides a \term{documentation string}
for the \term{slot}.  \kwd{documentation} can be supplied once at most 
for a given \term{slot}. \reviewer{Barmar: How is this retrieved?}
\endlist
 
Each class option is an option that refers to the \term{class} as a whole.
%!!! Barmar alleges that there are none of these:
%or to all class \term{slots}.
The following class options are available:
 
\beginlist
\itemitem{\bull}
The \kwd{default-initargs} class option is followed by a list of
alternating initialization argument \term{names} and default initial value
forms.  If any of these initialization arguments does not appear in
the initialization argument list supplied to \funref{make-instance}, the
corresponding default initial value form is evaluated, and the
initialization argument \term{name} and the \term{form}'s value are added to the end
of the initialization argument list before the instance is created;
\seesection\ObjectCreationAndInit.
The default initial value form is evaluated each time it is used.  The lexical
environment in which this \term{form} is evaluated is the lexical environment
in which the \macref{defclass} form was evaluated.  The dynamic
environment is the dynamic environment in which \funref{make-instance}
was called.  If an initialization argument \term{name} appears more than once
in a \kwd{default-initargs} class option, an error is signaled.
 
 
\itemitem{\bull} 
\issue{DOCUMENTATION-FUNCTION-BUGS:FIX}
The \kwd{documentation} class option causes a \term{documentation string} 
to be attached with the \term{class} \term{object},
and attached with kind \misc{type} to the \param{class-name}.
\kwd{documentation} can be supplied once at most.
\endissue{DOCUMENTATION-FUNCTION-BUGS:FIX}
 
\itemitem{\bull}
The \kwd{metaclass} class option is used to specify that
instances of the \term{class} being defined are to have a different metaclass
than the default provided by the system (\theclass{standard-class}).
 
\endlist
                             
Note the following rules of \macref{defclass} for \term{standard classes}:
 
\beginlist
 
\itemitem{\bull}
It is not required that the \term{superclasses} of a \term{class} be defined before
the \macref{defclass} form for that \term{class} is evaluated.
 
\itemitem{\bull}
All the \term{superclasses} of a \term{class} must be defined before 
an \term{instance} of the \term{class} can be made.
 
\itemitem{\bull}
A \term{class} must be defined before it can be used as a parameter
specializer in a \macref{defmethod} form.
 
\endlist
 
The \OS\ can be extended to cover situations where these rules are not
obeyed.
 
Some slot options are inherited by a \term{class} from its 
\term{superclasses}, and
some can be shadowed or altered by providing a local slot description.
No class options except \kwd{default-initargs} are inherited.  For a
detailed description of how \term{slots} and slot options are inherited, 
\seesection\SlotInheritance.
 
The options to \macref{defclass} can be extended.  It is required that
all implementations signal an error if they observe a class option or
a slot option that is not implemented locally.
 
It is valid to specify more than one reader, writer, accessor, or
initialization argument for a \term{slot}.  No other slot option can
appear
more than once in a single slot description, or an error is
signaled.
 
If no reader, writer, or accessor is specified for a \term{slot}, 
the \term{slot} can only be \term{accessed} by \thefunction{slot-value}.
 
%The macro \macref{defclass} defines a new named \term{class}.
 
%Defining a new \term{class} also causes a 
%\term{type} of the same \term{name} to be defined.
      
%The \kwd{reader} slot option causes an \term{unqualified method} 
%to be defined on the \term{generic function} named 
%\param{reader-function-name} to read the value of the given \term{slot}.
      
%The \kwd{writer} slot option causes an \term{unqualified method}
%to be defined on the \term{generic function} named 
%\param{writer-function-name} to write the value of the \term{slot}. 
      
%The \kwd{accessor} slot option causes an \term{unqualified method}
%to be defined on the \term{generic function} named 
%\param{reader-function-name} to read the value of the given \term{slot}
%and an \term{unqualified method} to be defined on the 
%\term{generic function} named {\tt (setf \param{reader-function-name})} to be
%used with \macref{setf} to modify the value of the \term{slot}.
      
%The \kwd{initarg} slot option declares an initialization argument
%named \param{initarg-name}.
 
\issue{COMPILE-FILE-HANDLING-OF-TOP-LEVEL-FORMS:CLARIFY}
% added qualification about top-level-ness  --sjl 5 Mar 92
If a \macref{defclass} \term{form} appears as a \term{top level form},
the \term{compiler} must make the \term{class} \term{name} be recognized as a
valid \term{type} \term{name} in subsequent declarations (as for \macref{deftype})
and be recognized as a valid \term{class} \term{name} for \macref{defmethod}
\term{parameter specializers} and for use as the \kwd{metaclass} option of a
subsequent \macref{defclass}.  The \term{compiler} must make 
%%!!! this doesn't look right. maybe "a class object"? -kmp 7-Jun-91
the \term{class} definition 
available to be returned by \funref{find-class} when its \param{environment}
\term{argument} is a value received as the \term{environment parameter} of a \term{macro}.
\endissue{COMPILE-FILE-HANDLING-OF-TOP-LEVEL-FORMS:CLARIFY}

\label Examples:\None.

\label Affected By:\None.
 
\label Exceptional Situations::
 
If there are any duplicate slot names, 
an error \oftype{program-error} is signaled.
                                                                          
If an initialization argument \term{name} appears more than once in 
\kwd{default-initargs} class option, 
an error \oftype{program-error} is signaled.
 
If any of the following slot options appears more than once in a
single slot description, an error \oftype{program-error}
is signaled: \kwd{allocation},
\kwd{initform}, \kwd{type}, \kwd{documentation}.
 
It is required that all implementations signal 
an error \oftype{program-error} if they observe a class option 
or a slot option that is not implemented locally.

%%gray's annotation
%>Other possible errors are an undefined metaclass or attempting to
%>redefine the name of an existing type with an incompatible metaclass
%>{e.g.  DEFCLASS for a name previously defined by DEFSTRUCT or DEFTYPE}.
 
 
\label See Also::
 
\funref{documentation},
\funref{initialize-instance},
\funref{make-instance},
\funref{slot-value},
{\secref\Classes},
{\secref\Inheritance},
{\secref\ClassReDef},
{\secref\DeterminingtheCPL},
{\secref\ObjectCreationAndInit}
 
\label Notes:\None.

\endcom

%%% ========== DEFGENERIC
\begincom{defgeneric}\ftype{Macro}
 
\issue{DECLS-AND-DOC}

\label Syntax::
 
\DefmacWithValuesNewline defgeneric
			 {function-name gf-lambda-list
	  		  \interleave{\down{option} | \stardown{method-description}}}
	 		 {new-generic}

\auxbnf{option}{\paren{\kwd{argument-precedence-order} \plusparam{parameter-name}}  |\CR
		\paren{\misc{declare} \plusparam{gf-declaration}}		    |\CR
	        \paren{\kwd{documentation} \param{gf-documentation}}		    |\CR
		\paren{\kwd{method-combination} 
		       \param{method-combination} 
		       \starparam{method-combination-argument}}			    |\CR
		\paren{\kwd{generic-function-class} \param{generic-function-class}} |\CR
		\paren{\kwd{method-class} \param{method-class}}}
\auxbnf{method-description}{\lparen\kwd{method}
			    \vtop{\hbox{\starparam{method-qualifier} 
					\param{specialized-lambda-list}}
			          \hbox{{\DeclsAndDoc} 
					\starparam{form}\rparen}}}
 
\label Arguments and Values::

\param{function-name}---a \term{function name}.

\param{generic-function-class}---a \term{non-nil} \term{symbol} naming a \term{class}.

\param{gf-declaration}---an \declref{optimize} \term{declaration specifier};
  other \term{declaration specifiers} are not permitted.
%% Barmar: What's the theory here?
%%         Are \declref{ignore} and \declref{dynamic-extent} in or out?}
%   \declref{special}, \declref{ftype}, \declref{function}, 
%   \declref{inline}, \declref{notinline}, and \declref{declaration} 
%  declarations are not permitted.

\param{gf-documentation}---a \term{string}; \noeval.

\param{gf-lambda-list}---a \term{generic function lambda list}.

\param{method-class}---a \term{non-nil} \term{symbol} naming a \term{class}.

\param{method-combination-argument}---an \term{object.}
%% Barmar: Redundant with info to follow.
%suitable as \term{arguments} to the \param{method-combination-name};
%the standard method combination type does not support
%any \term{arguments}.
%All types of method combination defined by the
%short form of \macref{define-method-combination} accept
%\kwd{order}, which defaults to \kwd{most-specific-first}.

\param{method-combination-name}---a \term{symbol} 
				  naming a \term{method combination} \term{type}.

\param{method-qualifiers},
\param{specialized-lambda-list},
\param{declarations},
\param{documentation},
\param{forms}---as per \macref{defmethod}.

\param{new-generic}---the \term{generic function} \term{object}.

\param{parameter-name}---a \term{symbol} that names a \term{required parameter} 
			 in the \param{lambda-list}.
  (If the \kwd{argument-precedence-order} option is specified,
   each \term{required parameter} in the \param{lambda-list}
   must be used exactly once as a \param{parameter-name}.)
 
\label Description::
 
The macro \macref{defgeneric} is used to define a \term{generic function}
or to specify options and declarations that pertain 
to a \term{generic function} as a whole.
 
%!!! Rewrite in terms of "fboundp" to avoid function calls?
If \param{function-name} is a 
\term{list} it must be of the form {\tt (setf \i{symbol})}.
If \f{(fboundp \param{function-name})} is \term{false}, a new
\term{generic function} is created.  
\issue{FUNCTION-NAME:LARGE}
If \f{(fdefinition \param{function-name})} is a \term{generic function}, that 
\endissue{FUNCTION-NAME:LARGE}
\term{generic function}
is modified.  If \param{function-name} names 
an \term{ordinary function},
a \term{macro}, or a \term{special operator}, 
an error is signaled.
 
The effect of the \macref{defgeneric} macro is as if the following three
steps were performed: first, 
\term{methods} defined by previous \macref{defgeneric} \term{forms} are removed; 
\reviewer{Barmar: Shouldn't this (second) be first?}
second, \funref{ensure-generic-function}
is called; and finally, \term{methods} specified by the current
\macref{defgeneric} \term{form} are added to the \term{generic function}. 
 
Each \param{method-description} defines a \term{method} on the \term{generic function}.
The \term{lambda list} of each \term{method} must be congruent with the 
\term{lambda list}
specified by the \param{gf-lambda-list} option.  
If no \term{method} descriptions are specified and a \term{generic function} of the same
name does not already exist, a \term{generic function} with no 
\term{methods} is created.
 
The \param{gf-lambda-list} argument of \macref{defgeneric} specifies the shape of
\term{lambda lists} for the \term{methods} on this \term{generic function}.
All \term{methods} on the resulting 
\term{generic function} must have
\term{lambda lists} that are congruent with this shape.  If a \macref{defgeneric}
form is evaluated and some 
\term{methods} for that \term{generic function}
have \term{lambda lists} that are not congruent with that given in
the \macref{defgeneric} form, an error is signaled.  For further details
on method congruence, \seesection\GFMethodLambdaListCongruency.
 
The \term{generic function} passes to the 
\term{method} all the argument values passed to
it, and only those; default values are not supported.
Note that optional and keyword arguments in method definitions, however,
can have default initial value forms and can use supplied-p parameters. 
 
The following options are provided.  
\issue{DEFGENERIC-DECLARE:ALLOW-MULTIPLE}
Except as otherwise noted, 
\endissue{DEFGENERIC-DECLARE:ALLOW-MULTIPLE}
a given option may occur only once.
 
\beginlist
 
\itemitem{\bull} 
The \kwd{argument-precedence-order} option is used to specify the
order in which the required arguments in a call to the \term{generic function}
are tested for specificity when selecting a particular
\term{method}. Each required argument, as specified in the \param{gf-lambda-list}
argument, must be included exactly once as a \param{parameter-name}
so that the full and unambiguous precedence order is
supplied.  If this condition is not met, an error is signaled.
\reviewer{Barmar: What is the default order?}
 
\itemitem{\bull}
The \misc{declare} option is used to specify declarations that pertain
to the \term{generic function}.

An \declref{optimize} \term{declaration specifier} is allowed.
It specifies whether method selection should be optimized for 
speed or space, but it has no effect on \term{methods}.
To control how a \term{method} is optimized, an \declref{optimize}
declaration must be placed directly in the \macref{defmethod} \term{form}
or method description.  The optimization qualities \misc{speed} and
\misc{space} are the only qualities this standard requires, but an
implementation can extend the \CLOS\ to recognize other qualities.  
A simple implementation that has only one method selection technique 
and ignores \declref{optimize} \term{declaration specifiers} is valid.
 
The \declref{special}, \declref{ftype}, \declref{function}, \declref{inline},
\declref{notinline}, and \declref{declaration} declarations are not permitted.
Individual implementations can extend the \misc{declare} option to
support additional declarations.
\editornote{KMP: Does ``additional'' mean including special, ftype, etc.?  
Or only other things that are not mentioned here?}
If an implementation notices a \term{declaration specifier} that it does
not support and that has not been proclaimed as a non-standard 
\term{declaration identifier} name in a \declref{declaration} \term{proclamation}, 
it should issue a warning. \editornote{KMP: The wording of this previous sentence,
particularly the word ``and'' suggests to me that you can `proclaim declaration'
of an unsupported declaration (e.g., ftype) in order to suppress the warning.
That seems wrong.  Perhaps it instead means to say ``does not support or 
is both undefined and not proclaimed declaration.''}

\issue{DEFGENERIC-DECLARE:ALLOW-MULTIPLE}
The \misc{declare} option may be specified more than once.
The effect is the same as if the lists of \term{declaration specifiers} 
had been appended together into a single list and specified as a 
single \misc{declare} option.
\endissue{DEFGENERIC-DECLARE:ALLOW-MULTIPLE}

\itemitem{\bull} 
The \kwd{documentation} argument is a \term{documentation string}
to be attached to the \term{generic function} \term{object}, 
and to be attached with kind \misc{function} to the \param{function-name}.
 
\itemitem{\bull} 
The \kwd{generic-function-class} option may be used to specify that
the \term{generic function} is to have a different \term{class} than
the default provided by the system (\theclass{standard-generic-function}).
The \param{class-name} argument is the name of a \term{class} that can be the
\term{class} of a \term{generic function}.  If \param{function-name} specifies
an existing \term{generic function} that has a different value for the
\kwd{generic-function-class} argument and the new generic function 
\term{class} is compatible with the old, \funref{change-class} is called 
to change the \term{class} of the \term{generic function}; 
otherwise an error is signaled.
 
\itemitem{\bull} 
The \kwd{method-class} option is used to specify that all \term{methods} on
this \term{generic function} are to have a different \term{class} from the 
default provided by the system (\theclass{standard-method}).
The \param{class-name} argument is the name of a \term{class} that is capable 
of being the \term{class} of a \term{method}.
\reviewer{Barmar: Is \funref{change-class} called on existing methods?}
 
\itemitem{\bull} 
The \kwd{method-combination} option is followed by a symbol that
names a type of method combination.  The arguments (if any) that
follow that symbol depend on the type of method combination.  Note
that the standard method combination type does not support any
arguments.  However, all types of method combination defined by the
short form of \macref{define-method-combination} accept an optional
argument named \param{order}, defaulting to \kwd{most-specific-first},
where a value of \kwd{most-specific-last} reverses
the order of the primary \term{methods} without affecting the order of the
auxiliary \term{methods}.
 
\endlist
 
The \param{method-description} arguments define \term{methods} that will
be associated with the \term{generic function}.  The \param{method-qualifier}
and \param{specialized-lambda-list} arguments in a method description
are the same as for \macref{defmethod}.
 
The \param{form} arguments specify the method body.  The body of the
\term{method} is enclosed in an \term{implicit block}.
If \param{function-name} is a \term{symbol}, this block bears the same name as
the \term{generic function}.  If \param{function-name} is a 
\term{list} of the
form {\tt (setf \param{symbol})}, the name of the block is \param{symbol}.  
 
Implementations can extend \macref{defgeneric} to include other options.
It is required that an implementation signal an error if
it observes an option that is not implemented locally.

\issue{COMPILE-FILE-HANDLING-OF-TOP-LEVEL-FORMS:CLARIFY}
\macref{defgeneric} is not required to perform any compile-time side effects.
In particular, the \term{methods} are not installed for invocation during 
compilation.  An \term{implementation} may choose to store information about
the \term{generic function} for the purposes of compile-time error-checking
(such as checking the number of arguments on calls, or noting that a definition
 for the function name has been seen).
\endissue{COMPILE-FILE-HANDLING-OF-TOP-LEVEL-FORMS:CLARIFY}

\label Examples::
 
 
\label Affected By:\None.
 
\label Exceptional Situations::
 
If \param{function-name} names an \term{ordinary function}, a \term{macro},
or a \term{special operator}, an error \oftype{program-error} is signaled.
 
Each required argument, as specified in the \param{gf-lambda-list}
argument, must be included exactly once as a \param{parameter-name},
or an error \oftype{program-error} is signaled.
 
The \term{lambda list} of each \term{method} specified by a 
\param{method-description} must be congruent with the \term{lambda list} specified
by the \param{gf-lambda-list} option, or
an error \oftype{error} is signaled.
 
If a \macref{defgeneric} form is evaluated and some \term{methods} for
that \term{generic function} have \term{lambda lists} that are not congruent with
that given in the \macref{defgeneric} form, 
an error \oftype{error} is signaled.
 
A given \param{option} may occur only once,
or an error \oftype{program-error} is signaled.
 
\reviewer{Barmar: This says that an error is signaled if you specify the same generic
    function class as it already has!}
If \param{function-name} specifies an existing \term{generic function} 
that has a different value for the \kwd{generic-function-class} 
argument and the new generic function \term{class} is compatible with the
old, \funref{change-class} is called to change the \term{class} of 
the \term{generic function}; otherwise an error \oftype{error} is signaled.
 
Implementations can extend \macref{defgeneric} to include other options.
It is required that an implementation 
signal an error \oftype{program-error} if
it observes an option that is not implemented locally.
 
\label See Also::
 
\macref{defmethod},
\funref{documentation},
\funref{ensure-generic-function},
\issue{GENERIC-FUNCTION-POORLY-DESIGNED:DELETE}
%\macref{generic-function},
\typeref{generic-function},
\endissue{GENERIC-FUNCTION-POORLY-DESIGNED:DELETE}
{\secref\GFMethodLambdaListCongruency}
 
\label Notes:\None.
 
\endissue{DECLS-AND-DOC}

\endcom

%%% ========== DEFMETHOD
\begincom{defmethod}\ftype{Macro}
 
\issue{DECLS-AND-DOC}

\label Syntax::
 
%!!! Barmar: \macref{defun} and \macref{defmacro} don't bother to show this detail
%      for the specialized lambda list, so why does \macref{defmethod}?

\DefmacWithValuesNewline {defmethod} 
			 {\vtop{\hbox{\i{function-name}
				      \star{\curly{\i{method-qualifier}}}
				      \i{specialized-lambda-list}}
				\hbox{{\DeclsAndDoc} \starparam{form}}}}
			 {new-method}

\Vskip1pc!\null
\i{function-name}::$=$ \curly{\term{symbol} 
$\vert$ {\tt (setf \term{symbol})}}
\Vskip1pc!\null
\i{method-qualifier}::$=$ \term{non-list}
\Vskip1pc!\null
\settabs\+\hskip\leftskip&\cr
\+&\i{specialized-lambda-list}::$=$ 
(&\star{\curly{\param{var}  $\vert$ {\rm (}{\param{var}
\i{parameter-specializer-name}}{\rm )}}}  \cr
\+&&\ttbrac{{\opt} 
\star{\curly{\param{var} $\vert$ {\rm (}var 
\ttbrac{\param{initform} {\brac{\param{supplied-p-parameter}}} }{\rm )}}}}  \cr
\+&&\ttbrac{{\tt\&rest} \param{var}} \cr
\+&&{\tt \lbracket}{\key{}}&\star{\curly{\param{var}  $\vert$
{\rm (}\curly{\param{var} $\vert$ {\rm (}\term{keyword}\param{var}{\rm )}}
\ttbrac{\param{initform} \brac{\param{supplied-p-parameter}} }{\rm )}}}\cr
\+&&&\brac{\keyref{allow-other-keys}} {\tt \rbracket} \cr
\+&&\ttbrac{{\tt\&aux} 
\star{\curly{\param{var} $\vert$ {\rm (}\param{var} 
\brac{\param{initform}} {\rm )}}}} {\rm )} \cr
\Vskip1pc!\null
\+&\i{parameter-specializer-name}::$=$ \term{symbol} 
$\vert$ {\rm (}{\tt eql} \param{eql-specializer-form}{\rm )}\cr
\Vskip 1pc!
 
\label Arguments and Values::
 
\param{declaration}---a \misc{declare} \term{expression}; \noeval.

\param{documentation}---a \term{string}; \noeval.

\param{var}---a \term{variable} \term{name}.

\param{eql-specializer-form}---a \term{form}.
 
\param{Form}---a \term{form}.

\param{Initform}---a \term{form}.

\param{Supplied-p-parameter}---variable name. 

\param{new-method}---the new \term{method} \term{object}.
 
\label Description::
 
The macro \macref{defmethod} defines a \term{method} on a 
\term{generic function}.  

%!!! Rewrite to use "fbound" ?
If {\tt (fboundp \i{function-name})} is \nil, a 
\term{generic function} is created with default values for 
the argument precedence order
(each argument is more specific than the arguments to its right
in the argument list),
for the generic function class (\theclass{standard-generic-function}),
for the method class (\theclass{standard-method}),
and for the method combination type (the standard method combination type).
The \term{lambda list} of the \term{generic function} is
congruent with the \term{lambda list} of the 
\term{method} being defined; if the
\macref{defmethod} form mentions keyword arguments, the \term{lambda list} of
the \term{generic function} 
will mention {\tt &key} (but no keyword
arguments).  If \i{function-name} names 
an \term{ordinary function},
a \term{macro}, or a \term{special operator}, 
an error is signaled.
 
If a \term{generic function} is currently named by {\it function-name},
the \term{lambda list} of the
\term{method} must be congruent with the \term{lambda list} of the 
\term{generic function}.
If this condition does not hold, an error is signaled.  
For a definition of congruence in this context, \seesection\GFMethodLambdaListCongruency.
 
%% gray says redundant with syntax spec.
%If \i{function-name} is a \term{list},
%it must be of the form {\tt (setf \i{symbol})}.  
%\i{Function-name} names the \term{generic function}
%on which the \term{method} is defined.

Each \i{method-qualifier} argument is an \term{object} that is used by
method combination to identify the given \term{method}.  
The method combination type might further
restrict what a method \term{qualifier} can be.
The standard method combination type allows for \term{unqualified methods} and
\term{methods} whose sole
\term{qualifier} is one of the keywords \kwd{before}, \kwd{after}, or \kwd{around}.
 
The \i{specialized-lambda-list} argument is like an ordinary
\term{lambda list} except that the \term{names} of required parameters can
be replaced by specialized parameters.  A specialized parameter is a
list of the form 
\f{(\param{var} \i{parameter-specializer-name})}.
Only required parameters can be
specialized.  If \i{parameter-specializer-name} is a \term{symbol} it names a
\term{class}; if it is a \term{list},
it is of the form \f{(eql \param{eql-specializer-form})}.  The parameter
specializer name \f{(eql \param{eql-specializer-form})} indicates
that the corresponding argument must be \funref{eql} to the \term{object} that
is the value of \param{eql-specializer-form} for the \term{method} to be applicable.  
%%gray/moon addition
The \param{eql-specializer-form} is evaluated at the time
that the expansion of the \macref{defmethod} macro is evaluated.  
%%
If no \term{parameter specializer name} is specified for a given
required parameter, the \term{parameter specializer} defaults to 
\theclass{t}.
For further discussion, \seesection\IntroToMethods.
 
The \param{form} arguments specify the method body.
The body of the \term{method} is enclosed in an \term{implicit block}.  If
\i{function-name} is a \term{symbol}, 
this block bears the same \term{name} as the \term{generic function}.  
If \i{function-name} is a \term{list} of the form 
{\tt (setf \i{symbol})}, the \term{name} of the block is \i{symbol}.  
 
The \term{class} of the \term{method} \term{object} that is created is that given by the 
method class option of the \term{generic function} 
on which the \term{method} is defined.
 
If the \term{generic function} already has a \term{method} that agrees with the
\term{method} being defined on \term{parameter specializers} and \term{qualifiers},
\macref{defmethod} replaces the existing \term{method} with the one now being
defined.
For a definition of agreement in this context.
\seesection\SpecializerQualifierAgreement.
 
The \term{parameter specializers} are derived from 
the \term{parameter specializer names} as described in
\secref\IntroToMethods.

The expansion of the \macref{defmethod} macro ``refers to'' each
specialized parameter (see the description of \declref{ignore} 
within the description of \misc{declare}).
This includes parameters that
have an explicit \term{parameter specializer name} of \t.  This means
that a compiler warning does not occur if the body of the \term{method} does
not refer to a specialized parameter, while a warning might occur
if the body of the \term{method} does not refer to an unspecialized parameter.
For this reason, a parameter that specializes on \t\ is not quite synonymous
with an unspecialized parameter in this context.
 
\issue{DEFMETHOD-DECLARATION-SCOPE:CORRESPONDS-TO-BINDINGS}
Declarations at the head of the method body that apply to the 
method's \term{lambda variables} are treated as \term{bound declarations}
whose \term{scope} is the same as the corresponding \term{bindings}.

Declarations at the head of the method body that apply to the 
functional bindings of \funref{call-next-method} or \funref{next-method-p}
apply to references to those functions within the method body \param{forms}.
Any outer \term{bindings} of the \term{function names} \funref{call-next-method} and
\funref{next-method-p}, and declarations associated with such \term{bindings}
are \term{shadowed}\meaning{2} within the method body \param{forms}.

The \term{scope} of \term{free declarations} at the head of the method body 
is the entire method body, 
which includes any implicit local function definitions
  but excludes \term{initialization forms} for the \term{lambda variables}.
\endissue{DEFMETHOD-DECLARATION-SCOPE:CORRESPONDS-TO-BINDINGS}

\issue{COMPILE-FILE-HANDLING-OF-TOP-LEVEL-FORMS:CLARIFY}
\macref{defmethod} is not required to perform any compile-time side effects.
In particular, the \term{methods} are not installed for invocation during 
compilation.  An \term{implementation} may choose to store information about
the \term{generic function} for the purposes of compile-time error-checking
(such as checking the number of arguments on calls, or noting that a definition
 for the function name has been seen).
\endissue{COMPILE-FILE-HANDLING-OF-TOP-LEVEL-FORMS:CLARIFY}

\issue{DOCUMENTATION-FUNCTION-BUGS:FIX}
\param{Documentation} is attached as a \term{documentation string}
to the \term{method} \term{object}.
\endissue{DOCUMENTATION-FUNCTION-BUGS:FIX}

\label Examples:\None.

\label Affected By::

The definition of the referenced \term{generic function}.
 
\label Exceptional Situations::
 
If \i{function-name} names an \term{ordinary function},
a \term{macro}, or a \term{special operator}, 
an error \oftype{error} is signaled.
 
If a \term{generic function} is currently named by {\it function-name},
the \term{lambda list} of the
\term{method} must be congruent with the \term{lambda list} of the 
\term{generic function}, or
an error \oftype{error} is signaled.

%% gray addition
%Also get an error for an undefined specializer class.
 
\label See Also::

\macref{defgeneric}, 
\funref{documentation},
{\secref\IntroToMethods},
{\secref\GFMethodLambdaListCongruency},
{\secref\SpecializerQualifierAgreement},
{\secref\DocVsDecls}

\label Notes:\None.
 
\endissue{DECLS-AND-DOC}

\endcom

%%% ========== FIND-CLASS
\begincom{find-class}\ftype{Accessor}
 
\label Syntax::
 
\DefunWithValues find-class {symbol {\opt} errorp environment} {class}
\Defsetf         find-class {symbol {\opt} errorp environment} {new-class}
 
\label Arguments and Values::
 
\param{symbol}---a \term{symbol}.
 
\param{errorp}---a \term{generalized boolean}.
 \Default{\term{true}}
 
\param{environment} -- same as the \keyref{environment} argument to
  macro expansion functions and is used to distinguish between 
  compile-time and run-time environments.
\issue{MACRO-ENVIRONMENT-EXTENT:DYNAMIC}
  The \keyref{environment} argument has 
  \term{dynamic extent}; the consequences are undefined if 
  the \keyref{environment} argument is 
  referred to outside the \term{dynamic extent} 
  of the macro expansion function.
\endissue{MACRO-ENVIRONMENT-EXTENT:DYNAMIC}
%!!! Default?
 
\param{class}---a \term{class} \term{object}, or \nil.
 
\label Description::
 
Returns the \term{class} \term{object} named by the \param{symbol}
in the \param{environment}.  If there is no such \term{class},
\nil\ is returned if \param{errorp} is \term{false}; otherwise,
if \param{errorp} is \term{true}, an error is signaled.

The \term{class} associated with a particular \term{symbol} can be changed by using
\macref{setf} with \funref{find-class};
\issue{SETF-FIND-CLASS:ALLOW-NIL}
or, if the new \term{class} given to \macref{setf} is \nil,
the \term{class} association is removed 
(but the \term{class} \term{object} itself is not affected).
\endissue{SETF-FIND-CLASS:ALLOW-NIL}
The results are undefined if the user attempts to change
\issue{SETF-FIND-CLASS:ALLOW-NIL}
or remove
\endissue{SETF-FIND-CLASS:ALLOW-NIL}
the \term{class} associated with a 
\term{symbol} that is defined as a \term{type specifier} in this standard.
\Seesection\IntegratingTypesAndClasses.

When using \SETFof{find-class}, any \term{errorp} argument is \term{evaluated}
for effect, but any \term{values} it returns are ignored; the \param{errorp}
\term{parameter} is permitted primarily so that the \param{environment} \term{parameter}
can be used.

The \param{environment} might be used to distinguish between a compile-time and a
run-time environment.

\label Examples:\None.

\label Affected By:\None.
 
\label Exceptional Situations::

If there is no such \term{class} and \param{errorp} is \term{true},
\funref{find-class} signals an error \oftype{error}.
 
\label See Also::

\macref{defmacro},
{\secref\IntegratingTypesAndClasses}

\label Notes:\None.

\endcom

%%% ========== NEXT-METHOD-P
\begincom{next-method-p}\ftype{Local Function}
 
\label Syntax::
 
\DefunWithValues next-method-p {\noargs} {generalized-boolean}
 
\label Arguments and Values::
 
\param{generalized-boolean}---a \term{generalized boolean}. 

\label Description::
 
The locally defined function \funref{next-method-p} can be used 
\issue{METHOD-INITFORM:FORBID-CALL-NEXT-METHOD}
within the body \term{forms} (but not the \term{lambda list})
\endissue{METHOD-INITFORM:FORBID-CALL-NEXT-METHOD}
defined by a \term{method-defining form} to determine
whether a next \term{method} exists.
 
\Thefunction{next-method-p} has \term{lexical scope} and \term{indefinite extent}.
 
\issue{LEXICAL-CONSTRUCT-GLOBAL-DEFINITION:UNDEFINED}
Whether or not \funref{next-method-p} is \term{fbound} in the
\term{global environment} is \term{implementation-dependent};
however, the restrictions on redefinition and \term{shadowing} of
\funref{next-method-p} are the same as for \term{symbols} in \thepackage{common-lisp}
which are \term{fbound} in the \term{global environment}.
The consequences of attempting to use \funref{next-method-p} outside
of a \term{method-defining form} are undefined.
\endissue{LEXICAL-CONSTRUCT-GLOBAL-DEFINITION:UNDEFINED}

\label Examples:\None.

\label Affected By:\None.
 
\label Exceptional Situations:\None.
 
\label See Also::
 
\funref{call-next-method},
\macref{defmethod},
\macref{call-method}
 
\label Notes:\None.
 
\endcom

%%% ========== CALL-METHOD
%%% ========== MAKE-METHOD
%!!! ACW wonders if one of these is a Local Function.
\begincom{call-method, make-method}\ftype{Local Macro}
 
\label Syntax::
 
\DefmacWithValues call-method {method {\optional} next-method-list} {\starparam{result}}
\DefmacWithValues make-method {form} {method-object}

\label Arguments and Values::
 
% Moon notes that arguments are not evaluated.

\param{method}---a \term{method} \term{object},
	      or a \term{list} (see below); \noeval.

\param{method-object}---a \term{method} \term{object}.

\param{next-method-list}---a \term{list} of \param{method} \term{objects}; \noeval.
 
\param{results}---the \term{values} returned by the \term{method} invocation.
 
\label Description::
 
The macro \macref{call-method} is used in method combination.  It hides
the \term{implementation-dependent} details of how 
\term{methods} are called. The
macro \macref{call-method} has \term{lexical scope} and 
can only be used within
an \term{effective method} \term{form}.
 
\editornote{KMP: This next paragraph still needs some work.}%!!!
\issue{LEXICAL-CONSTRUCT-GLOBAL-DEFINITION:UNDEFINED}
Whether or not \macref{call-method} is \term{fbound} in the
\term{global environment} is \term{implementation-dependent};
however, the restrictions on redefinition and \term{shadowing} of
\macref{call-method} are the same as for \term{symbols} in \thepackage{common-lisp}
which are \term{fbound} in the \term{global environment}.
The consequences of attempting to use \macref{call-method} outside
of an \term{effective method} \term{form} are undefined.
\endissue{LEXICAL-CONSTRUCT-GLOBAL-DEFINITION:UNDEFINED}


% Description of arguments in this paragraph changed by Moon:
% Description of next-methods in this paragraph shortened by Moon:
 
The macro \macref{call-method} invokes the specified \term{method},
supplying it with arguments and with definitions for
\funref{call-next-method} and for \funref{next-method-p}.
If the invocation of \macref{call-method} is lexically inside
of a \funref{make-method}, the arguments are those that
were supplied to that method.  Otherwise the arguments are
those that were supplied to the generic function.
The definitions
of \funref{call-next-method} and \funref{next-method-p} rely on
the specified \param{next-method-list}.
 
If \param{method} is a \term{list}, the first element of the \term{list}
must be the symbol \funref{make-method} and the second element must be
a \term{form}.  Such a \term{list} specifies a \term{method} \term{object}
whose \term{method} function has a body that is the given \term{form}.
 
\param{Next-method-list} can contain \term{method} \term{objects} or \term{lists},
the first element of which must be the symbol \funref{make-method} and the
second element of which must be a \term{form}.
 
% Added by Moon:
 
Those are the only two places where \funref{make-method} can be used.
The \term{form} used with \funref{make-method} is evaluated in
the \term{null lexical environment} augmented with a local macro definition
for \macref{call-method} and with bindings named by
symbols not \term{accessible} from \thepackage{common-lisp-user}.
 
The \funref{call-next-method} function available to \param{method} 
will call the first \term{method} in \param{next-method-list}.
The \funref{call-next-method} function
available in that \term{method}, in turn, will call the second
\term{method} in \param{next-method-list}, and so on, until
the list of next \term{methods} is exhausted.
 
%%--Changed in drafting committee by Moon
 
If \param{next-method-list} is not supplied, the
\funref{call-next-method} function available to
\param{method} signals an error \oftype{control-error}
and the \funref{next-method-p} function
available to \param{method} returns {\nil}.

\label Examples::
 
%!!! Barmar: This desperately needs examples.
%      I have a hard time understanding the use of MAKE-METHOD.
 
\label Affected By:\None.
 
\label Exceptional Situations:\None.
 
\label See Also:: 
 
\funref{call-next-method},
\macref{define-method-combination},
\funref{next-method-p}
 
\label Notes:\None.
 
\endcom

%%% ========== CALL-NEXT-METHOD
\begincom{call-next-method}\ftype{Local Function}
 
\label Syntax::
 
\DefunWithValues call-next-method {{\rest} args} {\starparam{result}}
 
\label Arguments and Values::
 
\param{arg}---an \term{object}.  
% These are objects that are appropriate as args to the methods.
 
\param{results}---the \term{values} returned by the \term{method} it calls.
 
\label Description::
 
\Thefunction{call-next-method} can be used 
\issue{METHOD-INITFORM:FORBID-CALL-NEXT-METHOD}
within the body \term{forms} (but not the \term{lambda list})
\endissue{METHOD-INITFORM:FORBID-CALL-NEXT-METHOD}
of a \term{method} defined by a \term{method-defining form} to call the 
\term{next method}.
 
If there is no next \term{method}, the generic function 
\funref{no-next-method} is called.
 
The type of method combination used determines which \term{methods}
can invoke \funref{call-next-method}.  The standard 
\term{method combination} type allows \funref{call-next-method} 
to be used within primary \term{methods} and \term{around methods}.
%%Barmar thinks this is not needed because it is said elsewhere (ch4.1). -kmp 22-Dec-90
% The standard 
% \term{method combination}
% type defines the next \term{method} as follows:
%  
% \beginlist
% \itemitem{\bull}
% If \funref{call-next-method} is used in an \term{around method},
% the next \term{method} is the next most specific \term{around method}, if one is
% applicable.
%  
% \itemitem{\bull}
% If there are no \term{around methods} at all or if 
% \funref{call-next-method} is called by the least specific \term{around method},
% other \term{methods} are called as follows:
% \beginlist 
% \itemitem{--} All the \term{before methods} are called, in
% most-specific-first order.  \Thefunction{call-next-method}
% cannot be used in \term{before methods}.
%  
% \itemitem{--} 
% The most specific primary \term{method} is called.  Inside the body of a
% primary \term{method}, \funref{call-next-method} can be used to pass control to
% the next most specific primary \term{method}.  The generic function
% \funref{no-next-method} is called if \funref{call-next-method} is used and there
% are no more primary \term{methods}.
%                                                                    
% \itemitem{--} All the \term{after methods} are called in
% most-specific-last order.  \Thefunction{call-next-method}
% cannot be used in \term{after methods}.
%  
% \endlist
% \endlist
%
For generic functions using a type of method combination defined by
the short form of \macref{define-method-combination},
\funref{call-next-method} can be used in \term{around methods} only.
 
When \funref{call-next-method} is called with no arguments, it passes the
current \term{method}'s original arguments to the next \term{method}.  Neither
argument defaulting, nor using \specref{setq}, nor rebinding variables
with the same \term{names} as parameters of the \term{method} affects the values
\funref{call-next-method} passes to the \term{method} it calls.
 
% A sentence was removed here by Moon since it was duplicated below, and another
% sentence was moved to be next to the duplicate:
 
When \funref{call-next-method} is called with arguments, the 
\term{next method} is called with those arguments.
 
If \funref{call-next-method} is called with arguments but omits
optional arguments, the \term{next method} called defaults those arguments.
 
% Further computation is possible after \funref{call-next-method} returns.

\Thefunction{call-next-method} returns any \term{values} that are
returned by the \term{next method}.
 
\Thefunction{call-next-method} has \term{lexical scope} and 
\term{indefinite extent} and can only be used within the body of a
\term{method} defined by a \term{method-defining form}.
 
\issue{LEXICAL-CONSTRUCT-GLOBAL-DEFINITION:UNDEFINED}
Whether or not \funref{call-next-method} is \term{fbound} in the
\term{global environment} is \term{implementation-dependent};
however, the restrictions on redefinition and \term{shadowing} of
\funref{call-next-method} are the same as for \term{symbols} in \thepackage{common-lisp}
which are \term{fbound} in the \term{global environment}.
The consequences of attempting to use \funref{call-next-method} outside
of a \term{method-defining form} are undefined.
\endissue{LEXICAL-CONSTRUCT-GLOBAL-DEFINITION:UNDEFINED}

\label Examples:\None.

\label Affected By::

\macref{defmethod}, \macref{call-method}, \funref{define-method-combination}.
 
\label Exceptional Situations::
 
% Grammar improved by Moon:
 
%% Removed per X3J13. -kmp 05-Oct-93
% If \funref{call-next-method} is used in a \term{method} whose 
% \term{method combination} does not support it, 
% an error \oftype{control-error} is \term{signaled}.
 
When providing arguments to \funref{call-next-method}, 
the following rule must be satisfied or an error \oftype{error} 
%% "is" => "should be" per X3J13. -kmp 05-Oct-93
%is
should be
signaled: 
the ordered set of \term{applicable methods} for a changed set of arguments
for \funref{call-next-method} must be the same as the ordered set of
\term{applicable methods} for the original arguments to the
\term{generic function}.
Optimizations of the error checking are possible, but they must not change
the semantics of \funref{call-next-method}.
 
\label See Also::
 
\macref{define-method-combination},
\macref{defmethod},
\funref{next-method-p},
\funref{no-next-method},
\macref{call-method},
{\secref\MethodSelectionAndCombination},
{\secref\StdMethComb},
{\secref\BuiltInMethCombTypes}

\label Notes:\None.
 
\endcom

%%% ========== COMPUTE-APPLICABLE-METHODS
\begincom{compute-applicable-methods}\ftype{Standard Generic Function}
 
\issue{COMPUTE-APPLICABLE-METHODS:GENERIC} 

\label Syntax::
 
\DefgenWithValues compute-applicable-methods {generic-function function-arguments} {methods}
 
\label Method Signatures::
 
\Defmeth compute-applicable-methods {\specparam{generic-function}{standard-generic-function}}
 
\endissue{COMPUTE-APPLICABLE-METHODS:GENERIC} 

\label Arguments and Values::
 
\param{generic-function}---a \term{generic function}.
                                          
\param{function-arguments}---a \term{list} of arguments for the \param{generic-function}.
 
\param{methods}---a \term{list} of \term{method} \term{objects}.

\label Description::
 
Given a \param{generic-function} and a set of 
\param{function-arguments}, the function
\funref{compute-applicable-methods} returns the set of \term{methods}
that are applicable for those arguments
sorted according to precedence order.
\Seesection\MethodSelectionAndCombination.
 
\label Affected By::

\macref{defmethod}
 
\label Exceptional Situations:\None.
 
\label See Also::
 
{\secref\MethodSelectionAndCombination}
 
\label Notes:\None.
 
\endcom

%%% ========== DEFINE-METHOD-COMBINATION
\begincom{define-method-combination}\ftype{Macro}
 
\issue{DECLS-AND-DOC}

\label Syntax::
 
% 88-002R p.2-34 said "new method combination object" was returned, but it's wrong,
% method-combination objects are created by the defgeneric :method-combination option.
% See 88-002R p.1-28. --Moon
% The "name" is returned.
% Barrett didn't believe this at first, but now does.
% Consensus comes slowly.
 
\DefmacWithValuesNewline define-method-combination 
			 {name \interleave{\down{short-form-option}}}
			 {name}
 
\DefmacWithValuesNewline define-method-combination
                         {\vtop{\hbox{name lambda-list}
                                \hbox{\paren{\starparam{method-group-specifier}}}
                                \hbox{\brac{\paren{\kwd{arguments} . args-lambda-list}}}
                                \hbox{\brac{\paren{\kwd{generic-function} 
                                                   generic-function-symbol}}}
                                \hbox{\DeclsAndDoc}
                                \hbox{\starparam{form}}}}
			 {name}

\auxbnf{short-form-option}{\kwd{documentation} \param{documentation} | \CR
			   \kwd{identity-with-one-argument}
			   \param{identity-with-one-argument} |\CR
			   \kwd{operator} \param{operator}}
\auxbnf{method-group-specifier}{\paren{name 
				       \curly{\plusparam{qualifier-pattern} $\vert$ predicate}
			               \interleave{\down{long-form-option}}}}
\auxbnf{long-form-option}{\kwd{description} \param{description} |\CR
			  \kwd{order} \param{order} 		|\CR
			  \kwd{required} \param{required-p}}
 
\label Arguments and Values::
 
\param{args-lambda-list}---%
%Moon thought :arguments for DEFINE-METHOD-COMBINATION took an ordinary lambda list,
%but Barrett (comment #3, first public review) observes that &whole is permissible.
%Time to make a new kind of list.
a \term{define-method-combination arguments lambda list}.

\param{declaration}---a \misc{declare} \term{expression}; \noeval.

\param{description}---a \term{format control}.

\param{documentation}---a \term{string}; \noeval.

\param{forms}---an \term{implicit progn} 
  that must compute and return the \term{form} that specifies how
  the \term{methods} are combined, that is, the \term{effective method}.

\param{generic-function-symbol}---a \term{symbol}.

\param{identity-with-one-argument}---a \term{generalized boolean}.

\param{lambda-list}---\term{ordinary lambda list}.

\param{name}---a \term{symbol}. 
  Non-\term{keyword}, \term{non-nil} \term{symbols} are usually used.

\param{operator}---an \term{operator}.
  \param{Name} and \param{operator} are often the \term{same} \term{symbol}.
  This is the default, but it is not required.

\param{order}---\kwd{most-specific-first} or \kwd{most-specific-last}; \eval.

%Not a function designator?
\param{predicate}---a \term{symbol} that names a \term{function} of one argument
		    that returns a \term{generalized boolean}.

\param{qualifier-pattern}---a \term{list},
			 or the \term{symbol} \misc{*}.

\param{required-p}---a \term{generalized boolean}.
 
\label Description::
 
The macro \macref{define-method-combination} is used to define new types
of method combination.

There are two forms of \macref{define-method-combination}.  The short
form is a simple facility for the cases that are expected
to be most commonly needed.  The long form is more powerful but more
verbose.  It resembles \macref{defmacro} in that the body is an
expression, usually using backquote, that computes a \term{form}.  Thus
arbitrary control structures can be implemented.  The long form also
allows arbitrary processing of method \term{qualifiers}.
 
%In both the short and long forms, \param{name} is a symbol.  By convention,
%non-keyword, \term{non-nil} symbols are usually used.
 
 
\beginlist
\itemitem{{\bf Short Form}}
 
The short form syntax of \macref{define-method-combination} is recognized
when the second \term{subform} is a \term{non-nil} symbol or is not present.
When the short form is used, \param{name} is defined as a type of
method combination that produces a Lisp form
\f{({\param{operator} \param{method-call} \param{method-call} $\ldots$})}.
The \param{operator} is a \term{symbol} that can be the \term{name} of a 
\term{function}, \term{macro}, or \term{special operator}.  
The \param{operator} can be supplied by a keyword option;
it defaults to \param{name}.
 
Keyword options for the short form are the following:
 
\beginlist
 
\itemitem{\bull}
The \kwd{documentation} option is used to document the method-combination type;
see description of long form below.
 
\itemitem{\bull}
The \kwd{identity-with-one-argument} option enables an optimization
when its value is \term{true} (the default is \term{false}).  If there is
exactly one applicable method and it is a primary method, that method
serves as the effective method and \param{operator} is not called.
This optimization avoids the need to create a new effective method and
avoids the overhead of a \term{function} call.  This option is designed to be
used with operators such as \specref{progn}, \macref{and}, \funref{$+$}, and
\funref{max}.
     
\itemitem{\bull}
The \kwd{operator} option specifies the \term{name} of the operator.  The
\param{operator} argument is a \term{symbol} that can be the 
\term{name} of a \term{function},
\term{macro}, or 
\term{special form}.  
 
%By convention, \param{name} and
%\param{operator} are often the same symbol.  This is the default,
%but it is not required.
\endlist
 
%None of the \term{subforms} is evaluated.
 
These types of method combination require exactly one \term{qualifier} per
method.  An error is signaled if there are applicable methods with no
\term{qualifiers} or with \term{qualifiers} that are not supported by 
the method combination type. 
 
A method combination procedure defined in this way recognizes two
roles for methods.  A method whose one \term{qualifier} is the symbol naming
this type of method combination is defined to be a primary method.  At
least one primary method must be applicable or an error is signaled.
A method with \kwd{around} as its one \term{qualifier} is an auxiliary
method that behaves the same as an \term{around method} in standard
method combination.  \Thefunction{call-next-method} can only be
used in \term{around methods}; it cannot be used in primary methods
defined by the short form of the \macref{define-method-combination} macro.
 
A method combination procedure defined in this way accepts an optional
argument named \param{order}, which defaults to 
\kwd{most-specific-first}.  A value of \kwd{most-specific-last} reverses
the order of the primary methods without affecting the order of the
auxiliary methods.
 
The short form automatically includes error checking and support for
\term{around methods}.
 
For a discussion of built-in method combination types, 
\seesection\BuiltInMethCombTypes.
 
\itemitem{{\bf Long Form}}
 
The long form syntax of \macref{define-method-combination} is recognized 
when the second \term{subform} is a list.  
 
The \param{lambda-list} 
receives any arguments provided after the \term{name} of the method
combination type in the \kwd{method-combination} option to
\macref{defgeneric}.
 
A list of method group specifiers follows.  Each specifier selects a subset
of the applicable methods to play a particular role, either by matching
their \term{qualifiers} against some patterns or by testing their \term{qualifiers} with
a \param{predicate}.   
These method group specifiers define all method \term{qualifiers}
that can be used with this type of method combination.
 
% Removed by Moon as the same information is repeated below
%If an applicable 
%method does not fall into any method group, the system signals the error
%that the method is invalid for the kind of method combination in use.
 
%%Rewritten per Barmar. -kmp 28-Dec-90
%Each method group specifier names a variable.
The \term{car} of each \param{method-group-specifier} is a \term{symbol}
which \term{names} a \term{variable}.
During the execution of
the \term{forms} in the body of \macref{define-method-combination}, this
\term{variable} is bound to a list of the \term{methods} in the method group.  The
\term{methods} in this list occur in the order specified by the 
\kwd{order} option.
 
If \param{qualifier-pattern} is a \term{symbol} it must be \misc{*}.  
A method matches
a \param{qualifier-pattern} if the method's 
list of \term{qualifiers} is \funref{equal}
to the \param{qualifier-pattern} (except that the symbol \misc{*} in a 
\param{qualifier-pattern} matches anything).  Thus 
a \param{qualifier-pattern} can be one of the
following:
 the \term{empty list}, which matches \term{unqualified methods};
 the symbol \misc{*}, which matches all methods;
 a true list, which matches methods with the same number of \term{qualifiers} 
   as the length of the list when each \term{qualifier} matches 
   the corresponding list element; or
 a dotted list that ends in the symbol \misc{*} 
   (the \misc{*} matches any number of additional \term{qualifiers}).
 
 
Each applicable method is tested against the \param{qualifier-patterns} and
\param{predicates} in left-to-right order.  
As soon as a \param{qualifier-pattern} matches
or a \param{predicate} returns true, the method becomes a member of the
corresponding method group and no further tests are made.  Thus if a method
could be a member of more than one method group, it joins only the first
such group.  If a method group has more than one 
\param{qualifier-pattern}, a
method need only satisfy one of the \param{qualifier-patterns} to be a member of
the group.
 
The \term{name} of a \param{predicate} function can appear instead of 
\param{qualifier-patterns} in a method group specifier.  
The \param{predicate} is called for
each method that has not been assigned to an earlier method group; it
is called with one argument, the method's \term{qualifier} \term{list}.
The \param{predicate} should return true if the method is to be a member of the
method group.  A \param{predicate} can be distinguished from a 
\param{qualifier-pattern}
because it is a \term{symbol} other than \nil\ or \misc{*}.
 
% Wording improved --Moon
 
If there is an applicable method that does not fall into any method group,
\thefunction{invalid-method-error} is called.
 
Method group specifiers can have keyword options following the
\term{qualifier} patterns or predicate.  Keyword options can be distinguished from
additional \term{qualifier} patterns because they are neither lists nor the symbol
\misc{*}.  The keyword options are as follows:
 
\beginlist
 
\itemitem{\bull}
The \kwd{description} option is used to provide a description of the
role of methods in the method group.  Programming environment tools
use
 {\tt (apply \#'format stream \param{format-control} (method-qualifiers \param{method}))}
to print this description, which
is expected to be concise.  This keyword
option allows the description of a method \term{qualifier} to be defined in
the same module that defines the meaning of the 
method \term{qualifier}.  In most cases, \param{format-control} will not contain any
\funref{format} directives, but they are available for generality.  
If \kwd{description} is not supplied, a default description is generated
based on the variable name and the \term{qualifier} patterns and on whether
this method group includes the \term{unqualified methods}.  
 
\itemitem{\bull}
The \kwd{order} option specifies the order of methods.  The \param{order}
argument is a \term{form} that evaluates to 
\kwd{most-specific-first} or \kwd{most-specific-last}.  If it evaluates
to any other value, an error is signaled.  
%This keyword option is a
%convenience and does not add any expressive power.
If \kwd{order} is not supplied, it defaults to 
\kwd{most-specific-first}.
                                 
\itemitem{\bull}
The \kwd{required} option specifies whether at least one method in
this method group is required.
If its value is \term{true} and the method group is empty 
(that is, no applicable methods match the \term{qualifier} patterns
or satisfy the predicate), 
an error is signaled.  
%This keyword option is a convenience and does not
%add any expressive power.  
If \kwd{required} is not supplied,
it defaults to \nil.  
 
\endlist
 
The use of method group specifiers provides a convenient syntax to
select methods, to divide them among the possible roles, and to perform the
necessary error checking.  It is possible to perform further filtering
of methods in the body \term{forms} by using normal list-processing operations
and the functions \funref{method-qualifiers} and 
\funref{invalid-method-error}.  It is permissible to use \specref{setq} on the
variables named in the method group specifiers and to bind additional
variables.  It is also possible to bypass the method group specifier
mechanism and do everything in the body \term{forms}.  This is accomplished
by writing a single method group with \misc{*} as its only 
\param{qualifier-pattern}; 
the variable is then bound to a \term{list} of all of the
\term{applicable methods}, in most-specific-first order.
 
% Modified by Moon to clarify lexical environment:
 
The body \param{forms} compute and return the \term{form} that specifies
how the methods are combined, that is, the effective method.
The effective method is evaluated in
the \term{null lexical environment} augmented with a local macro definition
for \funref{call-method} and with bindings named by
symbols not \term{accessible} from \thepackage{common-lisp-user}.
Given a method object in one of the 
\term{lists} produced by the method group
specifiers and a \term{list} of next methods,
\funref{call-method}
will invoke the method such that \funref{call-next-method} has available
the next methods.
 
When an effective method has no effect other than to call a single
method, some implementations employ an optimization that uses the
single method directly as the effective method, thus avoiding the need
to create a new effective method.  This optimization is active when
the effective method form consists entirely of an invocation of
the \funref{call-method} macro whose first \term{subform} is a method object and
whose second \term{subform} is \nil\ or unsupplied.  Each
\macref{define-method-combination} body is responsible for stripping off
redundant invocations of \specref{progn}, \macref{and},
\macref{multiple-value-prog1}, and the like, if this optimization is desired.
 
 
% One sentence was removed and replaced by Moon, since the specification
% about congruence was excessively vague.  Also discuss the consequences
% of modifying arguments:
 
The list {\tt (:arguments . \param{lambda-list})} can appear before
any declarations or \term{documentation string}.  This form is useful when
the method combination type performs some specific behavior as part of
the combined method and that behavior needs access to the arguments to
the \term{generic function}.  Each parameter variable defined by 
\param{lambda-list} is bound to a \term{form} that can be inserted into the
effective method.  When this \term{form} is evaluated during execution of the
effective method, its value is the corresponding argument to the
\term{generic function}; the consequences of using such a \term{form} as
the \param{place} in a \macref{setf} \term{form} are undefined.
\issue{METHOD-COMBINATION-ARGUMENTS:CLARIFY}
%If \param{lambda-list} is not congruent to the
%generic function's \term{lambda list}, additional ignored parameters are
%automatically inserted until it is congruent.
% If the arguments supplied to the \term{generic function} do not
% match \param{lambda-list}, extra arguments are ignored and missing
% arguments are defaulted to \nil\.
% Thus it is permissible
% for \param{lambda-list} to receive fewer arguments than the number of
% required arguments for the \term{generic function}.
Argument correspondence is computed by dividing the \kwd{arguments} \param{lambda-list}
and the \term{generic function} \param{lambda-list} into three sections:
     the \term{required parameters},
     the \term{optional parameters},
 and the \term{keyword} and \term{rest parameters}.
The \term{arguments} supplied to the \term{generic function} for a particular \term{call}
are also divided into three sections;
     the required \term{arguments} section contains as many \term{arguments}
      as the \term{generic function} has \term{required parameters},
     the optional \term{arguments} section contains as many arguments
      as the \term{generic function} has \term{optional parameters},
     and the keyword/rest \term{arguments} section contains the remaining arguments.
Each \term{parameter} in the required and optional sections of the 
\kwd{arguments} \param{lambda-list} accesses the argument at the same position
in the corresponding section of the \term{arguments}.
If the section of the \kwd{arguments} \param{lambda-list} is shorter,
 extra \term{arguments} are ignored. 
If the section of the \kwd{arguments} \param{lambda-list} is longer,
 excess \term{required parameters} are bound to forms that evaluate to \nil\ 
 and excess \term{optional parameters} are \term{bound} to their initforms.
The \term{keyword parameters} and \term{rest parameters} in the \kwd{arguments}
\param{lambda-list} access the keyword/rest section of the \term{arguments}.
If the \kwd{arguments} \param{lambda-list} contains \keyref{key}, it behaves as
if it also contained \keyref{allow-other-keys}.

In addition, \keyref{whole} \param{var} can be placed first in the \kwd{arguments}
\param{lambda-list}.  It causes \param{var} to be \term{bound} to a \term{form}
that \term{evaluates} to a \term{list} of all of the \term{arguments} supplied
to the \term{generic function}.  This is different from \keyref{rest} because it
accesses all of the arguments, not just the keyword/rest \term{arguments}.

\endissue{METHOD-COMBINATION-ARGUMENTS:CLARIFY}

Erroneous conditions detected by the body should be reported with
\funref{method-combination-error} or \funref{invalid-method-error}; these
\term{functions}
add any necessary contextual information to the error message and will
signal the appropriate error.
                                                        
The body \param{forms} are evaluated inside of the \term{bindings} created by
the
\term{lambda list} and method group specifiers. 
\reviewer{Barmar: Are they inside or outside the :ARGUMENTS bindings?}
Declarations at the head of
the body are positioned directly inside of \term{bindings} created by the
\term{lambda list} and outside of the \term{bindings} of the method group variables. 
Thus method group variables cannot be declared in this way.  \specref{locally} may be used
around the body, however.
 
Within the body \param{forms}, \param{generic-function-symbol}
is bound to the \term{generic function} \term{object}.

\issue{DOCUMENTATION-FUNCTION-BUGS:FIX}
\param{Documentation} is attached as a \term{documentation string} 
    to \param{name} (as kind \specref{method-combination})
and to the \term{method combination} \term{object}.
\endissue{DOCUMENTATION-FUNCTION-BUGS:FIX}
 
% Removed by Moon, appears to be redundant with what was stated earlier:
 
%The functions \funref{method-combination-error} and 
%\funref{invalid-method-error} can be called from the body \param{forms} or
%from \term{functions} called by the body \param{forms}.  The actions of these
%two \term{functions} can depend on \term{implementation-dependent} dynamic variables
%automatically bound before the generic function
%\funref{compute-effective-method} is called.
 
Note that two methods with identical specializers, but with different
\term{qualifiers}, are not ordered by the algorithm described in Step 2 of
the method selection and combination process described in 
\secref\MethodSelectionAndCombination.  Normally the two methods play
different roles in the effective method because they have different
\term{qualifiers}, and no matter how they are ordered in the result of Step
2, the effective method is the same.  If the two methods play the same
role and their order matters, 
\reviewer{Barmar: How does the system know when the order matters?}
an error is signaled.  This happens as
part of the \term{qualifier} pattern matching in
\macref{define-method-combination}.

\endlist

\issue{DEFINE-METHOD-COMBINATION:CLARIFY}
If a \macref{define-method-combination} \term{form} appears as a
\term{top level form}, the \term{compiler} must make the
\term{method combination} \term{name} be recognized as a valid
\term{method combination} \term{name} in subsequent \macref{defgeneric}
\term{forms}.  However, the \term{method combination} is executed
no earlier than when the \macref{define-method-combination} \term{form}
is executed, and possibly as late as the time that \term{generic functions}
that use the \term{method combination} are executed.
\endissue{DEFINE-METHOD-COMBINATION:CLARIFY}

\label Examples::
 
%% Examples changed by Moon to reflect that the second argument of
%% call-method is unsupplied when call-next-method is not allowed

Most examples of the long form of \macref{define-method-combination} also
illustrate the use of the related \term{functions} that are provided as part
of the declarative method combination facility.
 
\code
;;; Examples of the short form of define-method-combination
 
 (define-method-combination and :identity-with-one-argument t) 
  
 (defmethod func and ((x class1) y) ...)
 
;;; The equivalent of this example in the long form is:
 
 (define-method-combination and 
         (&optional (order :most-specific-first))
         ((around (:around))
          (primary (and) :order order :required t))
   (let ((form (if (rest primary)
                   `(and ,@(mapcar #'(lambda (method)
                                       `(call-method ,method))
                                   primary))
                   `(call-method ,(first primary)))))
     (if around
         `(call-method ,(first around)
                       (,@(rest around)
                        (make-method ,form)))
         form)))
  
;;; Examples of the long form of define-method-combination
 
;The default method-combination technique
 (define-method-combination standard ()
         ((around (:around))
          (before (:before))
          (primary () :required t)
          (after (:after)))
   (flet ((call-methods (methods)
            (mapcar #'(lambda (method)
                        `(call-method ,method))
                    methods)))
     (let ((form (if (or before after (rest primary))
                     `(multiple-value-prog1
                        (progn ,@(call-methods before)
                               (call-method ,(first primary)
                                            ,(rest primary)))
                        ,@(call-methods (reverse after)))
                     `(call-method ,(first primary)))))
       (if around
           `(call-method ,(first around)
                         (,@(rest around)
                          (make-method ,form)))
           form))))
  
;A simple way to try several methods until one returns non-nil
 (define-method-combination or ()
         ((methods (or)))
   `(or ,@(mapcar #'(lambda (method)
                      `(call-method ,method))
                  methods)))
  
;A more complete version of the preceding
 (define-method-combination or 
         (&optional (order ':most-specific-first))
         ((around (:around))
          (primary (or)))
   ;; Process the order argument
   (case order
     (:most-specific-first)
     (:most-specific-last (setq primary (reverse primary)))
     (otherwise (method-combination-error "~S is an invalid order.~@
     :most-specific-first and :most-specific-last are the possible values."
                                          order)))
   ;; Must have a primary method
   (unless primary
     (method-combination-error "A primary method is required."))
   ;; Construct the form that calls the primary methods
   (let ((form (if (rest primary)
                   `(or ,@(mapcar #'(lambda (method)
                                      `(call-method ,method))
                                  primary))
                   `(call-method ,(first primary)))))
     ;; Wrap the around methods around that form
     (if around
         `(call-method ,(first around)
                       (,@(rest around)
                        (make-method ,form)))
         form)))
  
;The same thing, using the :order and :required keyword options
 (define-method-combination or 
         (&optional (order ':most-specific-first))
         ((around (:around))
          (primary (or) :order order :required t))
   (let ((form (if (rest primary)
                   `(or ,@(mapcar #'(lambda (method)
                                      `(call-method ,method))
                                  primary))
                   `(call-method ,(first primary)))))
     (if around
         `(call-method ,(first around)
                       (,@(rest around)
                        (make-method ,form)))
         form)))
  
;This short-form call is behaviorally identical to the preceding
 (define-method-combination or :identity-with-one-argument t)
 
;Order methods by positive integer qualifiers
;:around methods are disallowed to keep the example small
 (define-method-combination example-method-combination ()
         ((methods positive-integer-qualifier-p))
   `(progn ,@(mapcar #'(lambda (method)
                         `(call-method ,method))
                     (stable-sort methods #'<
                       :key #'(lambda (method)
                                (first (method-qualifiers method)))))))
 
 (defun positive-integer-qualifier-p (method-qualifiers)
   (and (= (length method-qualifiers) 1)
        (typep (first method-qualifiers) '(integer 0 *))))
  
;;; Example of the use of :arguments
 (define-method-combination progn-with-lock ()
         ((methods ()))
   (:arguments object)
   `(unwind-protect
        (progn (lock (object-lock ,object))
               ,@(mapcar #'(lambda (method)
                             `(call-method ,method))
                         methods))
      (unlock (object-lock ,object))))
  
\endcode
 
\label Affected By:\None.
 
\label Side Effects::

\issue{COMPILE-FILE-HANDLING-OF-TOP-LEVEL-FORMS:CLARIFY}
The \term{compiler} is not required to perform any compile-time side-effects.
\endissue{COMPILE-FILE-HANDLING-OF-TOP-LEVEL-FORMS:CLARIFY}

\label Exceptional Situations::
 
Method combination types defined with the short form require exactly
one \term{qualifier} per method.  
An error \oftype{error} is signaled if there are
applicable methods with no \term{qualifiers} or with \term{qualifiers} that are not
supported by the method combination type.
At least one primary method must be applicable or 
an error \oftype{error} is signaled.
 
If an applicable method does not fall into any method group, the
system signals an error \oftype{error}
indicating that the method is invalid for the kind of
method combination in use.
 
If the value of the \kwd{required} option is \term{true}
and the method group is empty (that is, no applicable
methods match the \term{qualifier} patterns or satisfy the predicate), 
an error \oftype{error} is signaled.
 
If the \kwd{order} option evaluates to a value other than 
\kwd{most-specific-first} or \kwd{most-specific-last}, 
an error \oftype{error} is signaled.
 
\label See Also:: 
 
\funref{call-method},
\funref{call-next-method},
\funref{documentation},
\funref{method-qualifiers},
\funref{method-combination-error},
\funref{invalid-method-error},
\macref{defgeneric},
{\secref\MethodSelectionAndCombination},
{\secref\BuiltInMethCombTypes},
{\secref\DocVsDecls}

\label Notes::
 
% Added phrase to the end of this paragraph --Moon:
 
The \kwd{method-combination} option of \macref{defgeneric} is used to
specify that a \term{generic function} should use a particular method
combination type.  The first argument to the \kwd{method-combination}
option is the \term{name} of a method combination type and the remaining
arguments are options for that type.
 
\endissue{DECLS-AND-DOC}

\endcom

%%% ========== FIND-METHOD
\begincom{find-method}\ftype{Standard Generic Function}
 
\label Syntax::
 
\DefgenWithValuesNewline find-method
			 {generic-function method-qualifiers specializers {\opt} errorp}
			 {method}
 
\label Method Signatures::
 
\Defmeth find-method 
	 {\vtop{\hbox{\specparam{generic-function}{standard-generic-function}}
		\hbox{method-qualifiers specializers {\opt} errorp}}}
 
\label Arguments and Values::
 
\param{generic-function}---a \term{generic function}.
 
\param{method-qualifiers}---a \term{list}.
 
\param{specializers}---a \term{list}.
 
\param{errorp}---a \term{generalized boolean}.
 \Default{\term{true}}
 
\param{method}---a \term{method} \term{object}, or \nil.
 
\label Description::
 
The \term{generic function} \funref{find-method} takes a \term{generic function} 
and returns the \term{method} \term{object} that agrees on \term{qualifiers} 
and \term{parameter specializers} with the \param{method-qualifiers} and 
\param{specializers} arguments of \funref{find-method}.  
\param{Method-qualifiers}  contains the
method \term{qualifiers} for the \term{method}. 
The order of the method \term{qualifiers}
is significant.                                
For a definition of agreement in this context,
\seesection\SpecializerQualifierAgreement.
 
The \param{specializers} argument contains the parameter
specializers for the \term{method}. It must correspond in length to
the number of required arguments of the \term{generic function}, or
an error is signaled.  This means that to obtain the
default \term{method} on a given \param{generic-function},
a \term{list} whose elements are \theclass{t} must be given.
 
If there is no such \term{method} and \param{errorp} is \term{true},
\funref{find-method} signals an error.
If there is no such \term{method} and \param{errorp} is \term{false},
\funref{find-method} returns \nil.
 
\label Examples::
 
\code
 (defmethod some-operation ((a integer) (b float)) (list a b))
\EV #<STANDARD-METHOD SOME-OPERATION (INTEGER FLOAT) 26723357>
 (find-method #'some-operation '() (mapcar #'find-class '(integer float)))
\EV #<STANDARD-METHOD SOME-OPERATION (INTEGER FLOAT) 26723357>
 (find-method #'some-operation '() (mapcar #'find-class '(integer integer)))
\OUT Error: No matching method
 (find-method #'some-operation '() (mapcar #'find-class '(integer integer)) nil)
\EV NIL
\endcode

\label Affected By::

\funref{add-method},
\macref{defclass},
\macref{defgeneric},
\macref{defmethod}
 
\label Exceptional Situations::
 
If the \param{specializers} argument does not correspond in length to
the number of required arguments of the \param{generic-function}, an
an error \oftype{error} is signaled.  
 
If there is no such \term{method} and \param{errorp} is \term{true}, 
\funref{find-method} signals an error \oftype{error}.
 
\label See Also::
 
{\secref\SpecializerQualifierAgreement}
 
\label Notes:\None.
 
\endcom

%%% ========== ADD-METHOD
\begincom{add-method}\ftype{Standard Generic Function}
 
\label Syntax::
 
\DefgenWithValues add-method {generic-function method} {generic-function}
 
\label Method Signatures::
 
\Defmeth add-method {\vtop{\hbox{\specparam{generic-function}{standard-generic-function}}
			   \hbox{\specparam{method}{method}}}}
 
\label Arguments and Values::
 
\param{generic-function}---a \term{generic function} \term{object}.
 
\param{method}---a \term{method} \term{object}.
 
\label Description::
 
The generic function \funref{add-method} adds a \term{method}
to a \term{generic function}.
 
If \param{method} agrees with an existing \term{method} of \param{generic-function}
on \term{parameter specializers} and \term{qualifiers}, 
the existing \term{method} is replaced.
 
%% Per X3J13. -kmp 05-Oct-93
\label Examples:\None.
 
%!!!

\label Affected By:\None.
 
\label Exceptional Situations::
 
The \term{lambda list} of the method function of \param{method} must be
congruent with the \term{lambda list} of \param{generic-function}, 
or an error \oftype{error} is signaled.
 
If \param{method} is a \term{method} \term{object} of
another \term{generic function}, an error \oftype{error} is signaled.
 
                                     
\label See Also::
                      
\macref{defmethod},
\macref{defgeneric},
\funref{find-method},
\funref{remove-method},
{\secref\SpecializerQualifierAgreement}

\label Notes:\None.
 
\endcom
 

%%% ========== INITIALIZE-INSTANCE
\begincom{initialize-instance}\ftype{Standard Generic Function}
 
\label Syntax::
 
\issue{INITIALIZATION-FUNCTION-KEYWORD-CHECKING}
\DefgenWithValues initialize-instance 
		  {instance {\rest} initargs {\key} {\allowotherkeys}}
		  {instance}
\endissue{INITIALIZATION-FUNCTION-KEYWORD-CHECKING}
 
\label Method Signatures::
 
\Defmeth initialize-instance {\specparam{instance}{standard-object} {\rest} initargs}
 
\label Arguments and Values::
 
\param{instance}---an \term{object}.
 
\param{initargs}---a \term{defaulted initialization argument list}.
 
\label Description::
 
Called by \funref{make-instance} to initialize a newly created \term{instance}.
The generic function is called with the new \param{instance} 
and the \term{defaulted initialization argument list}.
 
The system-supplied primary \term{method} on \funref{initialize-instance}
initializes the \term{slots} of the \param{instance} with values according 
to the \param{initargs} and the \kwd{initform} forms of the \term{slots}.
It does this by calling the generic function \funref{shared-initialize}
with the following arguments: the \param{instance}, \t\ (this indicates
that all \term{slots} for which no initialization arguments are provided
should be initialized according to their \kwd{initform} forms), and
the \param{initargs}.
 
Programmers can define \term{methods} for \funref{initialize-instance} to
specify actions to be taken when an instance is initialized.  If only
\term{after methods} are defined, they will be run after the
system-supplied primary \term{method} for initialization and therefore will
not interfere with the default behavior of \funref{initialize-instance}.
 
\label Examples:\None.

\label Affected By:\None.
 
\label Exceptional Situations:\None.
 
\label See Also::
 
\funref{shared-initialize},
\funref{make-instance},
\funref{slot-boundp},
\funref{slot-makunbound},
{\secref\ObjectCreationAndInit},
{\secref\InitargRules},
{\secref\DeclaringInitargValidity}

\label Notes:\None.
 
\endcom

%%% ========== CLASS-NAME
\begincom{class-name}\ftype{Standard Generic Function}
 
\label Syntax::
 
\DefgenWithValues class-name {class} {name}
 
\label Method Signatures::
 
\Defmeth class-name {\specparam{class}{class}}
 
\label Arguments and Values::
 
\param{class}---a \term{class} \term{object}.

\param{name}---a \term{symbol}.
 
\label Description::

Returns the \term{name} of the given \param{class}.
 
\label Examples:\None.

\label Affected By:\None.
 
\label Exceptional Situations:\None.
 
\label See Also::
 
\funref{find-class},
{\secref\Classes}
 
\label Notes::
 
If $S$ is a \term{symbol} such that $S =${\tt (class-name $C$)}
and $C =${\tt (find-class $S$)}, then $S$ is the proper name of $C$.
For further discussion, \seesection\Classes.

The name of an anonymous \term{class} is \nil.

\endcom

%%% ========== (SETF CLASS-NAME)
\begincom{(setf class-name)}\ftype{Standard Generic Function}
 
\label Syntax::
 
\DefgenWithValues {(setf class-name)} {new-value class} {new-value}
 
\label Method Signatures::
                                                             
\Defmeth {(setf class-name)} {new-value \specparam{class}{class}}
 
\label Arguments and Values::
                 
\param{new-value}---a \term{symbol}.
 
\param{class}---a \term{class}.
 
\label Description::
 
The generic function \f{(setf class-name)} sets the name of 
a \param{class} object.

%The name of \param{class} is set to \param{new-value}.
 
 
\label Examples:\None.

\label Affected By:\None.
 
\label Exceptional Situations:\None.
 
\label See Also::
 
\funref{find-class},
\term{proper name},
{\secref\Classes}

%% Per X3J13. -kmp 05-Oct-93
\label Notes:\None.
 
%This info is in the glossary and elsewhere.  I added a cross reference above. -kmp 14-May-91
% If \i{S} is a symbol such that 
%     \f{\i{S} = (class-name \i{C})}
% and \f{\i{C} = (find-class \i{S})},
% then \i{S} is the \term{proper name} of \i{C}.  
%For further discussion, \seesection\Classes.
 
\endcom

%%% ========== CLASS-OF
\begincom{class-of}\ftype{Function}
 
\label Syntax::
 
\DefunWithValues class-of {object} {class}
 
\label Arguments and Values::
 
\param{object}---an \term{object}.
 
\param{class}---a \term{class} \term{object}.

\label Description::
 
Returns the \term{class} of which the \param{object} is 
%"an instance" -> "a direct instance" -kmp 15-Jan-91
a \term{direct instance}.
 
\label Examples::

\code
 (class-of 'fred) \EV #<BUILT-IN-CLASS SYMBOL 610327300>
 (class-of 2/3) \EV #<BUILT-IN-CLASS RATIO 610326642>
 
 (defclass book () ()) \EV #<STANDARD-CLASS BOOK 33424745>
 (class-of (make-instance 'book)) \EV #<STANDARD-CLASS BOOK 33424745>
 
 (defclass novel (book) ()) \EV #<STANDARD-CLASS NOVEL 33424764>
 (class-of (make-instance 'novel)) \EV #<STANDARD-CLASS NOVEL 33424764>

 (defstruct kons kar kdr) \EV KONS
 (class-of (make-kons :kar 3 :kdr 4)) \EV #<STRUCTURE-CLASS KONS 250020317>
\endcode

\label Affected By:\None!
 
\label Exceptional Situations:\None!
 
\label See Also::
 
\funref{make-instance},
\funref{type-of}

\label Notes:\None.
 
\endcom

%--------------------Object Errors--------------------

\begincom{unbound-slot}\ftype{Condition Type}

\issue{UNDEFINED-VARIABLES-AND-FUNCTIONS:COMPROMISE}

\label Class Precedence List::
\typeref{unbound-slot},
\typeref{cell-error},
\typeref{error},
\typeref{serious-condition},
\typeref{condition},
\typeref{t}

\label Description::

The \term{object} having the unbound slot is initialized by 
\theinitkeyarg{instance} to \funref{make-condition},
and is \term{accessed} by \thefunction{unbound-slot-instance}.
% \Thetype{unbound-slot} has an instance slot that can be
% initialized using the \kwd{instance} keyword to \funref{make-condition}.
\endissue{UNDEFINED-VARIABLES-AND-FUNCTIONS:COMPROMISE}

The name of the cell (see \typeref{cell-error}) is the name of the slot.

\label See Also::

\funref{cell-error-name},
\funref{unbound-slot-instance},
{\secref\ConditionSystemConcepts}

\endcom%{unbound-slot}\ftype{Condition Type}

%%% ========== UNBOUND-SLOT-INSTANCE
\begincom{unbound-slot-instance}\ftype{Function}

\issue{UNDEFINED-VARIABLES-AND-FUNCTIONS:COMPROMISE}

\label Syntax::

\DefunWithValues unbound-slot-instance {condition} {instance}
    
\label Arguments and Values::

\param{condition}---a \term{condition} \oftype{unbound-slot}.

\param{instance}---an \term{object}.

\label Description::

Returns the instance which had the unbound slot in the \term{situation}
represented by the \param{condition}.          

\label Examples:\None.

\label Affected By:\None.

\label Exceptional Situations:\None.

\label See Also::

\funref{cell-error-name},
\typeref{unbound-slot},
{\secref\ConditionSystemConcepts}

\label Notes:\None.

%Shouldn't be needed.
%It is an error to use \macref{setf} with \funref{unbound-slot-instance}.

\endissue{UNDEFINED-VARIABLES-AND-FUNCTIONS:COMPROMISE}
\endcom
