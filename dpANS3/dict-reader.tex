% -*- Mode: TeX -*-

% Reader
%  Reader Variables

%-------------------- Readtable Type --------------------

%% 2.7.0 1
%% 22.1.5 2
\begincom{readtable}\ftype{System Class}

\label Class Precedence List::
\typeref{readtable},
\typeref{t}

\label Description::

A \term{readtable} maps \term{characters} into \term{syntax types} for 
the \term{Lisp reader}; \seechapter\Syntax.
A \term{readtable} also 
     contains associations between \term{macro characters} 
		               and their \term{reader macro functions},
 and records information about the case conversion rules 
     to be used by the \term{Lisp reader} when parsing \term{symbols}.

\issue{CHARACTER-PROPOSAL:2-1-1}
%% 22.1.5 3
%It is not required that an implementation have characters with non-zero
%\param{bits} and \param{font} \term{attributes}
%syntax descriptions in the \term{readtable}.  
%!!! The following replacement still isn't unambiguous.
%    Mail sent to Quinquevirate asking for advice.
%    -kmp 25-May-91
Each \term{simple} \term{character} must be representable in the \term{readtable}.
It is \term{implementation-defined} whether \term{non-simple} \term{characters}
can have syntax descriptions in the \term{readtable}.
\endissue{CHARACTER-PROPOSAL:2-1-1}

\label See Also::

{\secref\Readtables},
{\secref\PrintingOtherObjects}

\endcom%{readtable}\ftype{System Class}


%-------------------- Reader --------------------

%%% ========== COPY-READTABLE
\begincom{copy-readtable}\ftype{Function}

\label Syntax::

\DefunWithValues {copy-readtable} {{\opt} from-readtable to-readtable} {readtable}

\label Arguments and Values::

\param{from-readtable}---a \term{readtable designator}.
 \Default{the \term{current readtable}}

\param{to-readtable}---a \term{readtable} or \nil.
 \Default{\nil}

\param{readtable}---the \param{to-readtable} if it is \term{non-nil},
  or else a \term{fresh} \term{readtable}.

\label Description::

%% 22.1.5 6
\funref{copy-readtable} copies \param{from-readtable}.

%% 22.1.5 7
If \param{to-readtable} is \nil, a new \term{readtable} is created and returned.  
Otherwise the \term{readtable} specified by \param{to-readtable} is modified and returned.

\issue{READ-CASE-SENSITIVITY:READTABLE-KEYWORDS}
\funref{copy-readtable} copies the setting of \funref{readtable-case}.
\endissue{READ-CASE-SENSITIVITY:READTABLE-KEYWORDS}

\label Examples::

%!!! This is an awful example, and relatively unsafe to just try blindly.
\code
 (setq zvar 123) \EV 123
 (set-syntax-from-char #\\z #\\' (setq table2 (copy-readtable))) \EV T
 zvar \EV 123
 (copy-readtable table2 *readtable*) \EV #<READTABLE 614000277>
 zvar \EV VAR
 (setq *readtable* (copy-readtable)) \EV #<READTABLE 46210223>
 zvar \EV VAR
 (setq *readtable* (copy-readtable nil)) \EV #<READTABLE 46302670>
 zvar \EV 123
\endcode

\label Affected By:\None.

\label Exceptional Situations:\None.

\label See Also::

\typeref{readtable},
\varref{*readtable*}

\label Notes::

\code
(setq *readtable* (copy-readtable nil))
\endcode
restores the input syntax to standard \clisp\ syntax, even if
the \term{initial readtable} has been clobbered 
(assuming it is not so badly clobbered that you cannot type in the above expression).

On the other hand,

\code
(setq *readtable* (copy-readtable))
\endcode
replaces the current \term{readtable} with a copy of itself.
This is useful if you want to save a copy of a readtable for later use,
protected from alteration in the meantime.  It is also useful if you want to
locally bind the readtable to a copy of itself, as in:

\code
(let ((*readtable* (copy-readtable))) ...)
\endcode

\endcom

%%% ========== MAKE-DISPATCH-MACRO-CHARACTER
\begincom{make-dispatch-macro-character}\ftype{Function}

\label Syntax::

\DefunWithValues {make-dispatch-macro-character} 
		 {char {\opt} non-terminating-p readtable}
		 {\t}

\label Arguments and Values::

\issue{ARGUMENTS-UNDERSPECIFIED:SPECIFY}
\param{char}---a \term{character}.
\endissue{ARGUMENTS-UNDERSPECIFIED:SPECIFY}

\param{non-terminating-p}---a \term{generalized boolean}.
 \Default{\term{false}}

\param{readtable}---a \term{readtable}.
 \Default{the \term{current readtable}}

\label Description::

%% 22.1.5 19
\funref{make-dispatch-macro-character} makes \param{char} 
be a \term{dispatching macro character} in \param{readtable}.

Initially, every \term{character} in the dispatch table
associated with the \param{char} has an associated function 
that signals an error \oftype{reader-error}.

If \param{non-terminating-p} is \term{true}, 
the \term{dispatching macro character} 
is made a \term{non-terminating} \term{macro character};
if \param{non-terminating-p} is \term{false},
the \term{dispatching macro character} 
is made a \term{terminating} \term{macro character}.

\label Examples::

\code
 (get-macro-character #\\\lbr) \EV NIL, \term{false}
 (make-dispatch-macro-character #\\\lbr) \EV T
 (not (get-macro-character #\\\lbr)) \EV \term{false}
\endcode

\label Side Effects:\None.
 
%% Sandra thinks this is excessive.
% A \term{pprint dispatch table} is created and initialized.

The \param{readtable} is altered.

\label Affected By:\None.

\label Exceptional Situations:\None.

\label See Also::

\varref{*readtable*}, \funref{set-dispatch-macro-character}

\label Notes:\None.

\endcom

%%% ========== READ
%%% ========== READ-PRESERVING-WHITESPACE
\begincom{read, read-preserving-whitespace}\ftype{Function}

\label Syntax::

\DefunWithValues read 
		 {{\opt} input-stream eof-error-p eof-value recursive-p}
		 {object}

\DefunWithValuesNewline
		 read-preserving-whitespace
       	         {{\opt} \vtop{\hbox{input-stream eof-error-p}
		               \hbox{eof-value recursive-p}}}
		 {object}

\label Arguments and Values::

\param{input-stream}---an \term{input} \term{stream designator}.

\param{eof-error-p}---a \term{generalized boolean}.
 \Default{\term{true}}

\issue{ARGUMENTS-UNDERSPECIFIED:SPECIFY}
\param{eof-value}---an \term{object}.
\endissue{ARGUMENTS-UNDERSPECIFIED:SPECIFY}
 \Default{\nil}

\param{recursive-p}---a \term{generalized boolean}.
 \Default{\term{false}}

\param{object}---an \term{object} (parsed by the \term{Lisp reader})
		 or the \param{eof-value}.

\label Description::

%% 22.2.1 9  
\funref{read} parses the printed representation of an \term{object}
from \param{input-stream} and builds such an \term{object}.

%% 22.2.1 14
\funref{read-preserving-whitespace} is like \funref{read} but preserves
any \term{whitespace}\meaning{2} \term{character}
that delimits the printed representation of the \term{object}.
%% 22.2.1 17                                          
\funref{read-preserving-whitespace} is exactly like \funref{read}
when the \param{recursive-p} \term{argument} to \funref{read-preserving-whitespace}
is \term{true}.

%% 22.2.1 10
%% 22.2.1 13
When \varref{*read-suppress*} is \term{false},
\funref{read} throws away the delimiting \term{character} required by
certain printed representations if it is a
\term{whitespace}\meaning{2} \term{character};
but \funref{read} preserves the character 
(using \funref{unread-char}) if it is
syntactically meaningful, because it could be the start of the next expression.
%
%% Already said in discussion of Double-Quote -kmp 25-Jan-92
% %% 2.5.2 3
% \funref{read} always constructs
% a \term{simple string} when it reads the \term{double-quote} syntax.
%
%% Moved to discussion of Double-Quote -kmp 25-Jan-92
% \issue{CHARACTER-PROPOSAL:2-1-1}
%   It is \term{implementation-dependent} which \term{attributes} are removed
%   from characters within \term{double-quotes}.
% \endissue{CHARACTER-PROPOSAL:2-1-1}
%
%% Already said in discussion of #* -kmp 25-Jan-92
%% 2.5.3 2
% \funref{read} always constructs a \term{simple bit vector} 
% when it reads the \f{\#*} syntax.

If a file ends in a \term{symbol} or a \term{number}
immediately followed by an \term{end of file}\meaning{1},
\funref{read} reads the \term{symbol} or \term{number} successfully;
when called again, it sees the \term{end of file}\meaning{1} and
only then acts according to \param{eof-error-p}.
If a file contains ignorable text at the end, such
as blank lines and comments, \funref{read} 
does not consider it to end in the
middle of an \term{object}.
% Moved to the discussion of tokenizing symbols. -kmp 25-Jan-92
% \issue{CHARACTER-PROPOSAL:2-1-1}
% It is \term{implementation-dependent} which
%   \term{attributes} are removed from \term{symbol} names.
% \endissue{CHARACTER-PROPOSAL:2-1-1}

If \param{recursive-p} is \term{true}, the call to \funref{read} is 
expected to be made
from within some function that itself
has been called from \funref{read} or from a similar input function, rather
than from the top level.  
                                  
           
%% 22.2.1 9
Both functions return the \term{object} read from \param{input-stream}.
\param{Eof-value} is returned if \param{eof-error-p} is \term{false} and end of file
is reached before the beginning of an \term{object}.

\label Examples::

%!!! This example needs work.
\code
 (read)
\OUT \IN{'a}
\EV (QUOTE A)
 (with-input-from-string (is " ") (read is nil 'the-end)) \EV THE-END
 (defun skip-then-read-char (s c n)
    (if (char= c #\\\{) (read s t nil t) (read-preserving-whitespace s))
    (read-char-no-hang s)) \EV SKIP-THEN-READ-CHAR
 (let ((*readtable* (copy-readtable nil)))
    (set-dispatch-macro-character #\\# #\\\{ #'skip-then-read-char)
    (set-dispatch-macro-character #\\# #\\\} #'skip-then-read-char)
    (with-input-from-string (is "#\{123 x #\}123 y")
      (format t "~S ~S" (read is) (read is)))) \EV #\\x, #\\Space, NIL
\endcode

%% 22.2.1 15
As an example, consider this \term{reader macro} definition:

%JonL thinks this is easier to read if LOOP is used.
%  (defun slash-reader (stream char)
%    (declare (ignore char))
%    (do ((path (list (read-preserving-whitespace stream t nil t))
%               (cons (progn (read-char stream t nil t)
%                            (read-preserving-whitespace
%                              stream t nil t))
%                     path)))
%        ((not (eql (peek-char nil stream nil nil t) #\\/))
%         (cons 'path (nreverse path)))))                          

\code
 (defun slash-reader (stream char)
   (declare (ignore char))
   `(path . ,(loop for dir = (read-preserving-whitespace stream t nil t)
                   then (progn (read-char stream t nil t)
                               (read-preserving-whitespace stream t nil t))
                   collect dir
                   while (eql (peek-char nil stream nil nil t) #\\/))))
 (set-macro-character #\\/ #'slash-reader)
\endcode

%% 22.2.1 16
Consider now calling \funref{read} on this expression:

\code
 (zyedh /usr/games/zork /usr/games/boggle)
\endcode
The \f{/} macro reads objects separated by more \f{/} characters;
thus \f{/usr/games/zork} is intended to read as \f{(path usr games zork)}.
The entire example expression should therefore be read as

\code
 (zyedh (path usr games zork) (path usr games boggle))
\endcode
However, if \funref{read} had been used instead of
\funref{read-preserving-whitespace}, then after the reading of the symbol
\f{zork}, the following space would be discarded; the next call
to \funref{peek-char} would see the following \f{/}, and the loop would
continue, producing this interpretation:

\code
 (zyedh (path usr games zork usr games boggle))
\endcode
There are times when \term{whitespace}\meaning{2} should be discarded.
If a command interpreter takes single-character commands,
but occasionally reads an \term{object} then if the \term{whitespace}\meaning{2}
after a \term{symbol} 
is not discarded it might be interpreted as a command
some time later after the \term{symbol} had been read.


\label Affected By::

\varref{*standard-input*},
\varref{*terminal-io*},
\varref{*readtable*},
\varref{*read-default-float-format*},
\varref{*read-base*},
\varref{*read-suppress*},
\varref{*package*},
\varref{*read-eval*}.

\label Exceptional Situations::

%% 22.2.1 3
\funref{read} signals an error \oftype{end-of-file}, 
regardless of \param{eof-error-p}, if
the file ends in the middle of an \term{object} representation.
For example, if a file does
not contain enough right parentheses to balance the left parentheses in
it, \funref{read} signals an error.
This is detected when \funref{read} or \funref{read-preserving-whitespace}
is called with \param{recursive-p} and \param{eof-error-p} \term{non-nil}, 
and end-of-file is reached before the beginning of an \term{object}.

If \param{eof-error-p} is \term{true}, an error \oftype{end-of-file}
is signaled at the end of file.

\label See Also::

\funref{peek-char},
\funref{read-char},
\funref{unread-char},
\funref{read-from-string},
\funref{read-delimited-list}, 
\funref{parse-integer}, 
{\chapref\Syntax},
{\chapref\ReaderConcepts}

\label Notes:\None.

%% Sandra: Just cross-reference "Reader Concepts".
% \param{Recursive-p} should be true if the call to \funref{read} is
% from within some function that itself
% has been called from \funref{read} or from a similar input function, rather
% than from the top level.  For
% instance, a reader macro function that has to read from the input stream
% beyond the \term{macro character} should specify \param{recursive-p} as true.  The
% reasons for this are as follows.  First of all, the scoping of the
% notations \f{\#\i{n}=} and \f{\#\i{n}\#} occurs within a
% top-level call, so calls to \funref{read} 
% from reader macro functions must
% specify \param{recursive-p} as true to ensure that these notations are
% interpreted correctly.  Second, for white space to be preserved correctly
% by low-level calls to \funref{read}  occurring within a call to 
% \funref{read-preserving-whitespace}, \param{recursive-p} must be true.
% Otherwise a low-level call to \funref{read} does not know that it needs to
% preserve white space for the higher-level call.
% Third, this is necessary to distinguish end-of-file within an \term{object}
% from end-of-file between \term{objects}.
\endcom

%%% ========== READ-DELIMITED-LIST
\begincom{read-delimited-list}\ftype{Function}

\label Syntax::

\DefunWithValues read-delimited-list
                 {char {\opt} input-stream recursive-p}
                 {list}

\label Arguments and Values::

\param{char}---a \term{character}.

\param{input-stream}---an \term{input} \term{stream designator}.
 \Default{\term{standard input}}

\param{recursive-p}---a \term{generalized boolean}.
 \Default{\term{false}}

\param{list}---a \term{list} of the \term{objects} read.

\label Description::

%% 22.2.1 18
\funref{read-delimited-list} reads \term{objects} from \param{input-stream} 
until the next character after an \term{object}'s
representation (ignoring \term{whitespace}\meaning{2} characters and comments) is \param{char}.

%% 22.2.1 19
\funref{read-delimited-list} looks ahead at each step
for the next non-\term{whitespace}\meaning{2} \term{character}
and peeks at it as if with \funref{peek-char}.
If it is \param{char}, 
then the \term{character} is consumed and the \term{list} of \term{objects} is returned.
If it is a \term{constituent} or \term{escape} \term{character},
then \funref{read} is used to read an \term{object},
which is added to the end of the \term{list}.
If it is a \term{macro character},
its \term{reader macro function} is called;
if the function returns a \term{value},
that \term{value} is added to the \term{list}.
The peek-ahead process is then repeated.

If \param{recursive-p} is \term{true},
this call is expected to be embedded in a higher-level call to \funref{read} 
or a similar function.

%% 22.2.1 25
It is an error to reach end-of-file during the operation of
\funref{read-delimited-list}.

The consequences are undefined
if \param{char} has a \term{syntax type} of \term{whitespace}\meaning{2}
in the \term{current readtable}.

\label Examples::
\code
 (read-delimited-list #\\\rbracket) 
\OUT \IN{1 2 3 4 5 6 \rbracket}
\EV (1 2 3 4 5 6)
\endcode
%% 22.2.1 22
Suppose you wanted \f{\#\{\i{a} \i{b} \i{c} $\ldots$ \i{z}\}}
to read as a list of all pairs of the elements \i{a}, \i{b}, \i{c}, 
$\ldots$, \i{z}, for example.

\code
 #\{p q z a\}  reads as  ((p q) (p z) (p a) (q z) (q a) (z a))
\endcode
This can be done by specifying a macro-character definition for \f{\#\{}
that does two things: reads in all the items up to the \f{\}},
and constructs the pairs.  \funref{read-delimited-list} performs
the first task.

% (
\code
 (defun |#\{-reader| (stream char arg)
   (declare (ignore char arg))
   (mapcon #'(lambda (x)
              (mapcar #'(lambda (y) (list (car x) y)) (cdr x)))
          (read-delimited-list #\\\} stream t))) \EV |#\{-reader|

 (set-dispatch-macro-character #\\# #\\\{ #'|#\{-reader|) \EV T 
 (set-macro-character #\\\} (get-macro-character #\\) \nil))
\endcode
Note that \term{true} is supplied for the \param{recursive-p} argument.

%% 22.2.1 23
It is necessary here to give a definition to the character \f{\}} as
well to prevent it from being a constituent.
If the line

% (
\code
 (set-macro-character #\\\} (get-macro-character #\\) \nil))
\endcode
shown above were not included, then the \f{\}} in

\code
 #\{ p q z a\}
\endcode
would be considered a constituent character, part of the symbol named
\f{a\}}.  This could be corrected by putting a space before
the \f{\}}, but it is better to call
\funref{set-macro-character}.
                          
%% 22.2.1 24
Giving \f{\}} the same	      % (
definition as the standard definition of the character \f{)} has the
twin benefit of making it terminate tokens for use with 
\funref{read-delimited-list} and also making it invalid for use in any
other context. Attempting to read a stray \f{\}} will signal an error.

\label Affected By::

\varref{*standard-input*},
\varref{*readtable*},
\varref{*terminal-io*}.

\label Exceptional Situations:\None.

\label See Also::

\funref{read},
\funref{peek-char},
\funref{read-char},
\funref{unread-char}.

\label Notes::

%% 22.2.1 20
\funref{read-delimited-list} is intended for use in implementing \term{reader macros}.
Usually it is desirable for \param{char} to be a \term{terminating} \term{macro character}
so that it can be used to delimit tokens; however, \funref{read-delimited-list}
makes no attempt to alter the syntax specified for \param{char} by the current
readtable.  The caller must make any necessary changes to the readtable syntax 
explicitly.

\endcom


%%% ========== READ-FROM-STRING
\begincom{read-from-string}\ftype{Function}

\label Syntax::

\DefunWithValuesNewline
		 read-from-string
                 {string \vtop{\hbox{{\opt} eof-error-p eof-value}
                               \hbox{{\key} start end preserve-whitespace}}}
                 {object, position}

\label Arguments and Values::

\param{string}---a \term{string}.

\param{eof-error-p}---a \term{generalized boolean}.
 \Default{\term{true}}

\issue{ARGUMENTS-UNDERSPECIFIED:SPECIFY}
\param{eof-value}---an \term{object}.
\endissue{ARGUMENTS-UNDERSPECIFIED:SPECIFY}
 \Default{\nil}

\issue{SUBSEQ-OUT-OF-BOUNDS}
\issue{RANGE-OF-START-AND-END-PARAMETERS:INTEGER-AND-INTEGER-NIL}
\param{start}, \param{end}---\term{bounding index designators} of \param{string}.
 \Defaults{\param{start} and \param{end}}{\f{0} and \nil}
\endissue{RANGE-OF-START-AND-END-PARAMETERS:INTEGER-AND-INTEGER-NIL}
\endissue{SUBSEQ-OUT-OF-BOUNDS}

\param{preserve-whitespace}---a \term{generalized boolean}.
 \Default{\term{false}}

\param{object}---an \term{object} (parsed by the \term{Lisp reader})
		 or the \param{eof-value}.

\param{position}---an \term{integer} greater than or equal to zero,
		   and less than or equal to 
		   one more than the \term{length} of the \param{string}. 

\label Description::

%% 22.2.1 40
Parses the printed representation of an \term{object}
from the subsequence of \param{string} \term{bounded} by \param{start} and \param{end},
as if \funref{read} had been called on an \term{input} \term{stream} 
containing those same \term{characters}.

% This seems unnecessary/redundant to me.
% %% 22.2.1 39
% The characters of \param{string} are given successively to the \term{Lisp reader},
% and the \term{object} is built by the reader.

%% 22.2.1 41
If \param{preserve-whitespace} is \term{true},
the operation will preserve \term{whitespace}\meaning{2}
as \funref{read-preserving-whitespace} would do.

%% 22.2.1 43
If an \term{object} is successfully parsed, the \term{primary value}, \param{object},
is the \term{object} that was parsed.
If \param{eof-error-p} is \term{false} and if the end of the \param{substring} is reached,
\param{eof-value} is returned.

The \term{secondary value}, \param{position}, is the index of the first \term{character}
in the \term{bounded} \param{string} that was not read.
The \param{position} may depend upon the value of \param{preserve-whitespace}.
% Amazing.  CLtL really says this.  I wonder why. -kmp 23-Jan-92
If the entire \param{string} was read, 
the \param{position} returned is either the \param{length} of the \param{string} 
or one greater than the \param{length} of the \param{string}.

\label Examples::

%% 22.2.1 44
\code
 (read-from-string " 1 3 5" t nil :start 2) \EV 3, 5
 (read-from-string "(a b c)") \EV (A B C), 7
\endcode

\label Side Effects:\None.

\label Affected By:\None.

\label Exceptional Situations::

%% 22.2.1 42
If the end of the supplied substring
occurs before an \term{object} can be read, an
error is signaled if \param{eof-error-p} is \term{true}.
An error is signaled if the end of the \param{substring} occurs
in the middle of an incomplete \term{object}.

\label See Also::

\funref{read},
\funref{read-preserving-whitespace}

\label Notes::
                                  
% Moon and Sandra seemed to believe this was the motivation.
% I felt we should record the fact. -kmp 24-Jan-92
The reason that \param{position} is allowed to be beyond the
\term{length} of the \param{string} is to permit (but not require)
the \term{implementation} to work by simulating the effect of a 
trailing delimiter at the end of the \term{bounded} \param{string}.
When \param{preserve-whitespace} is \term{true},
the \param{position} might count the simulated delimiter.

\endcom

%%% ========== READTABLE-CASE
\begincom{readtable-case}\ftype{Accessor}
\issue{READ-CASE-SENSITIVITY:READTABLE-KEYWORDS}

\label Syntax::

\DefunWithValues readtable-case {readtable} {mode}

% Fixed broken syntax.  --sjl 16 Mar 92
%\Defsetf readtable-case {readtable mode} {mode}
\Defsetf readtable-case {readtable} {mode}

\label Arguments and Values::

\param{readtable}---a \term{readtable}.

\param{mode}---a \term{case sensitivity mode}.

\label Description::

\term{Accesses} the \term{readtable case} of \param{readtable},
which affects the way in which the \term{Lisp Reader}  reads  \term{symbols}
          and the way in which the \term{Lisp Printer} writes \term{symbols}.

\label Examples::

\Seesection\ReadtableCaseReadExamples\ and \secref\ReadtableCasePrintExamples.

\label Affected By:\None.

\label Exceptional Situations::

\Shouldchecktype{readtable}{a \term{readtable}}
\Shouldchecktype{mode}{a \term{case sensitivity mode}}

\label See Also::

\varref{*readtable*},
\varref{*print-escape*},
{\secref\ReaderAlgorithm},
{\secref\ReadtableCaseReadEffect},
{\secref\ReadtableCasePrintEffect}

\label Notes::

\funref{copy-readtable} copies the \term{readtable case} of the \param{readtable}.

\endissue{READ-CASE-SENSITIVITY:READTABLE-KEYWORDS}
\endcom

%%% ========== READTABLEP
\begincom{readtablep}\ftype{Function}

\label Syntax::

\DefunWithValues readtablep {object} {generalized-boolean}

\label Arguments and Values:: 

\param{object}---an \term{object}.

\param{generalized-boolean}---a \term{generalized boolean}.

\label Description::

%% 22.1.5 8
\TypePredicate{object}{readtable}

\label Examples::

\code
 (readtablep *readtable*) \EV \term{true}
 (readtablep (copy-readtable)) \EV \term{true}
 (readtablep '*readtable*) \EV \term{false}
\endcode

\label Side Effects:\None.

\label Affected By:\None.

\label Exceptional Situations:\None!

\label See Also:\None.

\label Notes::

\code
 (readtablep \param{object}) \EQ (typep \param{object} 'readtable) 
\endcode

\endcom

%%% ========== GET-DISPATCH-MACRO-CHARACTER
%%% ========== SET-DISPATCH-MACRO-CHARACTER
\begincom{set-dispatch-macro-character, get-dispatch-macro-character}\ftype{Function}

\label Syntax::

%% 22.1.5 24
%!!! Name this function1 argument something better. -kmp 15-Oct-91
\DefunWithValues get-dispatch-macro-character
		 {disp-char sub-char {\opt} readtable}
		 {function}

%% 22.1.5 23
\DefunWithValues set-dispatch-macro-character
		 {disp-char sub-char new-function {\opt} readtable}
		 {\t}

\label Arguments and Values::

\param{disp-char}---a \term{character}.

\param{sub-char}---a \term{character}.

\issue{GET-MACRO-CHARACTER-READTABLE:NIL-STANDARD}
\param{readtable}---a \term{readtable designator}.
\endissue{GET-MACRO-CHARACTER-READTABLE:NIL-STANDARD}
 \Default{the \term{current readtable}}

\param{function}---a \term{function designator} or \nil.

%I made this a function designator.  See reasoning for get-macro-character. -kmp 20-Sep-91
\param{new-function}---a \term{function designator}.

\label Description::

\funref{set-dispatch-macro-character} causes \param{new-function} to be called
when \param{disp-char} followed by \param{sub-char} is read.  
%% 22.1.5 22
If \param{sub-char} is a lowercase letter, 
it is converted to its uppercase equivalent.
It is an error if \param{sub-char} is one of the ten decimal digits.

\funref{set-dispatch-macro-character} installs a \param{new-function} to be called
when a particular \term{dispatching macro character} pair is read.
\param{New-function} is installed as the dispatch function to be
called  when \param{readtable} is in use and when \param{disp-char} is followed by 
\param{sub-char}.

%% This information is now available elsewhere.
% 
% %!!! current input stream?  this means the stream on which the char was seen.
% %    be more specific so people don't think this is standard input or something
% %    weird like that.
% The three arguments to \param{new-function} are the current input \term{stream},
% \param{sub-char}, and
%  the \term{integer} whose decimal representation appeared between
%    \param{disp-char} and \param{sub-char},
%   or \nil\ if no decimal integer appeared there; .
For more information about how the \param{new-function} is invoked,
\seesection\MacroChars.

\funref{get-dispatch-macro-character} retrieves
the dispatch function associated with \param{disp-char} and \param{sub-char}
in \param{readtable}.

\funref{get-dispatch-macro-character} returns the macro-character function
for \param{sub-char} under \param{disp-char}, or \nil\ if there is no
function associated with \param{sub-char}.
If \param{sub-char} is a decimal digit, \funref{get-dispatch-macro-character}
returns \nil.

\label Examples::                                        

%!!! The use of (values) here is really questionable! -kmp 1-May-91
\code
 (get-dispatch-macro-character #\\# #\\\{) \EV NIL
 (set-dispatch-macro-character #\\# #\\\{        ;dispatch on #\{
    #'(lambda(s c n)
        (let ((list (read s nil (values) t)))  ;list is object after #n\{
          (when (consp list)                   ;return nth element of list
            (unless (and n (< 0 n (length list))) (setq n 0))
            (setq list (nth n list)))
         list))) \EV T
 #\{(1 2 3 4) \EV 1
 #3\{(0 1 2 3) \EV 3
 #\{123 \EV 123
\endcode
%% 22.1.5 27
If it is desired that \f{\#\$\i{foo}} :
as if it were \f{(dollars \i{foo})}.

\code
(defun |#$-reader| (stream subchar arg)
   (declare (ignore subchar arg))
   (list 'dollars (read stream t nil t))) \EV |#$-reader|
 (set-dispatch-macro-character #\\# #\\\$ #'|#\$-reader|) \EV T
\endcode
%\code
% (get-dispatch-macro-character #\\# #\\\{) \EV NIL 
% (unless (get-dispatch-macro-character #\\# #\\x)
%      (warn "Hexadecimal input (#x<ddd>) is disabled")) \EV NIL 
% (let ((previous-fun (get-dispatch-macro-character #\\# #\\\{)))
%      (when previous-fun
%        (set-dispatch-macro-character #\\# #\\\{
%          #'(lambda (stream char arg)
%               (setq stream *debug-io*)
%               (when *debug-macro-chars*
%                  (format *trace-output* 
%                         "~&Occurrence of ~C~C on stream ~S"
%                                         #\\# #\\\{ stream))
%              (list (funcall previous-fun stream char)))))) \EV NIL 
%\endcode

\label See Also::

{\secref\MacroChars}

\label Side Effects::

The \param{readtable} is modified.

\label Affected By::

\varref{*readtable*}.

\label Exceptional Situations::

%% 22.1.5 26
For either function, an error is signaled if \param{disp-char} is not 
a \term{dispatching macro character} in \param{readtable}.  

\label See Also::

\varref{*readtable*}

\label Notes::
It is necessary
to use \funref{make-dispatch-macro-character} to set up the
dispatch character before specifying its sub-characters.
\endcom

%%% ========== GET-MACRO-CHARACTER
%%% ========== SET-MACRO-CHARACTER
\begincom{set-macro-character, get-macro-character}\ftype{Function}

\label Syntax::

%% 22.1.5 13
\DefunWithValues get-macro-character 
		 {char {\opt} readtable} 
		 {function, non-terminating-p}

\DefunWithValues set-macro-character
		 {char new-function {\opt} non-terminating-p readtable}
		 {\t}

\label Arguments and Values::

\param{char}---a \term{character}.

\param{non-terminating-p}---a \term{generalized boolean}.
 \Default{\term{false}}

\issue{GET-MACRO-CHARACTER-READTABLE:NIL-STANDARD}
\param{readtable}---a \term{readtable designator}.
\endissue{GET-MACRO-CHARACTER-READTABLE:NIL-STANDARD}
 \Default{the \term{current readtable}}

\param{function}---\nil, 
		   or a \term{designator} for a \term{function} of two \term{arguments}.

\param{new-function}---a \term{function designator}.

\label Description::

%%function or function designator? -kmp 29-Apr-91
%%I made it be a function designator since this is consistent with the CLtL 
%%treatment of symbols as functions, and since some implementation might consider
%%it a feature to carry around the name instead of the definition, so the user
%%could redefine the function and have his new definition take automatically.
%% -kmp 20-Sep-91

\funref{get-macro-character} returns as its \term{primary value}, \param{function},
the \term{reader macro function} associated with \param{char} in \param{readtable} (if any),
or else \nil\ if \param{char} is not a \term{macro character} in \param{readtable}.
The \term{secondary value}, \param{non-terminating-p}, is \term{true} 
if \param{char} is a \term{non-terminating} \term{macro character};
otherwise, it is \term{false}.

\funref{set-macro-character} causes \param{char} to be a \term{macro character}
associated with the \term{reader macro function} \param{new-function} 
(or the \term{designator} for \param{new-function}) in \param{readtable}.
If \param{non-terminating-p} is \term{true},
\param{char} becomes a \term{non-terminating} \term{macro character};
otherwise it becomes a     \term{terminating} \term{macro character}.

%% This information can be gotten from the Reader Algorithm. -kmp 23-Jan-92
% In the simplest case, \param{new-function} may return an \term{object}.
% This \term{object} is taken to be that whose printed representation
% was \param{char} and any following characters read by \param{new-function}.
% %% 22.1.5 15
% \param{New-function} may choose instead to return zero values.
% In this case, \param{char} and whatever it may have read
% contribute nothing to the \term{object} being read.
% %% 22.1.5 16
% \param{New-function} should not have any side effects other than on the
% \term{stream};
% because of backtracking and restarting of the \funref{read} operation,
% front ends (such as editors and
% rubout handlers) to the reader may cause
% \param{new-function} to be called repeatedly during the
% reading of a single expression in which \param{char} only appears once.

\label Examples::

\code
 (get-macro-character #\\\lbr) \EV NIL, \term{false}
 (not (get-macro-character #\\;)) \EV \term{false}
\endcode

The following is a possible definition for the \term{single-quote} \term{reader macro} 
in \term{standard syntax}:

\code
 (defun single-quote-reader (stream char)
   (declare (ignore char))
   (list 'quote (read stream t nil t))) \EV SINGLE-QUOTE-READER
 (set-macro-character #\\' #'single-quote-reader) \EV T
\endcode

Here \f{single-quote-reader} reads an \term{object} following the \term{single-quote}
and returns a \term{list} of \specref{quote} and that \term{object}.
The \param{char} argument is ignored.

The following is a possible definition for the \term{semicolon} \term{reader macro}
in \term{standard syntax}:

\code
 (defun semicolon-reader (stream char)
   (declare (ignore char))
   ;; First swallow the rest of the current input line.
   ;; End-of-file is acceptable for terminating the comment.
   (do () ((char= (read-char stream nil #\\Newline t) #\\Newline)))
   ;; Return zero values.
   (values)) \EV SEMICOLON-READER
 (set-macro-character #\\; #'semicolon-reader) \EV T
\endcode


\label Side Effects::

The \param{readtable} is modified.

\label Affected By:\None.

\label Exceptional Situations:\None.

\label See Also::

\varref{*readtable*}

\label Notes:\None.

%% Sandra thinks this is useless.  Enough is said in the argument info
%% above for people to figure it out.  (I agree.) -kmp 23-Jan-92
% \issue{GET-MACRO-CHARACTER-READTABLE:NIL-STANDARD}
% \f{(get-macro-character \i{char} nil)} 
% 
% is equivalent to 
% 
% \f{(get-macro-character \i{char} (copy-readtable))}, 
% 
% but without \term{consing}.
% \endissue{GET-MACRO-CHARACTER-READTABLE:NIL-STANDARD}

\endcom

%%% ========== SET-SYNTAX-FROM-CHAR
\begincom{set-syntax-from-char}\ftype{Function}

\label Syntax::

\issue{RETURN-VALUES-UNSPECIFIED:SPECIFY}
\DefunWithValues set-syntax-from-char
		 {to-char from-char {\opt} to-readtable from-readtable}
		 {\t}
\endissue{RETURN-VALUES-UNSPECIFIED:SPECIFY}

\label Arguments and Values::

\issue{ARGUMENTS-UNDERSPECIFIED:SPECIFY}
\param{to-char}---a \term{character}.

\param{from-char}---a \term{character}.
\endissue{ARGUMENTS-UNDERSPECIFIED:SPECIFY}

\param{to-readtable}---a \term{readtable}. 
 \Default{the \term{current readtable}}

\param{from-readtable}---a \term{readtable designator}.
 \Default{the \term{standard readtable}}
 
\label Description::

%% 22.1.5 9
\funref{set-syntax-from-char} makes
the syntax of \param{to-char}   in \param{to-readtable}    be the same as
the syntax of \param{from-char} in \param{from-readtable}.

%% 22.1.5 10
\funref{set-syntax-from-char} copies the \term{syntax types} of \param{from-char}.
If \param{from-char} is a \term{macro character},
its \term{reader macro function} is copied also.
If the character is a \term{dispatching macro character}, 
its entire dispatch table of \term{reader macro functions} is copied.  
The \term{constituent traits} of \param{from-char} are not copied.

%% 22.1.5 12
A macro definition from a character such as
\f{"} can be copied to another character; the standard definition for \f{"}
looks for another character that is the same as the character that
invoked it.  The definition of \f{(} can not be meaningfully copied
to \f{\{}, on the other hand. 
The result is that \term{lists} are of the form
\f{\{a b c)}, not \f{\{a b c\}},
because the definition
always looks for a closing parenthesis, not a closing brace.  

\label Examples::
\code
 (set-syntax-from-char #\\7 #\\;) \EV T
 123579 \EV 1235
\endcode
                                                             
\label Side Effects::

The \param{to-readtable} is modified.

\label Affected By::

The existing values in the \param{from-readtable}.

\label Exceptional Situations:\None.

\label See Also::

\funref{set-macro-character},
\funref{make-dispatch-macro-character},
{\secref\CharacterSyntaxTypes}

\label Notes::                                        

The \term{constituent traits} of a \term{character} are ``hard wired'' 
into the parser for extended \term{tokens}.  For example, if the definition
of \f{S} is copied to \f{*}, then \f{*} will become a \term{constituent} 
that is \term{alphabetic}\meaning{2} but that cannot be used as a
\term{short float} \term{exponent marker}.
For further information, \seesection\ConstituentTraits.

\endcom

%%% ========== WITH-STANDARD-IO-SYNTAX
\begincom{with-standard-io-syntax}\ftype{Macro}

\label Syntax::

\DefmacWithValues with-standard-io-syntax 
		  {\starparam{form}}
		  {\starparam{result}}

\label Arguments and Values::

\param{forms}---an \term{implicit progn}.

\param{results}---the \term{values} returned by the \term{forms}.

\label Description::

Within the dynamic extent of the body of \param{forms}, all reader/printer control
variables, including any \term{implementation-defined} ones not specified by
this standard, are bound to values that produce standard read/print
behavior.  The values for the variables specified by this standard are listed in 
\thenextfigure.

\reviewer{Barrett: *print-pprint-dispatch* should probably be mentioned here, too.}%!!!
%  We've opened issue WITH-STANDARD-IO-SYNTAX-INCOMPLETE to clarify this. -kmp 1-Sep-91
% I added *print-lines*, *print-miser-width*, *print-pprint-dispatch*, and
% *print-right-margin* to this table.  --sjl 16 Mar 1992

\tablefigtwo{Values of standard control variables}{Variable}{Value}{
   \varref{*package*}                   & \Thepackage{cl-user}          \cr
   \varref{*print-array*}               & \t                            \cr
   \varref{*print-base*}                & \f{10}                        \cr
   \varref{*print-case*}                & \kwd{upcase}                  \cr
   \varref{*print-circle*}              & \nil                          \cr
   \varref{*print-escape*}              & \t                            \cr
   \varref{*print-gensym*}              & \t                            \cr
   \varref{*print-length*}              & \nil                          \cr
   \varref{*print-level*}               & \nil                          \cr
   \varref{*print-lines*}               & \nil                          \cr
   \varref{*print-miser-width*}         & \nil                          \cr
   \varref{*print-pprint-dispatch*}     & The \term{standard pprint dispatch table} \cr
   \varref{*print-pretty*}              & \nil                          \cr
   \varref{*print-radix*}               & \nil                          \cr
   \varref{*print-readably*}            & \t                            \cr
   \varref{*print-right-margin*}        & \nil                          \cr
   \varref{*read-base*}                 & \f{10}                        \cr
   \varref{*read-default-float-format*} & \typeref{single-float}        \cr
   \varref{*read-eval*}                 & \t                            \cr
   \varref{*read-suppress*}             & \nil                          \cr
   \varref{*readtable*}                 & The \term{standard readtable} \cr
}
 
\label Examples::
 
\code
 (with-open-file (file pathname :direction :output)
   (with-standard-io-syntax
     (print data file)))

;;; ... Later, in another Lisp:

 (with-open-file (file pathname :direction :input)
   (with-standard-io-syntax
     (setq data (read file))))
\endcode


\label Affected By:\None.

\label Exceptional Situations:\None.

\label See Also:\None.

\label Notes:\None.

\endcom

%-------------------- Reader Variables --------------------

%%% ========== *READ-BASE*
\begincom{*read-base*}\ftype{Variable}

\label Value Type::

a \term{radix}.

\label Initial Value::

\f{10}.

\label Description::

%% 22.1.2 21
Controls the interpretation of tokens by \funref{read} as being
\term{integers} or \term{ratios}.  

\Thevalueof{*read-base*}, called the \newterm{current input base},
is the radix in which \term{integers} and
\term{ratios} are to be read by the \term{Lisp reader}.
The parsing of other numeric \term{types} (\eg \term{floats}) is 
not affected by this option.

The effect of \varref{*read-base*} on the reading of any particular
\term{rational} number can be locally overridden by explicit use of the
\f{\#O}, \f{\#X}, \f{\#B}, or {\tt \#\i{n}R} syntax
or by a trailing decimal point.

\label Examples::

\code
 (dotimes (i 6)
   (let ((*read-base* (+ 10. i)))
     (let ((object (read-from-string "(\\\\DAD DAD |BEE| BEE 123. 123)")))
       (print (list *read-base* object)))))
\OUT (10 (DAD DAD BEE BEE 123 123))
\OUT (11 (DAD DAD BEE BEE 123 146))
\OUT (12 (DAD DAD BEE BEE 123 171))
\OUT (13 (DAD DAD BEE BEE 123 198))
\OUT (14 (DAD 2701 BEE BEE 123 227))
\OUT (15 (DAD 3088 BEE 2699 123 258))
\EV NIL
\endcode

\label Affected By:\None.

\label See Also:\None.

\label Notes::

%% 22.1.2 22
Altering the input radix can be useful when reading data files in special formats.

\endcom


%%% ========== *READ-DEFAULT-FLOAT-FORMAT*
\begincom{*read-default-float-format*}\ftype{Variable}

\label Value Type::

%% 22.2.1 11
one of the \term{atomic type specifiers} 
    \typeref{short-float},
    \typeref{single-float},
    \typeref{double-float},
 or \typeref{long-float}, 
 or else some other \term{type specifier} defined 
    by the \term{implementation} to be acceptable.

\label Initial Value::

The \term{symbol} \typeref{single-float}.

\label Description::

%% 22.2.1 12
Controls the floating-point format that is to be used when reading a
floating-point number that has no \term{exponent marker} or that has
\f{e} or \f{E} for an \term{exponent marker}.  Other \term{exponent markers}
explicitly prescribe the floating-point format to be used.

The printer uses \varref{*read-default-float-format*} to guide the
choice of \term{exponent markers} when printing floating-point numbers.

\label Examples::

\code
 (let ((*read-default-float-format* 'double-float))
   (read-from-string "(1.0 1.0e0 1.0s0 1.0f0 1.0d0 1.0L0)"))
\EV (1.0   1.0   1.0   1.0 1.0   1.0)   ;Implementation has float format F.
\EV (1.0   1.0   1.0s0 1.0 1.0   1.0)   ;Implementation has float formats S and F.
\EV (1.0d0 1.0d0 1.0   1.0 1.0d0 1.0d0) ;Implementation has float formats F and D.
\EV (1.0d0 1.0d0 1.0s0 1.0 1.0d0 1.0d0) ;Implementation has float formats S, F, D.
\EV (1.0d0 1.0d0 1.0   1.0 1.0d0 1.0L0) ;Implementation has float formats F, D, L.
\EV (1.0d0 1.0d0 1.0s0 1.0 1.0d0 1.0L0) ;Implementation has formats S, F, D, L.
\endcode

\label Affected By:\None.

\label See Also:\None.

\label Notes:\None.

\endcom

%%% ========== *READ-EVAL*
\begincom{*read-eval*}\ftype{Variable}

\issue{DATA-IO:ADD-SUPPORT}

\label Value Type::

a \term{generalized boolean}.

\label Initial Value:: 

\term{true}.

\label Description::

If it is \term{true}, the \f{\#.} \term{reader macro} has its normal effect.
Otherwise, that \term{reader macro} signals an error \oftype{reader-error}.

\label Examples:\None.

\label Affected By:\None.

\label See Also::

\varref{*print-readably*}

\label Notes::

If \varref{*read-eval*} is \term{false} and \varref{*print-readably*} is \term{true}, 
any \term{method} for \funref{print-object} that would output a reference
to the \f{\#.} \term{reader macro} either outputs something different
or signals an error \oftype{print-not-readable}.

\endissue{DATA-IO:ADD-SUPPORT}

\endcom

%%% ========== *READ-SUPPRESS*
\begincom{*read-suppress*}\ftype{Variable}

\label Value Type::

a \term{generalized boolean}.

\label Initial Value::

\term{false}.

\label Description::

%% 22.1.2 25
This variable is intended primarily to support the operation of the
read-time conditional notations \f{\#+} and \f{\#-}. It is important for the
\term{reader macros} which implement these notations
to be able to skip over the printed representation of an
\term{expression} despite the possibility that the syntax of the skipped
\term{expression} may not be entirely valid for the current implementation,
since \f{\#+} and \f{\#-} exist in order to allow the same program to be
shared among several \Lisp\ implementations (including dialects other than \clisp)
despite small incompatibilities of syntax.

%% 22.1.2 24
%% 22.1.2 26
If it is \term{false}, the \term{Lisp reader} operates normally.

\issue{READ-SUPPRESS-CONFUSING:GENERALIZE}
% Otherwise, many of the normal operations of the \term{Lisp reader}
% are suppressed; specifically:
% 
% %% 22.1.2 27
% \beginlist
% \itemitem{Extended tokens}
% 
% An extended token is discarded and treated as if it were \nil;
% that is, reading an extended token when \varref{*read-suppress*} is \term{true} returns \nil.
% 
% %% 22.1.2 28
% \itemitem{\f{\#}}
% 
% Any standard \f{\#} \term{dispatching macro character} notation
% that requires, permits, or disallows an infix numerical argument,
% such as \f{\#\i{n}R}, does not enforce any constraint on the presence,
% absence, or value of such an argument.
% 
% %% 22.1.2 29
% \itemitem{\tt \#\\}
% 
% The \f{\#\\} notation parses as \nil.
% It does not signal an error even if an unknown character name is seen.
% 
% %% 22.1.2 30
% \itemitem{\tt \#B, \#O, \#X, \#R}
% 
% Each of the \f{\#B}, \f{\#O}, \f{\#X}, and \f{\#R} notations 
% scans over a following token and produces the value \nil.
% None of these notations signals an error 
% even if the token does not have the syntax of a rational number.
% 
% %% 22.1.2 31
% \itemitem{\f{\#*}}
% 
% \issue{SHARP-STAR-DELIMITER:NORMAL-DELIMITER}
% The \f{\#*} notation scans a (possibly null) token until a normal delimiter appears
% and produces the value \nil.  It does not signal an error even if the token
% contains characters other of the characters \f{0} and \f{1}, or is of the wrong length.
% \endissue{SHARP-STAR-DELIMITER:NORMAL-DELIMITER}
% 
% %% 22.1.2 32
% \issue{SHARP-COMMA-CONFUSION:REMOVE}
% \itemitem{\tt \#.}                                        
% 
% The \f{\#.} notation reads the following \term{object} in 
% suppressed mode but does not evaluate it.
% The \term{object} is discarded and \nil\ is produced.
% \endissue{SHARP-COMMA-CONFUSION:REMOVE}
% 
% %% 22.1.2 33
% \itemitem{\tt \#A, \#C, \#S, \#:}
% 
% Each of the \f{\#A}, \f{\#C}, \f{\#S}, and \f{\#:} notations reads the following
% \term{object} in suppressed mode but does not interpret it in any way;
% it need not even be a \term{list} in the case of \f{\#C} or \f{\#S}, or a 
% \term{symbol} in the case of \f{\#:}.  The \term{object} is 
% discarded and \nil\ is produced.

If \thevalueof{*read-suppress*} is \term{true},
      \funref{read},
      \funref{read-preserving-whitespace},
      \funref{read-delimited-list},
  and \funref{read-from-string}
all return a \term{primary value} of \nil\ when they complete successfully;
however, they continue to parse the representation of an \term{object}
in the normal way, in order to skip over the \term{object},
and continue to indicate \term{end of file} in the normal way.
Except as noted below,
any \term{standardized} \term{reader macro}\meaning{2}
that is defined to \term{read}\meaning{2} 
a following \term{object} or \term{token} 
will do so,
but not signal an error if the \term{object}
read is not of an appropriate type or syntax.
The \term{standard syntax} and its associated \term{reader macros}
will not construct any new \term{objects} 
 (\eg when reading the representation of a \term{symbol},
      no \term{symbol} will be constructed or interned).

\endissue{READ-SUPPRESS-CONFUSING:GENERALIZE}
      
%% 22.1.2 27
\beginlist

\issue{READ-SUPPRESS-CONFUSING:GENERALIZE}

%% The cleanup issue did not ask me to put this entry for Extended
%% Tokens here, but people have complained to me personally that this
%% passage (present in CLtL1) was omitted in the dpANS and that they
%% were worried that these items were not guaranteed.  I believe that
%% issue READ-SUPPRESS-CONFUSING asks to omit this information because
%% it thinks it is redundant, not because it thinks it is incorrect.
%% But I disagree on the redundancy issue, so I am ignoring it and
%% including this data regardless.  Refer to p345 of CLtL1 for
%% supporting documentation.  -kmp 23-Aug-93

\itemitem{Extended tokens}

All extended tokens are completely uninterpreted.
Errors such as those that might otherwise be signaled due to
     detection of invalid \term{potential numbers},
     invalid patterns of \term{package markers},
 and invalid uses of the \term{dot} character are suppressed.

\itemitem{Dispatching macro characters (including \term{sharpsign})}

\term{Dispatching macro characters} continue to parse an infix numerical
argument, and invoke the dispatch function.  The \term{standardized}
\term{sharpsign} \term{reader macros} do not enforce any constraints
on either the presence of or the value of the numerical argument.

\endissue{READ-SUPPRESS-CONFUSING:GENERALIZE}

%% 22.1.2 34
\itemitem{\tt \#=}

The \f{\#=} notation is totally ignored.  It does not read
a following \term{object}. It produces no \term{object},
but is treated as \term{whitespace}\meaning{2}.

%% 22.1.2 35
\itemitem{\tt \#\#}

The \f{\#\#} notation always produces \nil.
\endlist

No matter what \thevalueof{*read-suppress*},
parentheses still continue to delimit and construct \term{lists};
the \f{\#(} notation continues to delimit \term{vectors};
and comments, \term{strings}, 
and the \term{single-quote} and \term{backquote} notations continue to be
interpreted properly.  Such situations as
\f{')},	\f{\#<}, % (
\f{\#)}, and \f{\#\SpaceChar} continue to signal errors.

\label Examples::

\issue{READ-SUPPRESS-CONFUSING:GENERALIZE}
% \code
%  (let ((*read-suppress* t))
%    (dotimes (i 4)
%      (format t "~&input here> ")
%      (format t "~&parsed as: ~S~%" (read))))
% \OUT input here> 101
% \OUT parsed as: NIL
% \OUT input here> (#\\a :test)
% \OUT parsed as: (NIL NIL)
% \OUT input here> '("xyz" #(a b c))
% \OUT parsed as: (QUOTE ("xyz" #(NIL NIL NIL)))
% \OUT input here> (list 1 2 '3)
% \OUT parsed as: (NIL NIL NIL (QUOTE NIL))
% \EV NIL
% \endcode
\code
 (let ((*read-suppress* t))
   (mapcar #'read-from-string
           '("#(foo bar baz)" "#P(:type :lisp)" "#c1.2"
             "#.(PRINT 'FOO)" "#3AHELLO" "#S(INTEGER)"
             "#*ABC" "#\\GARBAGE" "#RALPHA" "#3R444")))
\EV (NIL NIL NIL NIL NIL NIL NIL NIL NIL NIL)
\endcode
\endissue{READ-SUPPRESS-CONFUSING:GENERALIZE}

\label Affected By:\None.

\label See Also::

\funref{read},
{\chapref\Syntax}

\label Notes::

\issue{READ-SUPPRESS-CONFUSING:GENERALIZE}
\term{Programmers} and \term{implementations} that define additional
\term{macro characters} are strongly encouraged to make them respect
\varref{*read-suppress*} just as \term{standardized} \term{macro characters} do.
That is, when \thevalueof{*read-suppress*} is \term{true},
they should ignore type errors when reading a following \term{object}
and the \term{functions} that implement \term{dispatching macro characters}
should tolerate \nil\ as their infix \term{parameter} value even if a numeric
value would ordinarily be required.
\endissue{READ-SUPPRESS-CONFUSING:GENERALIZE}

\endcom

%%% ========== *READTABLE*
\begincom{*readtable*}\ftype{Variable}

\label Value Type::

a \term{readtable}.

\label Initial Value::

A \term{readtable} that conforms to the description of \clisp\ syntax in \chapref\Syntax.

\label Description::

%% 22.1.5 4
\Thevalueof{*readtable*} is called the \term{current readtable}.
It controls the parsing behavior of the \term{Lisp reader},
and can also influence the \term{Lisp printer} (\eg \seefun{readtable-case}).

\label Examples::

%!!! This example needs work.
\code
 (readtablep *readtable*) \EV \term{true}
 (setq zvar 123) \EV 123
 (set-syntax-from-char #\\z #\\' (setq table2 (copy-readtable))) \EV T
 zvar \EV 123
 (setq *readtable* table2) \EV #<READTABLE>
 zvar \EV VAR
 (setq *readtable* (copy-readtable nil)) \EV #<READTABLE>
 zvar \EV 123
\endcode

\label Affected By::

\funref{compile-file},
\funref{load}

\label See Also::

\funref{compile-file},
\funref{load},
\typeref{readtable},
{\secref\CurrentReadtable}

\label Notes:\None.

\endcom

%%% ========== READER-ERROR
\begincom{reader-error}\ftype{Condition Type}

\issue{PARSE-ERROR-STREAM:SPLIT-TYPES}

\label Class Precedence List::
\typeref{reader-error},
\typeref{parse-error},
\typeref{stream-error},
\typeref{error},
\typeref{serious-condition},
\typeref{condition},
\typeref{t}

\label Description::

\Thetype{reader-error} consists of
error conditions that are related to tokenization and parsing
done by the \term{Lisp reader}.

\label See Also::

\funref{read},
\funref{stream-error-stream},
{\secref\ReaderConcepts}

\endissue{PARSE-ERROR-STREAM:SPLIT-TYPES}

\endcom%{reader-error}\ftype{Condition Type}
