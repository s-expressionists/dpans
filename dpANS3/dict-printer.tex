% -*- Mode: TeX -*-

% Pretty Printer Concepts
% Printer
%  Printer Variables

\def\writekeys#1{\vtop{\hbox{array base case circle escape gensym}
		       \hbox{length level lines miser-width pprint-dispatch}
		       \hbox{pretty radix readably right-margin #1}}}

\def\writekeydescriptions#1{\param{array}---a \term{generalized boolean}.

\param{base}---a \term{radix}.

\param{case}---a \term{symbol} of \term{type} \f{(member :upcase :downcase :capitalize)}.

\param{circle}---a \term{generalized boolean}.

\param{escape}---a \term{generalized boolean}.

\param{gensym}---a \term{generalized boolean}.

\param{length}---a non-negative \term{integer}, or \nil.

\param{level}---a non-negative \term{integer}, or \nil.

\param{lines}---a non-negative \term{integer}, or \nil.

\param{miser-width}---a non-negative \term{integer}, or \nil.

\param{pprint-dispatch}---a \term{pprint dispatch table}.

\param{pretty}---a \term{generalized boolean}.

\param{radix}---a \term{generalized boolean}.

\param{readably}---a \term{generalized boolean}.

\param{right-margin}---a non-negative \term{integer}, or \nil.

#1}


%-------------------- Printer --------------------

%%% ========== COPY-PPRINT-DISPATCH

\begincom{copy-pprint-dispatch}\ftype{Function}

%KMP: I'd like to change the name of this function to COPY-PPRINT-DISPATCH-TABLE.

\issue{PRETTY-PRINT-INTERFACE}

\label Syntax::

\DefunWithValues copy-pprint-dispatch {{\opt} table} {new-table}

\label Arguments and Values::

\param{table}---a \term{pprint dispatch table}, or \nil.

\param{new-table}---a \term{fresh} \term{pprint dispatch table}.

\label Description::

Creates and returns a copy of the specified \param{table}, 
or of \thevalueof{*print-pprint-dispatch*} if no \param{table} is specified,
or of the initial \term{value} of \varref{*print-pprint-dispatch*} if \nil\ is specified.

\label Examples:\None.

\label Side Effects:\None.

%% Sandra thinks this is excessive.
%Creates a \term{pprint dispatch table}.

\label Affected By:\None.

\label Exceptional Situations::

Should signal an error \oftype{type-error} if \param{table} 
is not a \term{pprint dispatch table}.

\label See Also:\None.

\label Notes:\None.

\endissue{PRETTY-PRINT-INTERFACE}

\endcom

%%% ========== FORMATTER
\begincom{formatter}\ftype{Macro}

\issue{PRETTY-PRINT-INTERFACE}

\label Syntax::

\DefmacWithValues formatter {control-string} {function}

\label Arguments and Values::

\param{control-string}---a \term{format string}; \noeval.

\param{function}---a \term{function}.

\label Description::

Returns a \term{function} which has behavior equivalent to:

\code
  #'(lambda (*standard-output* &rest arguments)
      (apply #'format t \i{control-string} arguments)
      \i{arguments-tail})
\endcode

where \i{arguments-tail} is either the tail of \i{arguments} 
which has as its \term{car} the argument that would be processed next 
if there were more format directives in the \param{control-string}, 
or else \nil\ if no more \i{arguments} follow the most recently
processed argument.

\label Examples::

%This doesn't look like valid syntax to me.  --sjl 16 mar 92
%\code
%((formatter "~&~A~A") *standard-output* 'a 'b 'c)
%\OUT AB
%\EV (C)
\code
(funcall (formatter "~&~A~A") *standard-output* 'a 'b 'c)
\OUT AB
\EV (C)

(format t (formatter "~&~A~A") 'a 'b 'c)
\OUT AB
\EV NIL
\endcode

\label Side Effects:\None.

%% Sandra thinks this is excessive.
%Might create a \term{closure} at run time.

\label Affected By:\None.

\label Exceptional Situations::

Might signal an error (at macro expansion time or at run time) if the argument
is not a valid \term{format string}.

\label See Also::

\funref{format}

\label Notes:\None.

\endissue{PRETTY-PRINT-INTERFACE}

\endcom

%%% ========== PPRINT-DISPATCH

\begincom{pprint-dispatch}\ftype{Function}

\issue{PRETTY-PRINT-INTERFACE}

\label Syntax::

\DefunWithValues pprint-dispatch {object {\opt} table} {function, found-p}

\label Arguments and Values::

\param{object}---an \term{object}.

%!!! pprint dispatch table designator?? -kmp 16-Oct-91
\param{table}---a \term{pprint dispatch table}, or \nil.
 \Default{\thevalueof{*print-pprint-dispatch*}}

% It wasn't obvious if this should be a "function" or "function designator".
% I asked x3j13 with issue PPRINT-DISPATCH-RETURN-VALUE, but rather than vote
% on it, they just told me to use my discretion.  Which is basically the same
% as if they had approved the issue.  "function designator" it is. -kmp 11-Feb-92
\param{function}---a \term{function designator}.

\param{found-p}---a \term{generalized boolean}.

\label Description::

Retrieves the highest priority function in \param{table} that is
associated with a \term{type specifier} that matches \param{object}.
The function is chosen by finding all of the \term{type specifiers} in \param{table}
that match the \param{object} and 
selecting the highest priority function associated with any of these
\term{type specifiers}.  If there is more than one highest priority function,
an arbitrary choice is made.  If no \term{type specifiers} match the 
\param{object}, a function is returned that prints \param{object} 
\issue{GENERALIZE-PRETTY-PRINTER:UNIFY}
%with \varref{*print-pretty*} bound to \nil.
using \funref{print-object}.
\endissue{GENERALIZE-PRETTY-PRINTER:UNIFY}

The \term{secondary value}, \param{found-p}, is \term{true} if a matching
\term{type specifier} was found in \param{table}, or \term{false} otherwise.

If \param{table} is \nil,
retrieval is done in the
%initial value of \varref{*print-pprint-dispatch*}.
\term{initial pprint dispatch table}.

\label Examples:\None.

\label Side Effects:\None.

\label Affected By::

The state of the \param{table}.

\label Exceptional Situations::

Should signal an error \oftype{type-error} if \term{table} is neither a
\term{pprint-dispatch-table} nor \nil.

\label See Also:\None.

\label Notes::

\code
(let ((*print-pretty* t))
  (write object :stream s))
\EQ (funcall (pprint-dispatch object) s object)
\endcode

\endissue{PRETTY-PRINT-INTERFACE}

\endcom

%%% ========== PPRINT-EXIT-IF-LIST-EXHAUSTED
\begincom{pprint-exit-if-list-exhausted}\ftype{Local Macro}

\issue{PRETTY-PRINT-INTERFACE}

\label Syntax::

\DefmacWithValues pprint-exit-if-list-exhausted {\noargs} {\nil}

\label Arguments and Values:\None.

\label Description::

Tests whether or not the \param{list} passed to 
the \term{lexically current logical block}
has been exhausted; \seesection\DynamicControlofOutput.
If this \param{list} has been
reduced to \nil, \funref{pprint-exit-if-list-exhausted} terminates the execution
of the \term{lexically current logical block} except for the printing
of the suffix.  Otherwise \funref{pprint-exit-if-list-exhausted} returns \nil. 

\issue{LEXICAL-CONSTRUCT-GLOBAL-DEFINITION:UNDEFINED}
Whether or not \macref{pprint-exit-if-list-exhausted} is \term{fbound} in the
\term{global environment} is \term{implementation-dependent};
however, the restrictions on redefinition and \term{shadowing} of
\macref{pprint-exit-if-list-exhausted} are the same as for \term{symbols} in \thepackage{common-lisp}
which are \term{fbound} in the \term{global environment}.
The consequences of attempting to use \macref{pprint-exit-if-list-exhausted} outside
of \macref{pprint-logical-block} are undefined.
\endissue{LEXICAL-CONSTRUCT-GLOBAL-DEFINITION:UNDEFINED}

\label Examples:\None.

\label Side Effects:\None.

\label Affected By:\None.

\label Exceptional Situations::

An error is signaled (at macro expansion time or at run time) if
\funref{pprint-exit-if-list-exhausted} is used anywhere other than
%syntactically nested
lexically within a call on \macref{pprint-logical-block}.  
% It is undefined what happens if \funref{pprint-if-list-exhausted} is executed outside
% of the dynamic extent of this \funref{pprint-logical-block}.
Also, the consequences of executing \funref{pprint-if-list-exhausted} outside
of the dynamic extent of the \macref{pprint-logical-block} which lexically
contains it are undefined.

\label See Also::

\macref{pprint-logical-block}, \funref{pprint-pop}.

\label Notes:\None.

\endissue{PRETTY-PRINT-INTERFACE}

\endcom

%%% ========== PPRINT-FILL
%%% ========== PPRINT-LINEAR
%%% ========== PPRINT-TABULAR

\begincom{pprint-fill, pprint-linear, pprint-tabular}\ftype{Function}

\issue{PRETTY-PRINT-INTERFACE}

\label Syntax::

\DefunWithValues pprint-fill    {stream object {\opt} colon-p at-sign-p} {\nil}
\DefunWithValues pprint-linear  {stream object {\opt} colon-p at-sign-p} {\nil}
\DefunWithValues pprint-tabular {stream object {\opt} colon-p at-sign-p tabsize} {\nil}

\label Arguments and Values::

\param{stream}---an \term{output} \term{stream designator}.

\param{object}---an \term{object}.

\param{colon-p}---a \term{generalized boolean}.
  \Default{\term{true}}

\param{at-sign-p}---a \term{generalized boolean}.
  \Default{\term{implementation-dependent}}

\param{tabsize}---a non-negative \term{integer}.
  \Default{\f{16}}

\label Description::

The functions \funref{pprint-fill}, \funref{pprint-linear}, and
\funref{pprint-tabular} specify particular ways of \term{pretty printing}
a \term{list} to \param{stream}. 
Each function prints parentheses around the output if and only
if \param{colon-p} is \term{true}.
Each function ignores its \param{at-sign-p} argument.
(Both arguments are included even though only one is needed
so that these functions can be used via \formatOp{/.../}
and as \funref{set-pprint-dispatch} functions, as well as directly.)
Each function handles abbreviation and the detection of circularity
and sharing correctly, and uses \funref{write} to print \param{object}
when it is a \term{non-list}.
% KMP: The pretty printer proposal as adopted was ambiguous 
%  in the preceding sentence.  I tried to resolve the ambiguity
%  by changing the argument name and doing some rewording.
%  Dick Waters should double-check this interpretation.
% Waters: Is ok.

\issue{GENERALIZE-PRETTY-PRINTER:UNIFY}
If \param{object} is a \term{list} and 
if \thevalueof{*print-pretty*} is \term{false},
each of these functions prints \param{object}
using a minimum of \term{whitespace},
as described in \secref\PrintingListsAndConses.
Otherwise (if \param{object} is a \term{list} and 
	   if \thevalueof{*print-pretty*} is \term{true}):

\beginlist

\itemitem{\bull}
\Thefunction{pprint-linear} prints a \term{list} either all on one line, 
or with each \term{element} on a separate line.  

\itemitem{\bull}
\Thefunction{pprint-fill} prints a \term{list} with as many \term{elements}
as possible on each line.

\itemitem{\bull}
\Thefunction{pprint-tabular} is the same as \funref{pprint-fill}
except that it prints the \term{elements} so that they line up in columns.
The \param{tabsize} specifies the column spacing in \term{ems},
% KMP: Is this (1) the minimum intercolumn spacing, or 
%  (2) the total spacing from the left edge of one column 
%      to the left edge of the next?
% Waters: (2)
which is the total spacing from the leading edge of one column to
the leading edge of the next.
\endlist

\endissue{GENERALIZE-PRETTY-PRINTER:UNIFY}

\label Examples::

Evaluating the following with a line length of \f{25} produces the output shown.
 
\code
(progn (princ "Roads ") 
       (pprint-tabular *standard-output* '(elm main maple center) nil nil 8))
Roads ELM     MAIN
      MAPLE   CENTER
\endcode

\label Side Effects::

Performs output to the indicated \term{stream}.

\label Affected By::

The cursor position on the indicated \term{stream}, if it can be determined.

\label Exceptional Situations:\None.

\label See Also:\None.

\label Notes::

\Thefunction{pprint-tabular} could be defined as follows:

\code
(defun pprint-tabular (s list &optional (colon-p t) at-sign-p (tabsize nil))
  (declare (ignore at-sign-p))
  (when (null tabsize) (setq tabsize 16))
  (pprint-logical-block (s list :prefix (if colon-p "(" "")
                                :suffix (if colon-p ")" ""))
    (pprint-exit-if-list-exhausted)
    (loop (write (pprint-pop) :stream s)
          (pprint-exit-if-list-exhausted)
          (write-char #\\Space s)
          (pprint-tab :section-relative 0 tabsize s)
          (pprint-newline :fill s))))
\endcode

Note that it would have been inconvenient to specify this function
using \funref{format}, because of the need to pass its \param{tabsize} argument
through to a \formatdirective{:T} nested within an iteration over a list.

\endissue{PRETTY-PRINT-INTERFACE}

\endcom

%%% ========== PPRINT-INDENT

\begincom{pprint-indent}\ftype{Function}

\issue{PRETTY-PRINT-INTERFACE}

\label Syntax::

\DefunWithValues pprint-indent {relative-to n {\opt} stream} {\nil}

\label Arguments and Values::

\param{relative-to}---either \kwd{block} or \kwd{current}.

\param{n}---a \term{real}.

\param{stream}---an \term{output} \term{stream designator}.
  \Default{\term{standard output}}

\label Description::

\funref{pprint-indent} specifies the indentation to use in a logical block on \term{stream}.
\issue{GENERALIZE-PRETTY-PRINTER:UNIFY}
If \param{stream} is a \term{pretty printing stream}
   and \thevalueof{*print-pretty*} is \term{true},
\funref{pprint-indent} sets the indentation in the innermost 
dynamically enclosing logical block;
otherwise, \funref{pprint-indent} has no effect.
\endissue{GENERALIZE-PRETTY-PRINTER:UNIFY}

\param{N} specifies the indentation in
\term{ems}.  If \param{relative-to} is \kwd{block}, the indentation is set
to the horizontal position of the first character in the \term{dynamically current logical block} plus \param{n} 
\term{ems}. If \param{relative-to} is \kwd{current}, the indentation is set
to the current output position plus \param{n} \term{ems}. (For robustness
in the face of variable-width fonts, it is advisable to use \kwd{current}
with an \param{n} of zero whenever possible.)

\param{N} can be negative;
however, the total indentation cannot be moved 
left of the beginning of the line 
or left of the end of the rightmost per-line prefix---an attempt to move beyond
one of these limits is treated
% KMP: How?
%     (1) Is an error signaled, 
%  or (2) or do we just min out at 0?
% Waters: (2).
the same as an attempt to move to that limit.
Changes in indentation caused by \param{pprint-indent}
do not take effect until after the next line break.  In addition, in 
miser mode all calls to \funref{pprint-indent} are ignored, forcing the lines 
corresponding to the logical block to line up under the first character in
the block.

\label Examples:\None.

\label Side Effects:\None.

\label Affected By:\None.

\label Exceptional Situations::

An error is signaled if \param{relative-to} is any \term{object} other
than \kwd{block} or \kwd{current}.

\label See Also::

{\secref\TildeI}

\label Notes:\None.

\endissue{PRETTY-PRINT-INTERFACE}

\endcom

%%% ========== PPRINT-LOGICAL-BLOCK

\begincom{pprint-logical-block}\ftype{Macro}

\issue{DECLS-AND-DOC}

\issue{PRETTY-PRINT-INTERFACE}

\label Syntax::

\DefmacWithValuesNewline pprint-logical-block
		  	 {\vtop{\hbox{\paren{stream-symbol object 
					     {\key} prefix per-line-prefix suffix}}
				\hbox{\starparam{declaration} \starparam{form}}}}
			 {\nil}

\label Arguments and Values:: 

\param{stream-symbol}---a \term{stream variable designator}.

\param{object}---an \term{object}; \eval.

\kwd{prefix}---a \term{string}; \eval.
	       \HairyDefault.

\kwd{per-line-prefix}---a \term{string}; \eval.
		        \HairyDefault.

\kwd{suffix}---a \term{string}; \eval.
	       \Default{the \term{null} \term{string}}

\param{declaration}---a \misc{declare} \term{expression}; \noeval.

\param{forms}---an \term{implicit progn}.

\label Description::

Causes printing to be grouped into a logical block.

The logical block is printed to the \term{stream} that is the \term{value}
of the \term{variable} denoted by \param{stream-symbol}.
During the execution of the \term{forms},
that \term{variable} is \term{bound} to a \term{pretty printing stream}
that supports decisions about the arrangement of output
and then forwards the output to the destination stream.
\issue{GENERALIZE-PRETTY-PRINTER:UNIFY}
% If, during that context, \varref{*print-pretty*} becomes bound to \nil, 
% the stream ceases to behave as a \term{pretty printing stream}.
\endissue{GENERALIZE-PRETTY-PRINTER:UNIFY}
All the standard printing functions 
  (\eg \funref{write},
       \funref{princ},
   and \funref{terpri})
can be used to print output to the \term{pretty printing stream}.
All and only the output sent to this \term{pretty printing stream}
is treated as being in the logical block.
 
The \param{prefix} specifies a prefix to be printed before the beginning of
the logical block.
The \param{per-line-prefix} specifies a prefix that is printed before the block
and at the beginning of each new line in the block.
The \kwd{prefix} and \kwd{pre-line-prefix} \term{arguments} are mutually exclusive.
If neither \kwd{prefix} nor \kwd{per-line-prefix} is specified,
a \param{prefix} of the \term{null} \term{string} is assumed.

The \param{suffix} specifies a suffix that is printed just after the logical block.

The \param{object} is 
%interpreted as 
normally 
a \term{list} that the body \param{forms} are responsible for printing.
If \param{object} is not a \term{list},
it is printed using \funref{write}.
(This makes it easier to write printing functions that are robust
 in the face of malformed arguments.)
If \varref{*print-circle*}
%%There's no such var. -kmp 14-Jan-91
%(and possibly \varref{*print-shared*}) 
is \term{non-nil} and \param{object} is a circular (or shared) reference to a \term{cons},
then an appropriate ``\f{\#\i{n}\#}'' marker is printed.  (This
makes it easy to write printing functions that provide full support
for circularity and sharing abbreviation.)  If \varref{*print-level*} is not
\nil\ and the logical block is at a dynamic nesting depth of greater
than \varref{*print-level*} in logical blocks, ``\f{\#}'' is printed.
(This makes easy to write printing functions that provide full support for depth
abbreviation.)
 
If either of the three conditions above occurs, the indicated output is
printed on \param{stream-symbol} and the body \param{forms} are skipped 
along with the printing of the \kwd{prefix} and \kwd{suffix}.
(If the body \param{forms} are not to be responsible for printing a list,
then the first two tests above can be turned off by supplying \nil\ for
the \param{object} argument.)
 
In addition to the \param{object} argument of \macref{pprint-logical-block},
the arguments of the standard printing functions (such as \funref{write}, 
\funref{print}, \funref{prin1}, and \funref{pprint}, as well as the arguments
of the standard \term{format directives} such as \formatOp{A}, \formatOp{S},
(and \formatOp{W}) are all checked (when necessary) for circularity and sharing.
However, such checking is not applied to the arguments of the
functions \funref{write-line}, \funref{write-string}, and \funref{write-char}
or to the literal text output by \funref{format}.  A consequence of this is
that you must use one of the latter functions if you want to print some 
literal text in the output that is not supposed to be checked for circularity
or sharing.
% (See the examples below.)

The body \param{forms} of a \macref{pprint-logical-block} \term{form}
must not perform any side-effects on the surrounding environment; for 
example, no \term{variables} must be assigned which have not been
\term{bound} within its scope.  
% (See notes below for rationale.)

\issue{GENERALIZE-PRETTY-PRINTER:UNIFY}
\Themacro{pprint-logical-block} may be used regardless of \thevalueof{*print-pretty*}.
\endissue{GENERALIZE-PRETTY-PRINTER:UNIFY}

\label Examples:\None.

\label Side Effects:\None.

\label Affected By::

\varref{*print-circle*}, \varref{*print-level*}.

\label Exceptional Situations::

An error \oftype{type-error} is signaled if any of the \kwd{suffix}, 
\kwd{prefix}, or \kwd{per-line-prefix} is supplied but does not evaluate 
to a \term{string}.

An error is signaled if \kwd{prefix} and \kwd{pre-line-prefix} are both used.

\macref{pprint-logical-block} and the \term{pretty printing stream} it creates 
have \term{dynamic extent}.  The consequences are undefined if, outside
of this extent, output is attempted to the \term{pretty printing stream} it creates.

It is also unspecified what happens if, within this extent, any output is 
sent directly to the underlying destination stream.

\label See Also::

\macref{pprint-pop},
\macref{pprint-exit-if-list-exhausted},
{\secref\TildeLessThanLogicalBlock}

\label Notes::

\issue{GENERALIZE-PRETTY-PRINTER:UNIFY}
One reason for using \themacro{pprint-logical-block} when \thevalueof{*print-pretty*}
is \nil\ would be to allow it to perform checking for \term{dotted lists}, 
as well as (in conjunction with \macref{pprint-pop}) 
checking for \varref{*print-level*} or \varref{*print-length*} being exceeded.
\endissue{GENERALIZE-PRETTY-PRINTER:UNIFY}

Detection of circularity and sharing is supported by the \term{pretty printer}
by in essence performing requested output twice.  On the first pass,
circularities and sharing are detected and the actual outputting of characters
is suppressed.  On the second pass, the appropriate ``\f{\#\param{n}=}'' 
and ``\f{\#\param{n}\#}'' markers are inserted and characters are output.
This is why the restriction on side-effects is necessary.
Obeying this restriction is facilitated by using \macref{pprint-pop},
instead of an ordinary \macref{pop} when traversing a list being printed by 
the body \param{forms} of the \macref{pprint-logical-block} \term{form}.)

\endissue{PRETTY-PRINT-INTERFACE}

\endissue{DECLS-AND-DOC}

\endcom

%%% ========== PPRINT-NEWLINE

\begincom{pprint-newline}\ftype{Function}

\issue{PRETTY-PRINT-INTERFACE}

\label Syntax::

\DefunWithValues pprint-newline {kind {\opt} stream} {\nil}

\label Arguments and Values::

\param{kind}---one of \kwd{linear}, \kwd{fill}, \kwd{miser}, or \kwd{mandatory}.

\param{stream}---a \term{stream designator}.
 \Default{\term{standard output}}

\label Description::

%% This follows from being a stream designator. -kmp 27-Aug-93
% \param{Stream} defaults to \term{standard output}.
% If it is \nil, \term{standard output} is used instead.
% If it is \t, \term{terminal I/O} is used instead.

\issue{GENERALIZE-PRETTY-PRINTER:UNIFY}
If \param{stream} is a \term{pretty printing stream}
   and \thevalueof{*print-pretty*} is \term{true},
a line break is inserted in the output 
when the appropriate condition below is satisfied;
otherwise, \funref{pprint-newline} has no effect.
\endissue{GENERALIZE-PRETTY-PRINTER:UNIFY}

\param{Kind} specifies the style of conditional newline.
This \term{parameter} is treated as follows:

\beginlist

\item{\kwd{linear}}

This specifies a
``linear-style'' \term{conditional newline}.\idxtext{linear-style conditional newline}
A line break is inserted 
if and only if the immediately containing \term{section}
cannot be printed on one line.
The effect of this is that line breaks are
either inserted at every linear-style conditional newline in a logical block
or at none of them.

\item{\kwd{miser}}

This specifies a 
``miser-style'' \term{conditional newline}.\idxtext{miser-style conditional newline}
A line break is inserted
if and only if the immediately containing \term{section}
cannot be printed on one line
and miser style is in effect in the immediately containing logical block.
The effect of this is that miser-style conditional newlines
act like linear-style conditional newlines,
but only when miser style is in effect.
Miser style is in effect for a logical block if and only if 
the starting position of the logical block
is less than or equal to
\varref{*print-miser-width*} \term{ems} from the right margin.

\item{\kwd{fill}}

This specifies a
``fill-style'' \term{conditional newline}.\idxtext{fill-style conditional newline}
A line break is inserted if and only if
either (a) the following \term{section} cannot be printed 
	   on the end of the current line,
       (b) the preceding \term{section} was not printed on a single line,
    or (c) the immediately containing \term{section} cannot
	   be printed on one line and miser style is in effect
	   in the immediately containing logical block.
If a logical block is broken up into a number of subsections
by fill-style conditional newlines,
the basic effect is that the logical block
is printed with as many subsections as possible on each line.
However, if miser style is in effect,
fill-style conditional newlines act like linear-style conditional newlines.

\item{\kwd{mandatory}}

This specifies a
``mandatory-style'' \term{conditional newline}.\idxtext{mandatory-style conditional newline}
A line break is always inserted.
This implies that  none of the containing \term{sections}
can be printed on a single line and 
will therefore trigger the insertion of line breaks 
at linear-style conditional newlines in these \term{sections}.

\endlist

When a line break is inserted by any type of conditional newline, 
any blanks that immediately precede the conditional newline are omitted 
from the output and indentation is introduced at the beginning of the next line.
By default, the indentation causes the following line to begin 
in the same horizontal position 
as the first character in the immediately containing logical block.
(The indentation can be changed via \funref{pprint-indent}.)

There are a variety of ways unconditional newlines can be introduced into
the output (\ie via \funref{terpri} or by printing a string containing a newline
character).  As with mandatory conditional newlines, this prevents any of
the containing \term{sections} from being printed on one line.  In general, when
an unconditional newline is encountered, it is printed out without
suppression of the preceding blanks and without any indentation following
it.  However, if a per-line prefix has been specified (see
\macref{pprint-logical-block}), this prefix will always be printed no matter 
how a newline originates.

\label Examples::

\Seesection\PrettyPrinterExamples.

\label Side Effects::

Output to \param{stream}.

\label Affected By::

\varref{*print-pretty*}, \varref{*print-miser-width*}.
The presence of containing logical blocks.
The placement of newlines and conditional newlines.

\label Exceptional Situations::

An error \oftype{type-error} is signaled if \param{kind} 
is not one of \kwd{linear}, \kwd{fill}, \kwd{miser}, or \kwd{mandatory}.

\label See Also::

{\secref\TildeUnderscore},
{\secref\PrettyPrinterExamples}

\label Notes:\None.

\endissue{PRETTY-PRINT-INTERFACE}

\endcom

%%% ========== PPRINT-POP

\begincom{pprint-pop}\ftype{Local Macro}

\issue{PRETTY-PRINT-INTERFACE}

\label Syntax::

\DefmacWithValues pprint-pop {\noargs} {object}

\label Arguments and Values::

\param{object}---an \term{element} of the \term{list} 
		 being printed in the \term{lexically current logical block},
		 or \nil.

\label Description::

Pops one \term{element} from the \term{list} being printed
in the \term{lexically current logical block}, obeying \varref{*print-length*}
and \varref{*print-circle*} as described below.

Each time \macref{pprint-pop} is called, it pops the next value off the
\term{list} passed to the \term{lexically current logical block} and returns it.
However, before doing this, it performs three tests:

\beginlist
 \item{\bull}
    If the remaining `list' is not a \term{list}, 
    ``\f{. }''\idxterm{dot} is printed followed by the remaining `list.' 
    (This makes it easier to write printing functions that 
    are robust in the face of malformed arguments.)

 \item{\bull}
    If \varref{*print-length*} is \term{non-nil},
    and \macref{pprint-pop} has already been called \varref{*print-length*} times 
    within the immediately containing logical block,
    ``\f{...}''\idxtext{Dot Dot Dot}\idxcode{...} is printed.
    (This makes it easy to write printing functions that properly handle
    \varref{*print-length*}.)

 \item{\bull}
    If \varref{*print-circle*} is
    \term{non-nil}, and the remaining list is a circular (or shared) reference,
    then ``\f{. }'' is printed followed by an appropriate
    ``\f{\#\i{n}\#}''\idxtext{Sharpsign Sharpsign (reader macro)} marker.
    (This catches instances of \term{cdr} circularity and sharing in lists.)
\endlist

If either of the three conditions above occurs, the indicated output is
printed on the \term{pretty printing stream} created by the immediately containing
\macref{pprint-logical-block} and the execution of the immediately containing
\macref{pprint-logical-block} is terminated except for the printing of the suffix.

%KMP: I find this next paragraph fairly undecipherable.
If \macref{pprint-logical-block} is given a `list' argument of \nil---because
it is not processing a list---\macref{pprint-pop} can still be used to obtain 
support for \varref{*print-length*}.  
In this situation, the first and third tests above are disabled and
\macref{pprint-pop} always returns \nil.
\Seesection\PrettyPrinterExamples---specifically, the \funref{pprint-vector} example.

\issue{LEXICAL-CONSTRUCT-GLOBAL-DEFINITION:UNDEFINED}
Whether or not \macref{pprint-pop} is \term{fbound} in the
\term{global environment} is \term{implementation-dependent};
however, the restrictions on redefinition and \term{shadowing} of
\macref{pprint-pop} are the same as for \term{symbols} in \thepackage{common-lisp}
which are \term{fbound} in the \term{global environment}.
The consequences of attempting to use \macref{pprint-pop} outside
of \macref{pprint-logical-block} are undefined.
\endissue{LEXICAL-CONSTRUCT-GLOBAL-DEFINITION:UNDEFINED}


\label Examples:\None.

\label Side Effects::

Might cause output
%This next info supplied by Waters. -kmp 16-Oct-91
to the \term{pretty printing stream} associated with the lexically current logical block.

\label Affected By::

\varref{*print-length*}, \varref{*print-circle*}.

\label Exceptional Situations::

An error is signaled (either at macro expansion time or at run time)
if a usage of \macref{pprint-pop} occurs where there is no lexically
containing \macref{pprint-logical-block} \term{form}.

The consequences are undefined if \macref{pprint-pop} is executed outside
of the \term{dynamic extent} of this \macref{pprint-logical-block}.

\label See Also::

\macref{pprint-exit-if-list-exhausted}, \macref{pprint-logical-block}.

\label Notes::

It is frequently a good idea to call \macref{pprint-exit-if-list-exhausted} 
before calling \macref{pprint-pop}.

\endissue{PRETTY-PRINT-INTERFACE}

\endcom

%%% ========== PPRINT-TAB

\begincom{pprint-tab}\ftype{Function}

\issue{PRETTY-PRINT-INTERFACE}

\label Syntax::

\DefunWithValues pprint-tab {kind colnum colinc {\opt} stream} {\nil}

\label Arguments and Values::

\param{kind}---one of \kwd{line}, \kwd{section}, \kwd{line-relative},
	       or \kwd{section-relative}.

\param{colnum}---a non-negative \term{integer}.

\param{colinc}---a non-negative \term{integer}.

\param{stream}---an \term{output} \term{stream designator}.

\label Description::

Specifies tabbing to \param{stream} as performed by the standard \formatdirective{T}.
\issue{GENERALIZE-PRETTY-PRINTER:UNIFY}
If \param{stream} is a \term{pretty printing stream} and
   \thevalueof{*print-pretty*} is \term{true},
\endissue{GENERALIZE-PRETTY-PRINTER:UNIFY}
tabbing is performed;
otherwise, \funref{pprint-tab} has no effect.

The arguments \param{colnum} and \param{colinc} correspond to the two 
\param{parameters} to \formatOp{T} and are in terms of \term{ems}.
The \param{kind} argument specifies the style of tabbing.  It must be one of
  \kwd{line} (tab as by \formatOp{T}),
  \kwd{section} (tab as by \formatOp{:T}, 
		 but measuring horizontal positions relative to 
		 the start of the dynamically enclosing section),
  \kwd{line-relative} (tab as by \formatOp{@T}), or
  \kwd{section-relative} (tab as by \formatOp{:@T}, 
			  but measuring horizontal positions relative to
			  the start of the dynamically enclosing section).

\label Examples:\None.

\label Side Effects:\None.

\label Affected By:\None.

\label Exceptional Situations::

An error is signaled if \param{kind} is not one of \kwd{line},
\kwd{section}, \kwd{line-relative}, or \kwd{section-relative}.

\label See Also::

\macref{pprint-logical-block}

\label Notes:\None.

\endissue{PRETTY-PRINT-INTERFACE}

\endcom

%%% ========== PRINT-OBJECT
\begincom{print-object}\ftype{Standard Generic Function}
 
\label Syntax::
 
\DefgenWithValues print-object {object stream} {object}
 
\label Method Signatures::
 
\Defmeth {print-object} {\paren{\param{object} standard-object} \param{stream}}
\issue{DEFSTRUCT-PRINT-FUNCTION-AGAIN:X3J13-MAR-93}
\Defmeth {print-object} {\paren{\param{object} structure-object} \param{stream}}
\endissue{DEFSTRUCT-PRINT-FUNCTION-AGAIN:X3J13-MAR-93}

\label Arguments and Values::
                                
\param{object}---an \term{object}.
 
\param{stream}---a \term{stream}.
%% KMP: It should not be necessary to say this.
% it cannot be the \term{symbols} \t\ or \nil. 
 
\label Description::
 
\TheGF{print-object} writes the printed representation of \param{object}
to \param{stream}.
\Thefunction{print-object} is called by the \term{Lisp printer}; 
it should not be called by the user.

\issue{DEFSTRUCT-PRINT-FUNCTION-AGAIN:X3J13-MAR-93}
Each implementation is required to provide a \term{method} on
\theclass{standard-object} and on \theclass{structure-object}.
In addition, each \term{implementation} must provide 
\term{methods} on enough other \term{classes} 
so as to ensure that there is always an applicable \term{method}.
Implementations are free to add \term{methods} for other \term{classes}.
Users may write \term{methods} for \funref{print-object} for their own
\term{classes} if they do not wish to inherit an 
\term{implementation-dependent} \term{method}.

The \term{method} on \theclass{structure-object} prints the object in the
default \f{\#S} notation; \seesection\PrintingStructures.
\endissue{DEFSTRUCT-PRINT-FUNCTION-AGAIN:X3J13-MAR-93}

%\term{Methods} on \funref{print-object} must 
%obey the print control special variables.
%The specific details are the following:

\term{Methods} on \funref{print-object} are responsible for implementing
their part of the semantics of the \term{printer control variables}, as follows:
 
\beginlist
 
\issue{DATA-IO:ADD-SUPPORT}

\item{\varref{*print-readably*}}

All methods for \funref{print-object} must obey \varref{*print-readably*}.
This includes both user-defined methods and \term{implementation-defined} methods.
Readable printing of \term{structures} and \term{standard objects} 
is controlled by their \funref{print-object} method, 
not by their \funref{make-load-form} \term{method}.
\term{Similarity} for these \term{objects} is application dependent
and hence is defined to be whatever these \term{methods} do; 
\seesection\Similarity.
 
\endissue{DATA-IO:ADD-SUPPORT}

\item{\varref{*print-escape*}}

Each \term{method} must implement \varref{*print-escape*}. 
 
\item{\varref{*print-pretty*}} 

\issue{GENERALIZE-PRETTY-PRINTER:UNIFY}
% The \varref{*print-pretty*} control variable can be ignored
% by most \term{methods} other than the one for \term{lists}.
The \term{method} may wish to perform specialized line breaking
or other output conditional on \thevalueof{*print-pretty*}.
For further information,
see (for example) the \term{macro} \funref{pprint-fill}.
See also \secref\PPrintDispatchTables\ and \secref\PrettyPrinterExamples.
\endissue{GENERALIZE-PRETTY-PRINTER:UNIFY}
 
\item{\varref{*print-length*}}

\term{Methods} that produce output of indefinite length must obey
\varref{*print-length*}.
\issue{GENERALIZE-PRETTY-PRINTER:UNIFY}
%, but most \term{methods} other than the one for 
%\term{lists} can ignore it.
For further information,
see (for example) the \term{macros} \macref{pprint-logical-block}
and \macref{pprint-pop}.
See also \secref\PPrintDispatchTables\ and \secref\PrettyPrinterExamples.
\endissue{GENERALIZE-PRETTY-PRINTER:UNIFY} 

\item{\varref{*print-level*}}

The printer takes care of \varref{*print-level*} automatically,
provided that each \term{method} handles exactly one level of structure and
calls \funref{write} (or an equivalent \term{function}) recursively if
there are more structural levels.  The printer's decision of whether an
\term{object} has components (and therefore should not be printed when the
printing depth is not less than \varref{*print-level*}) is
\term{implementation-dependent}.  In some implementations its 
\funref{print-object} \term{method} is not called; 
in others the \term{method} is called, 
and the determination that the \term{object} has components is based on what
it tries to write to the \param{stream}.
     
\item{\varref{*print-circle*}}

\issue{PRINT-CIRCLE-STRUCTURE:USER-FUNCTIONS-WORK}
When \thevalueof{*print-circle*} is \term{true},
a user-defined 
\issue{DEFSTRUCT-PRINT-FUNCTION-AGAIN:X3J13-MAR-93}
%print function
\funref{print-object} \term{method}
\endissue{DEFSTRUCT-PRINT-FUNCTION-AGAIN:X3J13-MAR-93}
can print \term{objects} to the supplied \term{stream} 
using \funref{write},
      \funref{prin1},
      \funref{princ},
   or \funref{format}
and expect circularities to be detected 
and printed using the \f{\#\i{n}\#} syntax.
If a user-defined
\issue{DEFSTRUCT-PRINT-FUNCTION-AGAIN:X3J13-MAR-93}
%print function
\funref{print-object} \term{method}
\endissue{DEFSTRUCT-PRINT-FUNCTION-AGAIN:X3J13-MAR-93}
prints to a \term{stream} other than the one
that was supplied, then circularity detection starts over for that
\term{stream}.  See \varref{*print-circle*}.
\endissue{PRINT-CIRCLE-STRUCTURE:USER-FUNCTIONS-WORK}

\item{\varref{*print-base*},
      \varref{*print-radix*},
      \varref{*print-case*}, 
      \varref{*print-gensym*},
  and \varref{*print-array*}}

These \term{printer control variables} apply to specific types of \term{objects}
and are handled by the \term{methods} for those \term{objects}.

\endlist
 
If these rules are not obeyed, the results are undefined.
 
%!!! What is the motivation for this?
In general, the printer and the \funref{print-object} methods should not
rebind the print control variables as they operate recursively through the
structure, but this is \term{implementation-dependent}.
 
In some implementations the \param{stream} argument passed to a 
\funref{print-object} \term{method} is not the original \term{stream}, 
but is an intermediate \term{stream} that implements part of the printer.
\term{methods} should therefore not depend on the identity of this \term{stream}.
 
%All of the existing printing functions (\funref{write}, 
%\funref{prin1}, \funref{print}, \funref{princ}, \funref{pprint},
%\funref{write-to-string}, \funref{prin1-to-string}, 
%\funref{princ-to-string}, the \formatOp{S} and \formatOp{A}
%format operations, and the \formatOp{B}, \formatOp{D},
%\formatOp{E}, \formatOp{F}, \formatOp{G}, \formatOp{\$},
%\formatOp{O}, \formatOp{R}, and \formatOp{X}
%format operations when they encounter a non-numeric
%value) are required to be changed to go through the \funref{print-object}
%generic function.  Each implementation is required to
%replace its former implementation of printing with one or more
%\funref{print-object} methods.  
%% Flushed per X3J13. -kmp 05-Oct-93
% Exactly which \term{classes} have \term{methods} for \funref{print-object} 
% is not specified; it would be valid for an implementation to have one
% default \term{method} that is inherited by all system-defined \term{classes}.
 
%% Per X3J13. -kmp 05-Oct-93
\label Examples:\None.
 
\label Affected By:\None.
 
\label Exceptional Situations:\None.
 
\label See Also::
 
\funref{pprint-fill},
\macref{pprint-logical-block},
\macref{pprint-pop},
\funref{write},
\varref{*print-readably*},
\varref{*print-escape*},
\varref{*print-pretty*},
\varref{*print-length*},
{\secref\DefaultPrintObjMeths},
\issue{DEFSTRUCT-PRINT-FUNCTION-AGAIN:X3J13-MAR-93}
{\secref\PrintingStructures},
\endissue{DEFSTRUCT-PRINT-FUNCTION-AGAIN:X3J13-MAR-93}
{\secref\PPrintDispatchTables},
{\secref\PrettyPrinterExamples}

\label Notes:\None.
 
\endcom

%%% ========== PRINT-UNREADABLE-OBJECT
\begincom{print-unreadable-object}\ftype{Macro}
\issue{DATA-IO:ADD-SUPPORT}

\label Syntax::

\DefmacWithValues print-unreadable-object
		  {\paren{object stream {\key} type identity} \starparam{form}}
		  {\nil}

\label Arguments and Values::

\param{object}---an \term{object}; \eval.

\param{stream}---%
%% Changed per Barrett #8 (first public review) -kmp 14-May-93
% the \term{name} of a \term{lexical variable} or \term{dynamic variable},
% 		 the \term{value} of which is a \term{stream}; 
% 		 used both unevaluated and evaluated.
a \term{stream designator}; \eval.

\param{type}---a \term{generalized boolean}; \eval.

\param{identity}---a \term{generalized boolean}; \eval.

\param{forms}---an \term{implicit progn}.
%!!! it is evaluated. also used unevaluated to determine if any forms were present. hmmm...

\label Description::

Outputs a printed representation of \param{object} on \param{stream},
beginning with ``\f{\#<}'' and ending with ``\f{>}''.  
Everything output to \param{stream} by the body \param{forms}
is enclosed in the the angle brackets.
If \param{type} is \term{true}, the output from \param{forms}
is preceded by a brief description of the \param{object}'s
\term{type} and a space character.
If \param{identity} is \term{true},
the output from \param{forms} is followed by a space character
and a representation of the \param{object}'s identity, 
typically a storage address.
 
If either \param{type} or \param{identity} is not supplied, 
its value is \term{false}.  It is valid to omit the body \param{forms}.  
If \param{type} and \param{identity} are both true and there are no
body \param{forms}, only one space character separates the type 
and the identity.
 
\label Examples::
 
;; Note that in this example, the precise form of the output
;; is \term{implementation-dependent}.

\code
 (defmethod print-object ((obj airplane) stream)
   (print-unreadable-object (obj stream :type t :identity t)
     (princ (tail-number obj) stream)))

 (prin1-to-string my-airplane)
\EV "#<Airplane NW0773 36000123135>"
\OV "#<FAA:AIRPLANE NW0773 17>"
\endcode

\label Affected By:\None.

\label Exceptional Situations::

If \varref{*print-readably*} is \term{true}, \macref{print-unreadable-object}
signals an error \oftype{print-not-readable} without printing anything.
 
\label See Also:\None.

\label Notes:\None.

\endissue{DATA-IO:ADD-SUPPORT}

\endcom

%%% ========== SET-PPRINT-DISPATCH
\begincom{set-pprint-dispatch}\ftype{Function}

\issue{PRETTY-PRINT-INTERFACE}

\label Syntax::

\DefunWithValues set-pprint-dispatch {type-specifier function {\opt} priority table} {\nil}

\label Arguments and Values:: 

\param{type-specifier}---a \term{type specifier}.
% %This is now a valid type specifier! -kmp 30-Apr-91
% 		         or a list of the form 
% 			  \f{(cons \param{type-specifier} \param{type-specifier})}
% 			 or {\tt (cons \param{type-specifier})}.
%    \editornote{KMP: Can the type-specifier in a cons specifier also be a 
% 		    usage of CONS?  Why don't we extend the CONS type specifier
% 		    rather than make this special case?}

\param{function}---a \term{function}, a \term{function name}, or \nil.

\param{priority}---a \term{real}.
 \Default{\f{0}}

\param{table}---a \term{pprint dispatch table}.
 \Default{\thevalueof{*print-pprint-dispatch*}}

\label Description::

Installs an entry into the \term{pprint dispatch table} which is \param{table}.

\param{Type-specifier} 
%% already said above. -kmp
%must be a valid \term{type specifier} and
is the \term{key}
of the entry.  The first action of \funref{set-pprint-dispatch} is to remove any
pre-existing entry associated with \param{type-specifier}.  This guarantees that
there will never be two entries associated with the same \term{type specifier} 
in a given \term{pprint dispatch table}.  Equality of \term{type specifiers} is
tested by \funref{equal}.

Two values are associated with each \term{type specifier} in a 
\term{pprint dispatch table}: a \param{function} and a \param{priority}.
The \param{function} must accept two arguments: the \term{stream} to which output
is sent and the \term{object} to be printed.  The \param{function} should
\term{pretty print} the \term{object} to the \param{stream}.  The \param{function}
can assume that object satisfies the \term{type} given by \term{type-specifier}.
The \param{function} must obey \varref{*print-readably*}.
%(see issue DATA-IO)
Any values returned by the \param{function} are ignored.

\param{Priority} is a priority to resolve conflicts
when an object matches more than one entry.

It is permissible for \param{function} to be \nil.  In this situation, 
there will be no \param{type-specifier} entry in \param{table} after
\funref{set-pprint-dispatch} returns.

\label Examples:\None.

\label Side Effects:\None.

\label Affected By:\None.

\label Exceptional Situations::

An error is signaled if \param{priority} is not a \term{real}.

\label See Also:\None.

\label Notes::

Since \term{pprint dispatch tables} are often used to control the pretty
printing of Lisp code, it is common for the \param{type-specifier} to be
an \term{expression} of the form 

\code
 (cons \param{car-type} \param{cdr-type})
\endcode

This signifies that the corresponding object must be a cons cell 
whose \term{car} matches the \term{type specifier} \param{car-type} 
and whose \term{cdr} matches the \term{type specifier} \param{cdr-type}.
The \param{cdr-type} can be omitted in which case it defaults to \misc{t}.

%% I don't think this note had any content. -kmp 13-Jan-91
% ----------------------------------------------------------------------
% 
% Implementation note:  The restriction above is very useful to users
% without actually limiting what Common Lisp implementors can do.  It is
% possible for implementors to set up any kind of pretty printing they
% desire using the range of priorities available to them.
% 
% ----------------------------------------------------------------------


\endissue{PRETTY-PRINT-INTERFACE}

\endcom

%%% ========== PRIN1
%%% ========== PRINC
%%% ========== PRINT
%%% ========== PPRINT
%%% ========== WRITE
\begincom{write, prin1, print, pprint, princ}\ftype{Function}

%% 22.3.1 2
\label Syntax::

%% 22.3.1 5
\DefunWithValuesNewline write
			{\param{object} {\key} \writekeys{stream}}
			{object}

\DefunWithValues prin1  {object {\opt} output-stream} {object}
\DefunWithValues princ  {object {\opt} output-stream} {object}
\DefunWithValues print  {object {\opt} output-stream} {object}
\DefunWithValues pprint {object {\opt} output-stream} {\novalues}

%% 22.3.1 1                                
%% 22.3.1 3
\label Arguments and Values::

\param{object}---an \term{object}.

\param{output-stream}---an \term{output} \term{stream designator}.
		        \Default{\term{standard output}}

\writekeydescriptions{\param{stream}---an \term{output} \term{stream designator}.
\Default{\term{standard output}}}

\label Description::

\funref{write}, \funref{prin1}, \funref{princ}, \funref{print}, and \funref{pprint}
write the printed representation of \param{object} to \param{output-stream}.

%!!! Does WRITE actually bind all the vars?? -kmp 8-May-91
\funref{write} is the general entry point to the \term{Lisp printer}.
For each explicitly supplied \term{keyword parameter} named in \thenextfigure,
the corresponding \term{printer control variable} is dynamically bound to its \term{value}
while printing goes on;
for each \term{keyword parameter} in \thenextfigure\ that is not explicitly supplied,
the value of the corresponding \term{printer control variable} is the same as it was
at the time \funref{write} was invoked.
Once the appropriate \term{bindings} are \term{established},
the \term{object} is output by the \term{Lisp printer}.

\tablefigtwo{Argument correspondences for the WRITE function.}{Parameter}{Corresponding Dynamic Variable}{
 \param{array} & \varref{*print-array*} \cr
 \param{base} & \varref{*print-base*} \cr
 \param{case} & \varref{*print-case*} \cr
 \param{circle} & \varref{*print-circle*} \cr
 \param{escape} & \varref{*print-escape*} \cr
 \param{gensym} & \varref{*print-gensym*} \cr
 \param{length} & \varref{*print-length*} \cr
 \param{level} & \varref{*print-level*} \cr
 \param{lines} & \varref{*print-lines*} \cr
 \param{miser-width} & \varref{*print-miser-width*} \cr
 \param{pprint-dispatch} & \varref{*print-pprint-dispatch*} \cr
 \param{pretty} & \varref{*print-pretty*} \cr
 \param{radix} & \varref{*print-radix*} \cr
 \param{readably} & \varref{*print-readably*} \cr
 \param{right-margin} & \varref{*print-right-margin*} \cr
}

\issue{PRETTY-PRINT-INTERFACE}
\funref{prin1}, \funref{princ}, \funref{print}, and \funref{pprint} implicitly 
\term{bind} certain print parameters to particular values.  The remaining parameter
values are taken from 
     \varref{*print-array*},
     \varref{*print-base*},
     \varref{*print-case*}, 
     \varref{*print-circle*},
     \varref{*print-escape*},
     \varref{*print-gensym*},
     \varref{*print-length*},
     \varref{*print-level*},
     \varref{*print-lines*},
     \varref{*print-miser-width*},
     \varref{*print-pprint-dispatch*},
     \varref{*print-pretty*}, 
     \varref{*print-radix*}, 
 and \varref{*print-right-margin*}.
\endissue{PRETTY-PRINT-INTERFACE}

\funref{prin1} produces output suitable for input to \funref{read}.  
It binds \varref{*print-escape*} to \term{true}.
%% 2.5.2 3
%% 2.5.3 2
% I don't think this belongs here.  --sjl 16 Mar 92
%\funref{prin1} prints any \term{string}
%or \term{bit vector}, not just a simple one, but \funref{read} will 
%always construct a \term{simple string} or \term{simple bit vector} 
%when it reads the appropriate syntax.

%% 22.3.1 8
\funref{princ} is just like \funref{prin1} except that the
output has no \term{escape} \term{characters}.
It binds \varref{*print-escape*}   to \term{false}
\issue{PRINC-READABLY:X3J13-DEC-91}
     and \varref{*print-readably*} to \term{false}.
\endissue{PRINC-READABLY:X3J13-DEC-91}
% I don't think this belongs here.  --sjl 16 Mar 92
%A \term{symbol} is printed as the characters of its \term{name};
%a \term{string} is printed without surrounding \term{double-quotes};
%and there might be differences for other \term{types} as well.  
The general rule is that output from \funref{princ} is intended to look
good to people, while output from \funref{prin1} is intended to
be acceptable to \funref{read}.
% I don't think this belongs here.  --sjl 16 Mar 92
%\issue{PRINC-CHARACTER}
%For \term{characters},
%
%\code
% (princ character output-stream) \EQ (write-char character output-stream)
%\endcode
%\endissue{PRINC-CHARACTER}

%% 22.3.1 6
\funref{print} is just like \funref{prin1} 
except that the printed representation
of \param{object} is preceded by a newline 
and followed by a space.

%% 22.3.1 7
\funref{pprint} is just like \funref{print} except that the trailing
space is omitted and 
\param{object} is printed with the \varref{*print-pretty*} flag \term{non-nil} 
to produce pretty output.

\param{Output-stream} specifies the \term{stream} to which 
output is to be sent.  

\label Affected By::

\varref{*standard-output*},
\varref{*terminal-io*},
\varref{*print-escape*},
\varref{*print-radix*},
\varref{*print-base*},
\varref{*print-circle*},
\varref{*print-pretty*},
\varref{*print-level*},
\varref{*print-length*},
\varref{*print-case*},
\varref{*print-gensym*},
\varref{*print-array*},
\varref{*read-default-float-format*}.

\label Exceptional Situations:\None.

\label See Also::

\funref{readtable-case},
{\secref\FORMATPrinterOps}

\label Notes::

\Thefunctions{prin1} and \funref{print} do not bind \varref{*print-readably*}.

\code
 (prin1 object output-stream)
\EQ (write object :stream output-stream :escape t)
\endcode

\issue{PRINC-READABLY:X3J13-DEC-91}
\code
 (princ object output-stream)
\EQ (write object stream output-stream :escape nil :readably nil)
\endcode
\endissue{PRINC-READABLY:X3J13-DEC-91}

\code
 (print object output-stream)
\EQ (progn (terpri output-stream)
           (write object :stream output-stream
                         :escape t)
           (write-char #\\space output-stream))
\endcode

\code
 (pprint object output-stream)
\EQ (write object :stream output-stream :escape t :pretty t)
\endcode
\endcom

%%% ========== PRIN1-TO-STRING
%%% ========== PRINC-TO-STRING
%%% ========== WRITE-TO-STRING
\begincom{write-to-string, prin1-to-string, princ-to-string}\ftype{Function}

\label Syntax::

\DefunWithValuesNewline write-to-string
			{object {\key} \writekeys{}}
			{string}

\DefunWithValues prin1-to-string {object} {string}

\DefunWithValues princ-to-string {object} {string}

\label Arguments and Values:: 

\param{object}---an \term{object}.

\writekeydescriptions{}

\param{string}---a \term{string}.

\label Description::

\funref{write-to-string}, \funref{prin1-to-string}, and \funref{princ-to-string}
are used to create a \term{string} consisting of the printed representation 
of \param{object}.
%% 22.3.1 11
\param{Object} is effectively printed as if by \funref{write},
\funref{prin1}, or \funref{princ}, respectively,
and the \term{characters} that would be output are made 
into a \term{string}.

\funref{write-to-string} is the general output function.
It has the ability to specify all the parameters applicable 
to the printing of \param{object}.

\funref{prin1-to-string} acts like \funref{write-to-string} with 
\f{:escape t}, that is, escape characters are written where appropriate.

\funref{princ-to-string} acts like \funref{write-to-string} with 
\issue{PRINC-READABLY:X3J13-DEC-91}
\f{:escape nil :readably nil}.  
\endissue{PRINC-READABLY:X3J13-DEC-91}
Thus no \term{escape} \term{characters} are written.

All other keywords that would be specified to \funref{write-to-string} 
are default values when \funref{prin1-to-string}
or \funref{princ-to-string} is invoked.

The meanings and defaults for the keyword arguments to \funref{write-to-string}
are the same as those for \funref{write}.

\label Examples::

\code
 (prin1-to-string "abc") \EV "\\"abc\\""
 (princ-to-string "abc") \EV "abc"
\endcode

\label Side Effects:\None.

\label Affected By::

\varref{*print-escape*},
\varref{*print-radix*},
\varref{*print-base*},
\varref{*print-circle*},
\varref{*print-pretty*},
\varref{*print-level*},
\varref{*print-length*},
\varref{*print-case*},
\varref{*print-gensym*},
\varref{*print-array*},
\varref{*read-default-float-format*}.

\label Exceptional Situations:\None.%!!! Not likely.

\label See Also::

\funref{write}

\label Notes::       

\code
 (write-to-string \param{object} \star{\curly{\param{key} \param{argument}}})
\EQ (with-output-to-string (#1=#:string-stream) 
     (write object :stream #1# \star{\curly{\param{key} \param{argument}}}))

 (princ-to-string \param{object})
\EQ (with-output-to-string (string-stream)
     (princ \param{object} string-stream))

 (prin1-to-string \param{object})
\EQ (with-output-to-string (string-stream)
     (prin1 \param{object} string-stream))
\endcode
\endcom

%-------------------- Printer Variables --------------------

%%% ========== *PRINT-ARRAY*
\begincom{*print-array*}\ftype{Variable}

\label Value Type::

a \term{generalized boolean}.

\label Initial Value::

\term{implementation-dependent}.

\label Description::

%% 22.1.6 55

Controls the format in which \term{arrays} are printed.  
If it is \term{false}, the contents of \term{arrays} other than \term{strings}
are never printed.  Instead, \term{arrays} are printed in a concise form using
\f{\#<} that gives enough information for the user to be able to identify the
\term{array}, but does not include the entire \term{array} contents.
If it is \term{true}, non-\term{string} \term{arrays} are printed using
\f{\#(...)}, \f{\#*}, or \f{\#nA} syntax.

\label Examples:\None.

\label Affected By::

The \term{implementation}.

\label See Also::

{\secref\SharpsignLeftParen},
{\secref\SharpsignLeftAngle}

\label Notes:\None.

\endcom

%%% ========== *PRINT-BASE*
%%% ========== *PRINT-RADIX*
\begincom{*print-base*, *print-radix*}\ftype{Variable}

\label Value Type::

\varref{*print-base*}---a \term{radix}.
\varref{*print-radix*}---a \term{generalized boolean}.

\label Initial Value::

The initial \term{value} of \varref{*print-base*} is \f{10}.
%%It's implied. -kmp 12-Jan-91
% (decimal radix).
The initial \term{value} of \varref{*print-radix*} is \term{false}.

\label Description::

\varref{*print-base*} and \varref{*print-radix*} control the printing
of \term{rationals}.
\Thevalueof{*print-base*} is called the \newterm{current output base}.

%% 22.1.6 44
\Thevalueof{*print-base*} is the \term{radix} in which the printer 
will print \term{rationals}.   For radices above \f{10}, letters of
the alphabet are used to represent digits above \f{9}.

%% 22.1.6 46
If \thevalueof{*print-radix*} is \term{true},
the printer will print a radix specifier to indicate the \term{radix}
in which it is printing a \term{rational} number.  The radix specifier
is always printed using lowercase letters.  If \varref{*print-base*}
is \f{2}, \f{8}, or \f{16}, then the radix specifier used is \f{\#b},
\f{\#o}, or \f{\#x}, respectively.  For \term{integers}, base ten is
indicated by a trailing decimal point instead of a leading radix
specifier; for \term{ratios}, \f{\#10r} is used.  
                    
\label Examples::

\code
 (let ((*print-base* 24.) (*print-radix* t)) 
   (print 23.))
\OUT #24rN
\EV 23
 (setq *print-base* 10) \EV 10
 (setq *print-radix* nil) \EV NIL                                          
 (dotimes (i 35)
    (let ((*print-base* (+ i 2)))           ;print the decimal number 40 
      (write 40)                            ;in each base from 2 to 36
      (if (zerop (mod i 10)) (terpri) (format t " "))))
\OUT 101000
\OUT 1111 220 130 104 55 50 44 40 37 34
\OUT 31 2C 2A 28 26 24 22 20 1J 1I
\OUT 1H 1G 1F 1E 1D 1C 1B 1A 19 18
\OUT 17 16 15 14 
\EV NIL
 (dolist (pb '(2 3 8 10 16))               
    (let ((*print-radix* t)                 ;print the integer 10 and 
          (*print-base* pb))                ;the ratio 1/10 in bases 2, 
     (format t "~&~S  ~S~%" 10 1/10)))        ;3, 8, 10, 16
\OUT #b1010  #b1/1010
\OUT #3r101  #3r1/101
\OUT #o12  #o1/12
\OUT 10.  #10r1/10
\OUT #xA  #x1/A
\EV NIL
\endcode

\label Affected By::

Might be \term{bound} by \funref{format}, and \funref{write}, \funref{write-to-string}.

\label See Also::

\funref{format},
\funref{write},
\funref{write-to-string}

\label Notes:\None.

\endcom

%%% ========== *PRINT-CASE*
\begincom{*print-case*}\ftype{Variable}

\label Value Type::

One of the \term{symbols} \kwd{upcase}, \kwd{downcase}, or \kwd{capitalize}.

\label Initial Value::

The \term{symbol} \kwd{upcase}.

\label Description::

%% 22.1.6 47
\Thevalueof{*print-case*} controls the case (upper, lower, or mixed) in 
which to print any uppercase characters in the names of \term{symbols}
when vertical-bar syntax is not used.

\issue{PRINT-CASE-PRINT-ESCAPE-INTERACTION:VERTICAL-BAR-RULE-NO-UPCASE}
\issue{JUN90-TRIVIAL-ISSUES:3}
\varref{*print-case*} has an effect at all times when \thevalueof{*print-escape*}
is \term{false}. \varref{*print-case*} also has an effect when
\thevalueof{*print-escape*} is \term{true} unless inside an escape context
(\ie unless between \term{vertical-bars} or after a \term{slash}).
\issue{PRINT-CASE-BEHAVIOR:CLARIFY}
% Under no circumstance is
% any character that is lowercase in the internal representation ever presented
% as uppercase due to \varref{*print-case*}.
\endissue{PRINT-CASE-BEHAVIOR:CLARIFY}
\endissue{JUN90-TRIVIAL-ISSUES:3}
\endissue{PRINT-CASE-PRINT-ESCAPE-INTERACTION:VERTICAL-BAR-RULE-NO-UPCASE}

\label Examples::

\code
 (defun test-print-case ()
   (dolist (*print-case* '(:upcase :downcase :capitalize))
     (format t "~&~S ~S~%" 'this-and-that '|And-something-elSE|)))
\EV TEST-PC
;; Although the choice of which characters to escape is specified by
;; *PRINT-CASE*, the choice of how to escape those characters 
;; (i.e., whether single escapes or multiple escapes are used)
;; is implementation-dependent.  The examples here show two of the
;; many valid ways in which escaping might appear.
 (test-print-case) ;Implementation A
\OUT THIS-AND-THAT |And-something-elSE|
\OUT this-and-that a\\n\\d-\\s\\o\\m\\e\\t\\h\\i\\n\\g-\\e\\lse
\OUT This-And-That A\\n\\d-\\s\\o\\m\\e\\t\\h\\i\\n\\g-\\e\\lse
\EV NIL
 (test-print-case) ;Implementation B
\OUT THIS-AND-THAT |And-something-elSE|
\OUT this-and-that a|nd-something-el|se
\OUT This-And-That A|nd-something-el|se
\EV NIL
\endcode

\label Affected By:\None.

\label See Also::

\funref{write}

\label Notes::

%!!! Does issue READ-CASE-SENSITIVITY affect any of this?

\funref{read} normally converts lowercase characters appearing 
in \term{symbols} to corresponding uppercase characters,
so that internally print names normally contain only uppercase characters.

%% 22.1.6 48
If \varref{*print-escape*} is \term{true},
lowercase characters in the \term{name} of a \term{symbol}
are always printed in lowercase, and
are preceded by a single escape character 
or enclosed by multiple escape characters;
uppercase characters in the \term{name} of a \term{symbol}
are printed in upper case, in lower case, or in mixed case
so as to capitalize words, according to the value of
\varref{*print-case*}.  The convention for what constitutes
a ``word'' is the same as for \funref{string-capitalize}.
\endcom


%%% ========== *PRINT-CIRCLE*
\begincom{*print-circle*}\ftype{Variable}

\issue{PRINT-CIRCLE-SHARED:RESPECT-PRINT-CIRCLE}

\label Value Type::

a \term{generalized boolean}.

\label Initial Value::

\term{false}.

\label Description::

%% 22.1.6 42
Controls the attempt to detect circularity and sharing in an \term{object} 
being printed.

If \term{false},
the printing process merely proceeds by recursive descent without attempting 
to detect circularity and sharing.

%% 22.1.6 43
If \term{true},
the printer will endeavor to detect cycles and sharing
in the structure to be printed, 
and to use \f{\#\i{n}=} and \f{\#\i{n}\#}
syntax to indicate the circularities or shared components.

\issue{PRINT-CIRCLE-STRUCTURE:USER-FUNCTIONS-WORK}
If \term{true}, a user-defined 
\issue{DEFSTRUCT-PRINT-FUNCTION-AGAIN:X3J13-MAR-93}
%print function
\funref{print-object} \term{method}
\endissue{DEFSTRUCT-PRINT-FUNCTION-AGAIN:X3J13-MAR-93}
can print
\term{objects} to the supplied \term{stream} using \funref{write}, \funref{prin1}, 
\funref{princ}, or \funref{format} and expect circularities and sharing
to be detected and printed using the \f{\#\i{n}\#} syntax.
\issue{DEFSTRUCT-PRINT-FUNCTION-AGAIN:X3J13-MAR-93}
%This applies to \term{methods} on \funref{print-object} in addition to
%\kwd{print-function} options.
\endissue{DEFSTRUCT-PRINT-FUNCTION-AGAIN:X3J13-MAR-93}
If a user-defined 
\issue{DEFSTRUCT-PRINT-FUNCTION-AGAIN:X3J13-MAR-93}
%print function
\funref{print-object} \term{method}
\endissue{DEFSTRUCT-PRINT-FUNCTION-AGAIN:X3J13-MAR-93}
prints to a \term{stream} other than the one
that was supplied, then circularity detection starts over for that \term{stream}. 
\endissue{PRINT-CIRCLE-STRUCTURE:USER-FUNCTIONS-WORK}

Note that implementations should not use \f{\#\i{n}\#} notation 
when the \term{Lisp reader} would automatically assure sharing without it
(\eg as happens with \term{interned} \term{symbols}).

\label Examples::

\code
 (let ((a (list 1 2 3)))
   (setf (cdddr a) a)
   (let ((*print-circle* t))
     (write a)
     :done))
\OUT #1=(1 2 3 . #1#)
\EV :DONE
\endcode

\label Affected By:\None.

\label See Also::

\funref{write}

\label Notes::

An attempt to print a circular structure with \varref{*print-circle*} 
set to \nil\ may lead to looping behavior and failure to terminate.

\endissue{PRINT-CIRCLE-SHARED:RESPECT-PRINT-CIRCLE}

\endcom


%%% ========== *PRINT-ESCAPE*
\begincom{*print-escape*}\ftype{Variable}

\label Value Type::

a \term{generalized boolean}.

\label Initial Value::

%% 22.1.6 38
\term{true}.

\label Description::

%% 22.1.6 35
If \term{false},
escape characters and \term{package prefixes} are not output
when an expression is printed.
                      
%% 22.1.6 36
If \term{true}, an attempt is made to print an \term{expression}
in such a way that it can be read again to produce an \funref{equal} \term{expression}.
(This is only a guideline; not a requirement.  See \varref{*print-readably*}.)

For more specific details of how \thevalueof{*print-escape*}
affects the printing of certain \term{types},
\seesection\DefaultPrintObjMeths.

\label Examples::
\code
 (let ((*print-escape* t)) (write #\\a))
\OUT #\\a
\EV #\\a
 (let ((*print-escape* nil)) (write #\\a))
\OUT a
\EV #\\a
\endcode

\label Affected By::

\funref{princ}, \funref{prin1}, \funref{format}

\label See Also::

\funref{write}, \funref{readtable-case}

\label Notes::

\funref{princ} effectively binds \varref{*print-escape*} to \term{false}.
\funref{prin1} effectively binds \varref{*print-escape*} to \term{true}.
                    
\endcom


%%% ========== *PRINT-GENSYM*
\begincom{*print-gensym*}\ftype{Variable}

\label Value Type::

a \term{generalized boolean}.

\label Initial Value::

\term{true}.

\label Description::

%% 22.1.6 49
Controls whether the prefix ``\f{\#:}'' is printed before 
\term{apparently uninterned} \term{symbols}.  
The prefix is printed before such \term{symbols} 
if and only if \thevalueof{*print-gensym*} is \term{true}.

\label Examples::

\code
 (let ((*print-gensym* nil))
   (print (gensym)))
\OUT G6040 
\EV #:G6040
\endcode

\label Affected By:\None.

\label See Also::

\funref{write}, \varref{*print-escape*}

\label Notes:\None.

\endcom


%%% ========== *PRINT-LEVEL*
%%% ========== *PRINT-LENGTH*
\begincom{*print-level*, *print-length*}\ftype{Variable}

\label Value Type::

a non-negative \term{integer}, or \nil.

\label Initial Value::

\nil.

\label Description::

%% 22.1.6 50
\varref{*print-level*} controls how many levels deep a nested \term{object} will print.
If it is \term{false}, then no control is exercised.
Otherwise, it is an \term{integer} indicating the maximum level to be printed.
An \term{object} to be printed is at level \f{0};
its components (as of a \term{list} or \term{vector}) are at level \f{1};
and so on.
If an \term{object} to be recursively printed has components 
and is at a level equal to or greater than \thevalueof{*print-level*}, 
then the \term{object} is printed as ``\f{\#}''.

%% 22.1.6 51
\varref{*print-length*} controls how many elements at a given level are printed.
If it is \term{false}, there is no limit to the number of components printed.
Otherwise, it is an \term{integer} indicating the maximum number of \term{elements} 
of an \term{object} to be printed.  If exceeded, the printer will print 
``\f{...}'' in place of the other \term{elements}.  In the case of a \term{dotted list},
if the \term{list} contains exactly as many \term{elements} as \thevalueof{*print-length*},
the terminating \term{atom} is printed rather than printing ``\f{...}''

%% 22.1.6 52
\varref{*print-level*} and \varref{*print-length*} affect the printing 
of an any \term{object} printed with a list-like syntax.  They do not affect
the printing of \term{symbols}, \term{strings}, and \term{bit vectors}.

\label Examples::

%% 22.1.6 54
\medbreak
\code
 (setq a '(1 (2 (3 (4 (5 (6))))))) \EV (1 (2 (3 (4 (5 (6))))))
 (dotimes (i 8) 
   (let ((*print-level* i)) 
     (format t "~&~D -- ~S~%" i a)))
\OUT 0 -- #
\OUT 1 -- (1 #)
\OUT 2 -- (1 (2 #))
\OUT 3 -- (1 (2 (3 #)))
\OUT 4 -- (1 (2 (3 (4 #))))
\OUT 5 -- (1 (2 (3 (4 (5 #)))))
\OUT 6 -- (1 (2 (3 (4 (5 (6))))))
\OUT 7 -- (1 (2 (3 (4 (5 (6))))))
\EV NIL
\medbreak
 (setq a '(1 2 3 4 5 6)) \EV (1 2 3 4 5 6)
 (dotimes (i 7) 
   (let ((*print-length* i)) 
     (format t "~&~D -- ~S~%" i a))) 
\OUT 0 -- (...)
\OUT 1 -- (1 ...)
\OUT 2 -- (1 2 ...)
\OUT 3 -- (1 2 3 ...)
\OUT 4 -- (1 2 3 4 ...)
\OUT 5 -- (1 2 3 4 5 6)
\OUT 6 -- (1 2 3 4 5 6)
\EV NIL
\medbreak
(dolist (level-length '((0 1) (1 1) (1 2) (1 3) (1 4) 
                        (2 1) (2 2) (2 3) (3 2) (3 3) (3 4)))
 (let ((*print-level*  (first  level-length))
       (*print-length* (second level-length)))
   (format t "~&~D ~D -- ~S~%"
           *print-level* *print-length* 
           '(if (member x y) (+ (car x) 3) '(foo . #(a b c d "Baz"))))))
\OUT 0 1 -- #
\OUT 1 1 -- (IF ...)
\OUT 1 2 -- (IF # ...)
\OUT 1 3 -- (IF # # ...)
\OUT 1 4 -- (IF # # #)
\OUT 2 1 -- (IF ...)
\OUT 2 2 -- (IF (MEMBER X ...) ...)
\OUT 2 3 -- (IF (MEMBER X Y) (+ # 3) ...)
\OUT 3 2 -- (IF (MEMBER X ...) ...)
\OUT 3 3 -- (IF (MEMBER X Y) (+ (CAR X) 3) ...)
\OUT 3 4 -- (IF (MEMBER X Y) (+ (CAR X) 3) '(FOO . #(A B C D ...)))
\EV NIL
\endcode

\label Affected By:\None.

\label See Also::

\funref{write}

\label Notes:\None.

% KMP:
%   There has been some debate about how strictly these abbreviation guidelines
%   should be applied.
%
%      (write '(1 2 3) :length 1) prints (1 ...)   or (1 2 3)?
%      (write '(1 2 3) :length 2) prints (1 2 ...) or (1 2 3)?
%
% Barmar:
%   While I've always secretly wished that these limits were defined such
%   that they would be ignored when the length of the thing being replaced
%   with "..."  or "#" is less than or equal to the replacement text, I
%   don't seriously believe there's really a need for implementation
%   flexibility.  This is a fairly straightforward piece of the language.
%
% Dick:
%   As the original author of GPRINT, I am probably partly responsible for
%   the behavior you see.  I deliberately tried to do better than the
%   default length and depth processing did.  I was however copying what
%   the standard Lispm software did at the time.
%   Nevertheless, I agree with the point of view that there is no reason
%   to make this complicated.  Let's just have a simple abbreviation
%   algorithm and hold everybody to it.  It is not worth being more
%   creative than that.

\endcom

%%% ========== *PRINT-LINES*
\begincom{*print-lines*}\ftype{Variable}

\label Value Type::

a non-negative \term{integer}, or \nil.

\label Initial Value::

\nil.

\label Description::

When \thevalueof{*print-lines*} is other than \nil,
it is a limit on the number of output lines produced when something is pretty
printed.  If an attempt is made to go beyond that many lines,
``\f{..}''\idxtext{Dot Dot}\idxcode{..}
is printed at the end of the last line followed by all of the
suffixes (closing delimiters) that are pending to be printed.

\label Examples::

\code
 (let ((*print-right-margin* 25) (*print-lines* 3))
   (pprint '(progn (setq a 1 b 2 c 3 d 4))))
\OUT (PROGN (SETQ A 1
\OUT              B 2
\OUT              C 3 ..))
\EV \novalues
\endcode

\label See Also:\None.

\label Notes::

The ``\f{..}'' notation is intentionally different than
the ``\f{...}'' notation used for level abbreviation, so that the two
different situations can be visually distinguished.

This notation is used to increase the likelihood that the \term{Lisp reader}
will signal an error if an attempt is later made to read the abbreviated output.
Note however that if the truncation occurs in a \term{string}, 
as in \f{"This string has been trunc.."}, the problem situation cannot be
detected later and no such error will be signaled.

\endcom

%%% ========== *PRINT-MISER-WIDTH*
\begincom{*print-miser-width*}\ftype{Variable}

\issue{PRETTY-PRINT-INTERFACE}

\label Value Type::

a non-negative \term{integer}, or \nil.

\label Initial Value::

\term{implementation-dependent}

\label Description::

If it is not \nil, the \term{pretty printer} switches to a compact
style of output (called miser style) whenever the width available for
printing a substructure is less than or equal to this many \term{ems}.

\label Examples:\None.

\label See Also:\None.

\label Notes:\None.

\endissue{PRETTY-PRINT-INTERFACE}

\endcom

%%% ========== *PRINT-PPRINT-DISPATCH*
\begincom{*print-pprint-dispatch*}\ftype{Variable}

\issue{PRETTY-PRINT-INTERFACE}

\label Value Type::

a \term{pprint dispatch table}.

\label Initial Value::

\term{implementation-dependent}, but the initial entries all use a 
special class of priorities that have the property that they are less
than every priority that can be specified using \funref{set-pprint-dispatch},
so that the initial contents of any entry can be overridden.

\label Description::

The \term{pprint dispatch table} which currently controls the \term{pretty printer}.

\label Examples:\None.

\label See Also::

\varref{*print-pretty*},
{\secref\PPrintDispatchTables}

\label Notes::

The intent is that the initial \term{value} of this \term{variable} should
cause `traditional' \term{pretty printing} of \term{code}.
% Added on advice of Dick Waters:
In general, however, you can put a value in \varref{*print-pprint-dispatch*}
that makes pretty-printed output look exactly like non-pretty-printed output.
\issue{GENERALIZE-PRETTY-PRINTER:UNIFY}
% Setting \varref{*print-pretty*} to \term{true} means having \varref{*print-pprint-dispatch*}
% control printing---nothing more, and nothing less.
Setting \varref{*print-pretty*} to \term{true} 
just causes the functions contained in the \term{current pprint dispatch table} 
to have priority over normal \funref{print-object} methods;
it has no magic way of enforcing that those functions actually produce pretty
output.  For details, \seesection\PPrintDispatchTables.
\endissue{GENERALIZE-PRETTY-PRINTER:UNIFY}
\endissue{PRETTY-PRINT-INTERFACE}

\endcom

%%% ========== *PRINT-PRETTY*
\begincom{*print-pretty*}\ftype{Variable}

\label Value Type::

a \term{generalized boolean}.

\label Initial Value::

%% 22.1.6 41
\term{implementation-dependent}.

\label Description::

Controls whether the \term{Lisp printer} calls the \term{pretty printer}.

%% 22.1.6 39
If it is \term{false},
the \term{pretty printer} is not used and 
\issue{PRINTER-WHITESPACE:JUST-ONE-SPACE}
a minimum
%only a small amount 
\endissue{PRINTER-WHITESPACE:JUST-ONE-SPACE}
of \term{whitespace}\meaning{1}
is output when printing an expression.

%% 22.1.6 40
If it is \term{true},
the \term{pretty printer} is used, and the \term{Lisp printer} will endeavor
to insert extra \term{whitespace}\meaning{1} where appropriate to make \term{expressions} 
more readable.

\varref{*print-pretty*} has an effect even when \thevalueof{*print-escape*}
is \term{false}.

\label Examples::

\issue{FORMAT-PRETTY-PRINT:YES}
\code
 (setq *print-pretty* 'nil) \EV NIL
 (progn (write '(let ((a 1) (b 2) (c 3)) (+ a b c))) nil)
\OUT (LET ((A 1) (B 2) (C 3)) (+ A B C))
\EV NIL
 (let ((*print-pretty* t))
   (progn (write '(let ((a 1) (b 2) (c 3)) (+ a b c))) nil))
\OUT (LET ((A 1)
\OUT       (B 2)
\OUT       (C 3))
\OUT   (+ A B C))
\EV NIL
;; Note that the first two expressions printed by this next form
;; differ from the second two only in whether escape characters are printed.
;; In all four cases, extra whitespace is inserted by the pretty printer.
 (flet ((test (x)
          (let ((*print-pretty* t))
            (print x)
            (format t "~%~S " x)
            (terpri) (princ x) (princ " ")
            (format t "~%~A " x))))
  (test '#'(lambda () (list "a" #\b 'c #'d))))
\OUT #'(LAMBDA ()
\OUT     (LIST "a" #\b 'C #'D))
\OUT #'(LAMBDA ()
\OUT     (LIST "a" #\b 'C #'D))
\OUT #'(LAMBDA ()
\OUT     (LIST a b 'C #'D)) 
\OUT #'(LAMBDA ()
\OUT     (LIST a b 'C #'D))
\EV NIL
\endcode
\endissue{FORMAT-PRETTY-PRINT:YES}

\label Affected By:\None.

\label See Also::

\funref{write}

\label Notes:\None.

\endcom

%%% ========== *PRINT-READABLY*
\begincom{*print-readably*}\ftype{Variable}
\issue{DATA-IO:ADD-SUPPORT}

\label Value Type::

a \term{generalized boolean}.

\label Initial Value::

%% 22.1.6 38
\term{false}.

\label Description::

%% 22.1.6 36
If \varref{*print-readably*} is \term{true},
some special rules for printing \term{objects} go into effect.
Specifically, printing any \term{object} $O\sub 1$ produces a printed 
representation that, when seen by the \term{Lisp reader} 
%% This additional restriction added in response to Loosemore #14, first public review
while the \term{standard readtable} is in effect,
will produce 
an \term{object} $O\sub 2$ that is \term{similar} to $O\sub 1$.
The printed representation produced might or might not be the same as
the printed representation produced when \varref{*print-readably*} is \term{false}.
% If printing a readable printed representation of an \term{object} is not possible,
If printing an \term{object} \term{readably} is not possible,
an error \oftype{print-not-readable} is signaled rather than 
using a syntax (\eg the ``\f{\#<}'' syntax) that would not be readable by
the same \term{implementation}.
If the \term{value} of some other \term{printer control variable} is such
that these requirements would be violated, the \term{value} of that other
\term{variable} is ignored.

\issue{PRINT-READABLY-BEHAVIOR:CLARIFY}
%%!!! Glag.  I can't believe we voted this in.
%%    I want (and *print-readably* *print-escape*) implies ...!
%%    Grumbling mail sent to X3J13.
%%    -kmp 22-Aug-93
Specifically, if \varref{*print-readably*} is \term{true},
printing proceeds as if
      \varref{*print-escape*},
      \varref{*print-array*},
  and \varref{*print-gensym*} were also \term{true},
and as if 
      \varref{*print-length*},
      \varref{*print-level*},
  and \varref{*print-lines*} were \term{false}.
\endissue{PRINT-READABLY-BEHAVIOR:CLARIFY}

If \varref{*print-readably*} is \term{false},
the normal rules for printing and the normal interpretations
of other \term{printer control variables} are in effect.

Individual \term{methods} for \funref{print-object}, including user-defined
\term{methods}, are responsible for implementing these requirements.

If \varref{*read-eval*} is \term{false} and \varref{*print-readably*} is \term{true},
any such method that would output a reference to the ``\f{\#.}'' \term{reader macro}
will either output something else or will signal an error (as described above).

\label Examples::

\code
 (let ((x (list "a" '\\a (gensym) '((a (b (c))) d e f g)))
       (*print-escape* nil)
       (*print-gensym* nil)
       (*print-level* 3)
       (*print-length* 3))
   (write x)
   (let ((*print-readably* t))
     (terpri)
     (write x)
     :done))
\OUT (a a G4581 ((A #) D E ...))
\OUT ("a" |a| #:G4581 ((A (B (C))) D E F G))
\EV :DONE

;; This is setup code is shared between the examples
;; of three hypothetical implementations which follow.
 (setq table (make-hash-table)) \EV #<HASH-TABLE EQL 0/120 32005763> 
 (setf (gethash table 1) 'one) \EV ONE
 (setf (gethash table 2) 'two) \EV TWO

;; Implementation A
 (let ((*print-readably* t)) (print table))
 Error: Can't print #<HASH-TABLE EQL 0/120 32005763> readably.

;; Implementation B
;; No standardized #S notation for hash tables is defined, 
;; but there might be an implementation-defined notation.
 (let ((*print-readably* t)) (print table))
\OUT #S(HASH-TABLE :TEST EQL :SIZE 120 :CONTENTS (1 ONE 2 TWO))
\EV #<HASH-TABLE EQL 0/120 32005763>

;; Implementation C
;; Note that #. notation can only be used if *READ-EVAL* is true.
;; If *READ-EVAL* were false, this same implementation might have to
;; signal an error unless it had yet another printing strategy to fall
;; back on.
 (let ((*print-readably* t)) (print table))
\OUT #.(LET ((HASH-TABLE (MAKE-HASH-TABLE)))
\OUT     (SETF (GETHASH 1 HASH-TABLE) ONE)
\OUT     (SETF (GETHASH 2 HASH-TABLE) TWO)
\OUT     HASH-TABLE)
\EV #<HASH-TABLE EQL 0/120 32005763>
\endcode

\label Affected By:\None.

\label See Also::

\funref{write}, \macref{print-unreadable-object}

\label Notes::

The rules for ``\term{similarity}'' imply that 
\f{\#A} or \f{\#(} % )
syntax cannot be used for \term{arrays} of \term{element type}
other than \typeref{t}.  
An implementation will have to use another syntax 
or signal an error \oftype{print-not-readable}.  

\issue{PRINT-READABLY-BEHAVIOR:CLARIFY}
% The printing of interned \term{symbols}, of \term{strings}, 
% and of \term{bit vectors} is not affected by \varref{*print-readably*}.
\endissue{PRINT-READABLY-BEHAVIOR:CLARIFY}

\endissue{DATA-IO:ADD-SUPPORT}

\endcom

%%% ========== *PRINT-RIGHT-MARGIN*

\begincom{*print-right-margin*}\ftype{Variable}

\issue{PRETTY-PRINT-INTERFACE}

\label Value Type::

a non-negative \term{integer}, or \nil.

\label Initial Value::

\nil.

\label Description::

If it is \term{non-nil}, it specifies the right margin (as \term{integer}
number of \term{ems}) to use when the \term{pretty printer} is making
layout decisions.

If it is \nil, the right margin is taken to be the maximum line length 
such that output can be displayed without wraparound or truncation.
If this cannot be determined, an \term{implementation-dependent} value is used.

\label Examples:\None.

\label See Also:\None.

\label Notes::

This measure is in units of \term{ems} in order to be compatible with
\term{implementation-defined} variable-width fonts while still not 
requiring the language to provide support for fonts.

\endissue{PRETTY-PRINT-INTERFACE}

\endcom

%-------------------- Printer Errors --------------------


%%% ========== PRINT-NOT-READABLE
\begincom{print-not-readable}\ftype{Condition Type}

\issue{DATA-IO:ADD-SUPPORT}
\label Class Precedence List::
\typeref{print-not-readable},
\typeref{error},
\typeref{serious-condition},
\typeref{condition},
\typeref{t}

\label Description::

\Thetype{print-not-readable} consists of error conditions that occur during
output while \varref{*print-readably*} is \term{true}, as a result of attempting
to write a printed representation with the \term{Lisp printer}
that would not be correctly read back with the \term{Lisp reader}.
The object which could not be printed is initialized by 
\theinitkeyarg{object} to \funref{make-condition}, and is \term{accessed} by 
\thefunction{print-not-readable-object}.
\endissue{DATA-IO:ADD-SUPPORT}

\label See Also::

\funref{print-not-readable-object}

\endcom%{print-not-readable}\ftype{Condition Type}

%%% ========== PRINT-NOT-READABLE-OBJECT
\begincom{print-not-readable-object}\ftype{Function}
\issue{DATA-IO:ADD-SUPPORT}
\label Syntax::

\DefunWithValues print-not-readable-object {condition} {object}

\label Arguments and Values::

\param{condition}---a \term{condition} \oftype{print-not-readable}.

\param{object}---an \term{object}.

\label Description::

Returns the \term{object} that could not be printed readably 
in the situation represented by \param{condition}.

\label Examples:\None.

\label Affected By:\None.

\label Exceptional Situations:\None.

\label See Also::

\typeref{print-not-readable},
{\secref\Conditions}

\label Notes:\None.

%% Should not be needed.
%It is an error to use \macref{setf} with \funref{print-not-readable-object}. 

\endissue{DATA-IO:ADD-SUPPORT}
\endcom

%%% ========== FORMAT
\begincom{format}\ftype{Function}

\label Syntax::

\DefunWithValues {format} {destination control-string {\rest} args} {result}

\label Arguments and Values:: 

\param{destination}---\nil,
		      \t,
		      a \term{stream},
		   or a \term{string} with a \term{fill pointer}.

\issue{FORMAT-STRING-ARGUMENTS:SPECIFY}
\param{control-string}---a \term{format control}.
\endissue{FORMAT-STRING-ARGUMENTS:SPECIFY}

\param{args}---\term{format arguments} for \param{control-string}.

\param{result}---if \param{destination} is \term{non-nil}, then \nil;
		 otherwise, a \term{string}.

\label Description::

%% 22.3.2 6
\funref{format} produces formatted output by outputting the characters
of \param{control-string} and observing that a \term{tilde}
introduces a directive.  The character after the tilde, possibly preceded 
by prefix parameters and modifiers, specifies what kind of formatting 
is desired.  Most directives use one or more elements of \param{args} to
create their output.

If \param{destination} is a \term{string}, a \term{stream}, or \t, 
then the \param{result} is \nil.  Otherwise, the \param{result} is
a \term{string} containing the `output.'

%% 22.3.1 21
\funref{format} is useful for producing nicely formatted text, producing
good-looking messages, and so on.  \funref{format} can generate and return
a \term{string} or output to \param{destination}.

For details on how the \param{control-string} is interpreted,
\seesection\FormattedOutput.

\label Examples:\None.

\label Affected By::

\varref{*standard-output*},
\varref{*print-escape*},
\varref{*print-radix*},
\varref{*print-base*},
\varref{*print-circle*},
\varref{*print-pretty*},
\varref{*print-level*},
\varref{*print-length*},
\varref{*print-case*},
\varref{*print-gensym*},
\varref{*print-array*}.

\label Exceptional Situations::

\issue{STRING-OUTPUT-STREAM-BASHING:UNDEFINED}
If \param{destination} is a \term{string} with a \term{fill pointer},
the consequences are undefined if destructive modifications are performed
directly on the \term{string} during the \term{dynamic extent} of the call.
\endissue{STRING-OUTPUT-STREAM-BASHING:UNDEFINED}

\label See Also::

\funref{write},
{\secref\ImplementationDefinedScripts}

\label Notes:\None.

\endcom
