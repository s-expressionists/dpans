% -*- Mode: TeX -*-

% Conditions
%  Conditions
%   Signaling
%   Debugger
%   Handling
%   Condition Types
%    Condition Type Definition
%    Condition Instantiation
%    Condition Types/Accessors
%  Restarts
%   PreDefined Restarts


%%% ========== CONDITION
\begincom{condition}\ftype{Condition Type}

\issue{CONDITION-RESTARTS:BUGGY}
\reviewer{Barrett: I think CONDITION-RESTARTS is not fully integrated.}
\endissue{CONDITION-RESTARTS:BUGGY}

%!!! Barrett notes of the condition types:
%  There is little discussion in this descriptions of what the values 
%  of the initargs ought to be.  For example, package-error permits name
%  or package object.

\label Class Precedence List::
\typeref{condition},
\typeref{t}

\label Description::

All types of \term{conditions}, whether error or
non-error, must inherit from this \term{type}.
\issue{CLOS-CONDITIONS-AGAIN:ALLOW-SUBSET}
\issue{CLOS-CONDITIONS:INTEGRATE}
%%As noted by Symbolics, these just repeat the obvious:
%All types of \term{conditions} are \term{classes}.  
%All \term{condition} \term{objects} are \term{generalized instances} of one
%or more \term{classes}.
%    Barmar thinks the next statement is unnecessary and incomplete.
%    I think there's also a later cleanup issue which undoes it.  Work on this more. -kmp
%    Barrett also asks what about SLOT-VALUE and its setf.
%% Removed. This is explained better below.  -kmp 1-Oct-91
% \term{Slots} in \term{condition} \term{objects} can be accessed
% using \macref{with-slots}.
\endissue{CLOS-CONDITIONS:INTEGRATE}
\endissue{CLOS-CONDITIONS-AGAIN:ALLOW-SUBSET}

% %!!! Barrett points out that these look a little lonely all by themselves. -kmp 23-Oct-90
% %    (Well, they used to be somewhere different, so maybe they are not so 
% %    lonely now, but I'll leave the comment and think more about it later. -kmp 13-Nov-90)
% %!!! Barrett thinks this "disjointness" is probably false  and in any case uninteresting.
%% Finally commented out for Barrett. -kmp 2-Feb-92
% The \term{types} \typeref{simple-condition}, \typeref{warning}, 
% and \typeref{serious-condition} are \term{pairwise} \term{disjoint}.

No additional \term{subtype} relationships among the specified \subtypesof{condition}
are allowed, except when explicitly mentioned in the text; however
implementations are permitted to introduce additional \term{types}
and one of these \term{types} can be a \term{subtype} of any
number of the \subtypesof{condition}.

\issue{CONDITION-SLOTS:HIDDEN}
Whether a user-defined \term{condition} \term{type} has \term{slots} 
that are accessible by \macref{with-slots} is \term{implementation-dependent}.
Furthermore, even in an \term{implementation} 
in which user-defined \term{condition} \term{types} would have \term{slots}, 
it is \term{implementation-dependent} whether any \term{condition}
\term{types} defined in this document have such \term{slots} or, 
if they do, what their \term{names} might be;
only the reader functions documented by this specification may be relied
upon by portable code.
\endissue{CONDITION-SLOTS:HIDDEN}

\issue{CLOS-CONDITIONS-AGAIN:ALLOW-SUBSET}
\term{Conforming code} must observe the following restrictions related to
\term{conditions}:

\beginlist
\itemitem{\bull}
 \macref{define-condition}, not \macref{defclass}, must be used
 to define new \term{condition} \term{types}.

\itemitem{\bull}
 \funref{make-condition}, not \funref{make-instance}, must be used to
 create \term{condition} \term{objects} explicitly.

\itemitem{\bull}
 The \kwd{report} option of \macref{define-condition}, not \macref{defmethod}
 for \funref{print-object}, must be used to define a condition reporter.

\itemitem{\bull}
 \funref{slot-value}, \funref{slot-boundp}, \funref{slot-makunbound},
 and \macref{with-slots} must not be used on \term{condition} \term{objects}.
 Instead, the appropriate accessor functions (defined by \macref{define-condition})
 should be used.
\endlist
\endissue{CLOS-CONDITIONS-AGAIN:ALLOW-SUBSET}

\endcom%{condition}\ftype{Condition Type}

\begincom{warning}\ftype{Condition Type}

\label Class Precedence List::
\typeref{warning},
\typeref{condition},
\typeref{t}

\label Description::

\Thetype{warning} consists of all types of warnings.

\label See Also::

\typeref{style-warning}

\endcom%{warning}\ftype{Condition Type}

\issue{COMPILER-DIAGNOSTICS:USE-HANDLER}
\begincom{style-warning}\ftype{Condition Type}

\label Class Precedence List::
\typeref{style-warning},
\typeref{warning},
\typeref{condition},
\typeref{t}

\label Description::

\Thetype{style-warning} includes those \term{conditions} 
that represent \term{situations} involving \term{code} 
that is \term{conforming code} but that is nevertheless 
considered to be faulty or substandard.

\label See Also::

\funref{muffle-warning}

\label Notes::

An \term{implementation} might signal such a \term{condition}
if it encounters \term{code}
     that uses deprecated features 
  or that appears unaesthetic or inefficient.

An `unused variable' warning must be \oftype{style-warning}.

In general, the question of whether \term{code} is faulty or substandard
is a subjective decision to be made by the facility processing that \term{code}.
The intent is that whenever such a facility wishes to complain about
\term{code} on such subjective grounds, it should use this 
\term{condition} \term{type} so that any clients who wish to redirect or
muffle superfluous warnings can do so without risking that they will be
redirecting or muffling other, more serious warnings.

\endcom%{style-warning}\ftype{Condition Type}
\endissue{COMPILER-DIAGNOSTICS:USE-HANDLER}

\begincom{serious-condition}\ftype{Condition Type}

\label Class Precedence List::
\typeref{serious-condition},
\typeref{condition},
\typeref{t}

\label Description::

All \term{conditions} serious enough to require interactive intervention 
if not handled should inherit from \thetype{serious-condition}.
This condition type is provided
%for terminological convenience.
primarily so that it may be included as
a \term{superclass} of other \term{condition} \term{types};
it is not intended to be signaled directly.

\label Notes::

% It is
% conventional to use \funref{error} (or something built on
% \funref{error}) to \term{signal} \term{conditions} that are \oftype{serious-condition},
% and to use \funref{signal} to \term{signal} \term{conditions} 
% that are not of this \term{type}.

Signaling a \term{serious condition} does not itself force entry into
the debugger.   However, except in the unusual situation where the
programmer can assure that no harm will come from failing to
\term{handle} a \term{serious condition}, such a \term{condition} is
usually signaled with \funref{error} rather than \funref{signal} in
order to assure that the program does not continue without
\term{handling} the \term{condition}.  (And conversely, it is
conventional to use \funref{signal} rather than \funref{error} to signal
conditions which are not \term{serious conditions}, since normally the
failure to handle a non-serious condition is not reason enough for the
debugger to be entered.)

\endcom%{serious-condition}\ftype{Condition Type}

\begincom{error}\ftype{Condition Type}

\label Class Precedence List::
\typeref{error},
\typeref{serious-condition},
\typeref{condition},
\typeref{t}

\label Description::

\Thetype{error} consists of all \term{conditions} that represent \term{errors}.

\endcom%{error}\ftype{Condition Type}

\begincom{cell-error}\ftype{Condition Type}

\label Class Precedence List::
\typeref{cell-error},
\typeref{error},
\typeref{serious-condition},
\typeref{condition},
\typeref{t}

\label Description::

\Thetype{cell-error} consists of error conditions that occur during 
a location \term{access}.   The name of the offending cell is initialized by 
\theinitkeyarg{name} to \funref{make-condition},
and is \term{accessed} by \thefunction{cell-error-name}.

\label See Also::

\funref{cell-error-name}

\endcom%{cell-error}\ftype{Condition Type}

%%% ========== CELL-ERROR-NAME
\begincom{cell-error-name}\ftype{Function}

\issue{ACCESS-ERROR-NAME}

\label Syntax::

\DefunWithValues cell-error-name {condition} {name}

\label Arguments and Values::

\param{condition}---a \term{condition} \oftype{cell-error}.

\param{name}---an \term{object}.

\label Description::

Returns the \term{name} of the offending cell involved in the \term{situation}
represented by \param{condition}.

The nature of the result depends on the specific \term{type} of \param{condition}.
For example,
    if the \param{condition} is \oftype{unbound-variable}, the result is
     the \term{name} of the \term{unbound variable} which was being \term{accessed},
    if the \param{condition} is \oftype{undefined-function}, this is
     the \term{name} of the \term{undefined function} which was being \term{accessed},
and if the \param{condition} is \oftype{unbound-slot}, this is
     the \term{name} of the \term{slot} which was being \term{accessed}.

\label Examples:\None.

\label Affected By:\None.

\label Exceptional Situations:\None.

\label See Also::

\typeref{cell-error},
\typeref{unbound-slot},
\typeref{unbound-variable},
\typeref{undefined-function},
{\secref\ConditionSystemConcepts}

\label Notes:\None.

%Barmar: This should be ``consequences are unspecified.''
%Shouldn't be needed anyway--we don't provide an operator.
%It is an error to use \macref{setf} with \funref{cell-error-name}.

\endissue{ACCESS-ERROR-NAME}

\endcom

\begincom{parse-error}\ftype{Condition Type}

\issue{PARSE-ERROR-STREAM:SPLIT-TYPES}
\issue{TYPE-OF-AND-PREDEFINED-CLASSES:UNIFY-AND-EXTEND}
\issue{READER-ERROR:NEW-TYPE}

\label Class Precedence List::

\typeref{parse-error},
\typeref{error},
\typeref{serious-condition},
\typeref{condition},
\typeref{t}

\label Description::

\Thetype{parse-error} consists of 
error conditions that are related to parsing.

\label See Also::

\funref{parse-namestring},
\typeref{reader-error}

\endissue{READER-ERROR:NEW-TYPE}
\endissue{TYPE-OF-AND-PREDEFINED-CLASSES:UNIFY-AND-EXTEND}
\endissue{PARSE-ERROR-STREAM:SPLIT-TYPES}

\endcom%{parse-error}\ftype{Condition Type}

\begincom{storage-condition}\ftype{Condition Type}

\label Class Precedence List::
\typeref{storage-condition},
\typeref{serious-condition},
\typeref{condition},
\typeref{t}

\label Description::

\Thetype{storage-condition} consists of serious conditions that 
relate to problems with memory management that are potentially due to
\term{implementation-dependent} limits rather than semantic errors
in \term{conforming programs}, and that typically warrant entry to the 
debugger if not handled.  Depending on the details of the \term{implementation}, 
these might include such problems as 
  stack overflow,
  memory region overflow, 
and
  storage exhausted.

\label Notes::

While some \clisp\ operations might signal \term{storage-condition}
because they are defined to create \term{objects},
it is unspecified whether operations that are not defined to create
\term{objects} create them anyway 
and so might also signal \typeref{storage-condition}.
Likewise, the evaluator itself might create \term{objects}
and so might signal \typeref{storage-condition}.
(The natural assumption might be that such 
\term{object} creation is naturally inefficient, 
but even that is \term{implementation-dependent}.)
In general, the entire question of how storage allocation is done is
\term{implementation-dependent}, 
and so any operation might signal \typeref{storage-condition} at any time.
Because such a \term{condition} is indicative of a limitation 
   of the \term{implementation} 
or of the \term{image}
rather than an error in a \term{program},
\term{objects} \oftype{storage-condition} are not \oftype{error}.

\endcom%{storage-condition}\ftype{Condition Type}

%-------------------- Signaling --------------------

%%% ========== ASSERT
\begincom{assert}\ftype{Macro}

\label Syntax::

\DefmacWithValuesNewline assert 
		  	 {test-form \brac{\paren{\starparam{place}}
                                          \brac{datum-form
                                                \starparam{argument-form}}}}
		         {\nil}

\label Arguments and Values:: 

\param{test-form}---a \term{form}; always evaluated.

\param{place}---a \term{place}; evaluated if an error is signaled.

\issue{FORMAT-STRING-ARGUMENTS:SPECIFY}
\param{datum-form}---a \term{form} that evaluates to a \param{datum}.
  Evaluated each time an error is to be signaled, 
  or not at all if no error is to be signaled.

\param{argument-form}---a \term{form} that evaluates to an \param{argument}.
  Evaluated each time an error is to be signaled, 
  or not at all if no error is to be signaled.

\issue{ASSERT-ERROR-TYPE:ERROR}
\param{datum}, \param{arguments}---\term{designators} for a \term{condition} 
 of default type \typeref{error}.  (These \term{designators} are the
 result of evaluating \param{datum-form} and each of the \param{argument-forms}.)
\endissue{ASSERT-ERROR-TYPE:ERROR}
\endissue{FORMAT-STRING-ARGUMENTS:SPECIFY}


\label Description::

\macref{assert} assures that \param{test-form} evaluates to \term{true}.
If \param{test-form} evaluates to \term{false}, \macref{assert} signals a
\term{correctable} \term{error} (denoted by \param{datum} and \param{arguments}).
Continuing from this error using \therestart{continue} makes it possible
for the user to alter the values of the \param{places} before
\macref{assert} evaluates \param{test-form} again.
If the value of \param{test-form} is \term{non-nil},
\macref{assert} returns \nil.

The \param{places} are \term{generalized references} to data
upon which \param{test-form} depends, 
whose values can be changed by the user in attempting to correct the error.
\term{Subforms} of each \param{place} are only evaluated if an error is signaled, 
and might be re-evaluated if the error is re-signaled (after continuing without
actually fixing the problem).
\issue{PUSH-EVALUATION-ORDER:FIRST-ITEM}
The order of evaluation of the \param{places} is not specified;
\seesection\GenRefSubFormEval.\idxtext{order of evaluation}\idxtext{evaluation order}
\endissue{PUSH-EVALUATION-ORDER:FIRST-ITEM}
\issue{SETF-MULTIPLE-STORE-VARIABLES:ALLOW}
If a \param{place} \term{form} is supplied that produces more values than there
are store variables, the extra values are ignored. If the supplied 
\term{form} produces fewer values than there are store variables, 
the missing values are set to \nil.
\endissue{SETF-MULTIPLE-STORE-VARIABLES:ALLOW}
                                                        
\label Examples::
\code
 (setq x (make-array '(3 5) :initial-element 3))
\EV #2A((3 3 3 3 3) (3 3 3 3 3) (3 3 3 3 3))
 (setq y (make-array '(3 5) :initial-element 7))
\EV #2A((7 7 7 7 7) (7 7 7 7 7) (7 7 7 7 7))
 (defun matrix-multiply (a b)
   (let ((*print-array* nil))
     (assert (and (= (array-rank a) (array-rank b) 2)
                  (= (array-dimension a 1) (array-dimension b 0)))
             (a b)
             "Cannot multiply ~S by ~S." a b)
            (really-matrix-multiply a b))) \EV MATRIX-MULTIPLY
 (matrix-multiply x y)
\OUT Correctable error in MATRIX-MULTIPLY: 
\OUT Cannot multiply #<ARRAY ...> by #<ARRAY ...>.
\OUT Restart options:
\OUT  1: You will be prompted for one or more new values.
\OUT  2: Top level.
\OUT Debug> \IN{:continue 1}
\OUT Value for A: \IN{x}
\OUT Value for B: \IN{(make-array '(5 3) :initial-element 6)}
\EV #2A((54 54 54 54 54)
       (54 54 54 54 54)
       (54 54 54 54 54)
       (54 54 54 54 54)
       (54 54 54 54 54))
\endcode

\code
 (defun double-safely (x) (assert (numberp x) (x)) (+ x x))
 (double-safely 4) 
\EV 8
 
 (double-safely t)
\OUT Correctable error in DOUBLE-SAFELY: The value of (NUMBERP X) must be non-NIL.
\OUT Restart options:
\OUT  1: You will be prompted for one or more new values.
\OUT  2: Top level.
\OUT Debug> \IN{:continue 1}
\OUT Value for X: \IN{7}
\EV 14
\endcode

\label Affected By::

\varref{*break-on-signals*}
% Barrett thinks that \varref{*debug-io*} and others don't belong here because
% debugger might not be reached, and anyway it is a full turing machine and might
% depend on virtually anything.

The set of active \term{condition handlers}.

\label Exceptional Situations:\None.

%% Barrett says, and I agree, that this isn't an exceptional situation.
%% Anyway, it's said above. -kmp 2-Sep-91
% If the value of \param{test-form} is \nil,
% \macref{assert} signals a user-specified \term{condition},
% which defaults to \typeref{simple-error}.

\label See Also::

\macref{check-type}, \funref{error}, {\secref\GeneralizedReference}

\label Notes::

The debugger need not include the \param{test-form} in the error message,
and the \param{places} should not be included in the message, but they
should be made available for the user's perusal.  If the user gives the
``continue'' command, the values of any of the references can be altered.
The details of this depend on the implementation's style of user interface.
\endcom

%%% ========== ERROR
\begincom{error}\ftype{Function}

\label Syntax::

\DefunNoReturn error {datum {\rest} arguments}

\label Arguments and Values:: 

\param{datum}, \param{arguments}---\term{designators} for a \term{condition} 
 of default type \typeref{simple-error}.

\label Description::

\funref{error} effectively invokes \funref{signal} on the denoted \term{condition}.

If the \term{condition} is not handled, \f{(invoke-debugger \i{condition})} is done.  
As a consequence of calling \funref{invoke-debugger}, \funref{error} 
cannot directly return; the only exit from \funref{error}
can come by non-local transfer of control in a handler or by use of
an interactive debugging command.

\label Examples::

\code
 (defun factorial (x)
   (cond ((or (not (typep x 'integer)) (minusp x))
          (error "~S is not a valid argument to FACTORIAL." x))
         ((zerop x) 1)
         (t (* x (factorial (- x 1))))))
\EV FACTORIAL
(factorial 20)
\EV 2432902008176640000
(factorial -1)
\OUT Error: -1 is not a valid argument to FACTORIAL.
\OUT To continue, type :CONTINUE followed by an option number:
\OUT  1: Return to Lisp Toplevel.
\OUT Debug> 
\endcode

\code
 (setq a 'fred)
\EV FRED
 (if (numberp a) (1+ a) (error "~S is not a number." A))
\OUT Error: FRED is not a number.
\OUT To continue, type :CONTINUE followed by an option number:
\OUT  1: Return to Lisp Toplevel.
\OUT Debug> \IN{:Continue 1}
\OUT Return to Lisp Toplevel.
 
 (define-condition not-a-number (error) 
                   ((argument :reader not-a-number-argument :initarg :argument))
   (:report (lambda (condition stream)
              (format stream "~S is not a number."
                      (not-a-number-argument condition)))))
\EV NOT-A-NUMBER
 
 (if (numberp a) (1+ a) (error 'not-a-number :argument a))
\OUT Error: FRED is not a number.
\OUT To continue, type :CONTINUE followed by an option number:
\OUT  1: Return to Lisp Toplevel.
\OUT Debug> \IN{:Continue 1}
\OUT Return to Lisp Toplevel.
\endcode

\label Side Effects::

\term{Handlers} for the specified condition, if any, are invoked 
and might have side effects.
Program execution might stop, and the debugger might be entered.

\label Affected By::

Existing handler bindings.

\varref{*break-on-signals*}
%\varref{*debug-io*}.

\label Exceptional Situations:\None.

%% Not exceptional.
% The reason for using \funref{error} is to signal a \term{condition},
% the exact nature of which is specified by the arguments;  however,
% that situation is not considered exceptional.

\Checktypes{\param{datum} and \param{arguments}}{\term{designators} for a \term{condition}}

\label See Also::

\funref{cerror}, \funref{signal}, \funref{format}, 
\macref{ignore-errors}, \varref{*break-on-signals*}, 
\macref{handler-bind}, {\secref\ConditionSystemConcepts}

\label Notes::

Some implementations may provide debugger
commands for interactively returning from individual stack frames.
However, it should be possible for the programmer to feel confident
about writing code like:

\code
 (defun wargames:no-win-scenario ()
   (error "pushing the button would be stupid.")
   (push-the-button))
\endcode
In this scenario, there should be no chance that
\funref{error} will return
and the button will get pushed.
 
While the meaning of this program is clear and it might be proven `safe'
by a formal theorem prover, such a proof is no guarantee that the
program is safe to execute.  Compilers have been known to have bugs,
computers to have signal glitches, and human beings to manually
intervene in ways that are not always possible to predict.  Those kinds
of errors, while beyond the scope of the condition system to formally
model, are not beyond the scope of things that should seriously be
considered when writing code that could have the kinds of sweeping
effects hinted at by this example.

\endcom

%%% ========== CERROR
\begincom{cerror}\ftype{Function}

\label Syntax::

\DefunWithValues cerror {continue-format-control datum {\rest} arguments} {\nil}

\label Arguments and Values:: 

\issue{FORMAT-STRING-ARGUMENTS:SPECIFY}
\param{Continue-format-control}---a \term{format control}.
\endissue{FORMAT-STRING-ARGUMENTS:SPECIFY}

\reviewer{Barmar: What is continue-format-control used for??}

\param{datum}, \param{arguments}---\term{designators} for a \term{condition} 
 of default type \typeref{simple-error}.

\label Description::

\funref{cerror} effectively invokes \funref{error} on the
\term{condition} named by \param{datum}.  As with any function that
implicitly calls \funref{error}, if the \term{condition} is not handled,
\f{(invoke-debugger \i{condition})} is executed.  While signaling is going on,
and while in the debugger if it is reached, it is possible to continue
code execution (\ie to return from \funref{cerror}) using \therestart{continue}.
% If the debugger is interactively instructed to continue,
% the call to \funref{cerror} returns \nil.

If \param{datum} is a \term{condition}, \param{arguments} can be supplied,
but are used only in conjunction with the \param{continue-format-control}.

\label Examples::

%!!! I don't like the debugger typeout format here. -kmp 3-Sep-91
\code
 (defun real-sqrt (n)
   (when (minusp n)
     (setq n (- n))
     (cerror "Return sqrt(~D) instead." "Tried to take sqrt(-~D)." n))
   (sqrt n))

 (real-sqrt 4)
\EV 2.0

 (real-sqrt -9)
\OUT Correctable error in REAL-SQRT: Tried to take sqrt(-9).
\OUT Restart options:
\OUT  1: Return sqrt(9) instead.
\OUT  2: Top level.
\OUT Debug> \IN{:continue 1}
\EV 3.0
 
 (define-condition not-a-number (error)
   ((argument :reader not-a-number-argument :initarg :argument))
   (:report (lambda (condition stream)
              (format stream "~S is not a number." 
                      (not-a-number-argument condition)))))
 
 (defun assure-number (n)
   (loop (when (numberp n) (return n))
         (cerror "Enter a number."
                 'not-a-number :argument n)
         (format t "~&Type a number: ")
         (setq n (read))
         (fresh-line)))

 (assure-number 'a)
\OUT Correctable error in ASSURE-NUMBER: A is not a number.
\OUT Restart options:
\OUT  1: Enter a number.
\OUT  2: Top level.
\OUT Debug> \IN{:continue 1}
\OUT Type a number: \IN{1/2}
\EV 1/2

 (defun assure-large-number (n)
   (loop (when (and (numberp n) (> n 73)) (return n))
         (cerror "Enter a number~:[~; a bit larger than ~D~]."
                 "~*~A is not a large number." 
                 (numberp n) n)
         (format t "~&Type a large number: ")
         (setq n (read))
         (fresh-line)))
 
 (assure-large-number 10000)
\EV 10000

 (assure-large-number 'a)
\OUT Correctable error in ASSURE-LARGE-NUMBER: A is not a large number.
\OUT Restart options:
\OUT  1: Enter a number.
\OUT  2: Top level.
\OUT Debug> \IN{:continue 1}
\OUT Type a large number: \IN{88}
\EV 88

 (assure-large-number 37)
\OUT Correctable error in ASSURE-LARGE-NUMBER: 37 is not a large number.
\OUT Restart options:
\OUT  1: Enter a number a bit larger than 37.
\OUT  2: Top level.
\OUT Debug> \IN{:continue 1}
\OUT Type a large number: \IN{259}
\EV 259
 
 (define-condition not-a-large-number (error)
   ((argument :reader not-a-large-number-argument :initarg :argument))
   (:report (lambda (condition stream)
              (format stream "~S is not a large number." 
                      (not-a-large-number-argument condition)))))
 
 (defun assure-large-number (n)
   (loop (when (and (numberp n) (> n 73)) (return n))
         (cerror "Enter a number~3*~:[~; a bit larger than ~*~D~]."
                 'not-a-large-number
                 :argument n 
                 :ignore (numberp n)
                 :ignore n
                 :allow-other-keys t)
         (format t "~&Type a large number: ")
         (setq n (read))
         (fresh-line)))
 

 (assure-large-number 'a)
\OUT Correctable error in ASSURE-LARGE-NUMBER: A is not a large number.
\OUT Restart options:
\OUT  1: Enter a number.
\OUT  2: Top level.
\OUT Debug> \IN{:continue 1}
\OUT Type a large number: \IN{88}
\EV 88
 
 (assure-large-number 37)
\OUT Correctable error in ASSURE-LARGE-NUMBER: A is not a large number.
\OUT Restart options:
\OUT  1: Enter a number a bit larger than 37.
\OUT  2: Top level.
\OUT Debug> \IN{:continue 1}
\OUT Type a large number: \IN{259}
\EV 259
\endcode

\label Affected By::

\varref{*break-on-signals*}.
%\varref{*debug-io*}.

Existing handler bindings.

\label Exceptional Situations:\None.

% The reason for using \funref{cerror} is to signal a \term{condition},
% the exact nature of which is specified by the arguments.

%!!! Was this saying anything useful??? -kmp 3-Sep-91
% An error \oftype{type-error} is signaled if the \term{condition} 
% named by \param{datum} is not handled.

\label See Also::

\funref{error}, \funref{format}, \macref{handler-bind},
\varref{*break-on-signals*}, \typeref{simple-type-error}

\label Notes::

If \param{datum} is a \term{condition} \term{type} rather than a 
\term{string}, the \funref{format} directive {\tt ~*} may be especially
useful in the \param{continue-format-control} in order to ignore the
\term{keywords} in the \term{initialization argument list}.  For example:
 
\code
(cerror "enter a new value to replace ~*~s" 
        'not-a-number
        :argument a)
\endcode
 
\endcom

%%% ========== CHECK-TYPE
\begincom{check-type}\ftype{Macro}

\label Syntax::

\DefmacWithValues check-type {place typespec {\brac{\param{string}}}} {\nil}

\label Arguments and Values:: 

\param{place}---a \term{place}.

\param{typespec}---a \term{type specifier}.

\param{string}---a \term{string}; \eval. %!!! Really??

\label Description::

\macref{check-type} signals a \term{correctable} \term{error}
\oftype{type-error} if the contents of \param{place} are not 
of the type \param{typespec}.

\macref{check-type} can return only if \therestart{store-value} is invoked,
either explicitly from a handler 
    or implicitly as one of the options offered by the debugger.
If \therestart{store-value} is invoked,
\macref{check-type} stores the new value 
that is the argument to the \term{restart} invocation 
(or that is prompted for interactively by the debugger)
in \param{place} and starts over, 
checking the type of the new value
and signaling another error if it is still not of the desired \term{type}.

% This used to say:
%   The \term{subforms} of \param{place} are evaluated once in the 
%   order specified as follows: ...
% Barmar said this is wrong. The first time the PLACE is evaluated, it is
% evaluated using normal evaluation.  It is only evaluated as a SETF place if
% the type check fails and \therestart{store-value} is used.
% Barrett concurs.  Rewritten...

The first time \param{place} is \term{evaluated}, 
it is \term{evaluated} by normal evaluation rules.
It is later \term{evaluated} as a \term{place} 
if the type check fails and \therestart{store-value} is used;
\seesection\GenRefSubFormEval.

%% I think this is now unnecessary due to the previous paragraph. -kmp 21-Nov-91
% The \param{place} and its \term{subforms} might be evaluated again if
% the type check fails.

\term{string} should be an English description of the type, 
starting with an indefinite article (``a'' or ``an'').
If \term{string} is not supplied,
it is computed automatically from \param{typespec}.
The automatically generated message mentions
      \param{place},
      its contents,
  and the desired type.
An implementation may choose to generate 
a somewhat differently worded error message 
if it recognizes that \param{place} is of a particular form, 
such as one of the arguments to the function that called \macref{check-type}.
\term{string} is allowed because some applications of \macref{check-type} 
may require a more specific description of what is wanted
than can be generated automatically from \param{typespec}.

\label Examples::

\code
 (setq aardvarks '(sam harry fred))
\EV (SAM HARRY FRED)
 (check-type aardvarks (array * (3)))
\OUT Error: The value of AARDVARKS, (SAM HARRY FRED),
\OUT        is not a 3-long array.
\OUT To continue, type :CONTINUE followed by an option number:
\OUT  1: Specify a value to use instead.
\OUT  2: Return to Lisp Toplevel.
\OUT Debug> \IN{:CONTINUE 1}
\OUT Use Value: \IN{#(SAM FRED HARRY)}
\EV NIL
 aardvarks
\EV #<ARRAY-T-3 13571>
 (map 'list #'identity aardvarks)
\EV (SAM FRED HARRY)
 (setq aardvark-count 'foo)
\EV FOO
 (check-type aardvark-count (integer 0 *) "A positive integer")
\OUT Error: The value of AARDVARK-COUNT, FOO, is not a positive integer.
\OUT To continue, type :CONTINUE followed by an option number:
\OUT  1: Specify a value to use instead.
\OUT  2: Top level.
\OUT Debug> \IN{:CONTINUE 2}
\endcode

\code
 (defmacro define-adder (name amount)
   (check-type name (and symbol (not null)) "a name for an adder function")
   (check-type amount integer)
   `(defun ,name (x) (+ x ,amount)))
  
 (macroexpand '(define-adder add3 3))
\EV (defun add3 (x) (+ x 3))
 
 (macroexpand '(define-adder 7 7))
\OUT Error: The value of NAME, 7, is not a name for an adder function.
\OUT To continue, type :CONTINUE followed by an option number:
\OUT  1: Specify a value to use instead.
\OUT  2: Top level.
\OUT Debug> \IN{:Continue 1}
\OUT Specify a value to use instead.
\OUT Type a form to be evaluated and used instead: \IN{'ADD7}
\EV (defun add7 (x) (+ x 7))
 
 (macroexpand '(define-adder add5 something))
\OUT Error: The value of AMOUNT, SOMETHING, is not an integer.
\OUT To continue, type :CONTINUE followed by an option number:
\OUT  1: Specify a value to use instead.
\OUT  2: Top level.
\OUT Debug> \IN{:Continue 1}
\OUT Type a form to be evaluated and used instead: \IN{5}
\EV (defun add5 (x) (+ x 5))
 
\endcode
 
Control is transferred to a handler.

\label Side Effects::

The debugger might be entered.

\label Affected By::

\varref{*break-on-signals*}
%\varref{*debug-io*}

The implementation.

\label Exceptional Situations:\None.

%% Barrett: Not exceptional.
% An error \oftype{type-error} is signaled 
% if the contents of \param{place} are not 
% of the \term{type} specified by \param{typespec}.

\label See Also::

{\secref\ConditionSystemConcepts}

\label Notes::

\code
 (check-type \param{place} \param{typespec})
 \EQ (assert (typep \param{place} '\param{typespec}) (\param{place})
            'type-error :datum \param{place} :expected-type '\param{typespec})
\endcode

\endcom

\begincom{simple-error}\ftype{Condition Type}

\label Class Precedence List::

\issue{TYPE-OF-AND-PREDEFINED-CLASSES:UNIFY-AND-EXTEND}
\typeref{simple-error},
\typeref{simple-condition},
\typeref{error},
\typeref{serious-condition},
\typeref{condition},
\typeref{t}
\endissue{TYPE-OF-AND-PREDEFINED-CLASSES:UNIFY-AND-EXTEND}

\label Description::

\Thetype{simple-error} consists of \term{conditions} that
are signaled by \funref{error} or \funref{cerror} when a
\issue{FORMAT-STRING-ARGUMENTS:SPECIFY}%
\term{format control}
\endissue{FORMAT-STRING-ARGUMENTS:SPECIFY}%
is supplied as the function's first argument.
%The default condition type for \funref{error} and 
%\funref{cerror} is \typeref{simple-error}.

\endcom%{simple-error}\ftype{Condition Type}


%%% ========== INVALID-METHOD-ERROR
\begincom{invalid-method-error}\ftype{Function}
 
\label Syntax::
 
\DefunWithValues invalid-method-error 
	         {method format-control {\rest} args}
		 {\term{implementation-dependent}}

\label Arguments and Values::
 
\param{method}---a \term{method}.
 
\issue{FORMAT-STRING-ARGUMENTS:SPECIFY}
\param{format-control}---a \term{format control}.
\endissue{FORMAT-STRING-ARGUMENTS:SPECIFY}
 
\param{args}---\term{format arguments} for the \param{format-control}.
 
\label Description::
 
\Thefunction{invalid-method-error} is used to signal an error \oftype{error}
when there is an applicable \term{method} whose \term{qualifiers} are not valid for
the method combination type.  The error message is constructed by
using the \param{format-control} suitable for \funref{format}
and any \param{args} to it.  Because an
implementation may need to add additional contextual information to
the error message, \funref{invalid-method-error} should be called only
within the dynamic extent of a method combination function.
 
\Thefunction{invalid-method-error} is called automatically when a
\term{method} fails to satisfy every \term{qualifier} pattern and predicate in a
\macref{define-method-combination} \term{form}.  A method combination function
that imposes additional restrictions should call 
\funref{invalid-method-error} explicitly if it encounters a \term{method} 
it cannot accept.
 
%!!! What does this mean? -kmp 13-Feb-91
Whether \funref{invalid-method-error} returns to its caller or exits via
\specref{throw} is \term{implementation-dependent}.

\label Examples:\None.
 
\label Side Effects::

The debugger might be entered.

\label Affected By::

\varref{*break-on-signals*}
 
\label Exceptional Situations:\None.
 
\label See Also:: 
 
\macref{define-method-combination}
                           
%% Per X3J13. -kmp 05-Oct-93
\label Notes:\None.
 
\endcom

%%% ========== METHOD-COMBINATION-ERROR
\begincom{method-combination-error}\ftype{Function}
 
\label Syntax::
 
\DefunWithValues {method-combination-error} 
		 {format-control {\rest} args}
	         {\term{implementation-dependent}}

\label Arguments and Values:: 
 
\issue{FORMAT-STRING-ARGUMENTS:SPECIFY}
\param{format-control}---a \term{format control}.
\endissue{FORMAT-STRING-ARGUMENTS:SPECIFY}
 
\param{args}---\term{format arguments} for \param{format-control}.
 
\label Description::
 
\Thefunction{method-combination-error} is used to signal an error
in method combination.  
 
The error message is constructed by using a \param{format-control} suitable
for \funref{format} and any \param{args} to it.  Because an implementation may
need to add additional contextual information to the error message,
\funref{method-combination-error} should be called only within the
dynamic extent of a method combination function.
 
%!!! What does this mean? -kmp 13-Feb-91
Whether \funref{method-combination-error} returns to its caller or exits
via \specref{throw} is \term{implementation-dependent}.

%% Per X3J13. -kmp 05-Oct-93
\label Examples:\None.
 
\label Side Effects::

The debugger might be entered.

\label Affected By::

\varref{*break-on-signals*}
 
\label Exceptional Situations:\None.
 
\label See Also:: 
 
\macref{define-method-combination}
 
%% Per X3J13. -kmp 05-Oct-93
\label Notes:\None.
 
\endcom

%%% ========== SIGNAL
\begincom{signal}\ftype{Function}

\label Syntax::

\DefunWithValues signal {datum {\rest} arguments} {\nil}

\label Arguments and Values::                                               

\param{datum}, \param{arguments}---\term{designators} for a \term{condition} 
 of default type \typeref{simple-condition}.

\label Description::

\term{Signals} the \term{condition} denoted by the given \param{datum} and \param{arguments}.
If the \term{condition} is not handled, \funref{signal} returns \nil.

\label Examples::

\code
 (defun handle-division-conditions (condition)
   (format t "Considering condition for division condition handling~%")
   (when (and (typep condition 'arithmetic-error)
              (eq '/ (arithmetic-error-operation condition)))
     (invoke-debugger condition)))
HANDLE-DIVISION-CONDITIONS
 (defun handle-other-arithmetic-errors (condition)
   (format t "Considering condition for arithmetic condition handling~%")
   (when (typep condition 'arithmetic-error)
     (abort)))
HANDLE-OTHER-ARITHMETIC-ERRORS
 (define-condition a-condition-with-no-handler (condition) ())
A-CONDITION-WITH-NO-HANDLER
 (signal 'a-condition-with-no-handler)
NIL
 (handler-bind ((condition #'handle-division-conditions)
                  (condition #'handle-other-arithmetic-errors))
   (signal 'a-condition-with-no-handler))
Considering condition for division condition handling
Considering condition for arithmetic condition handling
NIL
 (handler-bind ((arithmetic-error #'handle-division-conditions)
                  (arithmetic-error #'handle-other-arithmetic-errors))
   (signal 'arithmetic-error :operation '* :operands '(1.2 b)))
Considering condition for division condition handling
Considering condition for arithmetic condition handling
Back to Lisp Toplevel
\endcode

\label Side Effects::

The debugger might be entered due to \varref{*break-on-signals*}.

Handlers for the condition being signaled might transfer control.

\label Affected By::

Existing handler bindings.

\varref{*break-on-signals*}
%\varref{*debug-io*}

\label Exceptional Situations:\None.

\label See Also::

\varref{*break-on-signals*},
\funref{error},
\typeref{simple-condition},
{\secref\CondSignalHandle}

\label Notes::

If \f{(typep \param{datum} *break-on-signals*)} \term{yields} \term{true},
the debugger is entered prior to beginning the signaling process.  
\Therestart{continue} can be used to continue with the signaling process.
This is also true for all other \term{functions} and \term{macros} that
should, might, or must \term{signal} \term{conditions}.

\endcom

%----------------------------------------

\begincom{simple-condition}\ftype{Condition Type}

\label Class Precedence List::
\typeref{simple-condition},
\typeref{condition},
\typeref{t}

\label Description::

\Thetype{simple-condition} represents \term{conditions} that are
signaled by \funref{signal} whenever a \param{format-control} is
supplied as the function's first argument.
%The default \term{condition} \term{type} for \funref{signal} 
%and \funref{warn} is \typeref{simple-condition}.
\issue{FORMAT-STRING-ARGUMENTS:SPECIFY}
The \term{format control} and \term{format arguments} are initialized with 
\theinitkeyargs{format-control} 
\endissue{FORMAT-STRING-ARGUMENTS:SPECIFY}
and \kwd{format-arguments} to \funref{make-condition}, and are
\term{accessed} by the \term{functions}
\issue{FORMAT-STRING-ARGUMENTS:SPECIFY}
\funref{simple-condition-format-control}
\endissue{FORMAT-STRING-ARGUMENTS:SPECIFY}
and \funref{simple-condition-format-arguments}.
If format arguments are not supplied to \funref{make-condition},
\nil\ is used as a default.

\label See Also::

\issue{FORMAT-STRING-ARGUMENTS:SPECIFY}
\funref{simple-condition-format-control},
\endissue{FORMAT-STRING-ARGUMENTS:SPECIFY}
\funref{simple-condition-format-arguments}

\endcom%{simple-condition}\ftype{Condition Type}

%%% ========== SIMPLE-CONDITION-FORMAT-ARGUMENTS
\begincom{simple-condition-format-control, simple-condition-format-arguments}\ftype{Function}

\issue{FORMAT-STRING-ARGUMENTS:SPECIFY}

\label Syntax::

\DefunWithValues simple-condition-format-control {condition} {format-control}
\DefunWithValues simple-condition-format-arguments {condition} {format-arguments}

\label Arguments and Values:: 

\param{condition}---a \term{condition} of \term{type} \typeref{simple-condition}.
%% Barmar: These are all subtypes of simple-condition...
% 				    or \typeref{simple-warning}
% 				    or \typeref{simple-error}
% 				    or \typeref{simple-type-error}.

\param{format-control}---a \term{format control}.

\param{format-arguments}---a \term{list}.

\label Description::

\funref{simple-condition-format-control} returns the \term{format control} needed to 
process the \param{condition}'s \term{format arguments}.

\funref{simple-condition-format-arguments} returns a \term{list} of \term{format arguments} 
needed to process the \param{condition}'s \term{format control}.

\label Examples::

\code
 (setq foo (make-condition 'simple-condition
                          :format-control "Hi ~S"
                          :format-arguments '(ho)))
\EV #<SIMPLE-CONDITION 26223553>
 (apply #'format nil (simple-condition-format-control foo)
                     (simple-condition-format-arguments foo))
\EV "Hi HO"
\endcode

\label Side Effects:\None.

\label Affected By:\None.

\label Exceptional Situations:\None.

\label See Also::

\typeref{simple-condition},
{\secref\ConditionSystemConcepts}

%% Per X3J13. -kmp 05-Oct-93
\label Notes:\None.

%% Shouldn't be needed. -kmp 1-Sep-91
%It is an error to use \macref{setf} with \funref{simple-condition-format-arguments}.
%It is an error to use \macref{setf} with \funref{simple-condition-format-control}.

\endissue{FORMAT-STRING-ARGUMENTS:SPECIFY}

\endcom

%%% ========== WARN
\begincom{warn}\ftype{Function}

\label Syntax::

\DefunWithValues warn {datum {\rest} arguments} {\nil}

\label Arguments and Values:: 

\param{datum}, \param{arguments}---\term{designators} for a \term{condition} 
 of default type \typeref{simple-warning}.

\label Description::

\term{Signals} a \term{condition} \oftype{warning}.
If the \term{condition} is not \term{handled},
reports the \term{condition} to \term{error output}.

The precise mechanism for warning is as follows:

\issue{BREAK-ON-WARNINGS-OBSOLETE:REMOVE}
%Discussion of *BREAK-ON-WARNINGS* removed.
\endissue{BREAK-ON-WARNINGS-OBSOLETE:REMOVE}

%!!! Barret wonders whether stylistically it wouldn't be better to have just
%    normal text instead of this indented stuff.
\beginlist

\itemitem{{\bf The warning condition is signaled}}

While the \typeref{warning} \term{condition} is being signaled,
\therestart{muffle-warning} is established for use by a \term{handler}.
If invoked, this \term{restart} bypasses further action by \funref{warn},
which in turn causes \funref{warn} to immediately return \nil.

\itemitem{{\bf If no handler for the warning condition is found}}

If no handlers for the warning condition are found,
or if all such handlers decline,
then the \term{condition} is reported to \term{error output}
by \funref{warn} in an \term{implementation-dependent} format.
%% Barrett points out that the details of this are already said elsewhere in
%% the concept info for conditions.
% (with possible implementation-specific extra
% output such as motion to a fresh line before and/or after the display
% of the warning, or supplying some introductory text that might mention
% the name of the function which called \funref{warn} 
% and/or the fact that this is a warning).

\itemitem{{\bf \nil\ is returned}}

The value returned by \funref{warn} if it returns is \nil.
\endlist

\label Examples::

\code
  (defun foo (x)
    (let ((result (* x 2)))
      (if (not (typep result 'fixnum))
          (warn "You're using very big numbers."))
      result))
\EV FOO
 
  (foo 3)
\EV 6
 
  (foo most-positive-fixnum)
\OUT Warning: You're using very big numbers.
\EV 4294967294
 
  (setq *break-on-signals* t)
\EV T
 
  (foo most-positive-fixnum)
\OUT Break: Caveat emptor.
\OUT To continue, type :CONTINUE followed by an option number.
\OUT  1: Return from Break.
\OUT  2: Abort to Lisp Toplevel.
\OUT Debug> :continue 1
\OUT Warning: You're using very big numbers.
\EV 4294967294
\endcode
 
\label Side Effects::

A warning is issued.  The debugger might be entered.

\label Affected By::

Existing handler bindings.

\varref{*break-on-signals*},
\varref{*error-output*}.         

\label Exceptional Situations::

If \param{datum} is a \term{condition}
and if the \term{condition} is not \oftype{warning},
or \param{arguments} is \term{non-nil}, an error \oftype{type-error} is signaled.

If \param{datum} is a condition type, 
the result of {\tt (apply #'make-condition datum arguments)} 
must be \oftype{warning} or an error \oftype{type-error} is signaled.

\label See Also::

\varref{*break-on-signals*},
\funref{muffle-warning},
\funref{signal}

\label Notes:\None.

\endcom

\begincom{simple-warning}\ftype{Condition Type}

\label Class Precedence List::

\issue{TYPE-OF-AND-PREDEFINED-CLASSES:UNIFY-AND-EXTEND}
\typeref{simple-warning},
\typeref{simple-condition},
\typeref{warning},
\typeref{condition},
\typeref{t}
\endissue{TYPE-OF-AND-PREDEFINED-CLASSES:UNIFY-AND-EXTEND}

\label Description::

\Thetype{simple-warning} represents \term{conditions} that 
are signaled by \funref{warn} whenever a 
\issue{FORMAT-STRING-ARGUMENTS:SPECIFY}
\term{format control} 
\endissue{FORMAT-STRING-ARGUMENTS:SPECIFY}
is supplied as the function's first argument.

\endcom%{simple-warning}\ftype{Condition Type}

%-------------------- Debugger --------------------

%%% ========== INVOKE-DEBUGGER
\begincom{invoke-debugger}\ftype{Function}

\label Syntax::

\DefunNoReturn invoke-debugger {condition}

\label Arguments and Values:: 

\param{condition}---a \term{condition} \term{object}.

\label Description::

\funref{invoke-debugger} attempts to enter the debugger with \param{condition}.

If \varref{*debugger-hook*} is not \nil, it should be a \term{function} 
(or the name of a \term{function}) to be called prior to entry to 
the standard debugger.  The \term{function} is called with
\varref{*debugger-hook*} bound to \nil, and the \term{function} 
must accept two arguments: the \param{condition} 
and \thevalueof{*debugger-hook*} prior to binding it to \nil. 
If the \term{function} returns normally,
the standard debugger is entered.
 
%!!! KMP: Maybe make glossary term of "standard debugger"?
The standard debugger never directly returns.  Return can occur only by a
non-local transfer of control, such as the use of a restart function.

\label Examples::

\code
 (ignore-errors ;Normally, this would suppress debugger entry
   (handler-bind ((error #'invoke-debugger)) ;But this forces debugger entry
     (error "Foo.")))
Debug: Foo.
To continue, type :CONTINUE followed by an option number:
 1: Return to Lisp Toplevel.
Debug>
\endcode
 

\label Side Effects::

\varref{*debugger-hook*} is bound to \nil,
program execution is discontinued,
and the debugger is entered.

\label Affected By::

\varref{*debug-io*} and \varref{*debugger-hook*}.

\label Exceptional Situations:\None.

\label See Also::

\funref{error}, \funref{break}

\label Notes:\None.

\endcom

%%% ========== BREAK
\begincom{break}\ftype{Function}

\label Syntax::

\DefunWithValues break {{\opt} format-control {\rest} format-arguments} {\nil}

\label Arguments and Values:: 

\issue{FORMAT-STRING-ARGUMENTS:SPECIFY}
\param{format-control}---a \term{format control}.
\endissue{FORMAT-STRING-ARGUMENTS:SPECIFY}
 \Default{\term{implementation-dependent}}

\param{format-arguments}---\term{format arguments} for the \param{format-control}.

\label Description::

%% 24.0.0 29

\funref{break} \term{formats} \param{format-control} and \param{format-arguments}
and then goes directly into the debugger without allowing any possibility of
interception by programmed error-handling facilities.

If \therestart{continue} is used while in the debugger,
\funref{break} immediately returns \nil\ without taking any unusual recovery action.

\issue{DEBUGGER-HOOK-VS-BREAK:CLARIFY}
\funref{break} binds \varref{*debugger-hook*} to \nil\ 
before attempting to enter the debugger.
\endissue{DEBUGGER-HOOK-VS-BREAK:CLARIFY}

\label Examples::

\code
 (break "You got here with arguments: ~:S." '(FOO 37 A))
\OUT BREAK: You got here with these arguments: FOO, 37, A.
\OUT To continue, type :CONTINUE followed by an option number:
\OUT  1: Return from BREAK.
\OUT  2: Top level.
\OUT Debug> :CONTINUE 1
\OUT Return from BREAK.
\EV NIL
 
\endcode
 
\label Side Effects::

The debugger is entered.

\label Affected By::

\varref{*debug-io*}.

\label Exceptional Situations:\None.

\label See Also::

\funref{error}, \funref{invoke-debugger}.

\label Notes::

%% 24.0.0 30
\funref{break} is used as a way of inserting temporary debugging
``breakpoints'' in a program, not as a way of signaling errors.  
For this reason, \funref{break} does not take the \param{continue-format-control}
\term{argument} that \funref{cerror} takes.
This and the lack of any possibility of interception by
\term{condition} \term{handling} are the only program-visible 
differences between \funref{break} and \funref{cerror}.

The user interface aspects of \funref{break} and \funref{cerror} are
permitted to vary more widely, in order to accomodate the interface
needs of the \term{implementation}. For example, it is permissible for a
\term{Lisp read-eval-print loop} to be entered by \funref{break} rather
than the conventional debugger.

\funref{break} could be defined by:

\issue{FORMAT-STRING-ARGUMENTS:SPECIFY}
\code
 (defun break (&optional (format-control "Break") &rest format-arguments)
   (with-simple-restart (continue "Return from BREAK.")
     (let ((*debugger-hook* nil))
       (invoke-debugger
           (make-condition 'simple-condition
                           :format-control format-control
                           :format-arguments format-arguments))))
   nil)
\endcode
\endissue{FORMAT-STRING-ARGUMENTS:SPECIFY}

\endcom

%%% ========== *DEBUGGER-HOOK*
\begincom{*debugger-hook*}\ftype{Variable}

\label Value Type::

%!!! Barrett: What if this is invoked directly instead of from invoke-debugger?
a \term{designator} for a \term{function} of two \term{arguments}
  (a \term{condition} and \thevalueof{*debugger-hook*} at the time 
   the debugger was entered),
or \nil.

\label Initial Value::

\nil.

\label Description::

When \thevalueof{*debugger-hook*} is \term{non-nil}, it is called prior to
normal entry into the debugger, either due to a call to \funref{invoke-debugger} 
or due to automatic entry into the debugger from a call to \funref{error} 
or \funref{cerror} with a condition that is not handled.  
The \term{function} may either handle the \term{condition}
(transfer control) or return normally (allowing the standard debugger to run).
To minimize recursive errors while debugging,
\varref{*debugger-hook*} is bound to \nil\ by \funref{invoke-debugger} 
prior to calling the \term{function}.

\label Examples::

\code
 (defun one-of (choices &optional (prompt "Choice"))
   (let ((n (length choices)) (i))
     (do ((c choices (cdr c)) (i 1 (+ i 1)))
         ((null c))
       (format t "~&[~D] ~A~%" i (car c)))
     (do () ((typep i `(integer 1 ,n)))
       (format t "~&~A: " prompt)
       (setq i (read))
       (fresh-line))
     (nth (- i 1) choices)))

 (defun my-debugger (condition me-or-my-encapsulation)
   (format t "~&Fooey: ~A" condition)
   (let ((restart (one-of (compute-restarts))))
     (if (not restart) (error "My debugger got an error."))
     (let ((*debugger-hook* me-or-my-encapsulation))
       (invoke-restart-interactively restart))))
 
 (let ((*debugger-hook* #'my-debugger))
   (+ 3 'a))
\OUT Fooey: The argument to +, A, is not a number.
\OUT  [1] Supply a replacement for A.
\OUT  [2] Return to Cloe Toplevel.
\OUT Choice: 1
\OUT  Form to evaluate and use: (+ 5 'b)
\OUT  Fooey: The argument to +, B, is not a number.
\OUT  [1] Supply a replacement for B.
\OUT  [2] Supply a replacement for A.
\OUT  [3] Return to Cloe Toplevel.
\OUT Choice: 1
\OUT  Form to evaluate and use: 1
\EV 9
\endcode

\label Affected By::

\funref{invoke-debugger}

\label See Also:\None.

\label Notes::

When evaluating code typed in by the user interactively, it is sometimes
useful to have the hook function bind \varref{*debugger-hook*} to the
\term{function} that was its second argument so that recursive errors
can be handled using the same interactive facility.
                                            
\endcom

%%% ========== *BREAK-ON-SIGNALS*
\begincom{*break-on-signals*}\ftype{Variable}
 
\label Value Type::

a \term{type specifier}.

\label Initial Value::

\nil.

\label Description::

%!!! Does this get involved in *debugger-hook*?  What kind of break is entered?
When \f{(typep \i{condition} *break-on-signals*)} returns \term{true},
calls to \funref{signal}, and to other \term{operators} such as \funref{error}
that implicitly call \funref{signal}, enter the debugger prior to
\term{signaling} the \term{condition}.

\Therestart{continue} can be used to continue with the normal
\term{signaling} process when a break occurs process due to
\varref{*break-on-signals*}.

\label Examples::

\issue{FORMAT-STRING-ARGUMENTS:SPECIFY}
\code
 *break-on-signals* \EV NIL
 (ignore-errors (error 'simple-error :format-control "Fooey!"))
\EV NIL, #<SIMPLE-ERROR 32207172>

 (let ((*break-on-signals* 'error))
   (ignore-errors (error 'simple-error :format-control "Fooey!")))
\OUT Break: Fooey!
\OUT BREAK entered because of *BREAK-ON-SIGNALS*.
\OUT To continue, type :CONTINUE followed by an option number:
\OUT  1: Continue to signal.
\OUT  2: Top level.
\OUT Debug> \IN{:CONTINUE 1}
\OUT Continue to signal.
\EV NIL, #<SIMPLE-ERROR 32212257>

 (let ((*break-on-signals* 'error))
   (error 'simple-error :format-control "Fooey!"))
\OUT Break: Fooey!
\OUT BREAK entered because of *BREAK-ON-SIGNALS*.
\OUT To continue, type :CONTINUE followed by an option number:
\OUT  1: Continue to signal.
\OUT  2: Top level.
\OUT Debug> \IN{:CONTINUE 1}
\OUT Continue to signal.
\OUT Error: Fooey!
\OUT To continue, type :CONTINUE followed by an option number:
\OUT  1: Top level.
\OUT Debug> \IN{:CONTINUE 1}
\OUT Top level.
\endcode
\endissue{FORMAT-STRING-ARGUMENTS:SPECIFY}

\label Affected By:\None.

\label See Also::

\funref{break},
\funref{signal}, \funref{warn}, \funref{error}, 
\funref{typep}, 
{\secref\ConditionSystemConcepts}

\label Notes::

\varref{*break-on-signals*} is intended primarily for use in debugging code that
does signaling.   When setting \varref{*break-on-signals*}, the user is
encouraged to choose the most restrictive specification that suffices.
Setting \varref{*break-on-signals*} effectively violates the modular handling of 
\term{condition} signaling.  In practice, the complete effect of setting
\varref{*break-on-signals*} might be unpredictable in some cases since the user
might not be aware of the variety or number of calls to \funref{signal} 
that are used in code called only incidentally.

\issue{BREAK-ON-WARNINGS-OBSOLETE:REMOVE}
% Reference to *BREAK-ON-WARNINGS* removed.
\endissue{BREAK-ON-WARNINGS-OBSOLETE:REMOVE}

\varref{*break-on-signals*} enables an early entry to the debugger but such an
entry does not preclude an additional entry to the debugger in the case of
operations such as \funref{error} and \funref{cerror}.

\endcom

%-------------------- Handling --------------------

%%% ========== HANDLER-BIND
\begincom{handler-bind}\ftype{Macro}

\label Syntax::

\DefmacWithValues handler-bind
		  {\paren{\stardown{binding}} 
		   \starparam{form}}
		  {\starparam{result}}

\auxbnf{binding}{\paren{type handler}}

\label Arguments and Values:: 

\param{type}---a \term{type specifier}.

\param{handler}---a \term{form}; evaluated to produce a \param{handler-function}.
                     
\param{handler-function}---a \term{designator} for a \term{function} of one \term{argument}.

\param{forms}---an \term{implicit progn}.

\param{results}---the \term{values} returned by the \term{forms}.

\label Description::

Executes \param{forms} in a \term{dynamic environment} where the indicated
\param{handler} \term{bindings} are in effect.

Each \param{handler} should evaluate to a \term{handler-function},
which is used to handle \term{conditions} of the given \param{type}
during execution of the \param{forms}.  This \term{function} should
take a single argument, the \term{condition} being signaled.

%!!! Barmar: The next two paragraphs belong in description of signaling,
%      not handling. [I agree. -kmp]
If more than one \param{handler} \term{binding} is supplied, 
the \param{handler} \term{bindings} are searched sequentially from 
top to bottom in search of a match (by visual analogy with \macref{typecase}).  
If an appropriate \term{type} is found, 
the associated handler is run in a \term{dynamic environment} where none of these
\param{handler} bindings are visible (to avoid recursive errors).  
If the \term{handler} \term{declines}, the search continues for another \term{handler}.

If no appropriate \term{handler} is found, other \term{handlers} are sought
from dynamically enclosing contours.  If no \term{handler} is found outside, 
then \funref{signal} returns or \funref{error} enters the debugger. 

\label Examples::
 
%Here's an example to think about as a possible replacement for the next
%couple of paragraphs...
% (defun test (x)
%   (handler-bind ((unbound-variable #'(lambda (c) (use-value 'unbound c)))
%                  (undefined-function 
%                    #'(lambda (c) 
%                        (use-value #'(lambda (&rest x) (cons (cell-error-name c) x)) c))))
%     (eval x)))
% (test '(frob (+ 1 2) hunoz))

In the following code, if an unbound variable error is
signaled in the body (and not handled by an intervening handler), 
the first function is called.  

\code
 (handler-bind ((unbound-variable #'(lambda ...))
                (error #'(lambda ...)))
   ...)
\endcode

If any other kind of error is signaled, the second function is called.
In either case, neither handler is active while executing the code
in the associated function.

\code
 (defun trap-error-handler (condition)
   (format *error-output* "~&~A~&" condition)
   (throw 'trap-errors nil))

 (defmacro trap-errors (&rest forms)
   `(catch 'trap-errors
      (handler-bind ((error #'trap-error-handler))
        ,@forms)))
 
 (list (trap-errors (signal "Foo.") 1)
       (trap-errors (error  "Bar.") 2)
       (+ 1 2))
\OUT Bar.
\EV (1 NIL 3)
\endcode

Note that ``Foo.'' is not printed because the condition made
by \funref{signal} is a \term{simple condition}, which is not \oftype{error}, 
so it doesn't trigger the handler for \typeref{error} set up by \f{trap-errors}.

\label Side Effects:\None.

\label Affected By:\None.

\label Exceptional Situations:\None.

\label See Also::

\macref{handler-case}

\label Notes:\None.

\endcom


%%% ========== HANDLER-CASE
\begincom{handler-case}\ftype{Macro}

\issue{DECLS-AND-DOC}

\label Syntax::
 
%!!! "expression" is a bad var name to use here.
\DefmacWithValues handler-case
		  {\param{expression}
		   \interleave{\stardown{error-clause} | \down{no-error-clause}}}
		  {\starparam{result}}
 
\auxbnf{clause}{\down{error-clause} | \down{no-error-clause}}
\auxbnf{error-clause}{\paren{typespec \paren{\ttbrac{var}} 
		      \starparam{declaration} \starparam{form}}}
\auxbnf{no-error-clause}{\paren{\kwd{no-error} \param{lambda-list} 
			 \starparam{declaration} \starparam{form}}}

%This should follow from the above BNF.
%Note: There can be no more than one \i{no-error-clause}.

\label Arguments and Values::

\param{expression}---a \term{form}.

\param{typespec}---a \term{type specifier}.

\param{var}---a \term{variable} \term{name}. 

\param{lambda-list}---an \term{ordinary lambda list}.

\param{declaration}---a \misc{declare} \term{expression}; \noeval.

\param{form}---a \term{form}.

\param{results}---In the normal situation, the values returned are those that result from
   the evaluation of \param{expression};
   in the exceptional situation when control is transferred to a \param{clause},
   the value of the last \param{form} in that \param{clause} is returned.
 
\label Description::

\macref{handler-case} executes \param{expression} in a \term{dynamic environment} where
various handlers are active.  Each \i{error-clause} specifies how to 
handle a \term{condition} matching the indicated \param{typespec}. 
A \i{no-error-clause} allows the specification of a particular action
if control returns normally.
 
%!!! It would be nice if this reference to (TYPEP ...) and all others
%    were rewritten in text fashion.
If a \term{condition} is signaled for which there is an appropriate
\i{error-clause} during the execution of \param{expression}
(\ie one for which \f{(typep \term{condition} '\param{typespec})}
returns \term{true}) and if there is no intervening handler for a 
\term{condition} of that \term{type}, then control is transferred to
the body of the relevant \i{error-clause}.  In this case, the 
dynamic state is unwound appropriately (so that the handlers established
around the \param{expression} are no longer active), and \param{var} is bound to
the \term{condition} that had been signaled.
%!!! Barmar: HANDLER-BIND describes this better...
If more than one case is provided, those cases are made accessible
in parallel.  That is, in
 
\code
  (handler-case \i{form}
    (\i{typespec1} (\i{var1}) \i{form1})
    (\i{typespec2} (\i{var2}) \i{form2}))
\endcode

if the first \i{clause} (containing \i{form1}) has been selected, 
the handler for the second is no longer visible (or vice versa).
 
   The \i{clauses}
are searched sequentially from top to bottom. If there is \term{type}
   overlap between \param{typespecs}, 
the earlier of the \i{clauses} is selected.
 
   If \param{var} 
is not needed, it can be omitted. That is, a \i{clause} such as:

\code
  (\param{typespec} (\param{var}) (declare (ignore \param{var})) \param{form})
\endcode

can be written
 \f{(\param{typespec} () \param{form})}.
 
%% Per X3J13. -kmp 05-Oct-93
%% 3 uses of HANDLER-BIND to be replaced by HANDLER-CASE in last para on page.
   If there are no \param{forms} in a selected \i{clause}, the case, and therefore
   \macref{handler-case}, returns \nil.
    If execution of \param{expression} 
returns normally and no \i{no-error-clause}
   exists, the values returned by 
\param{expression} are returned by \macref{handler-case}.
   If execution of 
\param{expression} returns normally and a \i{no-error-clause}
   does exist, the values returned are used as arguments to the function
   described by constructing
 \f{(lambda \param{lambda-list} \starparam{form})}
   from the \i{no-error-clause}, and the \term{values} of that function call are
   returned by \macref{handler-case}.
%The following was added to make Barmar happy. -kmp 1-Sep-91
The handlers which were established around the \param{expression} are no longer active at the time of this call.

\label Examples::

\code
 (defun assess-condition (condition)
   (handler-case (signal condition)
     (warning () "Lots of smoke, but no fire.")
     ((or arithmetic-error control-error cell-error stream-error)
        (condition)
       (format nil "~S looks especially bad." condition))
     (serious-condition (condition)
       (format nil "~S looks serious." condition))
     (condition () "Hardly worth mentioning.")))
\EV ASSESS-CONDITION
 (assess-condition (make-condition 'stream-error :stream *terminal-io*))
\EV "#<STREAM-ERROR 12352256> looks especially bad."
 (define-condition random-condition (condition) () 
   (:report (lambda (condition stream)
              (declare (ignore condition))
              (princ "Yow" stream))))
\EV RANDOM-CONDITION
 (assess-condition (make-condition 'random-condition))
\EV "Hardly worth mentioning."
\endcode
 
\label Affected By:\None.

\label Exceptional Situations:\None.

\label See Also::

\macref{handler-bind},
\macref{ignore-errors},
{\secref\ConditionSystemConcepts}

\label Notes::

\code
 (handler-case form
   (\i{type1} (\i{var1}) . \i{body1})
   (\i{type2} (\i{var2}) . \i{body2}) ...)
\endcode
is approximately equivalent to:

\code
 (block #1=#:g0001
   (let ((#2=#:g0002 nil))
     (tagbody
       (handler-bind ((\i{type1} #'(lambda (temp)
                                       (setq #1# temp)
                                       (go #3=#:g0003)))
                      (\i{type2} #'(lambda (temp)
                                       (setq #2# temp)
                                       (go #4=#:g0004))) ...)
       (return-from #1# form))
         #3# (return-from #1# (let ((\i{var1} #2#)) . \i{body1}))
         #4# (return-from #1# (let ((\i{var2} #2#)) . \i{body2})) ...)))
\endcode

\code
 (handler-case form
   (\i{type1} \i{(var1)} . \i{body1})
   ...
   (:no-error (\i{varN-1} \i{varN-2} ...) . \i{bodyN}))
\endcode
is approximately equivalent to:

\code

 (block #1=#:error-return
  (multiple-value-call #'(lambda (\i{varN-1} \i{varN-2} ...) . \i{bodyN})
     (block #2=#:normal-return
       (return-from #1#
         (handler-case (return-from #2# form)
           (\i{type1} (\i{var1}) . \i{body1}) ...)))))
\endcode

\endissue{DECLS-AND-DOC}

\endcom

%%% ========== IGNORE-ERRORS
\begincom{ignore-errors}\ftype{Macro}

\label Syntax::

\DefmacWithValues ignore-errors 
		  {\starparam{form}}
		  {\starparam{result}}

\label Arguments and Values::

\param{forms}---an \term{implicit progn}.
 
\param{results}---In the normal situation,
		  the \term{values} of the \term{forms} are returned;
		  in the exceptional situation,
		  two values are returned: \nil\ and the \term{condition}.

\label Description::

\macref{ignore-errors} is used to prevent \term{conditions} \oftype{error}
from causing entry into the debugger.

Specifically, \macref{ignore-errors} \term{executes} \term{forms}
in a \term{dynamic environment} where a \term{handler} for 
\term{conditions} \oftype{error} has been established;
if invoked, it \term{handles} such \term{conditions} by
returning two \term{values}, \nil\ and the \term{condition} that was \term{signaled},
from the \macref{ignore-errors} \term{form}.

If a \term{normal return} from the \term{forms} occurs, 
any \term{values} returned are returned by \macref{ignore-errors}.

\label Examples::

\code
 (defun load-init-file (program)
   (let ((win nil))
     (ignore-errors ;if this fails, don't enter debugger
       (load (merge-pathnames (make-pathname :name program :type :lisp)
                              (user-homedir-pathname)))
       (setq win t))
     (unless win (format t "~&Init file failed to load.~%"))
     win))
 
 (load-init-file "no-such-program")
\OUT Init file failed to load.
NIL
\endcode

\label Affected By:\None.

\label Exceptional Situations:\None.

\label See Also::

\macref{handler-case}, {\secref\ConditionSystemConcepts}

\label Notes::

\code
 (ignore-errors . \i{forms})
\endcode
 
   is equivalent to:
 
\code
 (handler-case (progn . \i{forms})
   (error (condition) (values nil condition)))
\endcode

Because the second return value is a \term{condition}
in the exceptional case, it is common (but not required) to arrange
for the second return value in the normal case to be missing or \nil\ so
that the two situations can be distinguished.
 
\endcom

%-------------------- Condition Type Definition --------------------

%%% ========== DEFINE-CONDITION
\begincom{define-condition}\ftype{Macro}

\issue{DEFINE-CONDITION-SYNTAX:INCOMPATIBLY-MORE-LIKE-DEFCLASS+EMPHASIZE-READ-ONLY}

\editornote{KMP: This syntax stuff is still very confused and needs lots of work.}

\label Syntax::

%!!! Consider renaming "parent-type" to "supertype".
\DefmacWithValuesNewline define-condition
		  {name \paren{\starparam{parent-type}}
	                \paren{\stardown{slot-spec}}
                        \starparam{option}}
		  {name}

\auxbnf{slot-spec}{slot-name | \paren{slot-name \down{slot-option}}}
\auxbnf{slot-option}{\begininterleave
		     \star{\curly{\kwd{reader}     \term{symbol}}}         | \CR
		     \star{\curly{\kwd{writer}     \down{function-name}}}  | \CR
		     \star{\curly{\kwd{accessor}   \term{symbol}}}         | \CR
		           \curly{\kwd{allocation} \down{allocation-type}} | \CR
		     \star{\curly{\kwd{initarg}    \term{symbol}}}         | \CR
                           \curly{\kwd{initform}   \term{form}}            | \CR
		           \curly{\kwd{type}       \param{type-specifier}} 
		     \endinterleave}
\auxbnf{option}{\begininterleave
		\paren{\kwd{default-initargs} \f{.} \param{initarg-list}} | \CR
		\paren{\kwd{documentation} \term{string}} | \CR
		\paren{\kwd{report}        \i{report-name}} \endinterleave}
\auxbnf{function-name}{\curly{\term{symbol} | {\tt (setf \term{symbol})}}}
\auxbnf{allocation-type}{\kwd{instance} | \kwd{class}}
\auxbnf{report-name}{\term{string} | \term{symbol} | \term{lambda expression}}

\label Arguments and Values::

\param{name}---a \term{symbol}.

%% Reworded per Barmar #9, First Public Review.
\param{parent-type}---a \term{symbol} naming a \term{condition} \term{type}.
  If no \param{parent-types} are supplied,
  the \param{parent-types} default to \f{(condition)}.
                     
\param{default-initargs}---a \term{list} of \term{keyword/value pairs}.

\editornote{KMP: This is all mixed up as to which is a slot option and which is
		 a main option.  I'll sort that out.  Also, some of this is implied
		 by the bnf and needn't be stated explicitly.}%!!!

\issue{CLOS-CONDITIONS:INTEGRATE}

\param{Slot-spec}---the \term{name} of a \term{slot} or a \term{list}
consisting of the \param{slot-name} followed by zero or more \param{slot-options}.
 
\param{Slot-name}---a slot name (a \term{symbol}),
the \term{list} of a slot name, or the 
\term{list} of slot name/slot form pairs.
 
\param{Option}---Any of the following:
 
\beginlist
 
\itemitem{\kwd{reader}}
            
\kwd{reader} can be supplied more than once for a given \term{slot} 
and cannot be \nil.
 
\itemitem{\kwd{writer}}
          
\kwd{writer} can be supplied more than once for a given \term{slot}
and must name a \term{generic function}.
 
\itemitem{\kwd{accessor}}
 
\kwd{accessor} can be supplied more than once for a given \term{slot}
and cannot be \nil.
 
\itemitem{\kwd{allocation}}
 
\kwd{allocation} can be supplied once at most for a given \term{slot}.
The default if \kwd{allocation} is not supplied is \kwd{instance}.
 
\itemitem{\kwd{initarg}} 
                                                           
\kwd{initarg} can be supplied more than once for a given \term{slot}.  
 
\itemitem{\kwd{initform}}  
     
\kwd{initform} can be supplied once at most for a given \term{slot}.  
 
\itemitem{\kwd{type}} 
 
\kwd{type} can be supplied once at most for a given \term{slot}. 
 
\itemitem{\kwd{documentation}} 
 
\kwd{documentation} can be supplied once at most for a given \term{slot}. 

\itemitem{\kwd{report}}

\kwd{report} can be supplied once at most.

% Removed:
%  \itemitem{\kwd{conc-name}} ...
%  \itemitem{\kwd{documentation}} ...
\endlist
\endissue{CLOS-CONDITIONS:INTEGRATE}

\label Description::

%!!! Barrett: Some of this stuff
\macref{define-condition} defines a new condition type called \param{name}, 
which is a \term{subtype} of 
\issue{CLOS-CONDITIONS:INTEGRATE}
the \term{type} or \term{types} named by
  \param{parent-type}.  
Each \param{parent-type} argument specifies a direct \term{supertype}
of the new \term{condition}. The new \term{condition}
inherits \term{slots} and \term{methods} from each of its direct
\term{supertypes}, and so on.
\endissue{CLOS-CONDITIONS:INTEGRATE}
%% Redundant. -kmp 3-Sep-91
% \term{Objects} of this \term{condition} type have all of the 
% indicated \param{slots}, plus
%   any additional slots that would be available in \term{objects}
% of type \param{parent-type}.

  If a slot name/slot form pair is supplied,
the slot form is a \term{form} that 
can be evaluated by \funref{make-condition} to
  produce a default value when an explicit value is not provided.  If no 
slot form
is supplied, the contents of the \param{slot} 
is initialized in an 
  \term{implementation-dependent} way.  

  If the \term{type} being defined and some other 
\term{type} from which it inherits
  have a slot by the same name, only one slot is allocated in the
  \term{condition}, 
but the supplied slot form overrides any slot form
  that might otherwise have been inherited from a \param{parent-type}.  If no 
slot form is supplied, the inherited slot form (if any) is still visible.

%% This looks suspicious to me. -kmp 14-May-91
%  Barrett agrees. -kmp 3-Sep-91
% Once the \term{condition} is defined,
% \funref{make-condition} accepts keywords (from \thepackage{keyword}) with the
% \term{name} of any of the designated \param{slots},
% and will initialize the corresponding \param{slots} in \term{conditions} it creates.

Accessors are created according to the same rules as used by 
%\macref{defstruct}.
\macref{defclass}.

A description of \param{slot-options} follows:

%!!! Isn't there a way of contracting this?

\beginlist

\issue{CLOS-CONDITIONS:INTEGRATE}
   
\itemitem{\kwd{reader}}

The \kwd{reader} slot option specifies that an \term{unqualified method} is
to be defined on the \term{generic function} named by the argument
to \kwd{reader} to read the value of the given \term{slot}.
 
% Removed:
% \itemitem{\kwd{writer}} 
% \itemitem{\kwd{accessor}} 
% \itemitem{\kwd{allocation}} 

\itemitem{\bull} The \kwd{initform} slot option is used to provide a default
initial value form to be used in the initialization of the \term{slot}.  This
\term{form} is evaluated every time it is used to initialize the 
\term{slot}.  The
\term{lexical environment} 
in which this \term{form} is evaluated is the lexical
\term{environment} in which the \macref{define-condition} 
form was evaluated.
Note that the \term{lexical environment} refers both to variables and to
\term{functions}.  
For \term{local slots}, the \term{dynamic environment} is the dynamic
\term{environment} 
in which \funref{make-condition} was called; for 
\term{shared slots}, the \term{dynamic environment} 
is the \term{dynamic environment} in which the
\macref{define-condition} form was evaluated.  
%\Seesection\ObjectCreationAndInit.
 
\reviewer{Barmar: Issue CLOS-CONDITIONS doesn't say this.}
No implementation is permitted to extend the syntax of \macref{define-condition}
to allow \f{(\param{slot-name} \param{form})} as an abbreviation for
\f{(\param{slot-name} :initform \param{form})}.
 
\itemitem{\kwd{initarg}}

The \kwd{initarg} slot option declares an initialization
argument named by its \term{symbol} argument
and specifies that this
initialization argument initializes the given \term{slot}.  If the
initialization argument has a value in the call to 
\funref{initialize-instance}, the value is stored into the given \term{slot},
and the slot's \kwd{initform} slot option, if any, is not
evaluated.  If none of the initialization arguments specified for a
given \term{slot} has a value, the \term{slot} is initialized according to the
\kwd{initform} slot option, if specified.  
 
\itemitem{\kwd{type}}

The \kwd{type} slot option specifies that the contents of the
\term{slot} is always of the specified \term{type}.  It effectively
declares the result type of the reader generic function when applied
to an \term{object} of this \term{condition} type.  
The consequences of attempting to store in a
\term{slot} a value that 
does not satisfy the type of the \term{slot} is undefined.
%The \kwd{type} slot option is further discussed in \secref\SlotInheritance.
 
\itemitem{\kwd{default-initargs}}

\editornote{KMP: This is an option, not a slot option.}%!!!

This option is treated the same as it would be \macref{defclass}.

\itemitem{\kwd{documentation}}

\editornote{KMP: This is both an option and a slot option.}%!!!

The \kwd{documentation} slot option provides a \term{documentation string}
for the \term{slot}.

%Removed:
%  \itemitem{\kwd{documentation}} ...
%  \itemitem{\kwd{conc-name}} ...
\endissue{CLOS-CONDITIONS:INTEGRATE}
\itemitem{\kwd{report}}

\editornote{KMP: This is an option, not a slot option.}%!!!

\term{Condition} reporting is mediated through the \funref{print-object}
method for the \term{condition} type in question, with \varref{*print-escape*}
always being \nil. Specifying \f{(:report \param{report-name})} 
in the definition of a condition type \f{C} is equivalent to:

\code
 (defmethod print-object ((x c) stream)
   (if *print-escape* (call-next-method) (\param{report-name} x stream)))
\endcode
 
     If the value supplied by the argument to \kwd{report} (\param{report-name})
is a \term{symbol} or a \term{lambda expression}, 
it must be acceptable to 
     \specref{function}. \f{(function \param{report-name})} 
is evaluated
     in the current \term{lexical environment}.  
It should return a \term{function} 
of two
     arguments, a \term{condition} and a \term{stream}, 
that prints on the \term{stream} a
     description of the \term{condition}. 
 This \term{function} is called whenever the
     \term{condition} is printed while \varref{*print-escape*} is \nil.

If \param{report-name} is a \term{string}, it is a shorthand for 

\code
 (lambda (condition stream)
   (declare (ignore condition))
   (write-string \param{report-name} stream))
\endcode

This option is processed after the new \term{condition} type has been defined,
so use of the \param{slot} accessors within the \kwd{report} function is permitted.
If this option is not supplied, information about how to report this
type of \term{condition} is inherited from the \param{parent-type}.

\endlist

% !!! Barmar: This is redundant because CLOS already says this.
%     KMP: I think that until the connection between conditions and standard-class is
%          clearer, it doesn't hurt to say it redundantly.
The consequences are unspecifed if an attempt is made to \term{read} a 
\param{slot} that has not been explicitly initialized and that has not 
been given a default value.

The consequences are unspecified if an attempt is made to assign the
\param{slots} by using \macref{setf}.

\issue{COMPILE-FILE-HANDLING-OF-TOP-LEVEL-FORMS:CLARIFY}
% added qualification about top-level-ness  --sjl 5 Mar 92
If a \macref{define-condition} \term{form} appears as a \term{top level form},
the \term{compiler} must make \param{name} recognizable as a valid \term{type} name,
and it must be possible to reference the \term{condition} \term{type} as the
\param{parent-type} of another \term{condition} \term{type} in a subsequent
\macref{define-condition} \term{form} in the \term{file} being compiled.
\endissue{COMPILE-FILE-HANDLING-OF-TOP-LEVEL-FORMS:CLARIFY}

\label Examples::

The following form defines a condition of \term{type} 
\f{peg/hole-mismatch} which inherits from a condition type
called \f{blocks-world-error}:

\code
(define-condition peg/hole-mismatch 
                  (blocks-world-error)
                  ((peg-shape  :initarg :peg-shape
                               :reader peg/hole-mismatch-peg-shape)
                   (hole-shape :initarg :hole-shape
                               :reader peg/hole-mismatch-hole-shape))
  (:report (lambda (condition stream)
             (format stream "A ~A peg cannot go in a ~A hole."
                     (peg/hole-mismatch-peg-shape  condition)
                     (peg/hole-mismatch-hole-shape condition)))))
\endcode

The new type has slots \f{peg-shape} and \f{hole-shape}, 
so \funref{make-condition} accepts \f{:peg-shape} and \f{:hole-shape} keywords.  
The \term{readers} \f{peg/hole-mismatch-peg-shape} and \f{peg/hole-mismatch-hole-shape} 
apply to objects of this type, as illustrated in the \kwd{report} information.

The following form defines a \term{condition} \term{type} named \f{machine-error}
which inherits from \typeref{error}: 

\code
(define-condition machine-error 
                  (error)
                  ((machine-name :initarg :machine-name
                                 :reader machine-error-machine-name))
  (:report (lambda (condition stream)
             (format stream "There is a problem with ~A."
                     (machine-error-machine-name condition)))))
\endcode

Building on this definition, a new error condition can be defined which
is a subtype of \f{machine-error} for use when machines are not available: 

\code
(define-condition machine-not-available-error (machine-error) ()
  (:report (lambda (condition stream)
             (format stream "The machine ~A is not available."
                     (machine-error-machine-name condition)))))
\endcode

This defines a still more specific condition, built upon 
\f{machine-not-available-error}, which provides a slot initialization form
for \f{machine-name} but which does not provide any new slots or report
information.  It just gives the \f{machine-name} slot a default initialization:

\code
(define-condition my-favorite-machine-not-available-error
                  (machine-not-available-error)
  ((machine-name :initform "mc.lcs.mit.edu")))
\endcode

Note that since no \kwd{report} clause was given, the information 
inherited from \f{machine-not-available-error} is used to
report this type of condition.

\code
 (define-condition ate-too-much (error) 
     ((person :initarg :person :reader ate-too-much-person)
      (weight :initarg :weight :reader ate-too-much-weight)
      (kind-of-food :initarg :kind-of-food
                    :reader :ate-too-much-kind-of-food)))
\EV ATE-TOO-MUCH
 (define-condition ate-too-much-ice-cream (ate-too-much)
   ((kind-of-food :initform 'ice-cream)
    (flavor       :initarg :flavor
                  :reader ate-too-much-ice-cream-flavor
                  :initform 'vanilla ))
   (:report (lambda (condition stream)
              (format stream "~A ate too much ~A ice-cream"
                      (ate-too-much-person condition)
                      (ate-too-much-ice-cream-flavor condition)))))
\EV ATE-TOO-MUCH-ICE-CREAM
 (make-condition 'ate-too-much-ice-cream
                 :person 'fred
                 :weight 300
                 :flavor 'chocolate)
\EV #<ATE-TOO-MUCH-ICE-CREAM 32236101>
 (format t "~A" *)
\OUT FRED ate too much CHOCOLATE ice-cream
\EV NIL
\endcode

\label Affected By:\None.

\label Exceptional Situations:\None.

\label See Also::

\funref{make-condition}, \macref{defclass}, {\secref\ConditionSystemConcepts}

%% Per X3J13. -kmp 05-Oct-93
\label Notes:\None.

\endissue{DEFINE-CONDITION-SYNTAX:INCOMPATIBLY-MORE-LIKE-DEFCLASS+EMPHASIZE-READ-ONLY}

\endcom

%-------------------- Condition Instantiation --------------------

%%% ========== MAKE-CONDITION
\begincom{make-condition}\ftype{Function}

\label Syntax::

\DefunWithValues make-condition {type {\rest} slot-initializations} {condition}

\label Arguments and Values::

%Barmar: a condition type or condition type name ?
%Barrett worried about the same thing.  I think the intent was that it be a subtype of
%condition, so I've required that explicitly. -kmp 3-Sep-91
\param{type}---a \term{type specifier} (for a \term{subtype} of \typeref{condition}).

\param{slot-initializations}---an \term{initialization argument list}.

\param{condition}---a \term{condition}.

\label Description::

Constructs and returns a \term{condition} of type \param{type} 
using \param{slot-initializations} for the initial values of the slots.  
The newly created \term{condition} is returned.

\label Examples::
\issue{FORMAT-STRING-ARGUMENTS:SPECIFY}
\code
 (defvar *oops-count* 0)

 (setq a (make-condition 'simple-error
                         :format-control "This is your ~:R error."
                         :format-arguments (list (incf *oops-count*))))
\EV #<SIMPLE-ERROR 32245104>
 
 (format t "~&~A~%" a)
\OUT This is your first error.
\EV NIL
 
 (error a)
\OUT Error: This is your first error.
\OUT To continue, type :CONTINUE followed by an option number:
\OUT  1: Return to Lisp Toplevel.
\OUT Debug> 
\endcode
\endissue{FORMAT-STRING-ARGUMENTS:SPECIFY}

\label Side Effects:\None.

%% Sandra thinks this is excessive.
%Creates a \term{condition}.

\label Affected By::

The set of defined \term{condition} \term{types}.

\label Exceptional Situations:\None.

\label See Also::

\macref{define-condition}, {\secref\ConditionSystemConcepts}

\label Notes:\None.

\endcom

%-------------------- Restarts --------------------

\begincom{restart}\ftype{System Class}

\label Class Precedence List::
\typeref{restart},
\typeref{t}

\label Description::

An \term{object} \oftype{restart} represents a \term{function} that can be
called to perform some form of recovery action, usually a transfer of control 
to an outer point in the running program.

An \term{implementation} is free to implement a \term{restart} in whatever 
manner is most convenient; a \term{restart} has only \term{dynamic extent}
relative to the scope of the binding \term{form} which \term{establishes} it.

\endcom%{restart}\ftype{System Class}

%%% ========== COMPUTE-RESTARTS
\begincom{compute-restarts}\ftype{Function}

\label Syntax::

\issue{CONDITION-RESTARTS:PERMIT-ASSOCIATION}
\DefunWithValues compute-restarts {{\opt} condition} {restarts}
\endissue{CONDITION-RESTARTS:PERMIT-ASSOCIATION}

\label Arguments and Values::

\issue{CONDITION-RESTARTS:PERMIT-ASSOCIATION}
\param{condition}---a \term{condition} \term{object}, or \nil.
\endissue{CONDITION-RESTARTS:PERMIT-ASSOCIATION}

\param{restarts}---a \term{list} of \term{restarts}.

\label Description::

\funref{compute-restarts} uses the dynamic state of the program to compute 
a \term{list} of the \term{restarts} which are currently active.

The resulting \term{list} is ordered so that the innermost
(more-recently established) restarts are nearer the head of the \term{list}.

\issue{CONDITION-RESTARTS:PERMIT-ASSOCIATION}
When \param{condition} is \term{non-nil}, only those \term{restarts}
are considered that are either explicitly associated with that \param{condition},
or not associated with any \term{condition}; that is, the excluded \term{restarts} 
are those that are associated with a non-empty set of \term{conditions} of 
which the given \param{condition} is not an \term{element}.
If \param{condition} is \nil, all \term{restarts} are considered.
\endissue{CONDITION-RESTARTS:PERMIT-ASSOCIATION}

\funref{compute-restarts} returns all 
%"valid" -> "applicable" per Barrett
\term{applicable restarts}, 
including anonymous ones, even if some of them have the same name as 
others and would therefore not be found by \funref{find-restart} 
when given a \term{symbol} argument.

Implementations are permitted, but not required, to return \term{distinct}
\term{lists} from repeated calls to \funref{compute-restarts} while in
the same dynamic environment.  
The consequences are undefined if the \term{list} returned by
\funref{compute-restarts} is every modified.

\label Examples::

\code
 ;; One possible way in which an interactive debugger might present
 ;; restarts to the user.
 (defun invoke-a-restart ()
   (let ((restarts (compute-restarts)))
     (do ((i 0 (+ i 1)) (r restarts (cdr r))) ((null r))
       (format t "~&~D: ~A~%" i (car r)))
     (let ((n nil) (k (length restarts)))
       (loop (when (and (typep n 'integer) (>= n 0) (< n k))
               (return t))
             (format t "~&Option: ")
             (setq n (read))
             (fresh-line))
       (invoke-restart-interactively (nth n restarts)))))

 (restart-case (invoke-a-restart)
   (one () 1)
   (two () 2)
   (nil () :report "Who knows?" 'anonymous)
   (one () 'I)
   (two () 'II))
\OUT 0: ONE
\OUT 1: TWO
\OUT 2: Who knows?
\OUT 3: ONE
\OUT 4: TWO
\OUT 5: Return to Lisp Toplevel.
\OUT Option: \IN{4}
\EV II
 
 ;; Note that in addition to user-defined restart points, COMPUTE-RESTARTS
 ;; also returns information about any system-supplied restarts, such as
 ;; the "Return to Lisp Toplevel" restart offered above.
 
\endcode
 

\label Side Effects:\None.

\label Affected By::

Existing restarts.

\label Exceptional Situations:\None.

\label See Also::

\funref{find-restart},
\funref{invoke-restart},
\macref{restart-bind}

\label Notes:\None.

\endcom

%%% ========== FIND-RESTART
\begincom{find-restart}\ftype{Function}

\label Syntax::

\issue{CONDITION-RESTARTS:PERMIT-ASSOCIATION}
\Defun find-restart {identifier {\opt} condition} {restart}
\endissue{CONDITION-RESTARTS:PERMIT-ASSOCIATION}

\label Arguments and Values::

\param{identifier}---a \term{non-nil} \term{symbol}, or a \term{restart}.

\issue{CONDITION-RESTARTS:PERMIT-ASSOCIATION}
\param{condition}---a \term{condition} \term{object}, or \nil.
\endissue{CONDITION-RESTARTS:PERMIT-ASSOCIATION}

\param{restart}---a \term{restart} or \nil.

\label Description::

\funref{find-restart} searches for a particular \term{restart} in the 
current \term{dynamic environment}.

\issue{CONDITION-RESTARTS:PERMIT-ASSOCIATION}
When \param{condition} is \term{non-nil}, only those \term{restarts}
are considered that are either explicitly associated with that \param{condition},
or not associated with any \term{condition}; that is, the excluded \term{restarts} 
are those that are associated with a non-empty set of \term{conditions} of 
which the given \param{condition} is not an \term{element}.
If \param{condition} is \nil, all \term{restarts} are considered.
\endissue{CONDITION-RESTARTS:PERMIT-ASSOCIATION}

If \param{identifier} is a \term{symbol}, then the innermost 
(most recently established) \term{applicable restart} with that \term{name} is returned.
\nil\ is returned if no such restart is found.

If \param{identifier} is a currently active restart, then it is returned.
Otherwise, \nil\ is returned.

\label Examples::

\code
 (restart-case
     (let ((r (find-restart 'my-restart)))
       (format t "~S is named ~S" r (restart-name r)))
   (my-restart () nil))
\OUT #<RESTART 32307325> is named MY-RESTART
\EV NIL
 (find-restart 'my-restart)
\EV NIL
\endcode

\label Side Effects:\None.

\label Affected By::

Existing restarts.

\macref{restart-case}, \macref{restart-bind}, \macref{with-condition-restarts}.

\label Exceptional Situations:\None.

\label See Also::

\funref{compute-restarts}

\label Notes::

\code
 (find-restart \param{identifier})
 \EQ (find \param{identifier} (compute-restarts) :key :restart-name)
\endcode

Although anonymous restarts have a name of \nil,
the consequences are unspecified if \nil\ is given as an \param{identifier}.  
Occasionally, programmers lament that \nil\ is not permissible as an
\param{identifier} argument.  In most such cases, \funref{compute-restarts}
can probably be used to simulate the desired effect.

\endcom

%%% ========== INVOKE-RESTART
\begincom{invoke-restart}\ftype{Function}

\label Syntax::

\DefunWithValues invoke-restart
		 {restart {\rest} arguments}
		 {\starparam{result}}

\label Arguments and Values::

\param{restart}---a \term{restart designator}.

\param{argument}---an \term{object}.

\param{results}---the \term{values} returned by the \term{function}
		   associated with \param{restart}, if that \term{function} returns.

\label Description::

Calls the \term{function} associated with \param{restart},
passing \param{arguments} to it.  
\param{Restart} must be valid in the current \term{dynamic environment}.  

\label Examples::
\code
 (defun add3 (x) (check-type x number) (+ x 3))
 
 (foo 'seven)
\OUT Error: The value SEVEN was not of type NUMBER.
\OUT To continue, type :CONTINUE followed by an option number:
\OUT  1: Specify a different value to use.
\OUT  2: Return to Lisp Toplevel.
\OUT Debug> \IN{(invoke-restart 'store-value 7)}
\EV 10
\endcode

\label Side Effects::

%!!! Barmar: This is true whenever calling random functions.
A non-local transfer of control might be done by the restart.

\label Affected By::

Existing restarts.

\label Exceptional Situations::

If \param{restart} is not valid, an error \oftype{control-error} is signaled.

\label See Also::

\funref{find-restart},
\macref{restart-bind},
\macref{restart-case},
\funref{invoke-restart-interactively}

\label Notes::

The most common use for \funref{invoke-restart} is in a \term{handler}.
It might be used explicitly, or implicitly through \funref{invoke-restart-interactively}
or a \term{restart function}.

\term{Restart functions} call \funref{invoke-restart}, not vice versa.  That is,
\term{invoke-restart} provides primitive functionality, and \term{restart functions}
are non-essential ``syntactic sugar.''

\endcom

%%% ========== INVOKE-RESTART-INTERACTIVELY
\begincom{invoke-restart-interactively}\ftype{Function}

\label Syntax::

\DefunWithValues invoke-restart-interactively {restart} {\starparam{result}}

\label Arguments and Values::

\param{restart}---a \term{restart designator}.

\param{results}---the \term{values} returned by the \term{function} 
		   associated with \param{restart}, if that \term{function} returns.

\label Description::

\funref{invoke-restart-interactively} calls the \term{function} associated
with \param{restart}, prompting for any necessary arguments. 
If \param{restart} is a name, it must be valid in the current \term{dynamic environment}.  

  \funref{invoke-restart-interactively} 
prompts for arguments by executing
  the code provided in the \kwd{interactive} keyword to 
\macref{restart-case} or 
  \kwd{interactive-function} keyword to \macref{restart-bind}.

%!!! Barrett: Make consistent with wrong # of args errors.
If no such options have been supplied in the corresponding
\macref{restart-bind} or \macref{restart-case}, 
then the consequences are undefined if the \param{restart} takes
  required arguments.  If the arguments are optional, an argument list of
  \nil\ is used.

  Once the arguments have been determined, 
\funref{invoke-restart-interactively}
  executes the following:

\code
 (apply #'invoke-restart \i{restart} \i{arguments})
\endcode
 

\label Examples::

\code
 (defun add3 (x) (check-type x number) (+ x 3))
 
 (add3 'seven)
\OUT Error: The value SEVEN was not of type NUMBER.
\OUT To continue, type :CONTINUE followed by an option number:
\OUT  1: Specify a different value to use.
\OUT  2: Return to Lisp Toplevel.
\OUT Debug> \IN{(invoke-restart-interactively 'store-value)}
\OUT Type a form to evaluate and use: \IN{7}
\EV 10
\endcode
 
\label Side Effects::

If prompting for arguments is necesary,
some typeout may occur (on \term{query I/O}).
 
%!!! Barmar: This is true whenever calling random functions.
A non-local transfer of control might be done by the restart.

\label Affected By::

\varref{*query-io*}, active \term{restarts}

\label Exceptional Situations::

If \param{restart} is not valid, an error \oftype{control-error}
is signaled.

\label See Also::

\funref{find-restart},
\funref{invoke-restart},
\macref{restart-case},
\macref{restart-bind}

\label Notes::

\funref{invoke-restart-interactively} is used internally by the debugger
and may also be useful in implementing other portable, interactive debugging 
tools.

\endcom

%%% ========== RESTART-BIND
\begincom{restart-bind}\ftype{Macro}

\label Syntax::

\DefmacWithValuesNewline restart-bind 
		  {\paren{\curly{\paren{name function
				        \stardown{key-val-pair}}}}
		   \starparam{form}}
		  {\starparam{result}}

%!!! Barmar: Somehow indicate that each may be supplied at most once.
\auxbnf{key-val-pair}{\kwd{interactive-function} {interactive-function} | \CR
		      \kwd{report-function} {report-function}           | \CR
		      \kwd{test-function} {test-function}}

\label Arguments and Values::

\param{name}---a \term{symbol}; \noeval.

\param{function}---a \term{form}; \eval.

\param{forms}---an \term{implicit progn}.

\param{interactive-function}---a \term{form}; \eval.

\param{report-function}---a \term{form}; \eval.

\param{test-function}---a \term{form}; \eval.

\param{results}---the \term{values} returned by the \term{forms}.

\label Description::

\macref{restart-bind} executes the body of \param{forms} 
in a \term{dynamic environment} where \term{restarts} with the given \param{names} are in effect.
 
If a \param{name} is \nil, it indicates an anonymous restart;
if a \param{name} is a \term{non-nil} \term{symbol}, it indicates a named restart.
 
The \param{function}, \param{interactive-function}, and \param{report-function}
are unconditionally evaluated in the current lexical and dynamic environment
prior to evaluation of the body. Each of these \term{forms} must evaluate to
a \term{function}.
 
If \funref{invoke-restart} is done on that restart,
the \term{function} which resulted from evaluating \param{function} 
is called, in the \term{dynamic environment} of the \funref{invoke-restart},
with the \term{arguments} given to \funref{invoke-restart}. 
The \term{function} may either perform a non-local transfer of control or may return normally.
 
If the restart is invoked interactively from the debugger 
(using \funref{invoke-restart-interactively}), 
the arguments are defaulted by calling the \term{function} 
which resulted from evaluating \param{interactive-function}.
That \term{function} may optionally prompt interactively on \term{query I/O}, 
and should return a \term{list} of arguments to be used by
\funref{invoke-restart-interactively} when invoking the restart. 
 
If a restart is invoked interactively but no \param{interactive-function} is used,
then an argument list of \nil\ is used. In that case, the \term{function}
must be compatible with an empty argument list.
 
If the restart is presented interactively (\eg by the debugger),
the presentation is done by calling the \term{function} which resulted
from evaluating \param{report-function}.
This \term{function} must be a \term{function} of one argument, a \term{stream}. 
It is expected to print a description of the action that the restart takes
to that \term{stream}. 
This \term{function} is called any time the restart is printed 
while \varref{*print-escape*} is \nil.
 
In the case of interactive invocation, 
the result is dependent on the value of \kwd{interactive-function}
as follows.

\beginlist
\itemitem{\kwd{interactive-function}}

  \param{Value} is evaluated in the current lexical environment and
  should return a \term{function} of no arguments which constructs a 
  \term{list} of arguments to be used by \funref{invoke-restart-interactively} 
  when invoking this restart.  The \term{function} may prompt interactively
  using \term{query I/O} if necessary.

\itemitem{\kwd{report-function}}

  \param{Value} is evaluated in the current lexical environment and
  should return a \term{function} of one argument, a \term{stream}, which
  prints on the \term{stream} a summary of the action that this restart
  takes.  This \term{function} is called whenever the restart is
  reported (printed while \varref{*print-escape*} is \nil).
% This next was added for Barmar. -kmp 1-Sep-91
  If no \kwd{report-function} option is provided, the manner in which the
  \term{restart} is reported is \term{implementation-dependent}.

\issue{CONDITION-RESTARTS:PERMIT-ASSOCIATION}
\itemitem{\kwd{test-function}}

  \param{Value} is evaluated in the current lexical environment and
  should return a \term{function} of one argument, a \term{condition}, which
  returns \term{true} if the restart is to be considered visible.
\endissue{CONDITION-RESTARTS:PERMIT-ASSOCIATION}

\endlist

% \label Examples::
% 
%\code
% (defun choose-an-interactive-restart ()
%   (restart-bind
%       ((optional-value
%          #'(lambda (&optional (x 'default)) x)
%          :report-function #'(lambda (stream)
%                               (format stream "Return an optional value")))\kern-3pt
%        (return-value
%          #'identity
%          :report-function #'(lambda (stream)
%                               (format stream "Return the given value"))
%          :interactive-function #'(lambda ()
%                                    (format t "Enter a value to return: ")
%                                  (list (eval (read))))))
%     (let ((cases (compute-restarts))
%           (*print-structure* t)
%           (index -1))
%       (dolist (case cases)
%         (format t "~&~D: ~A~%" (incf index) case))
%       (format t "Please enter restart to invoke: ")
%       (invoke-restart-interactively (nth (eval (read)) cases)))))
%\EV\ CHOOSE-AN-INTERACTIVE-RESTART
% (choose-an-interactive-restart)
%\EV\ 0: Return an optional value
%1: Return the given value
%2: Abort to Lisp Top Level
%Please enter restart to invoke: 0
%DEFAULT
% (choose-an-interactive-restart)
%\EV\ 0: Return an optional value
%1: Return the given value
%2: Abort to Lisp Top Level
%Please enter restart to invoke: 1
%Enter a value to return: t
%T
%\endcode

\label Side Effects:\None.

\label Affected By::

\varref{*query-io*}.

\label Exceptional Situations:\None.

\label See Also::

\macref{restart-case}, \macref{with-simple-restart}

\label Notes::

\macref{restart-bind} is primarily intended to be used to implement
\macref{restart-case} and  might be useful in implementing other
macros. Programmers who are uncertain about whether to use \macref{restart-case}
or \macref{restart-bind} should prefer \macref{restart-case} for the cases where
it is powerful enough, using \macref{restart-bind} only in cases where its full
generality is really needed.
 
\endcom

%%% ========== RESTART-CASE
\begincom{restart-case}\ftype{Macro}

\issue{DECLS-AND-DOC}

\label Syntax::

\DefmacWithValues restart-case 
		  {restartable-form {\curly{\down{clause}}}}
		  {\starparam{result}}

\issue{CONDITION-RESTARTS:PERMIT-ASSOCIATION}
\auxbnf{clause}{\lparen case-name lambda-list			       \CR
	        \ \interleave{\kwd{interactive} interactive-expression |
		      	      \kwd{report} report-expression           |
			      \kwd{test} test-expression}              \CR
		\ \starparam{declaration} \starparam{form}\rparen}
\endissue{CONDITION-RESTARTS:PERMIT-ASSOCIATION}

\label Arguments and Values::

\param{restartable-form}---a \term{form}.

\param{case-name}---a \term{symbol} or \nil.

\param{lambda-list}---an \term{ordinary lambda list}.

\param{interactive-expression}---a \term{symbol} or a \term{lambda expression}.

\param{report-expression}---a \term{string},
			     a \term{symbol},
			     or a \term{lambda expression}.                                 

\param{test-expression}---a \term{symbol} or a \term{lambda expression}.

\param{declaration}---a \misc{declare} \term{expression}; \noeval.

\param{form}---a \term{form}.

\param{results}---the \term{values} resulting from the \term{evaluation}
		   of \param{restartable-form}, 
		  or the \term{values} returned by the last \param{form}
		   executed in a chosen \term{clause},
		  or \nil.

\label Description::

\macref{restart-case} evaluates \param{restartable-form} in a \term{dynamic environment}
where the clauses have special meanings as points to which control may be transferred.  
If \param{restartable-form} finishes executing and returns any values, 
all values returned are returned by \macref{restart-case} and 
processing has completed. While \param{restartable-form} is executing, any code may
  transfer control to one of the clauses (see \funref{invoke-restart}).  
If a transfer
  occurs, the forms in the body of that clause is evaluated and any values
  returned by the last such form are returned by 
\macref{restart-case}.
In this case, the 
dynamic state is unwound appropriately (so that the restarts established
around the \param{restartable-form} are no longer active) prior to execution of the
clause.

  If there are no \param{forms} 
in a selected clause, \macref{restart-case} returns \nil.

If \param{case-name} is a \term{symbol}, it names this restart.

It is possible to have more than one clause use the same \param{case-name}.
In this case, the first clause with that name is found by \funref{find-restart}.  
The other clauses are accessible using \funref{compute-restarts}.

Each \param{arglist} is an \term{ordinary lambda list} to be bound during the 
execution of its corresponding \param{forms}.  These parameters are used 
by the \macref{restart-case} clause to receive any necessary data from a call
to \funref{invoke-restart}.

By default, \funref{invoke-restart-interactively} passes no arguments and
all arguments must be optional in order to accomodate interactive
restarting.  However, the arguments need not be optional if the
\kwd{interactive} 
keyword has been used to inform \funref{invoke-restart-interactively}
  about how to compute a proper argument list.

\param{Keyword} options have the following meaning.
\beginlist
\itemitem{\kwd{interactive}}
   
The \param{value} supplied by \f{:interactive \param{value}}
must be a suitable argument to \specref{function}. 
\f{(function \param{value})} is evaluated in the current lexical
    environment.  It should return a \term{function} of no arguments which 
    returns arguments to be used by 
\funref{invoke-restart-interactively} when it is invoked.
\funref{invoke-restart-interactively} 
is called in the dynamic
    environment available prior to any restart attempt, and uses 
\term{query I/O} for user interaction.

    If a restart is invoked interactively but no \kwd{interactive} option
    was supplied, the argument list used in the invocation is the empty
    list.

\itemitem{\kwd{report}}

If the \param{value} supplied by \f{:report \param{value}}
is a \term{lambda expression} or a \term{symbol}, it 
must be acceptable to \specref{function}.
\f{(function \param{value})} is evaluated in the current lexical
environment.  It should return a \term{function} of one
argument, a \term{stream}, which prints on the \term{stream} a 
description of the restart.  This \term{function} is called 
whenever the restart is printed while \varref{*print-escape*} is \nil.

If \param{value} is a \term{string}, it is a shorthand for

\code
 (lambda (stream) (write-string value stream))
\endcode

    If a named restart is asked to report but no report information has been
    supplied, the name of the restart is used in generating default report text.
  
    When \varref{*print-escape*} is \nil, the 
printer uses the report information for
    a restart.  For example, a debugger might announce the action of typing
    a ``continue'' command by:

\code
 (format t "~&~S -- ~A~%" ':continue some-restart)
\endcode
    which might then display as something like:

\code
 :CONTINUE -- Return to command level
\endcode

The consequences are unspecified if an unnamed restart is specified
but no \kwd{report} option is provided.

\issue{CONDITION-RESTARTS:PERMIT-ASSOCIATION}
\itemitem{\kwd{test}}

The \param{value} supplied by \f{:test \param{value}}
must be a suitable argument to \specref{function}. 
\f{(function \param{value})} is evaluated in the current lexical
    environment.  It should return a \term{function} of one \term{argument}, the
\term{condition}, that
returns \term{true} if the restart is to be considered visible.

The default for this option is equivalent to \f{(lambda (c) (declare (ignore c)) t)}.
\endissue{CONDITION-RESTARTS:PERMIT-ASSOCIATION}
\endlist

\issue{CONDITION-RESTARTS:PERMIT-ASSOCIATION}
If the \param{restartable-form} is a \term{list} whose \term{car} is any of
the \term{symbols} \funref{signal}, \funref{error}, \funref{cerror},
or \funref{warn} (or is a \term{macro form} which macroexpands into such a
\term{list}), then \macref{with-condition-restarts} is used implicitly
to associate the indicated \term{restarts} with the \term{condition} to be
signaled.
\endissue{CONDITION-RESTARTS:PERMIT-ASSOCIATION}

\label Examples::

\code
 (restart-case
     (handler-bind ((error #'(lambda (c)
                             (declare (ignore condition))
                             (invoke-restart 'my-restart 7))))
       (error "Foo."))
   (my-restart (&optional v) v))
\EV 7

 (define-condition food-error (error) ())
\EV FOOD-ERROR
 (define-condition bad-tasting-sundae (food-error) 
   ((ice-cream :initarg :ice-cream :reader bad-tasting-sundae-ice-cream)
    (sauce :initarg :sauce :reader bad-tasting-sundae-sauce)
    (topping :initarg :topping :reader bad-tasting-sundae-topping))
   (:report (lambda (condition stream)
              (format stream "Bad tasting sundae with ~S, ~S, and ~S"
                      (bad-tasting-sundae-ice-cream condition)
                      (bad-tasting-sundae-sauce condition)
                      (bad-tasting-sundae-topping condition)))))
\EV BAD-TASTING-SUNDAE
 (defun all-start-with-same-letter (symbol1 symbol2 symbol3)
   (let ((first-letter (char (symbol-name symbol1) 0)))
     (and (eql first-letter (char (symbol-name symbol2) 0))
          (eql first-letter (char (symbol-name symbol3) 0)))))
\EV ALL-START-WITH-SAME-LETTER
 (defun read-new-value ()
   (format t "Enter a new value: ")
   (multiple-value-list (eval (read))))
\EV READ-NEW-VALUE\eject
 (defun verify-or-fix-perfect-sundae (ice-cream sauce topping)
   (do ()
      ((all-start-with-same-letter ice-cream sauce topping))
     (restart-case
       (error 'bad-tasting-sundae
              :ice-cream ice-cream
              :sauce sauce
              :topping topping)
       (use-new-ice-cream (new-ice-cream)
         :report "Use a new ice cream."
         :interactive read-new-value  
         (setq ice-cream new-ice-cream))
       (use-new-sauce (new-sauce)
         :report "Use a new sauce."
         :interactive read-new-value
         (setq sauce new-sauce))
       (use-new-topping (new-topping)
         :report "Use a new topping."
         :interactive read-new-value
         (setq topping new-topping))))
   (values ice-cream sauce topping))
\EV VERIFY-OR-FIX-PERFECT-SUNDAE
 (verify-or-fix-perfect-sundae 'vanilla 'caramel 'cherry)
\OUT Error: Bad tasting sundae with VANILLA, CARAMEL, and CHERRY.
\OUT To continue, type :CONTINUE followed by an option number:
\OUT  1: Use a new ice cream.
\OUT  2: Use a new sauce.
\OUT  3: Use a new topping.
\OUT  4: Return to Lisp Toplevel.
\OUT Debug> \IN{:continue 1}
\OUT Use a new ice cream.
\OUT Enter a new ice cream: \IN{'chocolate}
\EV CHOCOLATE, CARAMEL, CHERRY
\endcode

\label Side Effects:\None.

\label Affected By:\None.

\label Exceptional Situations:\None.

\label See Also::

\macref{restart-bind}, \macref{with-simple-restart}.

\label Notes::

\code
 (restart-case \i{expression}
    (\i{name1} \i{arglist1} ...\i{options1}... . \i{body1})
    (\i{name2} \i{arglist2} ...\i{options2}... . \i{body2}))
\endcode
  is essentially equivalent to

\code
 (block #1=#:g0001
   (let ((#2=#:g0002 nil))
        (tagbody
        (restart-bind ((name1 #'(lambda (&rest temp)
                                (setq #2# temp)
                                (go #3=#:g0003))
                          ...\i{slightly-transformed-options1}...)
                       (name2 #'(lambda (&rest temp)
                                (setq #2# temp)
                                (go #4=#:g0004))
                          ...\i{slightly-transformed-options2}...))
        (return-from #1# \i{expression}))
          #3# (return-from #1#
                  (apply #'(lambda \i{arglist1} . \i{body1}) #2#))
          #4# (return-from #1#
                  (apply #'(lambda \i{arglist2} . \i{body2}) #2#)))))
\endcode

Unnamed restarts are generally only useful interactively
    and an interactive option which has no description is of little value.
    Implementations are encouraged to warn if 
an unnamed restart is used and no report information
    is provided
at compilation    time.  
At runtime, this error might be noticed when entering
      the debugger.  Since signaling an error would probably cause recursive
      entry into the debugger (causing yet another recursive error, etc.) it is
      suggested that the debugger print some indication of such problems when
      they occur but not actually signal errors.
  
\issue{CONDITION-RESTARTS:PERMIT-ASSOCIATION}
\code
 (restart-case (signal fred)
   (a ...)
   (b ...))
 \EQ
 (restart-case
     (with-condition-restarts fred 
                              (list (find-restart 'a) 
                                    (find-restart 'b))
       (signal fred))
   (a ...)
   (b ...))
\endcode
\endissue{CONDITION-RESTARTS:PERMIT-ASSOCIATION}

\endissue{DECLS-AND-DOC}

\endcom 

%%% ========== RESTART-NAME
\begincom{restart-name}\ftype{Function}

\label Syntax::

\DefunWithValues restart-name {restart} {name}

\label Arguments and Values:: 

\param{restart}---a \term{restart}.

\param{name}---a \term{symbol}.

\label Description::

Returns the name of the \param{restart},
or \nil\ if the \param{restart} is not named.

\label Examples::

\code
 (restart-case 
     (loop for restart in (compute-restarts)
               collect (restart-name restart))
   (case1 () :report "Return 1." 1)
   (nil   () :report "Return 2." 2)
   (case3 () :report "Return 3." 3)
   (case1 () :report "Return 4." 4))
\EV (CASE1 NIL CASE3 CASE1 ABORT)
 ;; In the example above the restart named ABORT was not created
 ;; explicitly, but was implicitly supplied by the system.
\endcode

\label Side Effects:\None.

\label Affected By:\None.

\label Exceptional Situations:\None.

\label See Also::

\funref{compute-restarts}
\funref{find-restart}

\label Notes:\None.

\endcom

%%% ========== WITH-CONDITION-RESTARTS

\begincom{with-condition-restarts}\ftype{Macro}

\issue{CONDITION-RESTARTS:PERMIT-ASSOCIATION}

\label Syntax::

\DefmacWithValuesNewline with-condition-restarts
		  {condition-form restarts-form \starparam{form}}
		  {\starparam{result}}

\label Arguments and Values::
 
\param{condition-form}---a \term{form}; \term{evaluated} to produce a \param{condition}.

\param{condition}---a \term{condition} \term{object} resulting from the 
		    \term{evaluation} of \param{condition-form}.

\param{restart-form}---a \term{form}; \term{evaluated} to produce a \param{restart-list}.

\param{restart-list}---a \term{list} of \term{restart} \term{objects} resulting 
		       from the \term{evaluation} of \param{restart-form}.

\param{forms}---an \term{implicit progn}; \eval.

\param{results}---the \term{values} returned by \param{forms}.

\label Description::

First, the \param{condition-form} and \param{restarts-form} are \term{evaluated}
in normal left-to-right order; the \term{primary values} yielded by these
\term{evaluations} are respectively called the \param{condition} 
and the \param{restart-list}.

Next, the \param{forms} are \term{evaluated} in a \term{dynamic environment}
in which each \term{restart} in \param{restart-list} is associated with
the \param{condition}.  \Seesection\AssocRestartWithCond.

\label Examples:\None.

\label Side Effects:\None.

\label Affected By:\None.

\label Exceptional Situations:\None.

\label See Also::

\macref{restart-case}

\label Notes::

Usually this \term{macro} is not used explicitly in code, 
since \macref{restart-case} handles most of the common cases
in a way that is syntactically more concise.

\endissue{CONDITION-RESTARTS:PERMIT-ASSOCIATION}

\endcom


%%% ========== WITH-SIMPLE-RESTART
\begincom{with-simple-restart}\ftype{Macro}

\label Syntax::

\DefmacWithValuesNewline with-simple-restart 
		  {\paren{name format-control \starparam{format-argument}}
		   \starparam{form}}
		  {\starparam{result}}

\label Arguments and Values::

\param{name}---a \term{symbol}.

\issue{FORMAT-STRING-ARGUMENTS:SPECIFY}
\param{format-control}---a \term{format control}.
\endissue{FORMAT-STRING-ARGUMENTS:SPECIFY}

\param{format-argument}---an \term{object} (\ie a \term{format argument}).

\param{forms}---an \term{implicit progn}.
 
\param{results}---in the normal situation,
   the \term{values} returned by the \param{forms};
   in the exceptional situation where the \term{restart} named \param{name} is invoked,
   two values---\nil\ and \t.
 
\label Description::

\macref{with-simple-restart} establishes a restart.  

If the restart designated by \param{name} is not invoked while executing \param{forms},
all values returned by the last of \param{forms} are returned. 
If the restart designated by \param{name} is invoked,
control is transferred to \macref{with-simple-restart},
which returns two values, \nil\ and \t.

If \param{name} is \nil, an anonymous restart is established.

The \param{format-control} and \param{format-arguments} are used 
report the \term{restart}.

\label Examples::

\code
 (defun read-eval-print-loop (level)
   (with-simple-restart (abort "Exit command level ~D." level)
     (loop
       (with-simple-restart (abort "Return to command level ~D." level)
         (let ((form (prog2 (fresh-line) (read) (fresh-line))))
           (prin1 (eval form)))))))
\EV READ-EVAL-PRINT-LOOP
 (read-eval-print-loop 1)
 (+ 'a 3)
\OUT Error: The argument, A, to the function + was of the wrong type.
\OUT        The function expected a number.
\OUT To continue, type :CONTINUE followed by an option number:
\OUT  1: Specify a value to use this time.
\OUT  2: Return to command level 1.
\OUT  3: Exit command level 1.
\OUT  4: Return to Lisp Toplevel.
\endcode

\code
 (defun compute-fixnum-power-of-2 (x)
   (with-simple-restart (nil "Give up on computing 2{\hat}~D." x)
     (let ((result 1))
       (dotimes (i x result)
         (setq result (* 2 result))
         (unless (fixnump result)
           (error "Power of 2 is too large."))))))
COMPUTE-FIXNUM-POWER-OF-2
 (defun compute-power-of-2 (x)
   (or (compute-fixnum-power-of-2 x) 'something big))
COMPUTE-POWER-OF-2
 (compute-power-of-2 10)
1024
 (compute-power-of-2 10000)
\OUT Error: Power of 2 is too large.
\OUT To continue, type :CONTINUE followed by an option number.
\OUT  1: Give up on computing 2{\hat}10000.
\OUT  2: Return to Lisp Toplevel
\OUT Debug> \IN{:continue 1}
\EV SOMETHING-BIG
\endcode

\label Side Effects:\None.

\label Affected By:\None.

\label Exceptional Situations:\None.

\label See Also::

\macref{restart-case}

\label Notes::

\macref{with-simple-restart} is shorthand for one of the most
common uses of \macref{restart-case}.

\macref{with-simple-restart} could be defined by:

\issue{FORMAT-STRING-ARGUMENTS:SPECIFY}
\code
 (defmacro with-simple-restart ((restart-name format-control
                                              &rest format-arguments)
                                &body forms)
   `(restart-case (progn ,@forms)
      (,restart-name ()
          :report (lambda (stream)
                    (format stream ,format-control ,@format-arguments))
         (values nil t))))
\endcode
\endissue{FORMAT-STRING-ARGUMENTS:SPECIFY}

Because the second return value is \t\ in the exceptional case,
it is common (but not required) to arrange for the second return value
in the normal case to be missing or \nil\ so that the two situations
can be distinguished.

\endcom

%-------------------- Pre-Defined Restarts --------------------


%%% ========== ABORT
\begincom{abort}\ftype{Restart}

\label Data Arguments Required::

None.

\label Description::

The intent of the \misc{abort} restart is to allow return to the
innermost ``command level.''  Implementors are encouraged to make 
sure that there is always a restart named \funref{abort} 
around any user code so that user code can call \funref{abort} 
at any time and expect something reasonable to happen;
exactly what the reasonable thing is may vary somewhat.  Typically,
in an interactive listener, the invocation of \funref{abort}
returns to the \term{Lisp reader} phase of the \term{Lisp read-eval-print loop},
though in some batch or multi-processing
situations there may be situations in which having it kill the running 
process is more appropriate.

\label See Also::

\secref\Restarts,
{\secref\InterfacesToRestarts},
\funref{invoke-restart},
\funref{abort} (\term{function})

\endcom%{abort}

%%% ========== CONTINUE
\begincom{continue}\ftype{Restart}

\label Data Arguments Required::

None.

\label Description::

\Therestart{continue} is generally part of protocols where there is
  a single ``obvious'' way to continue, such as in 
\funref{break} and \funref{cerror}.  Some
  user-defined protocols may also wish to incorporate it for similar reasons.
  In general, however, it is more reliable to design a special purpose restart
  with a name that more directly suits the particular application.

\label Examples::

\code
 (let ((x 3))
   (handler-bind ((error #'(lambda (c)
                             (let ((r (find-restart 'continue c)))
                               (when r (invoke-restart r))))))
     (cond ((not (floatp x))
            (cerror "Try floating it." "~D is not a float." x)
            (float x))
           (t x)))) \EV 3.0
\endcode   

\label See Also::

\secref\Restarts,
{\secref\InterfacesToRestarts},
\funref{invoke-restart},
\funref{continue} (\term{function}),
\macref{assert},
\funref{cerror}

\endcom%{continue}

%%% ========== MUFFLE-WARNING
\begincom{muffle-warning}\ftype{Restart}

\label Data Arguments Required::

None.

\label Description::

This \term{restart} is established by \funref{warn} so that \term{handlers}
of \typeref{warning} \term{conditions} have a way to tell \funref{warn} 
that a warning has already been dealt with and that no further action is warranted.

\label Examples::

\code
 (defvar *all-quiet* nil) \EV *ALL-QUIET*
 (defvar *saved-warnings* '()) \EV *SAVED-WARNINGS*
 (defun quiet-warning-handler (c)
   (when *all-quiet*
     (let ((r (find-restart 'muffle-warning c)))
       (when r 
         (push c *saved-warnings*)
         (invoke-restart r)))))
\EV CUSTOM-WARNING-HANDLER
 (defmacro with-quiet-warnings (&body forms)
   `(let ((*all-quiet* t)
          (*saved-warnings* '()))
      (handler-bind ((warning #'quiet-warning-handler))
        ,@forms
        *saved-warnings*)))
\EV WITH-QUIET-WARNINGS
 (setq saved
   (with-quiet-warnings
     (warn "Situation #1.")
     (let ((*all-quiet* nil))
       (warn "Situation #2."))
     (warn "Situation #3.")))
\OUT Warning: Situation #2.
\EV (#<SIMPLE-WARNING 42744421> #<SIMPLE-WARNING 42744365>)
 (dolist (s saved) (format t "~&~A~%" s))
\OUT Situation #3.
\OUT Situation #1.
\EV NIL
\endcode

\label See Also::

\secref\Restarts,
{\secref\InterfacesToRestarts},
\funref{invoke-restart},
\funref{muffle-warning} (\term{function}),
\funref{warn}

\endcom%{muffle-warning}

%%% ========== STORE-VALUE
\begincom{store-value}\ftype{Restart}

\label Data Arguments Required::

a value to use instead (on an ongoing basis).

\label Description::

\Therestart{store-value} is generally used by \term{handlers}
trying to recover from errors of \term{types} such as \typeref{cell-error} 
or \typeref{type-error}, which may wish to supply a replacement datum to
be stored permanently.

\label Examples::

\code
 (defun type-error-auto-coerce (c)
   (when (typep c 'type-error)
     (let ((r (find-restart 'store-value c)))
       (handler-case (let ((v (coerce (type-error-datum c)
                                      (type-error-expected-type c))))
                       (invoke-restart r v))
         (error ()))))) \EV TYPE-ERROR-AUTO-COERCE
 (let ((x 3))
   (handler-bind ((type-error #'type-error-auto-coerce))
     (check-type x float)
     x)) \EV 3.0
\endcode   

\label See Also::

{\secref\Restarts},
{\secref\InterfacesToRestarts},
\funref{invoke-restart},
\funref{store-value} (\term{function}),
\macref{ccase},
\macref{check-type},
\macref{ctypecase},
\funref{use-value} (\term{function} and \term{restart})

\endcom%{store-value}


%%% ========== USE-VALUE
\begincom{use-value}\ftype{Restart}

\label Data Arguments Required::

a value to use instead (once).

\label Description::

\Therestart{use-value} is generally used by \term{handlers} trying 
to recover from errors of \term{types} such as \typeref{cell-error}, 
where the handler may wish to supply a replacement datum for one-time use.

\label See Also::

{\secref\Restarts},
{\secref\InterfacesToRestarts},
\funref{invoke-restart},
\funref{use-value} (\term{function}),
\funref{store-value} (\term{function} and \term{restart})

\endcom%{use-value}

%%% ========== ABORT
%%% ========== CONTINUE
%%% ========== MUFFLE-WARNING
%%% ========== STORE-VALUE
%%% ========== USE-VALUE
\begincom{abort, continue, muffle-warning, store-value, use-value}\ftype{Function}
\idxref{abort}\idxref{continue}\idxref{muffle-warning}\idxref{store-value}\idxref{use-value}

\label Syntax::

%!!! KMP: Issue CONDITION-RESTARTS forgot to add the condition argument here,
%         but I have added it tentatively (pending x3j13 approval) since I'm 
%         sure it was intended.  (MUFFLE-WARNING-CONDITION-ARGUMENT was the technical
%         issue that was raised to fix this, but it was never voted upon.)

\issue{CONDITION-RESTARTS:PERMIT-ASSOCIATION}
\DefunNoReturn abort {{\opt} condition}
\DefunWithValues continue {{\opt} condition} {\nil}
\DefunNoReturn muffle-warning {{\opt} condition}
\DefunWithValues store-value {value {\opt} condition} {\nil}
\DefunWithValues use-value {value {\opt} condition} {\nil}
\endissue{CONDITION-RESTARTS:PERMIT-ASSOCIATION}

\label Arguments and Values::

\param{value}---an \term{object}.

\issue{MUFFLE-WARNING-CONDITION-ARGUMENT}
\issue{CONDITION-RESTARTS:PERMIT-ASSOCIATION}
\param{condition}---a \term{condition} \term{object}, or \nil.
\endissue{CONDITION-RESTARTS:PERMIT-ASSOCIATION}
\endissue{MUFFLE-WARNING-CONDITION-ARGUMENT}

\label Description::

Transfers control to the most recently established \term{applicable restart}
having the same name as the function.  That is,
  \thefunction{abort}    searches for an \term{applicable} \misc{abort}    \term{restart}, 
  \thefunction{continue} searches for an \term{applicable} \misc{continue} \term{restart},
and so on.

If no such \term{restart} exists, 
the functions
     \funref{continue},
     \funref{store-value}, 
 and \funref{use-value}
return \nil, and 
the functions
     \funref{abort}
 and \funref{muffle-warning}
signal an error \oftype{control-error}.

\issue{CONDITION-RESTARTS:PERMIT-ASSOCIATION}
When \param{condition} is \term{non-nil},
only those \term{restarts} are considered that are 
  either explicitly associated with that \param{condition},
      or not associated with any \term{condition};
that is, the excluded \term{restarts} are 
those that are associated with a non-empty set of \term{conditions}
of which the given \param{condition} is not an \term{element}.
If \param{condition} is \nil, all \term{restarts} are considered.
\endissue{CONDITION-RESTARTS:PERMIT-ASSOCIATION}

\label Examples::

\code
;;; Example of the ABORT retart

 (defmacro abort-on-error (&body forms)
   `(handler-bind ((error #'abort))
      ,@forms)) \EV ABORT-ON-ERROR
 (abort-on-error (+ 3 5)) \EV 8
 (abort-on-error (error "You lose."))
\OUT Returned to Lisp Top Level.

;;; Example of the CONTINUE restart

 (defun real-sqrt (n)
   (when (minusp n)
     (setq n (- n))
     (cerror "Return sqrt(~D) instead." "Tried to take sqrt(-~D)." n))
   (sqrt n))

 (real-sqrt 4) \EV 2
 (real-sqrt -9)
\OUT Error: Tried to take sqrt(-9).
\OUT To continue, type :CONTINUE followed by an option number:
\OUT  1: Return sqrt(9) instead.
\OUT  2: Return to Lisp Toplevel.
\OUT Debug> \IN{(continue)}
\OUT Return sqrt(9) instead.
\EV 3
 
 (handler-bind ((error #'(lambda (c) (continue))))
   (real-sqrt -9)) \EV 3

;;; Example of the MUFFLE-WARNING restart

 (defun count-down (x)
   (do ((counter x (1- counter)))
       ((= counter 0) 'done)
     (when (= counter 1)
       (warn "Almost done"))
     (format t "~&~D~%" counter)))
\EV COUNT-DOWN
 (count-down 3)
\OUT 3
\OUT 2
\OUT Warning: Almost done
\OUT 1
\EV DONE
 (defun ignore-warnings-while-counting (x)
   (handler-bind ((warning #'ignore-warning))
     (count-down x)))
\EV IGNORE-WARNINGS-WHILE-COUNTING
 (defun ignore-warning (condition)
   (declare (ignore condition))
   (muffle-warning))
\EV IGNORE-WARNING
 (ignore-warnings-while-counting 3)
\OUT 3
\OUT 2
\OUT 1
\EV DONE

;;; Example of the STORE-VALUE and USE-VALUE restarts

 (defun careful-symbol-value (symbol)
   (check-type symbol symbol)
   (restart-case (if (boundp symbol)
                     (return-from careful-symbol-value 
                                  (symbol-value symbol))
                     (error 'unbound-variable
                            :name symbol))
     (use-value (value)
       :report "Specify a value to use this time."
       value)
     (store-value (value)
       :report "Specify a value to store and use in the future."
       (setf (symbol-value symbol) value))))
 (setq a 1234) \EV 1234
 (careful-symbol-value 'a) \EV 1234
 (makunbound 'a) \EV A
 (careful-symbol-value 'a)
\OUT Error: A is not bound.
\OUT To continue, type :CONTINUE followed by an option number.
\OUT  1: Specify a value to use this time.
\OUT  2: Specify a value to store and use in the future.
\OUT  3: Return to Lisp Toplevel.
\OUT Debug> \IN{(use-value 12)}
\EV 12
 (careful-symbol-value 'a)
\OUT Error: A is not bound.
\OUT To continue, type :CONTINUE followed by an option number.
\OUT   1: Specify a value to use this time.
\OUT   2: Specify a value to store and use in the future.
\OUT   3: Return to Lisp Toplevel.
\OUT Debug> \IN{(store-value 24)}
\EV 24
 (careful-symbol-value 'a)
\EV 24

;;; Example of the USE-VALUE restart

 (defun add-symbols-with-default (default &rest symbols)
   (handler-bind ((sys:unbound-symbol
                    #'(lambda (c)
                        (declare (ignore c)) 
                        (use-value default))))
     (apply #'+ (mapcar #'careful-symbol-value symbols))))
\EV ADD-SYMBOLS-WITH-DEFAULT
 (setq x 1 y 2) \EV 2
 (add-symbols-with-default 3 'x 'y 'z) \EV 6


\endcode
 
\label Side Effects::

A transfer of control may occur if an appropriate \term{restart} is available,
or (in the case of \thefunction{abort} or \thefunction{muffle-warning})
execution may be stopped.

\label Affected By::

Each of these functions can be affected by 
the presence of a \term{restart} having the same name.

\label Exceptional Situations::

If an appropriate \misc{abort} \term{restart}
 is not available for \thefunction{abort},
or an appropriate \misc{muffle-warning} \term{restart} 
 is not available for \thefunction{muffle-warning},
an error \oftype{control-error} is signaled.

\label See Also::

\funref{invoke-restart},
{\secref\Restarts},
{\secref\InterfacesToRestarts},
\macref{assert},
\macref{ccase},
\funref{cerror},
\macref{check-type},
\macref{ctypecase},
\funref{use-value},
\funref{warn}

\label Notes::

\code
 (abort condition) \EQ (invoke-restart 'abort)
 (muffle-warning)  \EQ (invoke-restart 'muffle-warning)
 (continue)        \EQ (let ((r (find-restart 'continue))) (if r (invoke-restart r)))
 (use-value \param{x}) \EQ (let ((r (find-restart 'use-value))) (if r (invoke-restart r \param{x})))
 (store-value x) \EQ (let ((r (find-restart 'store-value))) (if r (invoke-restart r \param{x})))
\endcode

No functions defined in this specification are required to provide
a \misc{use-value} \term{restart}.

\endcom
