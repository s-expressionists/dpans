% -*- Mode: TeX -*-

\beginsubSection{Introduction to Streams}

%% 22.0.0 3
%All input/output operations are performed on \term{streams} of various kinds.

A \newterm{stream} is an \term{object} that can be used with an input or output
function to identify an appropriate source or sink of \term{characters} or 
\term{bytes} for that operation.
%% 21.0.0 3
%% 21.0.0 4
A \newterm{character} \newterm{stream} is a source or sink of \term{characters}.
A \newterm{binary} \newterm{stream} is a source or sink of \term{bytes}.

Some operations may be performed on any kind of \term{stream};
\thenextfigure\ provides a list of \term{standardized} operations
that are potentially useful with any kind of \term{stream}.

\displaytwo{Some General-Purpose Stream Operations}{
close&stream-element-type\cr
input-stream-p&streamp\cr
interactive-stream-p&with-open-stream\cr
output-stream-p&\cr
}

Other operations are only meaningful on certain \term{stream} \term{types}.
For example, \funref{read-char} is only defined for \term{character} \term{streams}
and \funref{read-byte} is only defined for \term{binary} \term{streams}.

\beginsubsubsection{Abstract Classifications of Streams}

\beginsubsubsubsection{Input, Output, and Bidirectional Streams}

%% 21.0.0 3
A \term{stream}, whether a \term{character} \term{stream} or a \term{binary} \term{stream},
can be an \newterm{input} \newterm{stream} (source of data),
       an \newterm{output} \newterm{stream} (sink for data),
       both, 
    or (\eg when ``\f{:direction :probe}'' is given to \funref{open}) neither.

\Thenextfigure\ shows \term{operators} relating to
\term{input} \term{streams}.

\DefineFigure{InputStreamOps}
\displaythree{Operators relating to Input Streams.}{
clear-input&read-byte&read-from-string\cr
listen&read-char&read-line\cr
peek-char&read-char-no-hang&read-preserving-whitespace\cr
read&read-delimited-list&unread-char\cr
}

\Thenextfigure\ shows \term{operators} relating to
\term{output} \term{streams}.

\DefineFigure{OutputStreamOps}
\displaythree{Operators relating to Output Streams.}{
clear-output&prin1&write\cr
finish-output&prin1-to-string&write-byte\cr
force-output&princ&write-char\cr
format&princ-to-string&write-line\cr
fresh-line&print&write-string\cr
pprint&terpri&write-to-string\cr
}

A \term{stream} that is both an \term{input} \term{stream} and an \term{output} \term{stream}
is called a \newterm{bidirectional} \newterm{stream}.
\Seefuns{input-stream-p} and \funref{output-stream-p}.

Any of the \term{operators} listed in \figref\InputStreamOps\ or \figref\OutputStreamOps\
can be used with \term{bidirectional} \term{streams}.  In addition, \thenextfigure\
shows a list of \term{operators} that relate specificaly to 
\term{bidirectional} \term{streams}.

\displaythree{Operators relating to Bidirectional Streams.}{
y-or-n-p&yes-or-no-p&\cr
}

\endsubsubsubsection%{Input, Output, and Bidirectional Streams}

\beginsubsubsubsection{Open and Closed Streams}
\DefineSection{OpenAndClosedStreams}

\term{Streams} are either \newterm{open} or \newterm{closed}.  

Except as explicitly specified otherwise,
operations that create and return \term{streams} return \term{open} \term{streams}.

The action of \term{closing} a \term{stream} marks the end of its use as a source
or sink of data, permitting the \term{implementation} to reclaim its internal data
structures, and to free any external resources which might have been locked by the
\term{stream} when it was opened.

Except as explicitly specified otherwise,
the consequences are undefined when a \term{closed} \term{stream} 
is used where a \term{stream} is called for.

Coercion of \term{streams} to \term{pathnames} 
is permissible for \term{closed} \term{streams};
in some situations, such as for a \term{truename} computation, 
the result might be different for an \term{open} \term{stream}
and for that same \term{stream} once it has been \term{closed}.

\endsubsubsubsection%{Open and Closed Streams}

\beginsubsubsubsection{Interactive Streams}
\DefineSection{InteractiveStreams}

An \newterm{interactive stream} is one on which it makes sense to perform
interactive querying.

The precise meaning of an \term{interactive stream} is
\term{implementation-defined}, and may depend on the underlying
operating system.  Some examples of the things that an
\term{implementation} might choose to use as identifying characteristics
of an \term{interactive stream} include:

\beginlist

\itemitem{\bull} 
  The \term{stream} is connected to a person (or equivalent) in such a way
  that the program can prompt for information and expect to receive different
  input depending on the prompt.

\itemitem{\bull}
  The program is expected to prompt for input and support ``normal input editing''.

\itemitem{\bull} 
  \funref{read-char} might wait for the user to type something before returning
  instead of immediately returning a character or end-of-file. 

\endlist 

The general intent of having some \term{streams} be classified as
\term{interactive streams} is to allow them to be distinguished from
streams containing batch (or background or command-file) input.
Output to batch streams is typically discarded or saved for later viewing, 
so interactive queries to such streams might not have the expected effect.

\term{Terminal I/O} might or might not be an \term{interactive stream}.

\endsubsubsubsection%{Interactive Streams}

\beginsubsubsubsection{File Streams}

%% 23.2.0 1
Some \term{streams}, called \newtermidx{file streams}{file stream}, provide access to \term{files}.
An \term{object} \ofclass{file-stream} is used to represent a \term{file stream}.

The basic operation for opening a \term{file} is \funref{open},
which typically returns a \term{file stream} 
(see its dictionary entry for details).
The basic operation for closing a \term{stream} is \funref{close}.
The macro \macref{with-open-file} is useful 
to express the common idiom of opening a \term{file} 
for the duration of a given body of \term{code}, 
and assuring that the resulting \term{stream} is closed upon exit from that body.

\endsubsubsubsection%{File Streams}

\endsubsubsection%{Abstract Classifications of Streams}

\beginsubsubsection{Other Subclasses of Stream}

\Theclass{stream} has a number of \term{subclasses} defined 
by this specification.  \Thenextfigure\ shows some information 
about these subclasses.

\tablefigtwo{Defined Names related to Specialized Streams}{Class}{Related Operators}{
 \typeref{broadcast-stream}    & \funref{make-broadcast-stream}        \cr
                               & \funref{broadcast-stream-streams}     \cr
 \typeref{concatenated-stream} & \funref{make-concatenated-stream}     \cr
                               & \funref{concatenated-stream-streams}  \cr
 \typeref{echo-stream}         & \funref{make-echo-stream}             \cr
                               & \funref{echo-stream-input-stream}     \cr
                               & \funref{echo-stream-output-stream}    \cr
 \typeref{string-stream}       & \funref{make-string-input-stream}     \cr
                               & \funref{with-input-from-string}       \cr
                               & \funref{make-string-output-stream}    \cr
                               & \macref{with-output-to-string}        \cr
                               & \funref{get-output-stream-string}     \cr
 \typeref{synonym-stream}      & \funref{make-synonym-stream}          \cr
                               & \funref{synonym-stream-symbol}        \cr
 \typeref{two-way-stream}      & \funref{make-two-way-stream}          \cr
                               & \funref{two-way-stream-input-stream}  \cr
                               & \funref{two-way-stream-output-stream} \cr
}

\endsubsubsection%{Other Subclasses of Stream}

\endsubSection%{Introduction to Streams}

\beginsubSection{Stream Variables}

\term{Variables} whose \term{values} must be \term{streams} are sometimes called 
\newtermidx{stream variables}{stream variable}.

Certain \term{stream variables} are defined by this specification 
to be the proper source of input or output in various \term{situations} 
where no specific \term{stream} has been specified instead.
A complete list of such \term{standardized} \term{stream variables}
appears in \thenextfigure.  
%Added by agreement of Barrett, Loosemore, and KMP. -kmp 14-Feb-92
The consequences are undefined if at any time
the \term{value} of any of these \term{variables} is not an \term{open} \term{stream}.

\DefineFigure{StandardizedStreamVars}
\tablefigtwo{Standardized Stream Variables}{Glossary Term}{Variable Name}{
 \term{debug I/O}       & \varref{*debug-io*}        \cr
 \term{error output}    & \varref{*error-output*}    \cr
 \term{query I/O}       & \varref{*query-io*}        \cr
 \term{standard input}  & \varref{*standard-input*}  \cr
 \term{standard output} & \varref{*standard-output*} \cr
 \term{terminal I/O}    & \varref{*terminal-io*}     \cr
 \term{trace output}    & \varref{*trace-output*}    \cr
}

Note that, by convention, \term{standardized} \term{stream variables} have names 
    ending in ``\f{-input*}''  if they must be \term{input} \term{streams},
    ending in ``\f{-output*}'' if they must be \term{output} \term{streams},
 or ending in ``\f{-io*}''     if they must be \term{bidirectional} \term{streams}.

User programs may \term{assign} or \term{bind} any \term{standardized} \term{stream variable}
except \varref{*terminal-io*}.

\endsubSection%{Stream Variables}

\beginsubSection{Stream Arguments to Standardized Functions}
\DefineSection{StreamArgsToStandardizedFns}

%This list conjured by KMP, Barrett, and Loosemore. -kmp 14-Feb-92
The \term{operators} in \thenextfigure\ accept \term{stream} \term{arguments} that
might be either \term{open} or \term{closed} \term{streams}.

\DefineFigure{OpenOrClosedStreamOps}
\displaythree{Operators that accept either Open or Closed Streams}{
broadcast-stream-streams&file-author&pathnamep\cr
close&file-namestring&probe-file\cr
compile-file&file-write-date&rename-file\cr
compile-file-pathname&host-namestring&streamp\cr
concatenated-stream-streams&load&synonym-stream-symbol\cr
delete-file&logical-pathname&translate-logical-pathname\cr
directory&merge-pathnames&translate-pathname\cr
directory-namestring&namestring&truename\cr
dribble&open&two-way-stream-input-stream\cr
echo-stream-input-stream&open-stream-p&two-way-stream-output-stream\cr
echo-stream-ouput-stream&parse-namestring&wild-pathname-p\cr
ed&pathname&with-open-file\cr
enough-namestring&pathname-match-p&\cr
}

%This list conjured by KMP, Barrett, and Loosemore. -kmp 14-Feb-92
The \term{operators} in \thenextfigure\ accept \term{stream} \term{arguments} that
must be \term{open} \term{streams}.

\displaythree{Operators that accept Open Streams only}{
clear-input&output-stream-p&read-char-no-hang\cr
clear-output&peek-char&read-delimited-list\cr
file-length&pprint&read-line\cr
file-position&pprint-fill&read-preserving-whitespace\cr
file-string-length&pprint-indent&stream-element-type\cr
finish-output&pprint-linear&stream-external-format\cr
force-output&pprint-logical-block&terpri\cr
format&pprint-newline&unread-char\cr
fresh-line&pprint-tab&with-open-stream\cr
get-output-stream-string&pprint-tabular&write\cr
input-stream-p&prin1&write-byte\cr
interactive-stream-p&princ&write-char\cr
listen&print&write-line\cr
make-broadcast-stream&print-object&write-string\cr
make-concatenated-stream&print-unreadable-object&y-or-n-p\cr
make-echo-stream&read&yes-or-no-p\cr
make-synonym-stream&read-byte&\cr
make-two-way-stream&read-char&\cr
}
 
\endsubSection%{Stream Arguments to Standardized Functions}

\beginsubSection{Restrictions on Composite Streams}

%KMP, Loosemore, and Barrett thought this stuff should be made explicit. -kmp 14-Feb-92

The consequences are undefined if any \term{component} of a \term{composite stream}
is \term{closed} before the \term{composite stream} is \term{closed}.

The consequences are undefined if the \term{synonym stream symbol} is not \term{bound}
to an \term{open} \term{stream} from the time of the \term{synonym stream}'s creation
until the time it is \term{closed}.

\endsubSection%{Restrictions on Composite Streams}
