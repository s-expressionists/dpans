% -*- Mode: TeX -*-

\newtermidx{Declarations}{declaration} provide a way of specifying information for use by
program processors, such as the evaluator or the compiler.

\newtermidx{Local declarations}{local declaration}
can be embedded in executable code using \misc{declare}.
\newtermidx{Global declarations}{global declaration}, 
or \newtermidx{proclamations}{proclamation},
are established by \funref{proclaim} or \macref{declaim}.

%% 9.3.0 1
\Thespecform{the} provides a shorthand notation for 
making a \term{local declaration} about the \term{type} of the
\term{value} of a given \term{form}.
%% 9.0.0 4
% This is redundant with the next sentence.  --sjl 3 Mar 92
% The consequences are undefined if a program violates a type declaration.
%% 9.2.0 1

%% 9.2.0 6
The consequences are undefined if a program violates a \term{declaration}
or a \term{proclamation}.

\beginSubsection{Minimal Declaration Processing Requirements}

In general, an \term{implementation} is free to ignore
\term{declaration specifiers} except for the
     \declref{declaration}\idxref{declaration},
     \declref{notinline}\idxref{notinline},
     \declref{safety}\idxref{safety},
 and \declref{special}\idxref{special} \term{declaration specifiers}.

A \declref{declaration} \term{declaration} must suppress warnings
about unrecognized \term{declarations} of the kind that it declares.
If an \term{implementation} does not produce warnings about
unrecognized declarations, it may safely ignore this \term{declaration}.

A \declref{notinline} \term{declaration} must be recognized by any \term{implementation}
that supports inline functions or \term{compiler macros} in order to disable those facilities.
An \term{implementation} that does not use inline functions or \term{compiler macros}
may safely ignore this \term{declaration}.

A \declref{safety} \term{declaration} that increases the current safety level 
must always be recognized.  An \term{implementation} that always processes 
code as if safety were high may safely ignore this \term{declaration}.

A \declref{special} \term{declaration} must be processed by all \term{implementations}.

\endSubsection%{Minimal Declaration Processing Requirements}

\beginSubsection{Declaration Specifiers}

A \newterm{declaration specifier} is an \term{expression} that can appear at
top level of a \misc{declare} expression or a \macref{declaim} form, or as 
the argument to \funref{proclaim}.
It is a \term{list} whose \term{car} is a \term{declaration identifier},
and whose \term{cdr} is data interpreted according to rules specific to
the \term{declaration identifier}.

\endSubsection%{Declaration Specifiers}

\beginSubsection{Declaration Identifiers}

\Thenextfigure\ shows a list of all 
\term{declaration identifiers}\idxterm{declaration identifier} 
defined by this standard.

\issue{DECLARE-FUNCTION-AMBIGUITY:DELETE-FTYPE-ABBREVIATION}
\displaythree{Common Lisp Declaration Identifiers}{
declaration&ignore&special\cr
dynamic-extent&inline&type\cr
ftype&notinline&\cr
ignorable&optimize&\cr
}
%FUNCTION removed.
\endissue{DECLARE-FUNCTION-AMBIGUITY:DELETE-FTYPE-ABBREVIATION}

%% 9.2.0 20
An implementation is free to support other (\term{implementation-defined})
\term{declaration identifiers} as well.  
% Sections 3.2.2.3 and 3.2.5 both classify this as an ordinary warning.
% --sjl 3 Mar 92
% A warning \oftype{style-warning} might be issued
A warning might be issued
if a \term{declaration identifier} 
is not among those defined above,
%Added for Barmar:   -kmp 11-Jan-91
is not defined by the \term{implementation},
is not a \term{type} \term{name}, 
and has not been declared in a \declref{declaration} \term{proclamation}.

% I can't figure out where this paragraph came from, and I'm convinced
% it's wrong.  Issue DECLARATION-SCOPE was intended to assign consistent
% scoping rules to all declarations based only on whether they are ``bound''
% or ``free''.  Allowing random scoping rules will also totally defeat
% the proposed define-declaration extensions.  --sjl 7 Mar 92
%For \term{implementation-defined} \term{declaration identifiers},
%the \term{scope} of \term{free declarations} and \term{bound declarations}
%is \term{implementation-defined}.

\beginsubsubsection{Shorthand notation for Type Declarations}

%Barrett says class objects are ok here, too.
A \term{type specifier} can be used as a \term{declaration identifier}.
\f{(\param{type-specifier} \starparam{var})} is taken as shorthand for
\f{(type \param{type-specifier} \starparam{var})}.

\endsubsubsection%{Shorthand notation for Type Declarations}

\endSubsection%{Declaration Identifiers}

\beginSubsection{Declaration Scope}
\DefineSection{DeclScope}

% Declarations can be divided into two kinds: those that concern 
% the \term{bindings} of variables, and those that do not.
% The \declref{special} declaration falls into both classes.
% Declarations that concern \term{variable} \term{bindings} apply
% only to the \term{bindings} made by the \term{form} at the head of 
% whose body they appear.  
% 
% All declarations introduced with \misc{declare} fall into two classes:
% \term{bound declarations} and \term{free declarations}.
% \term{Bound declarations} affect both a binding and any references;
% \term{free declarations} affect only references.
% Some declarations may be used in either way, depending on context.
%
%% The above rewritten with help from Sandra and Moon. -kmp 22-Aug-91

\term{Declarations} can be divided into two kinds: those that apply to the
\term{bindings} of \term{variables} or \term{functions}; and those that
do not apply to \term{bindings}.

A \term{declaration} that appears at the head of a binding \term{form} 
and applies to a \term{variable} or \term{function} \term{binding} 
made by that \term{form} is called a \newterm{bound declaration}; 
such a \term{declaration} affects both the \term{binding} and
any references within the \term{scope} of the \term{declaration}.  

\term{Declarations} that are not \term{bound declarations} are called
\newtermidx{free declarations}{free declaration}.

% \term{Free declarations} that apply to
% \term{bindings} affect only references to those \term{bindings}.  
A \term{free declaration} in a \term{form} $F1$ that applies to a \term{binding}
for a \term{name} $N$ \term{established} by some \term{form} $F2$
of which $F1$ is a \term{subform}
affects only references to $N$ within $F1$; it does not to apply to
other references to $N$ outside of $F1$, nor does it affect the manner
in which the \term{binding} of $N$ by $F2$ is \term{established}.

\term{Declarations} that do not apply to \term{bindings} can only appear 
as \term{free declarations}.

% Common Lisp prohibits binding the same name twice in the same binding form.
% It has been proposed that multiple bindings be permitted for LET*, DO*, PROG* 
% forms and for &AUX variables in lambda expressions, but never approved. 
% In an implementation which permits multiple bindings, `bound' declarations
% should probably be treated as if there were a separate `bound' declaration 
% for each of the bindings, but for us to say so would really go beyond the
% scope of this document.  As such, we'll just not say anything and leave it to
% any implementation which defines that circumstance to also define the relationship
% to bound declarations. -kmp 22-Aug-91

%% Rewritten by Sandra in response to Margolin #7, Dalton #3, Moon #7 (First Public Review)
% %% 9.1
% Some \term{forms} contain pieces of code that, properly speaking,
% are not part of the body of the \term{form}.  Examples of this
% are initialization forms that provide values for bound variables,
% and the result forms of iteration \term{forms}.
% 
% \issue{DECLARATION-SCOPE:NO-HOISTING}
%  
% The \term{scope} of a \term{declaration} located at the head of 
% a \term{special form}, \term{macro form}, or \term{lambda expression} is as follows:
% \beginlist
% \itemitem{1.}
% It always includes the body forms as well as any \term{step} or exit \term{forms}.
% \itemitem{2.}
% It also includes the \term{scope} of the name binding, if any, to which 
% it applies (\specref{let}, \misc{lambda},  \specref{flet},  \macref{do}, etc. 
% introduce name bindings; \specref{locally} does not).
% \endlist
% 
% %!!!! RPG: I'm tired but this doesn't make sense to me at all.
% This prescription depends on the fact that the \term{scope} of name bindings
% is already well-defined.
% Whether or not a particular declaration affects an initialization form 
% (such as for \specref{let} or \specref{let*}) 
% depends solely on whether it is
% applied to a variable or function name being bound whose \term{scope}
% includes such \term{forms}.  
% In this sense, the above specification limits the
% \term{scope} of declarations for name bindings to be exactly the 
% \term{scope} of the
% name binding itself. 
% There is no ``hoisting'' for declarations in \term{special forms} or 
% \term{lambda expressions}; 
% the only initialization forms affected by a declaration 
% are those included indirectly, by the effect, if any, that a 
% declaration has on a name binding. 
% Thus there is no
% ``hoisting'' of the special declarations in the following example:
% 
% % \code
% %  (defun bar (x y)           ;[1] 1st occurrence of x
% %    (let ((old-x x)          ;[2] 2nd occurrence of x 
% %          (x y))             ;[3] 3rd occurrence of x
% %      (declare (special x))
% %      (list old-x x)))
% % \endcode
% % 
% % Laubsch: ?
% % Barmar: Say what [the above] example is supposed to return.
% % RPG: Also, say explicitly which bindings and references are special.
% 
% \code
%  (let ((x 1))                ;[1] 1st occurrence of x
%    (declare (special x))     ;[2] 2nd occurrence of x
%    (let ((x 2))              ;[3] 3rd occurrence of x
%      (let ((old-x x)         ;[4] 4th occurrence of x
%            (x 3))            ;[5] 5th occurrence of x
%        (declare (special x)) ;[6] 6th occurrence of x
%        (list old-x x))))     ;[7] 7th occurrence of x
% \EV (2 3)
% \endcode
% 
% The first occurrence of \f{x} \term{establishes} a \term{dynamic binding}
% of \f{x} because of the \declref{special} \term{declaration} for \f{x}
% in the second line.  The third occurrence of \f{x} \term{establishes} a
% \term{lexical binding} of \f{x} (because there is no \declref{special}
% \term{declaration} in the corresponding \specref{let} \term{form}).
% The fourth occurrence of \f{x} \term{x} is a reference to the
% \term{lexical binding} of \f{x} established in the third line.
% The fifth occurrence of \f{x} \term{establishes} a \term{dynamic binding}
% of \term{x} for the body of the \specref{let} \term{form} that begins on
% that line because of the \declref{special} \term{declaration} for \f{x}
% in the sixth line. The reference to \f{x} in the fourth line is not
% affected by the \declref{special} \term{declaration} in the sixth line 
% because that reference is not within the ``would-be \term{lexical scope}''
% of the \term{variable} \f{x} in the fifth line.  The reference to \f{x}
% in the seventh line is a reference to the \term{dynamic binding} of \term{x}
% \term{established} in the fifth line.
% 
% Those declarations not correlated with any name \term{binding} do
% not cover any of the initialization forms; their \term{scope} only
% includes the body as well as any ``stepper'' or result forms.  In a
% sense, the above specification limits the \term{scope} of these
% kinds of declarations to be the same as an arbitrary name 
% \term{binding} in a \specref{let}, \specref{flet}, 
% \issue{WITH-ADDED-METHODS:DELETE}
% %\macref{with-added-methods},
% \endissue{WITH-ADDED-METHODS:DELETE}
% \issue{GENERIC-FLET-POORLY-DESIGNED:DELETE}
% %\specref{generic-flet},
% %\specref{generic-labels},
% \endissue{GENERIC-FLET-POORLY-DESIGNED:DELETE}
% and \specref{labels}
% \term{form}.
% %[See also the issue DECLARE-TYPE-FREE.]
% 
% In the following:\idxref{notinline}
% 
% \code
%  (lambda (&optional (x (foo 1))) ;[1]
%    (declare (notinline foo))     ;[2]
%    (foo x))                      ;[3]
% \endcode
% 
% the \term{call} to \f{foo} in the first line might be 
% compiled inline even though the \term{call} to \f{foo} in
% the third line must not be.  This is because
% the \declref{notinline} \term{declaration}
% for \f{foo} in the second line applies only to the body on the
% third line.  In order to suppress inlining for both \term{calls}, 
% one might write:\idxref{notinline}
% 
% \code
%  (locally (declare (notinline foo)) ;[1]
%    (lambda (&optional (x (foo 1)))  ;[2]
%      (foo x)))                      ;[3]
% \endcode
% 
% or, alternatively:\idxref{notinline}
% 
% \code
%  (lambda (&optional                               ;[1]
%             (x (locally (declare (notinline foo)) ;[2]
%                  (foo 1))))                       ;[3]
%    (declare (notinline foo))                      ;[4]
%    (foo x))                                       ;[5]
% \endcode
% 
% In the following:\idxterm{type declaration}
% 
% \code
%  (defun foo (x)                               ;[1]
%    (if (typep x 'integer)                     ;[2]
%        (list (let ((y (+ x 42)))              ;[3]
%                (declare (fixnum x y))         ;[4]
%                y)                             ;[5]
%              (+ x 42))                        ;[6]
%        `(foo ,x)))                            ;[7]
% \endcode
% 
% \f{x} is not initially (\eg in the first line) known to be a \term{fixnum} 
% since the scope of the \declref{fixnum} \term{declaration} for \f{x} in the fourth line
% covers only the body of the \specref{let} form in the fifth line, but not the
% \term{initialization form} for \f{y} in the third line.  The compiler can assume that
% \f{x} is not greater than the value of \f{(- most-positive-fixnum 42)} because \f{y}
% has been declared to be a \term{fixnum} in the fourth line.
% Even so, neither the \term{call} to \funref{+} in the third line 
%              nor the one in the sixth line 
% may be optimized into \term{calls} to \term{implementation-dependent} 
% \term{fixnum}-only arithmetic operators,
% just in case the call to \f{foo} looks something like:
% 
% \code
%  (foo (- most-negative-fixnum 1))
% \endcode
% 
% In following:\idxterm{type declaration}
% 
% \code
%  (defun foo (x)                               ;[1]
%    (if (typep x 'integer)                     ;[2]
%        (list (let ((y (+ x 42)))              ;[3]
%                (declare (fixnum x))           ;[4]
%                x                              ;[5]
%                y)                             ;[6]
%              (+ x 42))                        ;[7]
%        `(foo ,x)))                            ;[8]
% \endcode
% 
% \f{x} can be determined to be a \term{fixnum} throughout 
% the third through seventh lines, but only by inference
% from the fact that the reference to \f{x} in the fifth line
% (the only reference to which the \declref{fixnum} \term{declaration}
% in the fourth line applies) 
% is known to be a \term{fixnum}.  Since the compiler is capable of detecting that
% there are no \term{assignments} to \f{x}, it may reason that \f{x} is a \term{fixnum}
% throughout even though there is no explicit \term{declaration}.
% However, since there is no \declref{fixnum} \term{declaration} for \f{y} (as there
% was in the previous example), the compiler may not assume that the result of the
% addition in the third line is a \term{fixnum}.  Therefore, 
% neither \term{call} to \funref{+} (one the third and seventh lines) 
% may be optimized into \term{calls} to \term{implementation-dependent} 
% \term{fixnum}-only arithmetic operators,
% just in case the call to \term{foo} looks something like:
% 
% \code
%  (foo most-positive-fixnum)
% \endcode
% 
% However, in the following:\idxterm{type declaration}
% 
% \code
%  (defun foo (x)                               ;[1]
%    (if (typep x 'integer)                     ;[2]
%        (list (let ((y (the fixnum (+ x 42)))) ;[3]
%                (declare (fixnum x y))         ;[4]
%                x                              ;[5]
%                y)                             ;[6]
%              (+ x 42))                        ;[7]
%        `(foo ,x)))                            ;[8]
% \endcode
% 
% the compiler can infer that \f{x} is a \term{fixnum} throughout 
% the third through seventh lines by reasoning similar
% to that for the previous example.  Further, it can infer that the result of the 
% call to \funref{+} in the third line is a \term{fixnum} because of the \declref{fixnum}
% \term{declaration} in the fourth line.  Consequently, that \term{call} to \funref{+}
% may be optimized into a \term{call} to an \term{implementation-dependent} 
% \term{fixnum}-only arithmetic operator.  Further, the \term{call} to \funref{+}
% in the seventh line may be similarly optimized because the compiler can prove that
% the \f{x} in that line has the same \term{value}.
% 
% \endissue{DECLARATION-SCOPE:NO-HOISTING}

\issue{DECLARATION-SCOPE:NO-HOISTING}
\issue{WITH-ADDED-METHODS:DELETE}
\issue{GENERIC-FLET-POORLY-DESIGNED:DELETE}
The \term{scope} of a \term{bound declaration} is the same as the
%% Per X3J13. -kmp 5-Oct-93
%\term{scope}
\term{lexical scope}
of the \term{binding} to which it applies;
%% Added per X3J13. -kmp 5-Oct-93
for \term{special variables},
this means the \term{scope} that the \term{binding} 
would have had had it been a \term{lexical binding}.

Unless explicitly stated otherwise, the \term{scope} of a 
\term{free declaration} includes only the body \term{subforms} of 
the \term{form} at whose head it appears, and no other \term{subforms}.
The \term{scope} of \term{free declarations} specifically does not
include \term{initialization forms} for \term{bindings} established
by the \term{form} containing the \term{declarations}.

Some \term{iteration forms} include step, end-test, or result 
\term{subforms} that are also included in the \term{scope}
of \term{declarations} that appear in the \term{iteration form}.
Specifically, the \term{iteration forms} and \term{subforms} involved
are:

\beginlist
\item{\bull} \macref{do}, \macref{do*}:  
  \param{step-forms}, \param{end-test-form}, and \param{result-forms}.
\item{\bull} \macref{dolist}, \macref{dotimes}:
  \param{result-form}
\item{\bull} \macref{do-all-symbols}, \macref{do-external-symbols}, \macref{do-symbols}:
  \param{result-form}
\endlist
\endissue{GENERIC-FLET-POORLY-DESIGNED:DELETE}
\endissue{WITH-ADDED-METHODS:DELETE}

\beginsubsubsection{Examples of Declaration Scope}

Here is an example illustrating the \term{scope} of \term{bound declarations}.

\code
 (let ((x 1))                ;[1] 1st occurrence of x
   (declare (special x))     ;[2] 2nd occurrence of x
   (let ((x 2))              ;[3] 3rd occurrence of x
     (let ((old-x x)         ;[4] 4th occurrence of x
           (x 3))            ;[5] 5th occurrence of x
       (declare (special x)) ;[6] 6th occurrence of x
       (list old-x x))))     ;[7] 7th occurrence of x
\EV (2 3)
\endcode

The first occurrence of \f{x} \term{establishes} a \term{dynamic binding}
of \f{x} because of the \declref{special} \term{declaration} for \f{x}
in the second line.  The third occurrence of \f{x} \term{establishes} a
\term{lexical binding} of \f{x} (because there is no \declref{special}
\term{declaration} in the corresponding \specref{let} \term{form}).
The fourth occurrence of \f{x} \term{x} is a reference to the
\term{lexical binding} of \f{x} established in the third line.
The fifth occurrence of \f{x} \term{establishes} a \term{dynamic binding}
of \term{x} for the body of the \specref{let} \term{form} that begins on
that line because of the \declref{special} \term{declaration} for \f{x}
in the sixth line. The reference to \f{x} in the fourth line is not
affected by the \declref{special} \term{declaration} in the sixth line 
because that reference is not within the ``would-be \term{lexical scope}''
of the \term{variable} \f{x} in the fifth line.  The reference to \f{x}
in the seventh line is a reference to the \term{dynamic binding} of \term{x}
\term{established} in the fifth line.


Here is another example, to illustrate the \term{scope} of a
\term{free declaration}.  In the following:

\code
 (lambda (&optional (x (foo 1))) ;[1]
   (declare (notinline foo))     ;[2]
   (foo x))                      ;[3]
\endcode

the \term{call} to \f{foo} in the first line might be 
compiled inline even though the \term{call} to \f{foo} in
the third line must not be.  This is because
the \declref{notinline} \term{declaration}
for \f{foo} in the second line applies only to the body on the
third line.  In order to suppress inlining for both \term{calls}, 
one might write:

\code
 (locally (declare (notinline foo)) ;[1]
   (lambda (&optional (x (foo 1)))  ;[2]
     (foo x)))                      ;[3]
\endcode

or, alternatively:

\code
 (lambda (&optional                               ;[1]
            (x (locally (declare (notinline foo)) ;[2]
                 (foo 1))))                       ;[3]
   (declare (notinline foo))                      ;[4]
   (foo x))                                       ;[5]
\endcode


Finally, here is an example that shows the \term{scope} of
\term{declarations} in an \term{iteration form}.

\code
 (let ((x  1))                     ;[1]
   (declare (special x))           ;[2]
     (let ((x 2))                  ;[3]
       (dotimes (i x x)            ;[4]
         (declare (special x)))))  ;[5]
\EV 1
\endcode

In this example, the first reference to \f{x} on the fourth line is to
the \term{lexical binding} of \f{x} established on the third line.
However, the second occurrence of \f{x} on the fourth line lies within
the \term{scope} of the \term{free declaration} on the fifth line
(because this is the \param{result-form} of the \macref{dotimes})
and therefore refers to the \term{dynamic binding} of \f{x}.
\endissue{DECLARATION-SCOPE:NO-HOISTING}

\endsubsubsection%{Examples of Declaration Scope}

\endSubsection%{Declaration Scope}
