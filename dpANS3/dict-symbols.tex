% -*- Mode: TeX -*-

%!!! Sandra: Avoid talking about "symbols in the foo package"
%      This is ambiguous.  Does it mean "accessible in", "external in", 
%      "whose home package is", ...?

% Symbol Types
% Symbol Creation
% Symbol Accessors
% Symbol Properties
% Symbol Function/Value

%-------------------- Symbol Types --------------------

\begincom{symbol}\ftype{System Class}

\label Class Precedence List::
\typeref{symbol},
\typeref{t}

\label Description::

%% 11.0.0 11
%% 10.3.0 2
%% 10.3.0 3                                
%%% 11.0.0 12
% Some stuff about INTERN, IMPORT, SHADOWING-IMPORT, UNINTERN, etc. was commented out.

\term{Symbols} are used for their \term{object} identity to name various entities
in \clisp, including (but not limited to) linguistic entities such as
\term{variables} and \term{functions}.

\term{Symbols} can be collected together into \term{packages}.
A \term{symbol} is said to be \term{interned} in a \term{package} 
if it is \term{accessible} in that \term{package};
the same \term{symbol} can be \term{interned} in more than one \term{package}.
If a \term{symbol} is not \term{interned} in any \term{package}, 
it is called \term{uninterned}.

An \term{interned} \term{symbol} is uniquely identifiable by its \term{name} from 
any \term{package} in which it is \term{accessible}.

%!!! Sandra: I REALLY dislike this terminology.  Can't we call these "attributes" 
%     rather than "components" and leave out the discussion of "cells" entirely?

% My attempt at simplifying some of this convoluted discussion of ``cells''
% follows.  --sjl 16 Mar 92
%\term{Symbols} have the following components, or \term{cells}.
%(In each case, it is \term{implementation-dependent} how these \term{cells}
%are represented.  For example, they might be explicitly represented as a ``slots''
%in the \term{symbol} itself, or they might be represented as external associations 
%between the \term{symbol} and the \term{cell} \term{value} (if any) in some 
%\term{environment} or table, or in some cases they might not be explicitly
%represented at all.)

\term{Symbols} have the following attributes. 
%% "historically" => "historical" per Boyer/Kaufmann/Moore #11 (by X3J13 vote at May 4-5, 1994 meeting)
%% -kmp 9-May-94
For historical reasons,
these are sometimes referred to as \term{cells}, although the actual
internal representation of \term{symbols} and their attributes is
\term{implementation-dependent}.

\beginlist

%% 10.0.0 5
%% 10.2.0 1
\itemitem{\b{Name}}
 
The \term{name} of a \term{symbol} is a \term{string} used to identify the \term{symbol}.
Every \term{symbol} has a \term{name}, 
\issue{CHARACTER-PROPOSAL:2-6-3}
and the consequences are undefined if that \term{name} is altered.  
\endissue{CHARACTER-PROPOSAL:2-6-3}
The \term{name} is used as part of the external, printed representation of
the \term{symbol}; \seesection\CharacterSyntax.
\Thefunction{symbol-name} returns the \term{name} of a given \term{symbol}.
\issue{CHARACTER-PROPOSAL:2-6-2}
%% I replaced this on suggestion of KAB.  It doesn't match CHARACTER-PROPOSAL 2.6.2,
%% so who knows how it got here. -kmp 21-Jan-92
% The \term{characters} in a \term{symbol}'s \term{name} must be \term{graphic},
% and may come from any supported \term{character} \term{repertoire}.
A \term{symbol} may have any \term{character} in its \term{name}.
\endissue{CHARACTER-PROPOSAL:2-6-2}

%% 10.3.0 1
%% 10.0.0 3
%% 10.0.0 4
%% 10.1.0 1
%% 10.1.0 2
%% 10.1.0 3

%% 10.0.0 4
%% 10.3.0 4
%% 11.0.0 8
%% 11.0.0 10
%% 11.0.0 28
\itemitem{\b{Package}}

The \term{object} in this \term{cell} is called the \term{home package} 
of the \term{symbol}.  If the \term{home package} is \nil, the \term{symbol}
is sometimes said to have no \term{home package}.

When a \term{symbol} is first created, it has no \term{home package}.
When it is first \term{interned}, the \term{package} in which it is
initially \term{interned} becomes its \term{home package}.
The \term{home package} of a \term{symbol} can be \term{accessed}
by using \thefunction{symbol-package}.

If a \term{symbol} is \term{uninterned} from the \term{package} 
which is its \term{home package}, its \term{home package} is set to \nil.
Depending on whether there is another \term{package} in which the \term{symbol}
is \term{interned}, the symbol might or might not really be an \term{uninterned} \term{symbol}.
A \term{symbol} with no \term{home package} is therefore called 
\term{apparently uninterned}.

\issue{LISP-SYMBOL-REDEFINITION:MAR89-X3J13}
The consequences are undefined if an attempt is made to alter the \term{home package}
of a \term{symbol} 
%Added per Sandra -kmp 13-Dec-91
% ``accessible'' is wrong here -- sjl 16 Mar 92
%accessible
external
in \thepackage{common-lisp} or \thepackage{keyword}.
\endissue{LISP-SYMBOL-REDEFINITION:MAR89-X3J13}

%% 10.1.0 4
\itemitem{\b{Property list}}

The \term{property list} of a \term{symbol} provides a mechanism for
associating named attributes with that \term{symbol}.
The operations for adding and removing entries are \term{destructive}
to the \term{property list}.  \clisp\ provides \term{operators} both for
direct manipulation of \term{property list} \term{objects} 
 (\eg see \funref{getf}, \macref{remf}, and \funref{symbol-plist})
and for implicit manipulation of a \term{symbol}'s \term{property list} 
by reference to the \term{symbol} 
 (\eg see \funref{get} and \funref{remprop}).
The \term{property list} associated with a \term{fresh} \term{symbol} is 
initially \term{null}.
%!!! Sandra: What about the property list of symbols in the COMMON-LISP package?

%% 10.0.0 8                  
\itemitem{\b{Value}}

%!!! Sandra: You could avoid this awkwardness [in sentence 1] by not talking about
%     cells at all!  All you have to say is:
%      "If a symbol has a value attribute, it is said to be bound ..."
%!!! KAB: Does this first sentence really have to be said?
% My attempt at rewording this is below.  --sjl 16 Mar 92
%Conceptually, every \term{symbol} has a \term{value cell}, although
%no \term{implementation} is required to explicitly represent it as an \term{object}.
%A \term{symbol}'s \term{value cell} might or might not contain an \term{object}.
%If it does contain an \term{object}, it is said to be \term{bound},
If a symbol has a value attribute, it is said to be \term{bound},
and that fact can be detected by \thefunction{boundp}.
The \term{object} contained in the \term{value cell} of a \term{bound} \term{symbol}
is the \term{value} of the \term{global variable} named by that \term{symbol}, 
and can be \term{accessed} by \thefunction{symbol-value}.
A \term{symbol} can be made to be \term{unbound} by \thefunction{makunbound}.

The consequences are undefined if an attempt is made to change the \term{value}
of a \term{symbol} that names a \term{constant variable}, or to make such a 
\term{symbol} be \term{unbound}.

\itemitem{\b{Function}}

%!!! Sandra: See comment on "Value" above.
%!!! KAB: Does this first sentence really have to be said?
% My attempt at rewording this is below.  --sjl 16 Mar 92
%Conceptually, every \term{symbol} has a \term{function cell}, although
%no \term{implementation} is required to explicitly represent it as an \term{object}.
%A \term{symbol}'s \term{function cell} might or might not contain an \term{object}.
%If it does contain an \term{object}, it is said to be \term{fbound}, 
If a symbol has a function attribute, it is said to be \term{fbound},
and that fact can be detected by \thefunction{fboundp}.
If the \term{symbol} is the \term{name} of a \term{function} in the \term{global environment},
the \term{function cell} contains the \term{function}, 
and can be \term{accessed} by \thefunction{symbol-function}.
If the \term{symbol} is the \term{name} of either
   a \term{macro} in the \term{global environment} (see \funref{macro-function})
or a \term{special operator} (see \funref{special-operator-p}),
the \term{symbol} is \term{fbound}, 
and can be \term{accessed} by \thefunction{symbol-function},
but the \term{object} which the \term{function cell}
contains is of \term{implementation-dependent} \term{type} and purpose.
A \term{symbol} can be made to be \term{funbound} by \thefunction{fmakunbound}.

The consequences are undefined if an attempt is made to change the \term{functional value}
of a \term{symbol} that names a \term{special form}.

\endlist 

Operations on a \term{symbol}'s \term{value cell} and \term{function cell} are
sometimes described in terms of their effect on the \term{symbol} itself, but 
the user should keep in mind that there is an intimate relationship between the
contents of those \term{cells} and the \term{global variable} or 
global \term{function} definition, respectively.

\term{Symbols} are used as identifiers for \term{lexical variables} and 
lexical \term{function} definitions, but in that role, only their \term{object}
identity is significant.  \clisp\ provides no operation on a \term{symbol} that
can have any effect on a \term{lexical variable} or 
on a lexical \term{function} definition.

\label See Also::

{\secref\SymbolTokens},
{\secref\PotentialNumbersAsTokens},
{\secref\PrintingSymbols}

\endcom%{symbol}\ftype{System Class}

\begincom{keyword}\ftype{Type}

%Added per Barrett's suggestion. -kmp 23-Oct-90

\label Supertypes:: 

\typeref{keyword},
\typeref{symbol},
\typeref{t}

\label Description::

\Thetype{keyword} includes all \term{symbols} \term{interned} \thepackage{keyword}.

%!!! Non-standard use of "intern"?
\term{Interning} a \term{symbol} in \thepackage{keyword} has three automatic effects:

\beginlist
\item{1.} It causes the \term{symbol} to become \term{bound} to itself.
\item{2.} It causes the \term{symbol} to become an \term{external symbol}
	  of \thepackage{keyword}.
\item{3.} It causes the \term{symbol} to become a \term{constant variable}.
\endlist
 
\label See Also::

\funref{keywordp}

\endcom%{keyword}\ftype{Type}

%%% ========== SYMBOLP
\begincom{symbolp}\ftype{Function}

\label Syntax::

\DefunWithValues symbolp {object} {generalized-boolean}

\label Arguments and Values:: 

\param{object}---an \term{object}.

\param{generalized-boolean}---a \term{generalized boolean}.

\label Description::

%% 6.2.2 4
\TypePredicate{object}{symbol}

\label Examples::

\code
 (symbolp 'elephant) \EV \term{true}
 (symbolp 12) \EV \term{false}
 (symbolp nil) \EV \term{true}
 (symbolp '()) \EV \term{true}
 (symbolp :test) \EV \term{true}
 (symbolp "hello") \EV \term{false}
\endcode

\label Side Effects:\None.

\label Affected By:\None.

\label Exceptional Situations:\None!

\label See Also::

\funref{keywordp},
\typeref{symbol},
\funref{typep}

\label Notes::

\code
 (symbolp \param{object}) \EQ (typep \param{object} 'symbol)
\endcode

\endcom

%%% ========== KEYWORDP
\begincom{keywordp}\ftype{Function}

\label Syntax::

\DefunWithValues keywordp {object} {generalized-boolean}

\label Arguments and Values:: 

\param{object}---an \term{object}.

\param{generalized-boolean}---a \term{generalized boolean}.

\label Description::

%% 10.3.0 15
\Predicate{object}{a \term{keyword}\meaning{1}}

\label Examples::

\code
 (keywordp 'elephant) \EV \term{false}
 (keywordp 12) \EV \term{false}
 (keywordp :test) \EV \term{true}
 (keywordp ':test) \EV \term{true}
 (keywordp nil) \EV \term{false}
 (keywordp :nil) \EV \term{true}
 (keywordp '(:test)) \EV \term{false}
 (keywordp "hello") \EV \term{false}
 (keywordp ":hello") \EV \term{false}
 (keywordp '&optional) \EV \term{false}
\endcode

\label Side Effects:\None.

\label Affected By:\None.

\label Exceptional Situations:\None!

\label See Also::

\funref{constantp},
\typeref{keyword},
\funref{symbolp},
\funref{symbol-package}
                                               
\label Notes:\None.

\endcom

%-------------------- Symbol Creation --------------------

%%% ========== MAKE-SYMBOL
\begincom{make-symbol}\ftype{Function}

\label Syntax::

\DefunWithValues make-symbol {name} {new-symbol}

\label Arguments and Values:: 

%"simple string" => "string" per Barrett. -kmp 14-Feb-92
\param{name}---a \term{string}.

\param{new-symbol}---a \term{fresh}, \term{uninterned} \term{symbol}.

\label Description::

%% 10.3.0 5
\funref{make-symbol} creates and returns a \term{fresh}, \term{uninterned}
\term{symbol} whose \term{name} is the given \param{name}.
The \param{new-symbol} is neither \term{bound} nor \term{fbound} 
and has a \term{null} \term{property list}.

%% 10.3.0 6
It is \term{implementation-dependent} whether the \term{string} 
that becomes the \param{new-symbol}'s \term{name} is the given
\param{name} or a copy of it.  Once a \term{string}
has been given as the \param{name} \term{argument} to
\funref{make-symbol}, the consequences are undefined if a
subsequent attempt is made to alter that \term{string}.

\label Examples::

\code
 (setq temp-string "temp") \EV "temp"
 (setq temp-symbol (make-symbol temp-string)) \EV #:|temp|
 (symbol-name temp-symbol) \EV "temp"
 (eq (symbol-name temp-symbol) temp-string) \EV \term{implementation-dependent}
 (find-symbol "temp") \EV NIL, NIL
 (eq (make-symbol temp-string) (make-symbol temp-string)) \EV \term{false}
\endcode

\label Side Effects:\None.

% This sentence seems totally misplaced.  --sjl 16 Mar 92
%The \term{value} of the \term{place} named by \param{index}, if any, is modified.

%% Sandra thinks this is excessive.
%Creates a \term{fresh} \term{symbol}.

\label Affected By:\None.

\label Exceptional Situations::

%"simple string" => "string" per Barrett. -kmp 14-Feb-92
\Shouldchecktype{name}{a \term{string}}

\label See Also::

\funref{copy-symbol}

\label Notes::

No attempt is made by \funref{make-symbol} to convert the case
of the \term{name} to uppercase.  The only case conversion which ever 
occurs for \term{symbols} is done by the \term{Lisp reader}.
The program interface to \term{symbol} creation retains case,
and the program interface to interning symbols is case-sensitive.

\endcom

%%% ========== COPY-SYMBOL
\begincom{copy-symbol}\ftype{Function}

\label Syntax::

\DefunWithValues copy-symbol {symbol {\opt} copy-properties} {new-symbol}

\label Arguments and Values::

\param{symbol}---a \term{symbol}.

\param{copy-properties}---a \term{generalized boolean}.
  \Default{\term{false}}

\param{new-symbol}---a \term{fresh}, \term{uninterned} \term{symbol}.

\label Description::

%% 10.3.0 8
\issue{COPY-SYMBOL-PRINT-NAME:EQUAL}
\funref{copy-symbol} returns a \term{fresh}, \term{uninterned} \term{symbol},
the \term{name} of which is \funref{string=} to and possibly the \term{same} as
the \term{name} of the given \param{symbol}.
\endissue{COPY-SYMBOL-PRINT-NAME:EQUAL}

If \param{copy-properties} is \term{false},
the \param{new-symbol} is neither \term{bound} nor \term{fbound} 
and has a \term{null} \term{property list}.
If \param{copy-properties} is \term{true}, then
the initial \term{value} of \param{new-symbol} is
 the \term{value} of \param{symbol},
the initial \term{function} definition of \param{new-symbol} is
 the \term{functional value} of \param{symbol},
and the \term{property list} of \param{new-symbol} is
\issue{COPY-SYMBOL-COPY-PLIST:COPY-LIST}
 a \term{copy}\meaning{2} of the \term{property list} of \param{symbol}.
\endissue{COPY-SYMBOL-COPY-PLIST:COPY-LIST}

\label Examples::

\code
 (setq fred 'fred-smith) \EV FRED-SMITH
 (setf (symbol-value fred) 3) \EV 3
 (setq fred-clone-1a (copy-symbol fred nil)) \EV #:FRED-SMITH
 (setq fred-clone-1b (copy-symbol fred nil)) \EV #:FRED-SMITH
 (setq fred-clone-2a (copy-symbol fred t))   \EV #:FRED-SMITH
 (setq fred-clone-2b (copy-symbol fred t))   \EV #:FRED-SMITH
 (eq fred fred-clone-1a) \EV \term{false}
 (eq fred-clone-1a fred-clone-1b) \EV \term{false}
 (eq fred-clone-2a fred-clone-2b) \EV \term{false}
 (eq fred-clone-1a fred-clone-2a) \EV \term{false}
 (symbol-value fred) \EV 3
 (boundp fred-clone-1a) \EV \term{false}
 (symbol-value fred-clone-2a) \EV 3
 (setf (symbol-value fred-clone-2a) 4) \EV 4
 (symbol-value fred) \EV 3
 (symbol-value fred-clone-2a) \EV 4
 (symbol-value fred-clone-2b) \EV 3
 (boundp fred-clone-1a) \EV \term{false}
 (setf (symbol-function fred) #'(lambda (x) x)) \EV #<FUNCTION anonymous>
 (fboundp fred) \EV \term{true}
 (fboundp fred-clone-1a) \EV \term{false}
 (fboundp fred-clone-2a) \EV \term{false}
\endcode

\label Side Effects:\None.

%% Sandra thinks this is excessive.
%Creates a \term{fresh} \term{symbol}.

\label Affected By:\None.

\label Exceptional Situations::

\Shouldchecktype{symbol}{a \term{symbol}}

\label See Also::

\funref{make-symbol}

\label Notes::

Implementors are encouraged not to copy the \term{string} 
which is the \term{symbol}'s \term{name} unnecessarily.  
Unless there is a good reason to do so, the normal implementation
strategy is for the \param{new-symbol}'s \term{name} to
be \term{identical} to the given \param{symbol}'s \term{name}.

%% Barrett: This is only true if the implementation of MAKE-SYMBOL 
%   doesn't copy, which it might.
%
% \code
% (copy-symbol \param{x} nil) \EQ (make-symbol (symbol-name \param{x}))
% \endcode

\endcom

%%% ========== GENSYM
\begincom{gensym}\ftype{Function}

\label Syntax::

\DefunWithValues gensym {{\opt} x} {new-symbol}

\label Arguments and Values::

\param{x}---a \term{string} or a non-negative \term{integer}.
	    \HairyDefault.

\param{new-symbol}---a \term{fresh}, \term{uninterned} \term{symbol}.

\label Description::

%% 10.3.0 9
%% 10.3.0 10
%% 10.3.0 11

Creates and returns a \term{fresh}, \term{uninterned} \term{symbol},
as if by calling \funref{make-symbol}.  (The only difference between
\funref{gensym} and \funref{make-symbol} is in how the \param{new-symbol}'s 
\term{name} is determined.)

The \term{name} of the \param{new-symbol} is the concatenation 
of a prefix, which defaults to \f{"G"}, and
\issue{GENSYM-NAME-STICKINESS:LIKE-TEFLON}
a suffix, which is the decimal representation of a number that
defaults to \thevalueof{*gensym-counter*}.

If \param{x} is supplied, and is a \term{string}, then that \term{string} 
is used as a prefix instead of \f{"G"} for this call to \funref{gensym} only.

If \param{x} is supplied, and is an \term{integer}, then that \term{integer},
instead of \thevalueof{*gensym-counter*}, is used as the suffix
for this call to \funref{gensym} only.

If and only if no explicit suffix is supplied,
\varref{*gensym-counter*} is incremented after it is used.
\endissue{GENSYM-NAME-STICKINESS:LIKE-TEFLON}

\label Examples::

\code
 (setq sym1 (gensym)) \EV #:G3142
 (symbol-package sym1) \EV NIL
 (setq sym2 (gensym 100)) \EV #:G100
 (setq sym3 (gensym 100)) \EV #:G100
 (eq sym2 sym3) \EV \term{false}
 (find-symbol "G100") \EV NIL, NIL
 (gensym "T") \EV #:T3143
 (gensym) \EV #:G3144
\endcode

\label Side Effects::

%% Sandra thinks this is excessive.
%Creates a \term{fresh} \term{symbol}.

Might increment \varref{*gensym-counter*}.

\label Affected By::

\varref{*gensym-counter*}

\label Exceptional Situations::

\Shouldchecktype{x}{a \term{string} or a non-negative \term{integer}}

\label See Also::

\funref{gentemp},
\varref{*gensym-counter*}

\label Notes::

The ability to pass a numeric argument to \funref{gensym} has been deprecated;
explicitly \term{binding} \varref{*gensym-counter*} is now stylistically preferred.
(The somewhat baroque conventions for the optional argument are historical
in nature, and supported primarily for compatibility with older dialects
of \Lisp.   In modern code, it is recommended that the only kind of argument
used be a string prefix.  In general, though, to obtain more flexible control
of the \param{new-symbol}'s \term{name}, consider using \funref{make-symbol} instead.)

\endcom

%%% ========== *GENSYM-COUNTER*
\begincom{*gensym-counter*}\ftype{Variable}
\issue{GENSYM-NAME-STICKINESS:LIKE-TEFLON}

\label Value Type::

a non-negative \term{integer}.

\label Initial Value::

%A non-negative \term{integer}, the magnitude of which is 
\term{implementation-dependent}.

\label Description::

A number which will be used in constructing the \term{name} of 
the next \term{symbol} generated by \thefunction{gensym}.

\varref{*gensym-counter*} can be either \term{assigned} or \term{bound}
at any time, but its value must always be a non-negative \term{integer}.
 
\label Examples:\None.

\label Affected By::

\funref{gensym}.

\label See Also::

\funref{gensym}

\label Notes::

The ability to pass a numeric argument to \funref{gensym} has been deprecated;
explicitly \term{binding} \varref{*gensym-counter*} is now stylistically preferred.
\endissue{GENSYM-NAME-STICKINESS:LIKE-TEFLON}
\endcom

%%% ========== GENTEMP
\begincom{gentemp}\ftype{Function}

\label Syntax::

\DefunWithValues gentemp {{\opt} prefix package} {new-symbol}

\label Arguments and Values::

\param{prefix}---a \term{string}.
 \Default{\f{"T"}}

\param{package}---a \term{package designator}.
 \Default{the \term{current package}}

\param{new-symbol}---a \term{fresh}, \term{interned} \term{symbol}.

\label Description::

%% 10.3.0 14
\funref{gentemp} creates and returns a \term{fresh} \term{symbol},
\term{interned} in the indicated \param{package}.
%The \term{symbol} is guaranteed to be one that did not previously 
%exist in \param{package}. 
The \term{symbol} is guaranteed to be one that was not previously
\term{accessible} in \param{package}.
It is neither \term{bound} nor \term{fbound}, and has a \term{null}
\term{property list}.

The \term{name} of the \param{new-symbol} is the concatenation 
of the \param{prefix} and a suffix, which is taken from an internal
counter used only by \funref{gentemp}.  (If a \term{symbol} by that name
% already exists in \param{package}, the counter is incremented as
is already \term{accessible} in \param{package}, the counter is incremented as
many times as is necessary to produce a \term{name} that is not already the
\term{name} of a \term{symbol} \term{accessible} in \param{package}.)

\label Examples::

\code
 (gentemp) \EV T1298
 (gentemp "FOO") \EV FOO1299
 (find-symbol "FOO1300") \EV NIL, NIL
 (gentemp "FOO") \EV FOO1300
 (find-symbol "FOO1300") \EV FOO1300, :INTERNAL
 (intern "FOO1301") \EV FOO1301, :INTERNAL
 (gentemp "FOO") \EV FOO1302
 (gentemp) \EV T1303
\endcode

\label Side Effects::

Its internal counter is incremented one or more times.

\term{Interns} the \param{new-symbol} in \param{package}.

\label Affected By::

The current state of its internal counter, and
the current state of the \param{package}.

\label Exceptional Situations::

\Shouldchecktype{prefix}{a \term{string}}
\Shouldchecktype{package}{a \term{package designator}}

\label See Also::

\funref{gensym}

\label Notes::

The function \funref{gentemp} is deprecated.

If \param{package} is \thepackage{keyword},
the result is an \term{external symbol} of \param{package}.
Otherwise, the result is an \term{internal symbol} of \param{package}.

The \funref{gentemp} internal counter is independent of
\varref{*gensym-counter*}, the counter used by \funref{gensym}.  
There is no provision for accessing the \funref{gentemp} internal counter.

Just because \funref{gentemp} creates a \term{symbol} which did not
previously exist does not mean that such a \term{symbol} might not be
seen in the future (\eg in a data file---perhaps even created by the
same program in another session).  As such, this symbol is not truly
unique in the same sense as a \term{gensym} would be.  In particular,
programs which do automatic code generation should be careful not to
attach global attributes to such generated \term{symbols} (\eg
\declref{special} \term{declarations}) and then write them into a file
because such global attributes might, in a different session, end up
applying to other \term{symbols} that were automatically generated on
another day for some other purpose.

\endcom

%-------------------- Symbol Accessors --------------------

%%% ========== SYMBOL-FUNCTION
\begincom{symbol-function}\ftype{Accessor}

\label Syntax::

\DefunWithValues symbol-function {symbol} {contents}
\Defsetf         symbol-function {symbol} {new-contents}

\label Arguments and Values:: 

\issue{FUNCTION-TYPE:X3J13-MARCH-88}
% The symbol need not be fbound for the SETF form.  --sjl 16 Mar 92
%\param{symbol}---an \term{fbound} \term{symbol}.
\param{symbol}---a \term{symbol}.
\endissue{FUNCTION-TYPE:X3J13-MARCH-88}

\param{contents}---
\issue{FUNCTION-TYPE:X3J13-MARCH-88}
If the \param{symbol} is globally defined as a \term{macro} or a \term{special operator},
%% Removed per Barrett. -kmp 14-Feb-92
%and if the \param{symbol} is \term{fbound}, 
an \term{object} of \term{implementation-dependent} nature and identity is returned.
If the \param{symbol} is not globally defined as 
 either a \term{macro} or a \term{special operator},
and
 if the \param{symbol} is \term{fbound},
a \term{function} \term{object} is returned.
\endissue{FUNCTION-TYPE:X3J13-MARCH-88}

\issue{FUNCTION-TYPE:X3J13-MARCH-88}
\param{new-contents}---a \term{function}.
\endissue{FUNCTION-TYPE:X3J13-MARCH-88}
%% This shouldn't be needed, actually.
% \issue{FUNCTION-TYPE:X3J13-MARCH-88}
% The consequences of attempting to make the global \term{function} definition
% of a \term{symbol} be a \term{symbol}, a \term{list}, or the value returned
% by \funref{symbol-function} on the name of a \term{macro} or a
% \term{special form} are unspecified.
% \endissue{FUNCTION-TYPE:X3J13-MARCH-88}

\label Description::

%% 7.1.1 14
\term{Accesses} the \term{symbol}'s \term{function cell}.

\label Examples::

%Some examples involving FBOUNDP simplified/corrected per Moore #2 (first public review)
% -kmp 12-May-93
\code
 (symbol-function 'car) \EV #<FUNCTION CAR>
 (symbol-function 'twice) is an error   ;because TWICE isn't defined.
 (defun twice (n) (* n 2)) \EV TWICE
 (symbol-function 'twice) \EV #<FUNCTION TWICE>
 (list (twice 3)
       (funcall (function twice) 3)
       (funcall (symbol-function 'twice) 3))
\EV (6 6 6)
 (flet ((twice (x) (list x x)))
   (list (twice 3)
         (funcall (function twice) 3)
         (funcall (symbol-function 'twice) 3)))
\EV ((3 3) (3 3) 6)   
 (setf (symbol-function 'twice) #'(lambda (x) (list x x)))
\EV #<FUNCTION anonymous>
 (list (twice 3)
       (funcall (function twice) 3)
       (funcall (symbol-function 'twice) 3))
\EV ((3 3) (3 3) (3 3))
 (fboundp 'defun) \EV \term{true}
 (symbol-function 'defun)
\EV \term{implementation-dependent}
 (functionp (symbol-function 'defun))
\EV \term{implementation-dependent}
 (defun symbol-function-or-nil (symbol)
   (if (and (fboundp symbol) 
            (not (macro-function symbol))
            (not (special-operator-p symbol)))
       (symbol-function symbol)
       nil)) \EV SYMBOL-FUNCTION-OR-NIL
 (symbol-function-or-nil 'car) \EV #<FUNCTION CAR>
 (symbol-function-or-nil 'defun) \EV NIL
\endcode 

\label Side Effects:\None.

\label Affected By::

\macref{defun}

\label Exceptional Situations::

\Shouldchecktype{symbol}{a \term{symbol}}

Should signal \typeref{undefined-function} if \param{symbol} is not \term{fbound}
and an attempt is made to \term{read} its definition.  (No such error is signaled
on an attempt to \term{write} its definition.)

\label See Also::

\funref{fboundp},
\funref{fmakunbound},
\funref{macro-function},
\issue{SPECIAL-FORM-P-MISNOMER:RENAME}
\funref{special-operator-p}
\endissue{SPECIAL-FORM-P-MISNOMER:RENAME}

\label Notes::
%% 7.1.1 16
\funref{symbol-function} cannot \term{access} the value of a lexical function name
produced by \specref{flet} or \specref{labels}; it can \term{access} only
the global function value.

%!!! Sandra thinks this should be in the Description.  I'm not so sure. -kmp 13-Dec-91
\macref{setf} may be used with 
\funref{symbol-function} to replace a global function
definition when the \term{symbol}'s function definition 
does not represent a \term{special operator}.
 
\code
(symbol-function \param{symbol}) \EQ (fdefinition \param{symbol})
\endcode
However, \funref{fdefinition} accepts arguments other than just \term{symbols}.

\endcom

%%% ========== SYMBOL-NAME
\begincom{symbol-name}\ftype{Function}

\label Syntax::

\DefunWithValues symbol-name {symbol} {name}

\label Arguments and Values:: 

\param{symbol}---a \term{symbol}.

%"simple string" => "string" per Barrett. -kmp 14-Feb-92
\param{name}---a \term{string}.

\label Description::

%% 10.2.0 2
\funref{symbol-name} returns the \term{name} of \param{symbol}.
\issue{LISP-SYMBOL-REDEFINITION:MAR89-X3J13}
The consequences are undefined if \param{name} is ever modified.
\endissue{LISP-SYMBOL-REDEFINITION:MAR89-X3J13}

\label Examples::

\code
 (symbol-name 'temp) \EV "TEMP" 
 (symbol-name :start) \EV "START"
 (symbol-name (gensym)) \EV "G1234" ;for example
\endcode

\label Side Effects:\None.                                 

\label Affected By:\None.

\label Exceptional Situations::

\Shouldchecktype{symbol}{a \term{symbol}}

\label See Also:\None.

\label Notes:\None.

\endcom

%%% ========== SYMBOL-PACKAGE
\begincom{symbol-package}\ftype{Function}

\label Syntax::

\DefunWithValues symbol-package {symbol} {contents}

\label Arguments and Values:: 

\param{symbol}---a \term{symbol}.

\param{contents}---a \term{package} \term{object} or \nil.

\label Description::

%% 10.3.0 15
Returns the \term{home package} of \param{symbol}.

\label Examples::

\code
 (in-package "CL-USER") \EV #<PACKAGE "COMMON-LISP-USER">
 (symbol-package 'car) \EV #<PACKAGE "COMMON-LISP">
 (symbol-package 'bus) \EV #<PACKAGE "COMMON-LISP-USER">
 (symbol-package :optional) \EV #<PACKAGE "KEYWORD">
 ;; Gensyms are uninterned, so have no home package.
 (symbol-package (gensym)) \EV NIL
 (make-package 'pk1) \EV #<PACKAGE "PK1">
 (intern "SAMPLE1" "PK1") \EV PK1::SAMPLE1, NIL
 (export (find-symbol "SAMPLE1" "PK1") "PK1") \EV T
 (make-package 'pk2 :use '(pk1)) \EV #<PACKAGE "PK2">
 (find-symbol "SAMPLE1" "PK2") \EV PK1:SAMPLE1, :INHERITED
 (symbol-package 'pk1::sample1) \EV #<PACKAGE "PK1">
 (symbol-package 'pk2::sample1) \EV #<PACKAGE "PK1">
 (symbol-package 'pk1::sample2) \EV #<PACKAGE "PK1">
 (symbol-package 'pk2::sample2) \EV #<PACKAGE "PK2">
 ;; The next several forms create a scenario in which a symbol
 ;; is not really uninterned, but is "apparently uninterned",
 ;; and so SYMBOL-PACKAGE still returns NIL.
 (setq s3 'pk1::sample3) \EV PK1::SAMPLE3
 (import s3 'pk2) \EV T
 (unintern s3 'pk1) \EV T
 (symbol-package s3) \EV NIL
 (eq s3 'pk2::sample3) \EV T
\endcode

\label Side Effects:\None.

\label Affected By::

\funref{import},
\funref{intern},
\funref{unintern}

\label Exceptional Situations::

\Shouldchecktype{symbol}{a \term{symbol}}

\label See Also::

\funref{intern}

\label Notes:\None.

\endcom


%%% ========== SYMBOL-PLIST
\begincom{symbol-plist}\ftype{Accessor}

\label Syntax::

\DefunWithValues symbol-plist {symbol} {plist}
\Defsetf         symbol-plist {symbol} {new-plist}

\label Arguments and Values:: 

\param{symbol}---a \term{symbol}.

\param{plist}, \param{new-plist}---a \term{property list}.

\label Description::

%% 10.1.0 14
%% 10.1.0 16
\term{Accesses} the \term{property list} of \param{symbol}.

\label Examples::

\code
 (setq sym (gensym)) \EV #:G9723
 (symbol-plist sym) \EV ()
 (setf (get sym 'prop1) 'val1) \EV VAL1
 (symbol-plist sym) \EV (PROP1 VAL1)
 (setf (get sym 'prop2) 'val2) \EV VAL2
 (symbol-plist sym) \EV (PROP2 VAL2 PROP1 VAL1)
 (setf (symbol-plist sym) (list 'prop3 'val3)) \EV (PROP3 VAL3)
 (symbol-plist sym) \EV (PROP3 VAL3)
\endcode

\label Side Effects:\None.

\label Affected By:\None.

\label Exceptional Situations::

\Shouldchecktype{symbol}{a \term{symbol}}

\label See Also::

\funref{get},
\funref{remprop}
%!!! Barmar thinks there should be a concept section describing
%    the format of a plist.

\label Notes::

The use of \macref{setf} should be avoided, since a \term{symbol}'s
\term{property list} is a global resource that can contain information 
established and depended upon by unrelated programs in the same \term{Lisp image}.

\endcom


%%% ========== SYMBOL-VALUE
\begincom{symbol-value}\ftype{Accessor}

\label Syntax::

\DefunWithValues symbol-value {symbol} {value}
\Defsetf         symbol-value {symbol} {new-value}

\label Arguments and Values:: 

\param{symbol}---a \term{symbol} that must have a \term{value}.

\param{value}, \param{new-value}---an \term{object}.

\label Description::

\term{Accesses} the \term{symbol}'s \term{value cell}.

\label Examples::

\code
 (setf (symbol-value 'a) 1) \EV 1
 (symbol-value 'a) \EV 1
 ;; SYMBOL-VALUE cannot see lexical variables.
 (let ((a 2)) (symbol-value 'a)) \EV 1
 (let ((a 2)) (setq a 3) (symbol-value 'a)) \EV 1
 ;; SYMBOL-VALUE can see dynamic variables.
 (let ((a 2)) 
   (declare (special a)) 
   (symbol-value 'a)) \EV 2
 (let ((a 2)) 
   (declare (special a)) 
   (setq a 3)
   (symbol-value 'a)) \EV 3
 (let ((a 2))
   (setf (symbol-value 'a) 3)
   a) \EV 2
 a \EV 3
 (symbol-value 'a) \EV 3
 (let ((a 4))
   (declare (special a))
   (let ((b (symbol-value 'a)))
     (setf (symbol-value 'a) 5)
     (values a b))) \EV 5, 4
 a \EV 3
 (symbol-value :any-keyword) \EV :ANY-KEYWORD
 (symbol-value 'nil) \EV NIL
 (symbol-value '()) \EV NIL
 ;; The precision of this next one is \term{implementation-dependent}.
 (symbol-value 'pi) \EV 3.141592653589793d0  
\endcode

\label Side Effects:\None.

\label Affected By::

\funref{makunbound},
\funref{set},
\specref{setq}

\label Exceptional Situations::

\Shouldchecktype{symbol}{a \term{symbol}}

Should signal \typeref{unbound-variable} if \param{symbol} is \term{unbound}
and an attempt is made to \term{read} its \term{value}.  (No such error is signaled
on an attempt to \term{write} its \term{value}.)

\label See Also::

\funref{boundp}, \funref{makunbound}, \funref{set}, \specref{setq}

\label Notes::

\funref{symbol-value} can be used to get the value of a \term{constant variable}.
%% 7.1.1 12
\funref{symbol-value} cannot \term{access} the value of a \term{lexical variable}.

\endcom

%-------------------- Symbol Properties --------------------

%%% ========== GET
\begincom{get}\ftype{Accessor}

\label Syntax::

\DefunWithValues get {symbol indicator {\opt} default} {value}
\Defsetf         get {symbol indicator {\opt} default} {new-value}

\label Arguments and Values:: 

\param{symbol}---a \term{symbol}.

\param{indicator}---an \term{object}.

\param{default}---an \term{object}.
 \Default{\nil}

\param{value}---if the indicated property exists,
		   the \term{object} that is its \term{value};
		otherwise, the specified \param{default}.

\param{new-value}---an \term{object}.

\label Description::

%% 10.1.0 8
%% Barrett: Redundant with next paragraph.
% \funref{get} searches the \term{property list} of \param{symbol} for an
% indicator \funref{eq} to \param{indicator}.

% \funref{get} returns the corresponding value of an indicator from 
% the \term{property list} of \param{symbol} \funref{eq} to \param{indicator},
% if one is found.
\funref{get} finds a \term{property} 
on the \term{property list}\meaning{2} of \param{symbol} 
whose \term{property indicator} is \term{identical} to \param{indicator},
and returns its corresponding \term{property value}.
\issue{PLIST-DUPLICATES:ALLOW}
If there are multiple \term{properties}\meaning{1} with that \term{property indicator},
\funref{get} uses the first such \term{property}.
\endissue{PLIST-DUPLICATES:ALLOW}
If there is no \term{property} with that \term{property indicator},
\param{default} is returned.

% %% 10.1.0 12
% \macref{setf} may be used with \funref{get} to create a new property-value
% pair, possibly replacing an old pair with the same property name.
\macref{setf} of \funref{get} may be used to associate a new \term{object}
with an existing indicator already on the \param{symbol}'s \term{property list},
or to create a new assocation if none exists.
\issue{PLIST-DUPLICATES:ALLOW}
If there are multiple \term{properties}\meaning{1} with that \term{property indicator},
\macref{setf} of \funref{get} associates the \param{new-value} 
with the first such \term{property}.
\endissue{PLIST-DUPLICATES:ALLOW}
\issue{SETF-GET-DEFAULT:EVALUATED-BUT-IGNORED}
When a \funref{get} \term{form} is used as a \macref{setf} \param{place},
any \param{default} which is supplied is evaluated according to normal
left-to-right evaluation rules, but its \term{value} is ignored.
\endissue{SETF-GET-DEFAULT:EVALUATED-BUT-IGNORED}

\label Examples::

\code
 (defun make-person (first-name last-name)
   (let ((person (gensym "PERSON")))
     (setf (get person 'first-name) first-name)
     (setf (get person 'last-name) last-name)
     person)) \EV MAKE-PERSON
 (defvar *john* (make-person "John" "Dow")) \EV *JOHN*
 *john* \EV #:PERSON4603
 (defvar *sally* (make-person "Sally" "Jones")) \EV *SALLY*
 (get *john* 'first-name) \EV "John"
 (get *sally* 'last-name) \EV "Jones"
 (defun marry (man woman married-name)
   (setf (get man 'wife) woman)
   (setf (get woman 'husband) man)
   (setf (get man 'last-name) married-name)
   (setf (get woman 'last-name) married-name)
   married-name) \EV MARRY
 (marry *john* *sally* "Dow-Jones") \EV "Dow-Jones"
 (get *john* 'last-name) \EV "Dow-Jones"
 (get (get *john* 'wife) 'first-name) \EV "Sally"
 (symbol-plist *john*)
\EV (WIFE #:PERSON4604 LAST-NAME "Dow-Jones" FIRST-NAME "John")
 (defmacro age (person &optional (default ''thirty-something)) 
   `(get ,person 'age ,default)) \EV AGE
 (age *john*) \EV THIRTY-SOMETHING
 (age *john* 20) \EV 20
 (setf (age *john*) 25) \EV 25
 (age *john*) \EV 25
 (age *john* 20) \EV 25
\endcode

\label Side Effects:\None.

\label Affected By:\None.

\label Exceptional Situations::

\Shouldchecktype{symbol}{a \term{symbol}}

\label See Also::

\funref{getf},
\funref{symbol-plist},
\funref{remprop}

\label Notes::

\code
 (get x y) \EQ (getf (symbol-plist x) y)
\endcode

\term{Numbers} and \term{characters} are not recommended for use 
as \param{indicators} in portable code since \funref{get} tests 
with \funref{eq} rather than \funref{eql}, and consequently 
the effect of using such \param{indicators} is 
\term{implementation-dependent}.

% \code
%  (get 'clyde 'species) \EV NIL
%  (setf (get 'clyde 'species) 'elephant) \EV elephant
%  (get 'clyde 'species) \EV ELEPHANT
% \endcode
% \param{Default} may be supplied to \funref{get} in this context; 
% it is ignored by the \macref{setf} expander function
% for \funref{get}, but
% may be useful in such macros as \macref{push} 
% that are related to \macref{setf}:
% 
% \code
%  (push item (get sym 'token-stack '(initial-item)))
% \endcode
% means approximately the same as
% 
% \code
%  (setf (get sym 'token-stack '(initial-item))
%        (cons item (get sym 'token-stack '(initial-item))))
% \endcode
% which in turn would be treated as
% 
% \code
%  (setf (get sym 'token-stack)
%        (cons item (get sym 'token-stack '(initial-item))))
% \endcode
% 

There is no way using \funref{get} to distinguish an absent property from
one whose value is \param{default}.  However, see \funref{get-properties}.

% %% 10.1.0 15
% Using \funref{get} on the result of \funref{symbol-plist} does not work;
% the \term{symbol} itself must be given to \funref{get}. \funref{getf} can
% be used to extract properties from a disembodied property list.

\endcom 
                                                   

%%% ========== REMPROP
\begincom{remprop}\ftype{Function}

\label Syntax::

\DefunWithValues remprop {symbol indicator} {generalized-boolean}

\label Arguments and Values:: 

\param{symbol}---a \term{symbol}.

\param{indicator}---an \term{object}.

\param{generalized-boolean}---a \term{generalized boolean}.

\label Description::

%% 10.1.0 13
\funref{remprop} removes from the \term{property list}\meaning{2} of \param{symbol}
a \term{property}\meaning{1} with a \term{property indicator}
% EQ => identical -kmp 14-Jul-93
\term{identical} to \param{indicator}.
\issue{PLIST-DUPLICATES:ALLOW}
If there are multiple \term{properties}\meaning{1} with the \term{identical} key,
\funref{remprop} only removes the first such \term{property}.
\endissue{PLIST-DUPLICATES:ALLOW}
\funref{remprop} returns \term{false} if no such \term{property} was found,
or \term{true} if a property was found.

The \term{property indicator} 
and the corresponding \term{property value} 
are removed in an undefined order
by destructively splicing the property list.  
\issue{REMF-DESTRUCTION-UNSPECIFIED:X3J13-MAR-89}
The permissible side-effects correspond to those permitted for \macref{remf},
such that:

\code
 (remprop \i{x} \i{y}) \EQ (remf (symbol-plist \i{x}) \i{y})
\endcode
\endissue{REMF-DESTRUCTION-UNSPECIFIED:X3J13-MAR-89}

\label Examples::

\code
 (setq test (make-symbol "PSEUDO-PI")) \EV #:PSEUDO-PI
 (symbol-plist test) \EV ()
 (setf (get test 'constant) t) \EV T
 (setf (get test 'approximation) 3.14) \EV 3.14
 (setf (get test 'error-range) 'noticeable) \EV NOTICEABLE
 (symbol-plist test) 
\EV (ERROR-RANGE NOTICEABLE APPROXIMATION 3.14 CONSTANT T)
 (setf (get test 'approximation) nil) \EV NIL
 (symbol-plist test) 
\EV (ERROR-RANGE NOTICEABLE APPROXIMATION NIL CONSTANT T)
 (get test 'approximation) \EV NIL
 (remprop test 'approximation) \EV \term{true}
 (get test 'approximation) \EV NIL
 (symbol-plist test)
\EV (ERROR-RANGE NOTICEABLE CONSTANT T)
 (remprop test 'approximation) \EV NIL
 (symbol-plist test)
\EV (ERROR-RANGE NOTICEABLE CONSTANT T)
 (remprop test 'error-range) \EV \term{true}
 (setf (get test 'approximation) 3) \EV 3
 (symbol-plist test)
\EV (APPROXIMATION 3 CONSTANT T)
\endcode

\label Side Effects::

The \term{property list} of \param{symbol} is modified.

\label Affected By:\None.

\label Exceptional Situations::

\Shouldchecktype{symbol}{a \term{symbol}}

\label See Also::

\macref{remf}, \funref{symbol-plist}

\label Notes::

\term{Numbers} and \term{characters} are not recommended for use as
\param{indicators} in portable code since \funref{remprop} tests with
\funref{eq} rather than \funref{eql}, and consequently the effect of
using such \param{indicators} is \term{implementation-dependent}.  
Of course, if you've gotten as far as needing to remove such a
\term{property}, you don't have much choice---the time to have been
thinking about this was when you used \macref{setf} of \funref{get} to
establish the \term{property}.
%Barrett: Bletch.

\endcom

%-------------------- Symbol Function/Value --------------------

%%% ========== BOUNDP
\begincom{boundp}\ftype{Function}

\label Syntax::

\DefunWithValues boundp {symbol} {generalized-boolean}

\label Arguments and Values:: 

\param{symbol}---a \term{symbol}.

\param{generalized-boolean}---a \term{generalized boolean}.

\label Description::

%% 7.1.1 18   
\Predicate{symbol}{\term{bound}}

\label Examples::

\code
 (setq x 1) \EV 1
 (boundp 'x) \EV \term{true}
 (makunbound 'x) \EV X
 (boundp 'x) \EV \term{false}
 (let ((x 2)) (boundp 'x)) \EV \term{false}
 (let ((x 2)) (declare (special x)) (boundp 'x)) \EV \term{true}
\endcode

\label Affected By:\None.

\label Exceptional Situations::

\Shouldchecktype{symbol}{a \term{symbol}}

\label See Also::

\funref{set},
\specref{setq},
\funref{symbol-value},
\funref{makunbound}

\label Notes::

% I see no lexical environment here.  --sjl 16 Mar 92
%The \term{lexical environment} is ignored.
\Thefunction{boundp} determines only whether a \term{symbol} has a
value in the \term{global environment}; any \term{lexical bindings}
are ignored.

\endcom

%%% ========== MAKUNBOUND
\begincom{makunbound}\ftype{Function}

%!!! Barmar thinks that fmakunbound should move to dict-functions.

\label Syntax::

\DefunWithValues makunbound {symbol} {symbol}

\label Arguments and Values::

\param{symbol}---a \term{symbol}

\label Description::

Makes the \param{symbol} be \term{unbound},
regardless of whether it was previously \term{bound}.

\label Examples::

\code
 (setf (symbol-value 'a) 1)
 (boundp 'a) \EV \term{true}
 a \EV 1
 (makunbound 'a) \EV A
 (boundp 'a) \EV \term{false}
\endcode

\label Side Effects::

The \term{value cell} of \param{symbol} is modified.

\label Affected By:\None.

\label Exceptional Situations::

\Shouldchecktype{symbol}{a \term{symbol}}

\label See Also::

\funref{boundp}, \funref{fmakunbound}

\label Notes:\None.

\endcom

%%% ========== SET
\begincom{set}\ftype{Function}

\label Syntax::

\DefunWithValues set {symbol value} {value}

\label Arguments and Values:: 

\param{symbol}---a \term{symbol}.

%% 7.1.2 9
\issue{ARGUMENTS-UNDERSPECIFIED:SPECIFY}
\param{value}---an \term{object}.
\endissue{ARGUMENTS-UNDERSPECIFIED:SPECIFY}

\label Description::

%% 7.1.2 8
\funref{set} changes the contents of the \term{value cell} of \term{symbol}
to the given \term{value}.

\code
(set \param{symbol} \param{value}) \EQ (setf (symbol-value \param{symbol}) \param{value})
\endcode

\label Examples::

\code
 (setf (symbol-value 'n) 1) \EV 1
 (set 'n 2) \EV 2
 (symbol-value 'n) \EV 2
 (let ((n 3))
   (declare (special n))
   (setq n (+ n 1))
   (setf (symbol-value 'n) (* n 10))
   (set 'n (+ (symbol-value 'n) n))
   n) \EV 80
 n \EV 2
 (let ((n 3))
   (setq n (+ n 1))
   (setf (symbol-value 'n) (* n 10))
   (set 'n (+ (symbol-value 'n) n))
   n) \EV 4
 n \EV 44
 (defvar *n* 2)
 (let ((*n* 3))
   (setq *n* (+ *n* 1))
   (setf (symbol-value '*n*) (* *n* 10))
   (set '*n* (+ (symbol-value '*n*) *n*))
   *n*) \EV 80
  *n* \EV 2
 (defvar *even-count* 0) \EV *EVEN-COUNT*
 (defvar *odd-count* 0) \EV *ODD-COUNT*
 (defun tally-list (list)
   (dolist (element list)
     (set (if (evenp element) '*even-count* '*odd-count*)
          (+ element (if (evenp element) *even-count* *odd-count*)))))
 (tally-list '(1 9 4 3 2 7)) \EV NIL
 *even-count* \EV 6
 *odd-count* \EV 20
\endcode

\label Side Effects::

The \term{value} of \param{symbol} is changed.

\label Affected By:\None.

\label Exceptional Situations:\None.

\label See Also::

\specref{setq}, \specref{progv}, \funref{symbol-value}

\label Notes::

The function \funref{set} is deprecated.

%% 7.1.2 11
\funref{set} cannot change the value of a \term{lexical variable}.

\endcom

%-------------------- Symbol Errors --------------------

\begincom{unbound-variable}\ftype{Condition Type}

\label Class Precedence List::
\typeref{unbound-variable},
\typeref{cell-error},
\typeref{error},
\typeref{serious-condition},
\typeref{condition},
\typeref{t}

\label Description::

\Thetype{unbound-variable} consists of \term{error} \term{conditions}
that represent attempts to \term{read} the \term{value} of an \term{unbound variable}.
%Barrett: "name"?
%KMP: An "unbound variable" is by definition just a "name".  See glossary.

The name of the cell (see \typeref{cell-error}) is the \term{name} of the 
\term{variable} that was \term{unbound}.

\label See Also::

\funref{cell-error-name}

\endcom%{unbound-variable}\ftype{Condition Type}
